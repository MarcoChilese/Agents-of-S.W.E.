\subsection{Capitolato C2}

\subsubsection{Informazioni sul capitolato}

\begin{itemize}
	\item Nome: Colletta: piattaforma di raccolta dati mediante esercizi di grammatica;
	\item Proponente: Mivoq S.r.l.
\end{itemize}

\subsubsection{Descrizione Capitolato e Obiettivo Finale}
Il secondo capitolato propone di sviluppare una piattaforma collaborativa che serva simultaneamente per 
creare e svolgere esercizi di grammatica e per raccogliere dati. I dati raccolti saranno utilizzati per insegnare ad un elaboratore a svolgere esercizi, mediante tecniche di apprendimento automatico di tipo supervisionato\glossario.\\
L'obiettivo è creare un'applicazione Web o Mobile fruibile da insegnanti ed allievi. Gli insegnanti inseriranno gli esercizi da svolgere e la relativa soluzione. Il compito degli allievi è scegliere che tipo di esercizi svolgere e completarli. Tutti i dati che saranno inseriti da insegnanti e allievi saranno raccolti ed elaborati dal sistema di apprendimento, che, dopo una determinata mole di informazioni acquisite sarà in grado di creare e svolgere autonomamente esercizi grammaticali, così da ridurre notevolmente il lavoro dell'insegnante.

\subsubsection{Dominio Tecnologico}
\begin{itemize}
	\item \textbf{Firebase\footnote{\hyperref[Link al sito]{https://firebase.google.com/}}}: servizio esistente per immagazzinare dati;
	\item \textbf{Hunpus\footnote{\hyperref[Link al sito]{https://github.com/mivoq/hunpos}}, FreeLing\footnote{\hyperref[Link al sito]{http://nlp.lsi.upc.edu/freeling/}} o simili}: software opensource per lo svolgimento degli esercizi. 
\end{itemize}

\subsubsection{Valutazione del Capitolato}

\paragraph{Aspetti Positivi}
\begin{itemize}
	\item Viene fornito l'intero ambiente di sviluppo;
	\item Sistema apprendimento automatico preesistente;
	\item Chiarezza sull'ambito di utilizzo dell'applicativo.
\end{itemize}

\paragraph{Criticità}
\begin{itemize}
	\item Iterazione con sistemi di apprendimento automatico.
\end{itemize}

\paragraph{Conclusioni e Motivazioni della scelta} \-\\
Nonostante la valutazione positiva da parte del gruppo, sull'utilizzo di sistemi moderni di apprendimento automatico, l'ambito di utilizzo dell'applicativo ha influito notevolmente nella valutazione finale creando poco interesse a riguardo.\\
Il poco stimolo, quindi, ha avuto un ruolo fondamentale sulla decisione di non seguire la scelta del corrente capitolato.