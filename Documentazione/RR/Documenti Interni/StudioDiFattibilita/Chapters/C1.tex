\subsection{Capitolato C1}

\subsubsection{Informazioni sul capitolato}
\begin{itemize}
	\item Nome: Butterfly - monitor per processi CI/CD;
	\item Proponente: Imola Informatica.
\end{itemize}

\subsubsection{Descrizione Capitolato e Obiettivo Finale}
Il primo capitolato e il problema della notifica di eventuali problematiche nelle architetture di Continuos Integration e Continuos Delivery all'interno di realtà enterprise di grandi dimensioni. Ognuno degli strumenti utilizzati all'interno dell'architettura fornisce un proprio metodo specifico di notifica dei messaggi/problemi spesso con limitate capacità di configurazione. \\
La soluzione proposta, attraverso un pattern di Publisher/Subscriber, è quella di realizzare una serie di componenti software che si interfaccino con gli strumenti utilizzati dall'architettura di CI/CD, recuperino  le segnalazioni e provvedano a riportarle nella forma desiderata al client.  
\subsubsection{Dominio Tecnologico}

	\begin{itemize}
		\item \textbf{Redmine \footnote{\hyperref[Link al sito]{https://www.redmine.org/}}}: applicazione web per la gestione flessibile di progetti. 
		\item \textbf{Gitlab \footnote{\hyperref[Link al sito]{https://about.gitlab.com/}}}: piattaforma web per la gestione di repository git e di funzione trouble ticket.
		\item \textbf{Maven \footnote{\hyperref[Link al sito]{https://maven.apache.org/}}}: strumento di gestione software basato su Java e build automation. 
		\item \textbf{WireMock \footnote{\hyperref[Link al sito]{http://wiremock.org/}}}: simulatore di HTTP-based API. 
		\item \textbf{SonarQube \footnote{\hyperref[Link al sito]{https://www.sonarqube.org/}}}: strumento di Continuous Inspection per la qualità del codice. 
		\item \textbf{JFrog Artifactory \footnote{\hyperref[Link al sito]{https://jfrog.com/artifactory/}}}: manager di repository universale. 
		\item \textbf{Docker \footnote{\hyperref[Link al sito]{https://www.docker.com/}}}: container virtuale per deployment automatizzati. 
		\item \textbf{Telegram \footnote{\hyperref[Link al sito]{https://telegram.org/}}}: servizio di messaggistica istantanea basato su cloud. 
		\item \textbf{Slack \footnote{\hyperref[Link al sito]{https://slack.com/}}}: hub collaborativo. 
		\item \textbf{Apache Kafka \footnote{\hyperref[Link al sito]{https://kafka.apache.org/}}}: piattaforma di streaming distribuito, basata su un'astrazione di registro di commit distribuito. 
		\item \textbf{Java \footnote{\hyperref[Link al sito]{https://www.java.com/it/}}}:  linguaggio di programmazione ad alto livello, orientato agli oggetti e a tipizzazione statica
		\item \textbf{JUnit \footnote{\hyperref[Link al sito]{https://junit.org/junit5/}}}: framework di unit testing per Java
		\item \textbf{Python \footnote{\hyperref[Link al sito]{https://www.python.org/}}}: linguaggio di programmazione  ad alto livello orientato agli oggetti. 
		\item \textbf{NodeJS \footnote{\hyperref[Link al sito]{https://nodejs.org/it/}}}: piattaforma open source event-driven per l'esecuzione di codice JavaScript server-side
	\end{itemize}
	
\subsubsection{Valutazione del Capitolato}

\paragraph{Aspetti Positivi} ~\\
\begin{itemize}
	\item Molte tecnologie di rilievo da apprendere e utilizzare, sopratutto per quanto riguarda il linguaggio di programmazione: Python, molto richiesto e utilizzato in tutti gli ambiti informatici;
	\item Integrazione e cooperazione di più tecnologie usate simultaneamente. 
\end{itemize}

\paragraph{Criticità} ~\\
\begin{itemize}
	\item Molte tecnologie sconosciute rendono difficile una stima dei tempi e dei costi finali, aumentato i rischi di fallimento o di prolungamento del progetto; 
	\item Per alcune tecnologie la documentazione fornita non è di facile apprendimento per i neofiti o parzialmente assente. 
\end{itemize}


\paragraph{Conclusioni e Motivazioni della scelta} ~\\
In conclusione il progetto è sembrato interessante al gruppo, ma troppo rischioso da implementare poiché molte delle tecnologie richieste sono a noi sconosciute e di difficile apprendimento, 
e tale rischio potrebbe allungare i tempi di consegna prestabiliti dal gruppo.  

