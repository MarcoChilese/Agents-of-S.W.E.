\subsection{Capitolato C4}\label{C4}

\subsubsection{Descrizione generale}
\begin{itemize}
	\item Nome: MegAlexa: arricchitore di skill di Amazon Alexa;
	\item Proponente: Zero12 S.r.l.
\end{itemize}

\subsubsection{Descrizione Capitolato e Obiettivo Finale}
Il progetto consiste nello sviluppo di un'estensione delle capacità di Amazon Alexa\glossario, in grado di dare all'utente la possibilità di creare delle routine personalizzate formate da varie attività già esistenti.
Nello specifico, è richiesto di sviluppare un applicativo mobile o web, che, dati dei connettori ad attività, potrà inserirli all'interno di uno workflow, eseguibile tramite comando vocale, il quale li eseguirà in sequenza.

\subsubsection{Dominio Tecnologico}
\begin{itemize}
	\item \textbf{Amazon Web Services(AWS)\footnote{\url{https://aws.amazon.com/}}\glossario}: piattaforma di servizi cloud di Amazon;
	\item \textbf{API Gateway\footnote{\url{https://aws.amazon.com/it/api-gateway/}}\glossario}: servizio AWS per la gestione delle API;
	\item \textbf{Lambda\footnote{\url{https://aws.amazon.com/it/lambda/}}\glossario}: servizio AWS per l’esecuzione di codice senza la gestione del lato server;
	\item \textbf{DynamoDB\footnote{\url{https://aws.amazon.com/it/dynamodb/}}\glossario}: database AWS non relazionale con garanzie di prestazioni affidabili su ogni scala;
	\item \textbf{HTML\footnote{\url{https://www.w3.org/html/}}\glossario \& CSS\footnote{\url{https://www.w3.org/Style/CSS/}}\glossario}: linguaggi utilizzati per lo sviluppo del front-end del plugin;
	\item \textbf{NodeJS\footnote{\url{https://nodejs.org/it/}}\glossario}: piattaforma di sviluppo JavaScript;
	\item \textbf{Twitter Bootstrap\footnote{\url{https://getbootstrap.com/}}\glossario}: strumenti per la creazione di siti e applicazioni web responsive;
	\item \textbf{Javascript\footnote{\url{https://www.javascript.com/}}}: linguaggio di scripting web lato client;
	\item \textbf{Swift\footnote{\url{https://developer.apple.com/swift/}}\glossario o Kotlin\footnote{\url{https://kotlinlang.org/}}\glossario}: linguaggi per lo sviluppo di applicazioni mobile.
\end{itemize}

\subsubsection{Valutazione del Capitolato}
\paragraph{Aspetti Positivi}
\begin{itemize}
	\item Argomento molto interessante;
	\item Vasta gamma di tecnologie da usare;
	\item Certe tecnologie sono già conosciute da alcuni membri del gruppo.
\end{itemize}

\paragraph{Aspetti negativi}
\begin{itemize}
	\item Molte tecnologie e molto specifiche non conosciute dal gruppo;
	\item Da quanto è stato capito dal gruppo, non viene fornita la possibilità di usare un Amazon Alexa per testare il prodotto;
	\item Apparente semplicità del progetto.
\end{itemize}

\paragraph{Conclusioni e Motivazioni della scelta}\-\\
Ad un primo sguardo, il progetto era parso interessante al gruppo, ma dopo successive discussioni è stato sorpassato da altri capitolati. Resta comunque una delle seconde scelte.