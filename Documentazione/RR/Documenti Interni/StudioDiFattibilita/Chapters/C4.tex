\subsection{Capitolato C4}

\subsubsection{Informazioni sul capitolato}
Il progetto consiste nello sviluppo di un’estensione delle capacità di Alexa Amazon, in grado di dare all’utente la possibilità di creare la propria routine personalizzata formata da varie attività già presenti.
\subsubsection{Descrizione Capitolato e Obiettivo Finale}
Nello specifico, è richiesto di sviluppare un applicativo mobile o web, che, dati dei connettori ad attività, potrà inserirli all’interno di uno workflow, eseguibile tramite comando vocale, il quale li eseguirà in sequenza.
\subsubsection{Dominio Tecnologico}
•	Amazon Web Services : piattaforma di servizi cloud di Amazon.
•	API Gateway : servizio AWS per la gestione delle API.
•	Lambda :  servizio AWS per l’esecuzione di codice senza la gestione del lato server.
•	DynamoDB : database AWS non relazionale con garanzie di prestazioni affidabili su ogni scala.
•	NodeJS : piattaforma di sviluppo JavaScript.
•	Twitter Bootstrap : strumenti per la creazione di siti e applicazioni web responsive.
•	HTML5 , CSS3 e Javascript : linguaggi per lo sviluppo web.
•	Swift o kotlin : linguaggi per lo sviluppo mobile.	

\subsubsection{Valutazione del Capitolato}

\paragraph{Aspetti Positivi}

\paragraph{Criticità}

\paragraph{Conclusioni e Motivazioni della scelta}
Il progetto è parso interessante per il gruppo, in particolare per l’argomento trattato, posizionandolo nei primi posti tra le scelte, vista anche la conoscenza di molte delle tecnologie richieste. La troppa specificità delle altre tecnologie richieste, ha invece ha portato invece la preferenza di altri capitolati che richiedessero tecnologie più generiche.