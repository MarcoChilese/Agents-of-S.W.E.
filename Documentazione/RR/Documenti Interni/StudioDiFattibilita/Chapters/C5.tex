\subsection{Capitolato C5}

\subsubsection{Informazioni sul capitolato}
\begin{itemize}
  \item{Nome}: P2PCS: Piattaforma di Car Sharing Condominiale Peer to Peer;
  \item{Proponente}: GaiaGo S.r.l.
\end{itemize}

\subsubsection{Descrizione Capitolato e Obiettivo Finale}
Lo scopo finale del capitolato è la realizzazione di un'applicazione Android\glossario che permette la condivisione, in ambito condominiale, della propria automobile durante i periodi nei quali non è utilizzata.\\
All'interno dell'applicazione è possibile segnare, attraverso un calendario, i giorni in cui l'autoveicolo è a disposizione in quanto il proprietario non ha necessità di utilizzarlo.

\subsubsection{Dominio Tecnologico}
\begin{itemize}
  \item \textbf{Henshin - Movens\footnote{\texttt{http://henshingroup.com/}}}: piattaforma software per la mobilità, smart cities e IoT management;
  \item \textbf{Kotlin/Java}: utilizzate per lo sviluppo di applicazioni Android;
  \item \textbf{Google Cloud e NodeJS}: gestione back-end;
  \item \textbf{Octalysis framework}\footnote{\texttt{https://yukaichou.com/gamification-examples/octalysis-complete-gamification-framework}}: framework di Gamification\glossario.
  
\end{itemize}

\subsubsection{Valutazione del Capitolato}
\paragraph{Aspetti Positivi}
  \begin{itemize}
    \item Chiarezza sul prodotto desiderato e sulle aspettative;
    \item Ampio materiale fornito per lo sviluppo dell'applicativo, tra cui l'ambiente di test.
  \end{itemize}
\paragraph{Criticità}
  \begin{itemize}
    \item Poca chiarezza sull'ambito di utilizzo finale dell'applicazione.
  \end{itemize}

\paragraph{Conclusioni e Motivazioni della scelta}\-\\
Il gruppo ha valutato il capitolato in esame come non particolarmente stimolante, nonostante lo abbia giudicato di fattibile realizzazione. Sebbene il prodotto finale sia risultato chiaro alla comprensione generale, sono sorti alcuni dubbi sul fruitore finale del prodotto da realizzare, nonché sul suo effettivo ambito di utilizzo.
