\subsection{Capitolato C6}\label{C6}

\subsubsection{Informazioni sul capitolato}
\begin{itemize}
  \item{Nome}: Soldino;
  \item{Proponente}: RedBabel di Milo Ertola \& Alessandro Maccagnan.
\end{itemize}
\subsubsection{Descrizione Capitolato e Obiettivo Finale}
Il capitolato in esame richiede lo sviluppo di una piattaforma per il calcolo ed il pagamento del VAT (l'equivalente I.V.A. italiana) applicata ad una criptovaluta, Cubit\glossario. \\
Il risultato finale si propone di essere una piattaforma Web/UI avente come attori principali: il Governo, i proprietari di aziende ed i cittadini. In questo ambito ogni attore può:
\begin{itemize}
  \item \textit{Governo}:
  \begin{itemize}
    \item Minare e distribuire Cubit;
    \item Gestire la lista delle aziende registrate;
    \item Controllare le tasse pagate dalle aziende.
  \end{itemize}
  \item \textit{Proprietari di aziende}:
  \begin{itemize}
    \item Registrare la propria azienda presso la lista mantenuta dal Governo;
    \item Gestire i servizi o prodotti offerti dall'azienda;
    \item Acquistare i servizi o prodotti offerti dalle altre aziende;
    \item Creare un documento PDF contenente l'assessment della VAT;
    \item Calcolare e scaricare la ricevuta di pagamento della VAT;
    \item Pagare la VAT.
  \end{itemize}
  \item \textit{Cittadino:}
  \begin{itemize}
    \item Comparare servizi o prodotti dalle aziende.
  \end{itemize}
\end{itemize}

\subsubsection{Dominio Tecnologico}

\begin{itemize}
  \item \textbf{Ethereum}: piattaforma decentralizzata per la creazione e pubblicazione peer-to-peer di contratti intelligenti;
  \item \textbf{Blockchain Technology};
  \item \textbf{JavaScript, HTML/CSS, React, Redux, SCSS}: linguaggi utilizzati per la realizzazione della parte front-end del prodotto;
  \item \textbf{MetaMusk\footnote{\texttt{https://metamask.io/}}}: per eseguire l'applicazione Ethereum direttamente sul broswer;
  \item \textbf{Etherscan\footnote{\texttt{https://etherscan.io/}}}: piattaforma per Ethereum per la ricerca di smart contracts;
  \item \textbf{Zeppelin\footnote{\texttt{https://blog.zeppelin.solutions/}}}: permette l'analisi tra i dati raccolti ed alcuni linguaggi di programmazione;
  \item \textbf{EIP-712}: standard di compilazione.
\end{itemize}

\subsubsection{Valutazione del Capitolato}
\paragraph{Aspetti Positivi}
\begin{itemize}
  \item Tecnologia innovativa e molto ricerca.
\end{itemize}

\paragraph{Aspetti negativi}
\begin{itemize}
  \item Ambito complesso sia per sviluppo che per realizzazione del prodotto;
  \item Nessuna conoscenza pregressa del dominio di applicazione, all'infuori di alcuni linguaggi utilizzati per lo sviluppo del front-end;
\end{itemize}

\paragraph{Conclusioni e Motivazioni della scelta}\-\\
Sebbene il gruppo abbia valutato positivamente le tecnologie innovative coinvolte nello sviluppo del prodotto, la totale mancanza di conoscenza pregressa e l'ampio numero di nuove tecnologie utilizzate hanno influito notevolmente nella valutazione finale del capitolato in esame. \\
Il gruppo ha giudicato che la difficoltà complessiva nella realizzazione di quanto desiderato da Red Babel fosse eccessiva, visto il background di conoscenze del gruppo.
