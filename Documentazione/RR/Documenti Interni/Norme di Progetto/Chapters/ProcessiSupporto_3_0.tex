\subsection{Documentazione}

\subsubsection{Descrizione}
Questo capitolo descrive i dettagli su come deve essere redatta e verificata la documentazione durante il ciclo di vita del software. Le norme sono tassativamente valide per tutti i documenti formali.
\subsubsection{Ciclo di vita documentazione}
Il ciclo di vita previsto della documentazione si può suddividere principalmente in tre processi: 
\begin{itemize}
	\item \textbf{Sviluppo}: è il processo di stesura, eseguita dal redattore, dove descrive il ticket\glossario assegnato dal responsabile. Una volta 	terminata la fase di scrittura del documento, il redattore lo segnala al responsabile, che assegnerà a un verificatore il compito di	analizzare il lavoro svolto;
 	\item \textbf{Verifica}: è il processo eseguito dai verificatori designati dal responsabile, il loro compito è controllare che il redattore abbia scritto il documento nella norma e in maniera corretta grammaticalmente e strutturalmente;
 	\item \textbf{Approvato}: è il processo conclusivo, in cui il verificatore ha terminato il suo compito di controllo e comunica al responsabile il termine del lavoro. Il responsabile procederà a confermare il documento e ad eseguire il rilascio.
\end{itemize} 

\subsubsection{Template}
Il \texttt{gruppo} ha deciso di strutturare un template \LaTeX per dare uniformità a tutti i documenti. Il template facilita e velocizza la stesura, poiché i redattori devono concentrarsi solo ed esclusivamente al contenuto e non alla layout.  

\subsubsection{Struttura documenti}
Ogni documento segue una determinata struttura, predefinita e accordata dal \texttt{gruppo}:

\paragraph{Prima pagina}:
\paragraph{Piè di pagina}:
\paragraph{Nomenclatura}:
\paragraph{Tabelle}:
\paragraph{Indice Sezioni}:


\subsubsection{Documenti Correnti}
Sono descritti brevemente i documenti formali da consegnare:
\begin{itemize}
	\item \textbf{Analisi dei Requisiti}:
	\item \textbf{Glossario}:
	\item \textbf{Norme di Progetto}:
	\item \textbf{Piano di Progetto}:
	\item \textbf{Piano di Qualifica}:
	\item \textbf{Studio di Fattibilità}:
\end{itemize}


\subsubsection{Norme}

\paragraph{Struttura dei documenti}
\paragraph{Norme tipografiche}

\subsubsection{Struttura documentazione}

\subsubsection{Gestione termini Glossario}

\subsubsection{Ambiente}

\subsection{Qualità}

\subsubsection{Descrizione}

\subsubsection{Classificazione dei processi}

\subsubsection{Procedure}


