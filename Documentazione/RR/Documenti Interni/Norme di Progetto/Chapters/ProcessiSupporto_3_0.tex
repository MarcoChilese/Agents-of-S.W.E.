\subsection{Documentazione}

\subsubsection{Descrizione}
Questo capitolo descrive i dettagli su come deve essere redatta e verificata la documentazione durante il ciclo di vita del software. Le norme sono tassativamente valide per tutti i documenti formali.
\subsubsection{Ciclo di vita documentazione}
Il ciclo di vita previsto della documentazione si può suddividere principalmente in tre processi: 
\begin{itemize}
	\item \textbf{Sviluppo}: è il processo di stesura, eseguita dal redattore, dove descrive il ticket\glossario assegnato dal responsabile. Una volta 	terminata la fase di scrittura del documento, il redattore lo segnala al responsabile, che assegnerà a un verificatore il compito di	analizzare il lavoro svolto;
 	\item \textbf{Verifica}: è il processo eseguito dai verificatori designati dal responsabile, il loro compito è controllare che il redattore abbia scritto il documento nella norma e in maniera corretta grammaticalmente e strutturalmente;
 	\item \textbf{Approvato}: è il processo conclusivo, in cui il verificatore ha terminato il suo compito di controllo e comunica al responsabile il termine del lavoro. Il responsabile procederà a confermare il documento e ad eseguire il rilascio.
\end{itemize} 

\subsubsection{Template}
Il \texttt{gruppo} ha deciso di strutturare un template \LaTeX per dare uniformità a tutti i documenti. Il template facilita e velocizza la stesura, poiché i redattori devono concentrarsi solo ed esclusivamente al contenuto e non alla layout.  

\subsubsection{Struttura documenti}
Ogni documento segue una determinata struttura, predefinita e accordata dal \texttt{gruppo}:

\paragraph{Prima pagina}
La prima pagina di ogni documento ha la stessa struttura: il logo del \texttt{gruppo} centrato in alto, con sotto, sempre centrato, il nome del \texttt{gruppo} e il capitolato scelto. Appena sotto è posizionato il titolo del documento e una tabella contenente informazioni relative al documento, ovvero, la versione, i nome dei redattori e dei verificatori, lo stato (che può essere "confermato" o "work in progress"), l'utilizzo che avrà nel progetto (interno o esterno) e i destinatari.

\paragraph{Piè di pagina}
Il fondo pagina di tutti i documenti è molto pulito, contiene solamente il numero della pagina e, se presenti, i riferimenti bibliografici alle fonti utilizzate nella pagina corrente.

\paragraph{Nomenclatura}:
Le seguenti regole valgono per tutti i documenti eccetto per la lettera di presentazione. La nomenclatura è un aspetto fondamentale che abbiamo deciso di struttura nel seguente modo: 
\begin{itemize}
	\item \textbf{vX.Y.Z }: rappresenta la versione del documento con X , Y e Z numeri non negativi:
	\begin{itemize}
		\item \textbf{X}: rappresenta il numero di pubblicazioni ufficiali del documento in passato; se il valore è 0 significa che il documento non è mai stato pubblicato. Ogni qualvolta viene pubblicato Y e Z vengono azzerati e X viene incrementato di una unità;
		\item \textbf{Y}: identifica il numero di verifiche avvenute con successo, ogni qualvolta viene effettuata una verifica il valore di Z viene azzerato;
		\item \textbf{Z}: identifica il numero di volte che il documento è stato modificato prima di una pubblicazione e/o verifica.
	\end{itemize}
	\item Il formato dei file è .tex durante la fase di sviluppo, mentre dopo l'approvazione da parte del responsabile verrà creato un file con formato .pdf che rappresenta la pubblicazione in via ufficiosa.
\end{itemize}

\paragraph{Tabelle}
TODO : da definire nel prossimo incontro

\paragraph{Indice}
L'indice è strutturato nel seguente modo: titolo, argomento, e numero pagina. Ovviamente ogni titolo dell'argomento è un link alla pagina contenente lo stesso.

\subsubsection{Documenti Correnti}
Sono descritti brevemente i documenti formali da consegnare:
\begin{itemize}
	\item \textbf{Analisi dei Requisiti}:
	\item \textbf{Glossario}:
	\item \textbf{Norme di Progetto}:
	\item \textbf{Piano di Progetto}:
	\item \textbf{Piano di Qualifica}:
	\item \textbf{Studio di Fattibilità}:
\end{itemize}


\subsubsection{Norme}

\paragraph{Struttura dei documenti}
\paragraph{Norme tipografiche}

\subsubsection{Struttura documentazione}

\subsubsection{Gestione termini Glossario}

\subsubsection{Ambiente}

\subsection{Qualità}

\subsubsection{Descrizione}

\subsubsection{Classificazione dei processi}

\subsubsection{Procedure}


