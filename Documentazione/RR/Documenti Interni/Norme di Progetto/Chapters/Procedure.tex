\subsubsection{Gestione degli Strumenti di Versionamento}\label{ProcessiOrganizzativi_Procedure_GestioneStrumentiVersionamento}
	Come strumento per la gestione del versionamento dei files si è scelto di utilizzare una \textit{Repository}\glossario su \textit{GitHub}\glossario. La gestione di tale \textit{Repository}\glossario è compito dell'\textit{Amministratore di Progetto}.

\paragraph{Uso dei Branch} ~\\
	Al fine di agevolare il più possibile il parallelismo, evitando al contempo quanto più possibile eventuali problematiche in fase di \textit{merge}\glossario, sono stati creati i seguenti branch\glossario:
	\begin{itemize}
	\item \textbf{master}: questo branch contiene solamente i documenti che si trovano in stato \textbf{"Approvato"}, i quali formano dunque la \textit{baseline}\glossario;
	\item \textbf{develop}: questo è il branch di "sviluppo". Esso contiene tutti i documenti che, sebbene siano considerati ultimati per quanto riguarda la loro stesura, sono in attesa di approvazione o in fase di verifica;
	\item \textbf{feature/"nomeDocumento"}: vi sono quattro branch\glossario distinti questo tipo, nello specifico: "feature/analisiRequisiti", "feature/normeProgetto, "feature/pianoProgetto" e "feature/pianoQualifica". Ciascuno di questi è specializzato nell'avanzamento della stesura del documento a cui si riferisce;
	\item \textbf{feature/revisione"nomeDocumento"}: questi branch vengono sfruttati qualora un documento presente nel branch \textit{develop}, in sede di verifica, dovesse necessitare di modifiche non immediate.
	\end{itemize}

\paragraph{Norme delle Commit} ~\\
	Ogni \textit{commit}\glossario, ovvero ciascuna modifica alla repository\glossario \textit{Agents-of-S.W.E.} deve essere caratterizzata da una descrizione sensata, eventualmente accompagnata ad un riferimento esplicito ad una \textit{issue}\glossario aperta, al fine di agevolare una piena, e non onerosa, comprensione da parte di ogni membro del gruppo.

\paragraph{Norme dei Merge tra Branch} ~\\
	Al fine di rispettare la caratterizzazione che si è deciso di dare al branch \textbf{master}, e per agevolare un lavoro sistematico ed organizzato del lavoro si sono stabilite le seguenti norme in sede di \textit{merge}\glossario tra diversi branch\glossario:
	\begin{itemize}
	\item \textbf{Merge develop-feature/"nomeDocumento"}: questo merge avviene solo quando gli incaricati alla stesura del documento a cui si riferisce il branch: feature/"nomeDocumento" ritengono ultimata questa prima attività. Il documento in questione, dunque, si considera completo di ogni sua parte essenziale, eccezzion fatta per eventuali modifiche, anche cospicue, da apportare a seguito di un'attività di verifica e/o approvazione;
	\item \textbf{Merge develop-feature/revisione"nomeDocumento"}: questo merge avviene qualora le modifche necessarie al documento in questione siano risolte con successo;
	\item \textbf{Merge master-develop}: questo merge avviene solo quando ogni documento contenuto nel branch: "develop" è stato verificato ed approvato, e si considera dunque pronto per il rilascio.
	\end{itemize}


\subsubsection{Gestione degli Strumenti di Coordinamento}

\paragraph{Tasks} ~\\
	La suddivisione del lavoro in \textit{tasks}\glossario è compito del \textit{Responsabile di Progetto}. Lo strumento scelto per la creazione e gestione di questi \textit{tasks}\glossario è lo stesso Git\glossario, il quale mette a disposizione l'utile strumento delle \textit{issue}\glossario.\\
	La creazione di un \textit{task}\glossario da parte del \textit{Responsabile di Progetto}, risulterà dunque essere l'istanziazione di una \textit{issue}\glossario caratterizzata dalle seguenti proprietà:
	\begin{itemize}
	\item \textit{Titolo} significativo;
	\item \textit{Descrizione} concisa ma caratteristica ed esplicativa del problema da affrontare;
	\item \textit{Uno o più tags} associati a particolarità del \textit{task} in questione, e/o al/ai documento/i a cui si riferiscono. Tali etichette consentono una rapida catalogazione delle \textit{issue} stesse;
	\item \textit{Data di scadenza} che rappresenta il termine ultimo entro cui tale \textit{issue} deve essere chiusa.
	\end{itemize}
	E' importante far notare che, sebbene l'onere della suddivisione del lavoro, e dunque la creazione e gestione dei \textit{tasks}\glossario e dei conseguenti \textit{tickets}\glossario, ricada sul \textit{Responsabile di progetto}, ciascun membro del gruppo ha la facoltà di creare \textit{tasks}\glossario, a patto che tale compito veda lui come unico assegnatario. Tali \textit{tasks}\glossario dovranno inoltre essere approvati dal \textit{Responsabile} per essere validi.

\paragraph{Tickets} ~\\
	I \textit{tickets}\glossario rappresentano l'operazione di assegnazione dei \textit{tasks}\glossario, e quindi in questo caso delle \textit{issue}\glossario, ad uno o più specifici membri del gruppo. Tale operazione è responsabilità unica del \textit{Responsabile di Progetto} a meno che non si tratti di \textit{tasks}\glossario (approvati dal Responsabile) creati da un membro del gruppo, ed assegnati autonomamente a sè stesso.\\
	Questa operazione di \textit{ticketing} può avvenire in due modalità distinte:
	\begin{itemize}
	\item \textbf{Proattivamente}: nel caso in cui l'assegnatario del task in questione sia già noto, ed indicato come tale, alla creazione della issue da parte del \textit{Responsabile}. E' importante notare come questa sia l'unica modalità di \textit{ticketing} possibile nel caso in cui il creatore del task sia un membro diverso dal \textit{Responsabile di Progetto};
	\item \textbf{Retroattivamente}: nel caso in cui uno o più membri del gruppo vengano designati, in un secondo momento, come assegnatari di una issue già precedentemente esistente. Questa modalità di ticketing consente di gestire situazioni in cui non è utile individuare subito un assegnatario, come nel caso di task di importanza manginale e/o scandeze molto permissive, oppure invece si tratti di un compito le cui complessità sono emerse solo in seconda battuta, e necessiti dunque di un maggior apporto lavorativo per rispettarne le scadenze.
	\end{itemize}
