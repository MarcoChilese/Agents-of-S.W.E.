\section{Processi di supporto}\label{ProcessiSupporto}

\subsection{Documentazione}\label{ProcessiSupporto_Documentazione}

\subsubsection{Descrizione}
Questo capitolo descrive i dettagli su come deve essere redatta e verificata la documentazione durante il ciclo di vita del software. Le norme sono tassativamente valide per tutti i documenti formali.
\subsubsection{Ciclo di vita documentazione}
Il ciclo di vita previsto della documentazione si può suddividere principalmente in tre processi: 
\begin{itemize}
	\item \textbf{Sviluppo}: è il processo di stesura, eseguita dal redattore, dove sviluppa il task\glossario assegnato dal responsabile. Una volta terminata la fase di scrittura del documento, il redattore lo segnalerà al responsabile, il quale assegnerà ad un verificatore il compito di analizzare il lavoro svolto;
 	\item \textbf{Verifica}: è il processo eseguito dai verificatori designati dal responsabile. Il loro compito è quello di controllare che il redattore abbia scritto il documento adottando le regole indicate nel documento \textit{Norme di Progetto} ed in maniera grammaticalmente e strutturalmente corretta;
 	\item \textbf{Approvato}: è il processo conclusivo, in cui il verificatore ha terminato il suo compito di controllo e comunica al responsabile il termine del lavoro. Il responsabile procederà a confermare il documento ed ad eseguirne il rilascio.
\end{itemize} 

\subsubsection{Template}
Il gruppo ha deciso di strutturare un template \LaTeX  per dare uniformità a tutti i documenti. Il template facilita e velocizza la stesura, poiché i redattori devono concentrarsi solo ed esclusivamente sul contenuto e non sul layout.  

\subsubsection{Struttura documenti}\label{ProcessiSupporto_Documentazione_StrutturaDocumenti}
Ogni documento segue una determinata struttura, predefinita e accordata dal gruppo:

\paragraph{Prima pagina} \-\\
La prima pagina di ogni documento ha la stessa struttura: il logo del gruppo centrato in alto con sotto, sempre centrato, il nome del gruppo ed il capitolato scelto. Appena sotto è posizionato il titolo del documento ed una tabella contenente informazioni relative al documento, ovvero la versione, i nome dei redattori e dei verificatori, lo stato (che può essere "confermato" o "work in progress"), l'utilizzo che avrà nel progetto (interno o esterno) ed i destinatari.

\paragraph{Nomenclatura} \-\\
Le seguenti regole valgono per tutti i documenti eccetto per la lettera di presentazione. La nomenclatura è un aspetto fondamentale che abbiamo deciso di strutturare nel seguente modo: 
\begin{itemize}
	\item \textbf{vX.Y.Z }: rappresenta la versione del documento con X, Y e Z numeri non negativi:
	\begin{itemize}
		\item \textbf{X}: rappresenta il numero di pubblicazioni ufficiali del documento in passato; se il valore è 0 significa che il documento non è mai stato pubblicato. Ogni qualvolta viene pubblicato Y e Z vengono azzerati e X viene incrementato di una unità;
		\item \textbf{Y}: identifica il numero di verifiche avvenute con successo, ogni qualvolta viene effettuata una verifica il valore di Z viene azzerato;
		\item \textbf{Z}: identifica il numero di volte che il documento è stato modificato prima di una pubblicazione e/o verifica.
	\end{itemize}
	\item Il formato dei file è \textit{.tex} durante la fase di sviluppo. Dopo l'approvazione da parte del responsabile, verrà creato un file con formato \textit{.pdf} che rappresenta la pubblicazione ufficiale.
\end{itemize}

\paragraph{Struttura indice} \-\\
In tutta la documentazione, fatta eccezione per i verbali, dopo la pagina di presentazione è presente l'indice del documento. La struttura è standard: numero e titolo del capitolo, eventuali sottosezioni e paragrafi con indicato a fianco la pagina del contenuto. Ogni titolo è un link alla pagina del contenuto. 

\paragraph{Struttura tabelle} \-\\
Il gruppo ha deciso di standardizzare il formato delle tabelle come descritto:
\begin{itemize}
	\item Intestazione a sfondo blu e scritte bianche;
	\item Corpo con righe alternate di colore bianco e grigio, per facilitare la lettura.
\end{itemize}

\paragraph{Elenco tabelle} \-\\
In tutti i documenti è presente un elenco delle tabelle utilizzate, ove presenti. \'E posizionato appena dopo l'indice.  

\paragraph{Elenco figure} \-\\
In tutti i documenti è presente un elenco delle figure utilizzate, ove presenti. \'E posizionato appena dopo l'elenco delle tabelle.

\paragraph{Elenco Riferimenti} \-\\
In tutti i documenti, se necessario, è presente un elenco dei riferimenti alle fonti utilizzate per stilare il documento in esame. \'E posizionato all'inizio del documento, dopo l'elenco delle tabelle e delle figure.

\paragraph{Piè di pagina} \-\\ 
Le pagine con indice e presentazione del documento contengono dei numeri progressivi romani, mentre le pagine con il contenuto il numero della pagina; possono essere presenti riferimenti bibliografici;

\paragraph{Registro Modifiche} \-\\
Ogni documento, eccezion fatta per verbali e lettera di presentazione, presenta un registro delle modifiche chiamato "Changelog". \'E strutturato sotto forma di tabella, che contiene in ordine cronologico tutte le modifiche identificabili dalla versione.  Ogni riga contiene la versione del documento, la data, il nome di chi ha effettuato la modifica, con il suo corrispettivo ruolo, ed infine una breve descrizione della modifica effettuata.

%Per adesso i link sono fucsia

\subsubsection{Norme tipografiche}
\begin{itemize}
	\item \textbf{Glossario}: i termini contenuti nel glossario si possono identificare dal carattere G maiuscolo e corsivo a pedice della parola interessata, per esempio Norme\textsubscript{G};	 		
	\item \textbf{Nome gruppo}: in qualsiasi documento, quando si fa riferimento al gruppo si è deciso di adottare il seguente font: \texttt{Agents of S.W.E.};
	\item \textbf{Elenchi puntati}: ogni elemento di un elenco puntato deve essere seguito da un punto e virgola, eccezion fatta per l'ultimo che sarà seguito dal punto. La prima lettera di ogni item dovrà essere maiuscola;
	\item \textbf{Stile testo}:
	\begin{itemize}
		\item \textbf{Corsivo}: è utilizzato per citare tecnologie esterne, per riferimenti a documentazione (es. Analisi dei Requisiti, Piano di Progetto, ecc.), estensione dei file (es. .pdf, .tex, ecc.) e per identificare i ruoli del progetto (Responsabile,Analista, ecc.);	
		\item \textbf{Grassetto}: le parole in grassetto identificano il titolo di una sezione, sottosezione, paragrafo e un elemento di un elenco puntato;
		\item \textbf{URI}:	i link esterni sono evidenti per il colore blu.
	\end{itemize}
	\item \textbf{Formati}:
	\begin{itemize}
		\item \textbf{Date}: sono scritte seguendo il formato YYYY-MM-DD, dove YYYY rappresenta l'anno, MM il mese e DD il giorno;
		\item \textbf{Valuta}: si è adottato il formato XXX.YY con Y che denota le cifre decimali, X le cifre intere ed il punto (.) come delimitatore tra decimale e parte intera.
	\end{itemize}
\end{itemize}


\subsubsection{Documenti Correnti}
\label{ProcessiSupporto_Documentazione_DocumentiCorrenti}
Sono descritti brevemente i documenti formali da consegnare:
\begin{itemize}
	\item \textbf{Analisi dei Requisiti}: ha un utilizzo prettamente esterno, inoltre ha l'obiettivo di esporre e scomporre i requisiti\glossario del progetto. Contiene i casi d'uso ed i diagrammi di interazione con l'utente. Viene redatto dagli analisti dopo una profonda analisi del capitolato ed eventuali colloqui con la proponente; 
	\item \textbf{Glossario}: ha un utilizzo prettamente esterno ed ha lo scopo di dare una definizione ai termini più specifici usati nei documenti formali;
	\item \textbf{Norme di Progetto}: è utilizzato internamente, ed espone gli standard e le direttive utilizzate dal gruppo per sviluppare il progetto in tutta la sua interezza;
	\item \textbf{Piano di Progetto}: ha un utilizzo esterno, ed espone come il gruppo ha deciso di impiegare le risorse di tempo ed umane;
	\item \textbf{Piano di Qualifica}: ha un utilizzo esterno, descrive gli standard e gli obiettivi che il gruppo dovrà raggiungere per garantire la qualità di prodotto e processo;
	\item \textbf{Studio di Fattibilità}: è un documento interno, espone pregi e difetti di tutti i capitolati in esame. Descrive i motivi per cui il gruppo ha scelto il capitolato corrente.
\end{itemize}

\subsubsection{Ambiente}\label{ProcessiSupporto_Documentazione_Ambiente} 
Per uniformare e strutturare al meglio la scrittura dei documenti il gruppo ha adottato il formato \LaTeX. 
\textit{TexMaker\footnote{\texttt{\url{http://www.xm1math.net/texmaker/}}}} è utilizzato per la stesura, si è optato per questo editor perchè open-source ed integra un controllo ortografico della lingua italiana. 
Per la costruzione dei diagrammi è stato optato per l'utilizzo di script in \textit{Python}, scritti e testati dal gruppo, per uniformare la procedura in tutta la documentazione.

\subsection{Qualità} \label{qualita}

\subsubsection{Introduzione}
Il contenuto di questa sezione descrive le metriche e i criteri di qualità di processo e prodotto che vengono utilizzati nel documento \textit{Piano di Qualifica}.

\subsubsection{Classificazione Processi}
Per garantire una corretta struttura, il gruppo ha deciso di introdurre una nomenclatura per identificare i processi descritti nel documento. I processi saranno identificati così:
\begin{center}
	\textbf{PRC[num]}
\end{center}
dove:
\begin{itemize}
	\item \textbf{num}: rappresenta il numero identificativo del processo formato da due cifre intere a partire da 1, unico per tutto il documento.
\end{itemize}

\subsubsection{Classificazione Metriche}
Per garantire una corretta struttura, il gruppo ha deciso di introdurre una nomenclatura per identificare le metriche utilizzate. Si potranno quindi identificare cosi:
\begin{center}
	\textbf{MT[mcat][cat][num]}
\end{center}
\begin{itemize}
		\item \textbf{mcat}: identifica le metriche in base a quale macrocategoria andranno a misurare. Può assumere i seguenti valori:
	\begin{itemize}
		\item \textbf{PC}: per indicare le metriche di processo;
		\item \textbf{PD}: per indicare le metriche di prodotto;
		\item \textbf{TS}: per indicare le metriche di test.
	\end{itemize}
	\item \textbf{cat}: identifica la categoria di appartenenza, se esiste, altrimenti è vuota. Per le metriche di prodotto può assumere i seguenti valori:
	\begin{itemize}
		\item \textbf{D}: per indicare i documenti;
		\item \textbf{S}: per indicare il software.
	\end{itemize}

	\item \textbf{num}: identificativo univoco formato da due cifre intere a partire da 1.
\end{itemize}

\subsubsection{Controllo qualità di processo e metriche} \label{ControlloQualita_Processo}
La qualità di processo è raggiunta tramite l'utilizzo di metodi e modelli che garantiscono il corretto procedimento delle fasi di sviluppo del processo. Il team sfrutta le potenzialità del metodo PDCA, descritto nell'appendice \ref{PDCASection}, ottenendo miglioramenti continui nelle qualità di processo e verifica in modo da avere una maggiore qualità nel prodotto risultante. Inoltre verrà utilizzato lo standard ISO/IEC 15504, comunemente conosciuto con l'acronimo SPICE, che misurerà il livello di maturità dei processi.\-\\
Le seguenti metriche sono utilizzate per la valutazione dell'efficacia e dell'efficienza degli stessi.

\paragraph{MTPC01 Schedule Variance (SV)}\-\\
\'E un indice che consente di rilevare se l'andamento del progetto è in linea con quanto pianificato nella baseline\glossario. Può essere utile al cliente per verificare quantitativamente se il team di sviluppo del progetto è in linea con i tempi. \-\\
Calcolo:\-\\
\begin{center}
	\item \textbf{SV = BCWP - BCWS}
\end{center}
dove:
\begin{itemize}
	\item BCWP: valore delle attività realizzate alla data odierna (in giorni);
	\item BCWS: costo pianificato per realizzare le attività alla data odierna (in giorni).
\end{itemize}
Se SV > 0 significa che il progetto sta producendo con maggiore velocità prevista.

\paragraph{MTPC02 Budget Variance (BV)}\-\\
\'E un indice calcolato che permette di rilevare la differenza tra i costi previsti e quelli reali alla data odierna. \\
Calcolo:
\begin{center}
	\item \textbf{BV = BCWS - ACWP}
\end{center} 
dove:
\begin{itemize}
	\item BCWS: costo pianificato per realizzare le attività alla data odierna (in euro);
	\item ACWP: costo effettivo sostenuto per completare le attività alla data odierna (in euro).
\end{itemize}
Se BV > 0 significa che il progetto sta risparmiando sui costi prestabiliti, se BV = 0 significa che il progetto sta mantenendo i costi prefissati, se BV < 0 significa che il progetto sta superando il budget imposto.

\paragraph{MTPC03 Estimated At Completion (EAC)}\-\\
Indice che rappresenta la stima dei costi mancanti. Viene calcolato man mano che il progetto procede ed è fondamentale per attività di pianificazione. \\
Calcolo:
\begin{center}
	\item \textbf{EAC = ACWP + ETC}
\end{center}
dove:
\begin{itemize}
	\item ACWP: costo effettivo sostenuto per completare le attività alla data odierna (in euro);
	\item ETC: valore stimato per la realizzazione delle attività mancanti (in euro).
\end{itemize}

\iffalse
\paragraph{MTPC04 Cost Variance(CV)}\-\\
Indice che quantifica la produttività o efficienza monitorando se il valore del costo realmente maturato è minore, maggiore o uguale al costo effettivo.
Calcolo:
\begin{center}
	\item \textbf{CV = BCWP - ACWP}
\end{center}
dove:
\begin{itemize}
	\item BCWP: valore delle attività realizzate alla data odierna(in euro o giorni);
	\item ACWP: costo effettivo sostenuto per completare le attività alla data odierna(in euro o giorni).
\end{itemize}
Se CV > 0 significa che il progetto produce con maggior efficienza e minor costo.
\fi

\paragraph{Code Coverage}\-\\
Le operazioni di verifica della qualità sul software vengono eseguite seguendo le metriche principali sotto elencate:
\begin{itemize}
	\item \textbf{MTPC04 Function Coverage}: verificare che una funzione sia chiamata;
	\item \textbf{MTPC05 Statement Coverage}: verificare che ogni statement del codice sia eseguito, e non ci sia quindi codice che non verrà mai preso in considerazione;
	\item \textbf{MTPC06 Branch Coverage}: verificare che tutti i possibili percorsi delle strutture di controllo (del tipo \texttt{if, case}, ecc.) siano stati eseguiti.
	\item \textbf{MTPC07 Condition Coverage}: verificare che ogni condizione booleana sia considerata sia vera che falsa.
\end{itemize}

\paragraph{MTPC08 Rischi non preventivati}\-\\
Indice numerico incrementale a partire da 0. Indica il numero di rischi non preventivati che si verificano e vengono considerati durante la corrente fase del progetto. La misurazione avviene incrementando il valore per ogni rischio (non individuato precedentemente) che viene rilevato, il valore viene azzerato per ogni fase del progetto.

\paragraph{MTTS9 Percentuale di test passati}\-\\
Indica la percentuale di test passati, utile per misurare l'avanzamento qualitativo. La misurazione avviene:
\begin{center}
	\item $PTP = \frac{TP}{TT}*100$
\end{center}
dove PTP è la percentuale finale di test passati, TP sono i test passati e TT i test totali eseguiti.

\paragraph{MTTS10 Percentuale di test falliti}\-\\
Indica la percentuale di test falliti. La misurazione avviene:
\begin{center}
	\item $PTF = \frac{TF}{TT}*100$
\end{center}
dove PTF è la percentuale finale di test falliti, TF sono i test passati e TT i test totali eseguiti.

\paragraph{MTTS11 Percentuale di difetti sistemati}\-\\
Indica la percentuale di bug/difetti sistemati, indice utile per misurare l'avanzamento. La misurazione avviene:
\begin{center}
	\item $PDS = \frac{DS}{DT}*100$
\end{center}
dove PDS è la percentuale calcolata, DS i difetti sistemati e DT i difetti totali.

\paragraph{MTTS12 Tempo medio di risoluzione degli errori}\-\\
Indice calcolato sul tempo medio della risoluzione di bug creati dal team durante lo sviluppo. Utile per considerare abilità e tempo di assorbimento di un bug nel sistema. La misurazione avviene:
\begin{center}
	\item $TMRE = \frac{TRE}{NE}$
\end{center}
dove TMRE è il tempo medio calcolato, TRE il tempo totale per la risoluzione degli errori e NE il numero degli errori.

\paragraph{MTTS13 Numero medio di bug trovati per test}\-\\
Indica il numero medio di bug trovati eseguendo i test. \'E un indice utile per verificare la qualità dei test e del sistema in generale. La misurazione avviene:
\begin{center}
	\item $MBT = \frac{NB}{NT}$
\end{center}
dove MBT è la media calcolata, NB il numero di bug rilevati e NT il numero dei test.

\iffalse 
\paragraph{MTTS14 Difetti trovati per requisito}\-\\
Indice numerico che rappresenta il numero medio di test eseguiti per requisito. Utile per verificare che il sistema soddisfi i requisiti a pieno. La misurazione avviene:
\begin{center}
	\item $MDR = \frac{TDR}{TR}$
\end{center}
dove MDR è la media calcolata, TDR il numero totale di difetti trovati e TR il numero totale di requisiti.
\fi

\subsubsection{Metriche per la qualità di prodotto}
Le seguenti metriche sono utilizzate per misurare qualitativamente il prodotto.

\iffalse
\paragraph{MTTS15 Numero di test eseguiti per requisito}\-\\
\fi

\paragraph{MTPDD14 Indice di Gulpease}\-\\
Il gruppo ha deciso di utilizzare l'\textit{indice di Gulpease}\glossario\footnote{\url{https://it.wikipedia.org/wiki/Indice\_Gulpease}} per misurare la leggibilità di un testo. \'E stato sviluppato appositamente uno script in \textit{Python} per automatizzare la procedura e allo stesso modo velocizzarla. La procedura verrà utilizzata a documento terminato e completo così da valutare il lavoro svolto dai redattori.

\paragraph{MTPDD15 Correttezza Ortografica}\-\\
La correttezza ortografica è un aspetto importante che non accetta errori, i documenti al momento della pubblicazione sono corretti. Verranno utilizzati appositi strumenti che supporteranno la correzione. Il software \textit{TexMaker}, per esempio, integra un segnalatore automatico degli errori grammaticali. 

\paragraph{MTPDS16 Soddisfacimento Requisiti Obbligatori}\-\\
Indicatore percentuale che verifica che tutti i requisiti obbligatori siano soddisfatti. Condizione necessaria al fine di rispettare il contratto. La misurazione avviene:
\begin{center}
	\item $PRO = \frac{ROS}{ROT}*100$
\end{center}
dove PRO è la percentuale di requisiti obbligatori soddisfatti, ROS i requisiti obbligatori soddisfatti e ROT i requisiti obbligatori totali.

\paragraph{MTPDS17 Soddisfacimento Requisiti Opzionali Scelti}\-\\
Indicatore percentuale che verifica se tutti i requisiti opzionali scelti siano soddisfatti. Condizione necessaria al fine di rispettare il contratto. La misurazione avviene:
\begin{center}
	\item $PRP = \frac{RPS}{RPT}*100$
\end{center}
dove PRP è la percentuale di requisiti opzionali scelti soddisfatti, RPS i requisiti opzionali scelti soddisfatti e RPT i requisiti opzionali scelti totali.

\paragraph{MTPDS18 Tempo Medio di Comprensione}\-\\
Indica il tempo medio che l'utente impiega per comprendere cosa può svolgere il sistema. \'E misurato in minuti ed è rilevato attraverso test a persone esterne al team di sviluppo.

\paragraph{MTPDS19 Tempo Medio di Apprendimento}\-\\
Indica il tempo medio che l'utente impiega per riuscire a utilizzare a pieno il software e tutte le sue funzionalità. \'E misurato in minuti ed è rilevato tramite test a persone esterne al team di sviluppo. 

\iffalse
\paragraph{MTPDS25 Percentuale Commenti/Codice}\-\\
Indica le righe di commenti presente rispetto al codice. \'E calcolato per ogni procedura e non per l'intera codebase. La misurazione avviene:
\begin{center}
	\item $PC = \frac{RC}{RT}*100$
\end{center}
dove PC è la percentuale calcolata, RC il numero di righe di commento e RT il numero di righe totale.
\fi



\subsection{Versionamento}\label{ProcessiSupporto_Versionamento}
La necessità di più componenti del gruppo di cooperare su uno stesso documento, porta la necessità di utilizzare un sistema di versionamento distribuito. 

%Per le parti in cui più componenti del \texttt{gruppo} devono cooperare sugli stessi file, 
%Per le parti in cui più componenti del \texttt{gruppo} devono operare contemporaneamente su gli stessi file, in questa fase di \textit{RR}, si è scelto un sistema di 
%versionamento distribuito come, \textbf{git} e col supporto hosting di \textbf{GitHub}. 


%Nella fase di RR, non in questa fase di RR
\subsubsection{Controllo di versione}
Il sistema di versionamento, utilizzato nella fase di RR, è \textit{git}, con il supporto hosting di \textit{GitHub}.

\paragraph{Struttura del repository} \-\\ 
La struttura del repository segue il workflow \textit{gitflow} di \textit{Driessen at nvie}, idealizzato attorno il concetto di release del prodotto. Questo produce un framework robusto attorno al quale si possono gestire progetti di grandi dimensioni. I due branch principali sono il "master" ed in parallelo ad esso il "develop". 
Il master viene considerato il branch main, dove il codice sorgente della testa riflette sempre lo stato di \textit{"production-ready"},
mentre il ramo di develop è considerato il branch principale dove vengono effettuate le ultime modifiche per il prossimo incremento del prodotto.\\
Più precisamente la repository\glossario "Agents-of-S.W.E." è caratterizzata da una cartella principale "Documentazione", la cui sottocartella "RR" rappresenta la principale struttura in cui sono organizzati i file su cui il gruppo \texttt{Agents of S.W.E.} ha lavorato in vista della RR.\\
	Nello specifico la sottocartella "RR" è caratterizzata dalla seguente struttura di folder:
	\begin{itemize}
	\item \textbf{Documenti Esterni}:
		\begin{itemize}
		\item \textit{Analisi dei Requisiti};
		\item \textit{Glossario};
		\item \textit{Piano di Progetto};
		\item \textit{Piano di Qualifica};
		\item \textit{Verbali Esterni}.
		\end{itemize}
	\item \textbf{Documenti Interni}:
		\begin{itemize}
		\item \textit{Norme di Progetto};
		\item \textit{Studio di Fattibilità};
		\item \textit{Verbali Interni}.
		\end{itemize}
	\item \textbf{Corrispondenza};
	\item \textbf{Utility}.
	\end{itemize}
	Ciascuna sottocartella, corrispondente ad un documento, contiene a sua volta un file LaTeX \textit{.tex} che assume il nome del documento e il corrispettivo file \textit{.pdf} ottenuto attraverso la compilazione.\\
	La cartella "Corrispondenza" contiene una copia delle mail inviate all'azienda proponente attraverso l'indirizzo agentsofswe@gmail.com, la cartella "Utility" invece contiene gli Script\glossario realizzati dal gruppo.

%\paragraph{Configurazione sistema}

\paragraph{Processo di implementazione} \-\\
L'implementazione dei documenti avviene tramite gli strumenti utilizzati nel paragrafo §\ref{ProcessiSupporto_Documentazione}\\
Quest'ultimi, in fase di compilazione producono dei file di poca rilevanza con estensioni come \textit{.log, .out, .idx, .aux, .gz, .toc}, i quali verranno ignorati come da configurazione. I file con maggior rilevanza, come \textit{.pdf, .tex}, verranno versionati dal sistema di git preinstallato. \\
Una volta creati e/o modificati i documenti, si procede con il commit\glossario di essi. Il commit riporta un \textit{cambiamento} al file, con un messaggio allegato ad esso che ne 
descrive le modifiche apportate o un comando apposito per chiudere alcune task con tale commit. Dopo di che il commit viene pushato\glossario nel branch appropriato, a seconda dei criteri descritti nel prossimo paragrafo. 

\paragraph{Ciclo di vita dei branch} \-\\
\begin{enumerate}
	
	\item \textbf{Master}: branch main del repository, esso rappresenta lo stato di \textit{production-ready} del prodotto. Questo branch ha una durata di vita quanto il repository stesso o infinita;
	\item \textbf{Develop}: branch di sviluppo parallelo al "master" sul quale vengono aggiunte le feature provenienti appunto dai branch "features", e dal quale inizia il branch di "release". Ha la stessa durata di vita del branch master;
	 
	\item \textbf{Release}: branch di preparazione per un nuovo rilascio o aggiornamento del prodotto. 
	Utilizzato per risolvere piccoli errori e configurare le impostazioni di rilascio. Una volta rilasciato il 
	prodotto, esso si riversa nel branch "master" e "develop". Ha una durata breve in quanto il rilascio deve essere effettuato il prima possibile;

	\item \textbf{Feature}: branch usato per sviluppare nuove feature per il prossimo rilascio a breve o lungo termine. Il suo tempo di vita dura quanto lo sviluppo della nuova feature
	fintanto che non avviene il merge\glossario sul branch "develop";

	\item \textbf{Hotfix}: branch molto simili a quello di release, con l'obbiettivo di risolvere immediatamente un bug del prodotto in produzione o release. Una volta risolto il bug, 
	esso si riversa sui branch "master" e "develop", aggiornandoli nel minor tempo possibile. Ha un tempo di vita breve, in quanto viene creato per la necessità di risolvere 
	un problema sul prodotto rilasciato. 
	
\end{enumerate}


\paragraph{Rilascio di versione} \-\\
Il rilascio di una nuova versione del prodotto avviene nel momento in cui si raggiunge un certo numero di features implementate e testate. 
Dal branch "develop" si avvia un processo di verifica, che sfocia in una nuovo branch di "release", il quale porta il nome della release e che nella sua ultima fase rilascia la nuova versione sul branch "master" e applica le modifiche effettuate nel frattempo anche nel branch "develop", portando i due allo stesso livello di produzione. 

\subsubsection{Configurazione versionamento}

\paragraph{Remoto} \-\\
	La configurazione di GitHub avviene nel portale www.github.com, dove si inseriscono le chiavi SSH\glossario per ciascun collaboratore del nuovo repository. 
	Una volta creato il repository nel server remoto ed inserite le chiavi SSH, si procedere con la configurazione in locale.
	
\paragraph{Locale} \-\\
	In locale, si devono generare le chiavi SSH, che permettono il collegamento con il server remoto dove viene gestito il repository. 
	Una volta generate le chiavi, seguendo le varie procedure specifiche per ogni sistema operativo, vengono caricate sul portale apposito del gestore GitHub.
	L'ultima fase prevede la clonazione con uno dei seguenti metodi: 

	\begin{itemize}
		\item \textbf{GitHub Desktop}: gestore di versionamento a interfaccia grafica per sistemi Windows \& MacOS; 
		\item \textbf{GitKraken}: gestore di versionamento a interfaccia grafica per sistemi Windows, MacOS \& Linux; 		
		\item \textbf{Terminale(Bash)}: gestore di versionamento a riga di comando per i sistemi MacOS \& Linux.
	\end{itemize}
		