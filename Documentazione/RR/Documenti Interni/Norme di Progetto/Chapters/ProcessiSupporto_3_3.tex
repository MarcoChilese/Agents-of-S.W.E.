\subsection{Versionamento}
La necessità di più componenti del \texttt{gruppo} di cooperare su uno stesso documento, porta alla soluzione di utilizzare un sistema di versionamento distribuito. 

%Per le parti in cui più componenti del \texttt{gruppo} devono cooperare sugli stessi file, 
%Per le parti in cui più componenti del \texttt{gruppo} devono operare contemporaneamente su gli stessi file, in questa fase di \textit{RR}, si è scelto un sistema di 
%versionamento distribuito come, \textbf{git} e col supporto hosting di \textbf{GitHub}. 

\subsubsection{Controllo di versione}
Il sistema di versionamento, utilizzato in questa fase di \textbf{RR}, è \textbf{git}, con il supporto hosting di \textbf{GitHub}.

\paragraph{Struttura del repository}
La struttura del repository segue il workflow \textbf{gitflow} di \textbf{Driessen at nvie}, idealizzato attorno il concetto di \textbf{release} del prodotto. Questo produce
un framework robusto attorno al quale si possono gestire progetti di grandi dimensioni. I due branch principali sono il \textbf{master} e in parallelo ad esso il \textbf{develop}. 
Il \textbf{master} viene considerato il branch \textit{main}, dove il codice sorgente della \textit{testa} riflette sempre lo stato di \textit{"production-ready"},
mentre il ramo di \textbf{develop} è considerato il branch principale dove vengono effettuate le ultime modifiche per il prossimo rilascio del prodotto. 

%\paragraph{Configurazione sistema}

\paragraph{Processo di implementazione}
L'implementazione dei documenti avviene tramite gli strumenti utilizzati nel paragrafo --INSERIRE PARAGRAFO PARTE DI FAVARO--\\
% L'implementazione dei documenti in \textbf{RR} avviene tramite editor quali: 

%\begin{itemize}
	%\item \textbf{TexMaker};
	%\item \textbf{TexStudio}.
%\end{itemize} 
%

Quest'ultimi, in fase di compilazione producono dei file di poca rilevanza con estensioni come: \textit{.log, .out, .idx, .aux, .gz, .aux, .toc},
i quali verranno ignorati come da configurazione, ma soprattutto file di più rilevanza
come \textit{.pdf, .tex}, i quali verranno versionati dal sistema di git preinstallato.  
Una volta creati/modificati i documenti, si procede con il \glossario{commit} di essi. Il commit riporta un \textit{cambiamento} al file,con un messaggio allegato ad esso che ne 
descrive le modifiche apportate o un commando apposito per chiudere alcune task con tale commit. Dopo di che, il commit viene \glossario{pushato} nel branch appropriato, 
a seconda dei criteri descritti nel prossimo paragrafo. 

\paragraph{Ciclo di vita dei branch}
\begin{enumerate}
	
	\item \textbf{Master}: branch \textit{main} del repository, esso rappresenta lo stato di \textit{"production-ready"} del prodotto. Questo branch ha una durata di vita quanto il repository stesso o infinita;
	\item \textbf{Develop}: branch di sviluppo parallelo al \textbf{master} sul quale vengono aggiunte le feature provenienti appunto dai branch \textbf{feature},
	 e dal quale inizia il branch di \textbf{realse}. Ha una durata di vita quanto il branch master;
	 
	\item \textbf{Release}: branch di preparazione per un nuovo rilascio o aggiornamento del prodotto. 
	Utilizzato per risolvere piccoli errori e configurare le impostazioni di rilascio. Una volta rilasciato il 
	prodotto, esso si riversa sul branch \textbf{master} e \textbf{develop}. Ha una durata breve in quanto il rilascio deve essere effettuato il prima possibile;

	\item \textbf{Feature}: branch usato per sviluppare nuove feature per il prossimo rilascio a breve o lungo tempo. Il suo tempo di vita dura quanto lo sviluppo della nuova feature
	fintanto che non avviene il \glossario{merge} sul branch di \textbf{develop};

	\item \textbf{Hotfix}: branch molto simili a quelli di release, con l'obbiettivo di risolvere immediatamente un bug del prodotto in produzione o release. Una volta risolto il bug, 
	esso si riversa sui branch \textbf{master} e \textbf{develop}, aggiornandoli nel minor tempo possibile. Ha un tempo di vita breve, in quanto viene creato per la necessità di risolvere 
	un problema sul prodotto rilasciato. 
	
\end{enumerate}

\paragraph{Rilascio di versione}

\subsubsection{Configurazione versionamento}

\paragraph{Remoto}
	% Come configurare github
\paragraph{Locale}
	% configurare gitkraken terminale gitflow 
		
\subsection{Gestione di progetto}

\subsubsection{Configurazione strumenti di organizzazione}
	\paragraph{Inizializzazione}
	\paragraph{Aggiunta tasks}
	\paragraph{Aggiunta milestones}

\subsubsection{Ciclo di vita delle tasks}
	\paragraph{Apertura}
	\paragraph{Completamento}
	\paragraph{Richiesta di revisione}

	\paragraph{Completamento}

%\subsubsection

\subsection{Verifica}
	\subsubsection{Descrizione}
	\subsubsection{Analisti Statica}
	\subsubsection{Verifica Diagrammi UML}
	\subsubsection{Strumenti}
	
\subsection{Validazione}
	\subsubsection{Descrizione}
	\subsubsection{Procedure}


\subsection{Risoluzione problematiche}

	\subsubsection{Descrizione}
	
	\subsubsection{Ciclo di vita}
		\paragraph{Individuazione}
		\paragraph{Analisi problematiche}
		\paragraph{Scelta soluzione migliore}
		\paragraph{Reinserimento nella gestione di processo}
	








