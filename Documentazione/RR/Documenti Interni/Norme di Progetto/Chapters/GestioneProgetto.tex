La gestione di progetto avviene tramite un sistema di task integrato nel servizio di hosting GitHub. 
Esso permette l'integrazione delle task con il repository stesso, dando la possibilità ai vari commit di chiudere con comandi appositi determinate task, aumentando cosi l'automazione di tutto il processo. 

\subsubsection{Configurazione strumenti di organizzazione}
	La configurazione di tutto il processo di organizzazione avviene nel portale di GitHub, dove si crea una project board per ogni categoria di processo. 	

\paragraph{Inizializzazione} \-\\
 L'inizializzazione della project board avviene tramite un'istanza vuota oppure selezionando un template di ciclo di vita fornito da GitHub largamente utilizzate in molti progetti, quindi testati ed affidabili. Tra i template forniti abbiamo: 

\begin{itemize}
		\item \textbf{Basic Kanban}: presenta le fasi di \textit{ToDo, In Progress, e Done}; 
		\item \textbf{Automated Kanban}: presenta trigger\glossario predefiniti che permettono lo spostamento di task
		 automatici nei vari cicli di vita, utilizzando il meccanismo di chiusura dei commit;
		\item \textbf{Automated Kanban with Reviews}: tutto cioè che viene incluso nel template \textit{Automated Kanban} con l'aggiunta di trigger per la revisione di nuove componenti; 
		\item \textbf{Bug Triage}: template centrato sulla gestione degli errori, fornendo un ciclo di vita per essi che varia tra \textit{ToDo, Alta Priorità, Bassa Priorità e Chiusi}. 
\end{itemize}
	  	
\paragraph{Aggiunta milestone} \-\\
	Le milestone\glossario sono gruppi di task mirate ad un obbiettivo comune tra esse.
	Possono essere aggiunte in qualsiasi momento, sia prima che dopo la creazione di una task.

\subsubsection{Ciclo di vita delle task}

\paragraph{Apertura} \-\\
	Da una specifica project board si possono creare le task o le issue\glossario, le quali possono essere assegnate ad uno o più individui che collaborano al repository, inoltre ogni task può far parte di una milestone, che raggruppa un insieme di task o issue per il raggiungimento di un obbiettivo comune. \\
	Ad ognuna di esse può essere assegnato un colore che ne identifica il tipo come per esempio: bug, ToDo, miglioramenti, ecc.,\\
	Si può creare una nuova task senza l'obbligo di assegnarla ad una project board, mantenendo comunque tutte le funzionalità descritte precedentemente. 

\paragraph{Completamento} \-\\\label{ProcessiOrganizzativi_GestioneProgetto_CicloTask_Completamento}
	Il completamento di una task avviene in diversi modi, a seconda delle impostazioni della project board. 
	Se la project board è automatizzata, il completamento di una task può avvenire tramite commit utilizzando il codice di chiusura. \\
	Questo metodo collega direttamente l'implementazione richiesta alla task. \\
	Se la project board non è automatizzata, il completamento dalla task deve essere manuale spostandola nello stato di "Concluso". 
	
\paragraph{Richiesta di revisione} \-\\
	Accumulate un certo numero di task o di milestone, si avvia la procedura di revisione da parte del verificatore. \\ Questa può essere notificata e pianificata in modo automatico a seconda del livello di automatizzazione della project board, oppure può essere totalmente gestita dal verificatore. 
	
	
	\paragraph{Chiusura} \-\\
	Una volta che le task o le milestone sono state approvate dal verificatore, esse concludono il loro ciclo di vita nello stato di chiusura, le quali verranno spostate manualmente dal verificatore o automaticamente dalla project board se il merge è avvenuto con successo. 
	
\paragraph{Riapertura} \-\\ \label{ProcessiOrganizzativi_Riapertura}
	Le task in stato di "Chiusura" possono essere riaperte e spostate nello stato di "Apertura" se esse non soddisfanno tutti i parametri di qualità richiesti.
	
\subsubsection{Ruoli di Progetto}\label{ProcessiOrganizzativi_RuoliProgetto}
	Nell'ottica di un lavoro ben organizzato e collaborativo tra i membri del gruppo, ad ogni componente, in ogni 					momento, è attribuito un ruolo per un periodo di tempo limitato.\\
	Questi ruoli, che corrispondono ad una figura aziendale ben precisa, sono:
	\begin{itemize}
	\item \textit{Responsabile di Progetto};
	\item \textit{Amministratore di Progetto};
	\item \textit{Analista};
	\item \textit{Progettista};
	\item \textit{Programmatore};
	\item \textit{Verificatore}.
	\end{itemize}
	\paragraph{\textit{Responsabile di Progetto}} ~\\
	Detto anche \textit{"Project Manager"}, è il rappresentate del progetto\glossario, agli occhi sia del committente che del 			fornitore. Egli risulta dunque essere, in primo luogo, il responsabile ultimo dei risultati del proprio gruppo. 			Figura di grande responsabilità, partecipa al progetto per tutta la sua durata, ha il compito di prendere 						decisioni 	e approvare scelte collettive.\\
	Nello specifico egli ha la responsabilità di:
	\begin{itemize}
	\item Coordinare le attività del gruppo, attraverso la gestione delle risorse umane;
	\item Approvare i documenti redatti, e verificati, dai membri del gruppo;
	\item Elaborare piani e scadenze, monitorando i progressi nell'avanzamento del progetto;
	\item Redigere l'organigramma del gruppo e il \textit{Piano di Progetto v2.0.0}\glossario.
	\end{itemize}

\paragraph{\textit{Amministratore di Progetto}} ~\\
	L'\textit{Amministrator}e è la figura chiave per quanto concerne la produttività. Egli ha infatti come primaria 							responsabilità il garantire l'efficienza\glossario del gruppo, fornendo strumenti utili e occupandosi 								dell'operatività delle risorse. Ha dunque il compito di gestire l'ambiente lavorativo.\\
	Tra le sue responsbilità specifiche figurano:
	\begin{itemize}
	\item Redigere documenti che normano l'attività lavorativa, e la loro verifica\glossario;
	\item Redigere le \textit{Norme di Progetto v2.0.0}\glossario;
	\item Scegliere ed amministrare gli strumenti di versionamento\glossario;
	\item Ricercare strumenti che possano agevolare il lavoro del gruppo;
	\item Attuare piani e procedure di gestione della qualità\glossario.
	\end{itemize}

\paragraph{\textit{Analista}} ~\\
	L'\textit{Analista} deve essere dotato di un'ottima conoscenza riguardo al dominio del problema. Egli ha infatti il 					compito di analizzare tale dominio e comprenderlo appieno, affinchè possa avvenire una corretta 											progettazione\glossario.\\
	Ha il compito di:
	\begin{itemize}
	\item Comprendere al meglio il problema, per poi poterlo esporre in modo chiaro attraverso specifici 									requisiti\glossario;
	\item Redarre lo \textit{Studio di Fattibilità v1.0.0} e l'\textit{Analisi dei Requisiti v2.0.0}\glossario.
	\end{itemize}

\paragraph{\textit{Progettista}} ~\\
	Il \textit{Progettista} è responsabile delle attività di progettazione\glossario attraverso la gestione degli aspetti tecnici 	del progetto.\\
	Più nello specifico si occupa di:
	\begin{itemize}
	\item Definire l'Architettura\glossario del prodotto\glossario, applicando quanto più possibile norme di 							best practice\glossario e prestando attenzione alla manutenibilità del prodotto;
	\item Suddividere il problema, e di conseguenza il sistema, in parti di complessità trattabile.
	\end{itemize}

\paragraph{\textit{Programmatore}} ~\\
	Il \textit{Programmatore} si occupa delle attività di codifica, le quali portano alla realizzazione effettiva del 						prodotto.\\
	Egli ha dunque il compito di:
	\begin{itemize}
	\item Implementare l'architettura definita dal \textit{Progettista}, prestando attenzione a scrivere codice 						coerente con ciò che è stato stabilito nelle \textit{Norme di Progetto v2.0.0};
	\item Produrre codice documentato e manutenibile;
	\item Realizzare le componenti necessarie per la verifica e la validazione\glossario del codice;
	\item Redarre il \textit{Manuale Utente v1.0.0}.
	\end{itemize}

\paragraph{\textit{Verificatore}} ~\\
	Il \textit{Verificatore}, figura presente per l'intera durata del progetto, è responsabile delle attività di 				verifica.\\
	Nello specifico egli:
	\begin{itemize}
	\item Verifica l'applicazione ed il rispetto delle \textit{Norme di Progetto v2.0.0};
	\item Segnala al \textit{Responsabile di Progetto} l'emergere di eventuali discordanze tra quanto presentato nel 			\textit{Piano di Progetto v2.0.0} e quanto effettivamente realizzato;
	\item Ha il compito di redarre il \textit{Piano di Qualifica v2.0.0}.
	\end{itemize}

\paragraph{Rotazione dei Ruoli} ~\\
	Come da istruzioni ogni membro del gruppo dovrà ricoprire, per un periodo di tempo limitato, ciascun ruolo, nel 			rispetto delle seguenti regole:
	\begin{itemize}
	\item Ciascun membro dovrà svolgere esclusivamente le attività proprie del ruolo a lui assegnato;
	\item Al fine di evitare conflitti di interesse nessun membro potrà ricoprire un ruolo che preveda la 									verifica di quanto da lui svolto, nell'immediato passato;
	\item Vista l'ampia differenza di compiti e mansioni tra i vari ruoli, e al fine di valorizzare l'attività 						collaborativa all'interno del gruppo, ogni componente che abbia ricoperto in precedenza un ruolo ora destinato 			a qualcun altro dovrà fornire supporto al compagno in caso di necessità, fornendogli consigli e, se possibile, 			affiancandolo in situazioni critiche.
	\end{itemize}

	