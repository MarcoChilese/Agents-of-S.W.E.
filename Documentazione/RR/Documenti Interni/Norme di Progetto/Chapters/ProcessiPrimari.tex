\section{Processi Primari}

\subsection{Fornitura}

\subsection{Sviluppo}
\subsubsection{Analisi dei requisiti}
\paragraph{Classificazione dei requisiti}
\paragraph{Classificazione dei casi d'uso}

\subsubsection{Progettazione}
\subsubsection{Codifica}
Di seguito vengono definite delle norme che devono essere adottate dai Programmatori per garantire una buona leggibilità e manutenibilità del codice. Le prime norme che seguiranno sono le più generali, da adottarsi per ogni linguaggio di programmazione adottato all'interno del progetto, in seguito quelle più specifiche per i linguaggi JavaScript\glossario, HTML\glossario e CSS\glossario.\\
Ogni norma è caratterizzata da un paragrafo di appartenenza, da un titolo, una breve descrizione, e se il caso lo richiede, un esempio.\\
Il rispetto delle seguenti norme è fondamentale per garantire uno stile di codifica uniforme all'interno del progetto, oltre che per massimizzare la leggibilità e agevolare la manutenzione, la verifica\glossario e la validazione\glossario.

\paragraph{Convenzioni per i nomi:}
\begin{itemize}	
	\item I Programmatori devono adottare come notazione per la definizione di cartelle, file, metodi, funzioni e variabili il CamelCase\glossario.\\
	Di seguito un esempio di corretta nomenclatura:
	\begin{tcolorbox}
		\begin{center}
			INSERIRE ESEMPIO 
		\end{center}
	\end{tcolorbox}

	\item Tutti i nomi devono essere \textbf{unici} ed \textbf{autoesplicativi}, ciò per evitare ambiguità e limitare la complessità.
\end{itemize}
\paragraph{Convenzioni per la documentazione:}
\begin{itemize}	
	\item Tutti i nomi ed i commenti al codice vanno scritti in \textbf{inglese};
	\item Nel codice è possibile utilizzare un commento con denominazione \textbf{TODO} in cui si vanno ad indicare compiti da svolgere;
	\item L'intestazione di ogni file deve essere la seguente:
	\begin{tcolorbox}
		\begin{center}
			INSERIRE INTESTAZIONE FILE 
		\end{center}
	\end{tcolorbox}
	\item La versione del file nell'intestazione, deve rispettare la seguente formulazione: $Y.K$, dove Y rappresenta la versione principale, K la versione parziale della relativa versione principale.\\ I numeri di versione del tipo $Y.0$, dalla $1.0$, vengono considerate versioni stabili, e quindi versioni da testare per saggiarne la qualità.
	
\paragraph{JavaScript:}
\end{itemize}

