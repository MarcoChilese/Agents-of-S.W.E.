\section{Processi Primari}\label{ProcessiPrimari}

\subsection{Fornitura}
In questa sezione del documento vengono trattate le norme che il team \texttt{Agents of S.W.E.} decide e si impegna a rispettare, con lo scopo di proporsi e divenire fornitori nei confronti dell'azienda proponente, \textit{Zucchetti S.p.A.} e dei committenti Prof. Tullio Vardanega e Prof. Riccardo Cardin nell'ambito della progettazione, sviluppo e consegna del prodotto "\textit{G\&B}".

\subsection{Studio di fattibilità} \label{ProcessiPrimari_Sviluppo_StudioFattibilità}
In seguito all'incontro con le aziende il 16 novembre 2018, il gruppo si è riunito per discutere la scelta del capitolato. Dopo un'opportuna analisi il team ha deciso di scegliere il capitolato C2 ovvero "\textit{G\&B}". A questo punto gli analisti hanno redatto il documento "\textit{Studio di fattibilità}" dove sono contenute le analisi dei capitolati, specificando pro e contro di ognuno e motivando la scelta.
Tale documento è così strutturato: 
\begin{itemize}
	\item \textbf{Informazioni del capitolato}: sezione che contiene il nome del fornitore e il nome del capitolato;
	\item \textbf{Descrizione Capitolato e Obiettivo Finale}: descrizione obiettivo finale del capitolato e ambito di utilizzo del prodotto;
	\item \textbf{Dominio Tecnologico}: definisce il dominio tecnologico, ovvero le tecnologie richieste dall'azienda;
	\item \textbf{Valutazione del Capitolato}: definisce aspetti positivi, negativi, e criticità, motivando la scelta del capitolato.  
\end{itemize} 
\subsection{Sviluppo}
\subsubsection{Analisi dei requisiti}\label{ProcessiPrimari_Sviluppo_AnalisiRequisiti}
\'E un documento stilato dagli Analisti con lo scopo di riportare tutti i requisiti del capitolato. Le informazioni possono essere recuperate da più fonti, tra cui: 
\begin{itemize}
	\item \textbf{Riunioni esterne:} incontri con il cliente con lo scopo di approfondire aspetti critici;
	\item \textbf{Riunioni interne:} riunioni del gruppo dove l'obiettivo è discutere i casi d'uso del capitolato e studiare il dominio di utilizzo;
	\item \textbf{Documentazione:} analisi della documentazione offerta dall'azienda.
\end{itemize}
Il risultato dello studio sarà un documento verificato che contiene:
\begin{itemize}
	\item Descrizione generale del prodotto;
	\item Argomentazioni precise ed affidabili per i Progettisti;
	\item Diagrammi UML che rappresentano i casi d'uso;
	\item Funzionalità finali accordate con il cliente;
	\item Stima dei costi.      
\end{itemize}

\paragraph{Classificazione dei requisiti} \-\\
Il requisito si può definire come:
\begin{itemize}
	\item Capacità necessaria a un utente per raggiungere un obiettivo;
	\item Capacità del sistema per soddisfare un obbligo da contratto;
	\item Descrizione documentata di una capacità.
\end{itemize}
La classificazione dei requisiti aiuta a mettere ordine e a facilitare la comprensione e il mantenimento futuro del sistema. Si possono dividere in:
\begin{itemize}
	\item \textbf{Attributi di Prodotto}: specificano "\textit{cosa}" bisogna svolgere e definiscono i requisiti funzionali, prestazionali e di qualità;
	\item \textbf{Attributi di Processo}: specificano "\textit{come}" bisogna svolgere definendo norme contrattuali e realizzative. 
\end{itemize}
Ogni requisito, nel documento, deve seguire le seguenti regole di identificazione:
\begin{itemize}
	\item TODO PROSSIMA RIUNIONE
\end{itemize}

\paragraph{Classificazione dei casi d'uso} \-\\
Un caso d'uso consiste nel valutare ogni requisito con lo scopo di definire gli attori(utenti esterni) all'interno di un scenario che hanno un obiettivo finale comune. Il gruppo, nel documento, oltre a fornire una spiegazione verbosa e dettagliata dei casi d'uso, fornirà una rappresentazione grafica nel linguaggio UML.
Ogni caso d'uso avrà le seguenti informazioni:
\begin{itemize}
	\item Nome o identificativo: TODO
	\item Scenario: sequenza di passi che descrivono interazioni tra gli attori e il sistema, possono creare scenari secondari a seconda del comportamento dell'utente, per esempio dalla log-in passare alla registrazione;
	\item Pre-condizioni: identifica le condizioni sempre vere prima degli eventi del caso d'uso;
	\item Post-condizioni: identifica le condizioni sempre vere dopo gli eventi del caso d'uso;
	\item Trigger: evento scatenante del caso d'uso;
	\item Attori: identifica gli attori principali e attori secondari del caso d'uso. 
\end{itemize}

\subsubsection{Progettazione}
\subsubsection{Codifica}
Di seguito vengono definite delle norme che devono essere adottate dai Programmatori per garantire una buona leggibilità  e manutenibilità  del codice. Le prime norme che seguiranno sono le più generali, da adottarsi per ogni linguaggio di programmazione adottato all'interno del progetto, in seguito quelle più specifiche per i linguaggi ECMAScript 6\glossario, HTML\glossario e CSS\glossario.\\
Ogni norma è caratterizzata da un paragrafo di appartenenza, da un titolo, una breve descrizione e, se il caso lo richiede, un esempio.\\
Il rispetto delle seguenti norme è fondamentale per garantire uno stile di codifica uniforme all'interno del progetto, oltre che per massimizzare la leggibilità  e agevolare la manutenzione, la verifica\glossario e la validazione\glossario.

\paragraph{Convenzioni per i nomi:} \label{Nomi}
\begin{itemize}	
	\item I Programmatori devono adottare come notazione per la definizione di cartelle, file, metodi, funzioni e variabili il CamelCase\glossario.\\
	Di seguito un esempio di corretta nomenclatura:
	\begin{lstlisting}[language=JavaScript]
//Cartelle 
./thisIsAFolder	//OK
./ThisIsAFolder //NO

//File
myFile.extension //OK
MyFile.extension //NO

//Funzioni
myFunction() { ... } //OK
MyFunction() { ... } //NO
	\end{lstlisting}

	\item Tutti i nomi devono essere \textbf{unici} ed \textbf{autoesplicativi}, ciò per evitare ambiguità  e limitare la complessità .
\end{itemize}
\paragraph{Convenzioni per la documentazione:}
\begin{itemize}	
	\item Tutti i nomi ed i commenti al codice vanno scritti in \textbf{inglese};
	\item Nel codice è possibile utilizzare un commento con denominazione \textbf{TODO} in cui si vanno ad indicare compiti da svolgere;
	\item L'intestazione di ogni file deve essere la seguente:
	\begin{lstlisting}[language=JavaScript]
/**
* File: nameFile
* Type: fileType
* Creation date: yyyy-mm-gg
* Author: Name Surname
* Author e-mail: email@example.com
* Version: versionNumber 
*
* Changelog:
* #entry || Author || Date || Description
*/
	\end{lstlisting}
	\item La versione del file nell'intestazione deve rispettare la seguente formulazione: $X.Y.Z$, dove X rappresenta la versione principale, Y la versione parziale della relativa versione principale e Z l'avanzamento rispetto ad Y.\\ I numeri di versione del tipo $X.0.0$, dalla $1.0.0$, vengono considerate versioni stabili e quindi versioni da testare per saggiarne la qualità .
\end{itemize}
\paragraph{ECMAScript 6}\label{EcmaScript6} \-\\
Seguendo le indicazioni presenti nella documentazione\footnote{\texttt{http://docs.grafana.org/plugins/developing/development/}} dell'azienda fornitrice di \textit{Grafana}, la piattaforma  per cui si intende sviluppare il plug-in, il team ha deciso di adottare come linguaggio di programmazione principale ECMAScript 6\footnote{Linguaggio divenuto standard ISO: ISO/IEC 16262:2011, e relativo aggiornamento ISO/IEC 22275:2018.}.\\
ECMAScript 6 viene stardardizzato da \textbf{ECMA}\glossario\footnote{\texttt{http://www.ecma-international.org/}} nel giugno 2015 con la sigla \textbf{ECMA-262\footnote{\texttt{https://www.ecma-international.org/ecma-262/6.0/}}}.\\
Come stile di codifica si adottano le linee guida proposte da \textbf{Airbnb JavaScript Style Guide}\footnote{\texttt{https://github.com/airbnb/javascript}}. Per la verifica dell'adesione a tali norme, i Programmatori devono utilizzare, come suggerito dalla documentazione proposta da \textit{Grafana}, \textbf{ESLint}\glossario\footnote{\texttt{https://eslint.org/}}.\\
In particolare i Programmatori devono rispettare 5 linee guida proposte dalla documentazione ufficiale di \textit{Grafana}:
\begin{enumerate}
	\item Se una variabile non viene riutilizzata, deve essere dichiarata come \texttt{\textbf{const}};
	\item Utilizzare preferibilmente, per la definizione di variabili, la keyword  \texttt{\textbf{let}}, anziché  \texttt{\textbf{var}};
	\item Utilizzare il marcatore freccia (\texttt{\textbf{=>}}), in quanto non oscura il \texttt{\textbf{this}}:
	\begin{lstlisting}[language=JavaScript]
testDatasource() {
  return this.getServerStatus()
  .then(status => {
    return this.doSomething(status);
  })
}	
	\end{lstlisting}
	Invece che:
	\begin{lstlisting}[language=JavaScript]
testDatasource() {
  var self = this;
  return this.getServerStatus()
    .then(function(status) {
  return self.doSomething(status);
  })
}
	\end{lstlisting}
	\item Utilizzare l'oggetto \textit{Promise}:
	\begin{lstlisting}[language=JavaScript]
metricFindQuery(query) {
  if (!query) {
    return Promise.resolve([]);
  }
}	
	\end{lstlisting}
	Invece che:
	\begin{lstlisting}[language=JavaScript]
metricFindQuery(query) {
  if (!query) {
    return this.$q.when([]);
  }
}
	\end{lstlisting}
	%conseguenti?
	\item Se si utilizza \textit{Lodash}, per coerenza, lo si preferisca alle array function native di ES6.
\end{enumerate}
Verranno esaminate di seguito le norme in merito allo stile di codifica che i Programmatori dovranno adottare.

\subparagraph{Identazione}\-\\
\textbf{Norma 1}
L'identazione è da eseguirsi con tabulazione la cui larghezza sia impostata a due (2) spazi per ogni livello.\\
Di seguito un esempio da ritenersi corretto:
\begin{lstlisting}[language=JavaScript]
function() {
..let x = 2;
..if (x > 0)
....return true;
..else
....return false;
}
\end{lstlisting}
Qualsiasi altro tipo di indentazione è da ritenersi scorretta.\\
\-\\
\textbf{Norma 2}
Dopo la graffa principale va inserito uno (1) spazio. Nel seguente modo:
\begin{lstlisting}[language=JavaScript]
function() { ... }
\end{lstlisting}
\-\\
\textbf{Norma 3}
Dopo la keyword di un dato statement (\texttt{if, while}, etc.) va inserito uno (1) spazio. Per un esempio corretto si veda la norma successiva.\\
\-\\
\textbf{Norma 4}
Prima dell'apertura della graffa negli statement di controllo va inserito uno (1) spazio. Nel seguente modo:
\begin{lstlisting}[language=JavaScript]
function() {
  if (condition) {
    ...  
  }
  while (condition) {
    ...
  }
}
\end{lstlisting}
\-\\
\textbf{Norma 5}
Negli statement di controllo (\texttt{if, while}, etc) le condizioni concatenate o annidate, mediante operatori logici, che diventano eccessivamente lunghe NON vanno espresse in un'unica linea, bensì spezzate in più righe. Nel seguente modo:
\begin{lstlisting}[language=JavaScript]
function() {
  if (condition && condition) {
    ...  
  }
  
  if (
   veryLongCondition
   && longCondition
   && condition
    ) {
    doSomething();
  }
}
\end{lstlisting}
\-\\
\textbf{Norma 6}
Dopo blocchi, o prima di un nuovo statement va lasciata una riga vuota. Nel seguente modo:
\begin{lstlisting}[language=JavaScript]
function1() {
  if (condition) {
    doSomething():  
  }
  
  return toReturn;  
}

function2(){
  ...
}
\end{lstlisting}
\-\\
\textbf{Norma 7}
I blocchi di codice multi-riga devono essere contenuti all'interno di graffe. Blocchi costituiti da una singola riga non è necessario che siano contenuti tra graffe: nel caso non vengano utilizzate, la definizione deve essere \textit{inline}, cioè sulla stessa riga.\\
Nel seguente modo:
\begin{lstlisting}[language=JavaScript]
if (condition) return true;

if (condtion) {
  return true;
}
\end{lstlisting}

\subparagraph{Commenti al codice}\-\\
Il codice va commentato nel seguente modo:
\begin{itemize}
	\item "\texttt{//}" se il commento occupa una sola riga;
	\item "\texttt{/** ... */} " se il commento occupa più righe.
\end{itemize}
Nel seguente modo:
\begin{lstlisting}[language=JavaScript]
// single line comment
if (condition) return true;

/**
* multi line comment, line 1
* multi line comment, line 2
*/
if (condtion) {
  return true;
}
\end{lstlisting}

\subparagraph{Variabili}\-\\
\textbf{Norma 1}
Fare riferimento alle norme 1 e 2, all'inizio della sezione §\ref{EcmaScript6}.

\-\\
\textbf{Norma 2}
Non utilizzare dichiarazioni multiple di variabili, dichiarare una variabile per riga.\\
Nel seguente modo:
\begin{lstlisting}[language=JavaScript]
// OK
var x = 1;
var y = 0;

// NO
var x = 1, y = 0;
\end{lstlisting}


%va linkato il paragrafo
\subparagraph{Nomi}\-\\
\textbf{Norma 1} Oltre a quanto enunciato nel secondo punto del paragrafo §\ref{Nomi}, tutti i nomi di funzioni o variabili composti da una singola lettera, o che indichino temporaneità della variabile sono \textit{vietati}: ogni nome deve essere significativo.\\
\-\\
\textbf{Norma 2} 
\begin{enumerate}
	\item I nomi delle variabili, funzioni ed istanze devono utilizzare il CamelCase;
	\item I nomi delle classi deve avere lo stile \texttt{capWords}.
\end{enumerate}
Nel seguente modo:
\begin{lstlisting}[language=JavaScript]
// OK
var thisIsAVariable;

function thisIsAFunction() { ... }

class ThisIsAClass() {
  ...
}

// NO
var Variable;

function Function() { ... }

class myClass() {
  ...
}
\end{lstlisting}

