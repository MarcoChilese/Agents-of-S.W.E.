\section{Processi Organizzativi}

\subsection{Processi di Coordinamento}
\subsubsection{Gestione Comunicazioni}
	In questa sezione vengono descritte le norme che regolano le comunicazioni del gruppo \texttt{Agents of S.W.E.}, 		sia interne, tra i suoi componenti, sia verso entità esterne, come committenti e proponenti.

\paragraph{Comunicazioni Interne} ~\\
	Le comunicazioni interne ai membri del gruppo vengono gestite principalmente attraverso un gruppo 										\textit{Telegram}\glossario, presso il quale vengono discusse le tematiche riguardanti gli aspetti più generali o collettivi riguardanti il progetto. \\
	Per facilitare una comunicazione più specifica, monotematica, ed efficiente tra alcuni membri del gruppo, e per 			gestire meglio la stesura dei documenti, sono stati inoltre predisposti svariati canali tematici all'interno 				dell'app di messaggistica: \textit{Slack}\glossario. Tali canali sono:
	\begin{itemize}
	\item \textbf{\#general}: Per discussioni riguardanti rotazione di ruoli e decisioni degli argomenti principali 				da discure nelle riunioni;
	\item \textbf{\#normeprogetto}: Per discutere riguardo le regole del \textit{Way of Working} del gruppo, le norme 		da seguire e, di conseguenza, la stesura in collaborazione del documento \textit{Norme di Progetto}\glossario;
	\item \textbf{\#pianoprogetto}: Per confrontarsi riguardo il monte ore dei vari ruoli e per facilitare la stesura 		del documento \textit{Piano di Progetto}\glossario;
	\item \textbf{\#analisirequisiti}: Per discutere gli \textit{Use Case}\glossario e i requisiti necessari alla 				stesura dell'\textit{Analisi dei Requisiti};
	\item \textbf{\#pianoqualifica}: Per discutere strategie da attuare per garantire qualità attraverso 									verifica\glossario e validazione\glossario.
	\end{itemize}

\paragraph{Comunicazioni Esterne} ~\\
	Le comunicazioni esterne avvengono attraverso la casella di posta elettronica del gruppo: 														\textit{agentsofswe@gmail.com}, gestita principalmente dal \textit{Responsabile} del gruppo, ma accessibile da 			ogni membro e configurata per eseguire un inoltro automatico delle mail ricevute ad ogni membro del gruppo.


\subsubsection{Gestione Riunioni}
	Durante ogni riunione, interna o esterna, verrà nominato, tra i componenti del gruppo, un segretario che avrà il 		compito di far rispettare l'ordine del giorno, stilato dal \textit{Responsabile di Progetto}, ed occuparsi della 		stesura del	Verbale di Riunione\glossario.

\paragraph{Riunioni Interne} ~\\
	E' compito del \textit{Responsabile di Progetto} organizzare riunioni interne al gruppo \texttt{Agents of 						S.W.E.}. Ciò prevede, più nello specifico, la stesura dell'ordine del giorno, stabilire data, orario e luogo di 			incontro, ed assicurarsi, attraverso la comunicazione mediante i mezzi propri del gruppo, che ogni componente sia 	pienamente a conoscenza della riunione in tutti i suoi dettagli. \\
	D'altro canto ogni membro del gruppo deve presentarsi puntuale agli appuntamenti, e comunicare in anticipo 					eventuali ridardi o assenze adeguatamente giustificate. \\
	Una riunione non è da ritenersi valida se i partecipanti risultino essere in numero inferiore a cinque.

\paragraph{Riunioni Esterne} ~\\
	E' nuovamente compito del \textit{Responsabile di Progetto} organizzare riunioni esterne. Nello specifico egli 			deve preoccuparsi di contattare l'azienda proponente per fissare incontri qualora sia necessario, tenendo conto anche delle preferenze 			di date e orario espresse dagli altri membri del gruppo. La partecipazione a tali riunioni deve essere, a meno di 	casi eccezionali, unanime.\\
	Ogni membro del gruppo può, inoltre, esprimere al \textit{Responsabile} una richiesta, adeguatamente motivata, 			di fissare una riunione esterna. A questo punto sarà compito dello stesso \textit{Responsabile} giudicare come 			valida o meno la richiesta presentatagli ed agire di conseguenza.

\paragraph{Verbale di Riunione} ~\\
\label{VdR}
	Ad ogni riunione, interna o esterna, è compito del Segretario designato redigere il Verbale di Riunione 							corrispondente, che deve essere poi approvato dal \textit{Responsabile}.\\
	Tale Verbale avrà la seguente \textbf{Struttura:}
	\begin{itemize}
	\item \textbf{Informazioni Generali:} questa prima sezione composta di:
		\begin{itemize}
		\item \textbf{Luogo};
		\item \textbf{Data};
		\item \textbf{Ora};
		\item \textbf{Membri del team partecipanti};
		\item \textbf{Segretario}.
		\end{itemize}
	\item \textbf{Ordine del Giorno:} sotto forma di elenco puntato rappresentante gli argomenti discussi;
	\item \textbf{Resoconto}: tale sezione rappresenta il riassunto, redatto dal Segretario, punto per punto di tutti 		gli argomenti di discussione, sia quelli preventivamente presenti nell'ordine del giorno sia eventuali spunti 				di riflessione maturati autonomamente durante la riunione;
	\item \textbf{Tracciamento delle Decisioni:} consiste in una tabella per il tracciamento delle decisioni effettuate durante l'incontro, in modo tale da avere un prospetto più chiaro ed immediato di ciò che è stato trattato. \\
	Ogni decisione sarà identificata da un codice nel formato \textbf{VER-DATA.X} dove VER indica la parola verbale, DATA rappresenta la data del verbale nel formato formato YYYY-MM-DD e X rappresenta un numero sequenziale della decisione presa.
	
	\end{itemize}


\subsection{Ruoli di Progetto}
	Nell'ottica di un lavoro ben organizzato e collaborativo tra i membri del gruppo, ad ogni componente, in ogni 					momento, è attribuito un ruolo per un periodo di tempo limitato.\\
	Questi ruoli, che corrispondono ad una figura aziendale ben precisa, sono:
	\begin{itemize}
	\item \textbf{Responsabile di Progetto};
	\item \textbf{Amministratore di Progetto};
	\item \textbf{Analista};
	\item \textbf{Progettista};
	\item \textbf{Programmatore};
	\item \textbf{Verificatore}.
	\end{itemize}
	\paragraph{\textit{Responsabile di Progetto}} ~\\
	Detto anche \textit{"Project Manager"}, è il rappresentate del progetto\glossario, agli occhi sia del committente che del 			fornitore. Egli risulta dunque essere, in primo luogo, il responsabile ultimo dei risultati del proprio gruppo. 			Figura di grande responsabilità, partecipa al progetto per tutta la sua durata, ha il compito di prendere 						decisioni 	e approvare scelte collettive.\\
	Nello specifico egli ha la responsabilità di:
	\begin{itemize}
	\item Coordinare le attività del gruppo, attraverso la gestione delle risorse umane;
	\item Approvare i documenti redatti, e verificati, dai membri del gruppo;
	\item Elaborare piani e scadenze, monitorando i progressi nell'avanzamento del progetto;
	\item Redigere l'organigramma del gruppo e il \textit{Piano di Progetto v2.0.0}\glossario.
	\end{itemize}

\paragraph{\textit{Amministratore di Progetto}} ~\\
	L'\textit{Amministrator}e è la figura chiave per quanto concerne la produttività. Egli ha infatti come primaria 							responsabilità il garantire l'efficienza\glossario del gruppo, fornendo strumenti utili e occupandosi 								dell'operatività delle risorse. Ha dunque il compito di gestire l'ambiente lavorativo.\\
	Tra le sue responsbilità specifiche figurano:
	\begin{itemize}
	\item Redigere documenti che normano l'attività lavorativa, e la loro verifica\glossario;
	\item Redigere le \textit{Norme di Progetto v2.0.0}\glossario;
	\item Scegliere ed amministrare gli strumenti di versionamento\glossario;
	\item Ricercare strumenti che possano agevolare il lavoro del gruppo;
	\item Attuare piani e procedure di gestione della qualità\glossario.
	\end{itemize}

\paragraph{\textit{Analista}} ~\\
	L'\textit{Analista} deve essere dotato di un'ottima conoscenza riguardo al dominio del problema. Egli ha infatti il 					compito di analizzare tale dominio e comprenderlo appieno, affinchè possa avvenire una corretta 											progettazione\glossario.\\
	Ha il compito di:
	\begin{itemize}
	\item Comprendere al meglio il problema, per poi poterlo esporre in modo chiaro attraverso specifici 									requisiti\glossario;
	\item Redarre lo \textit{Studio di Fattibilità v1.0.0} e l'\textit{Analisi dei Requisiti v2.0.0}\glossario.
	\end{itemize}

\paragraph{\textit{Progettista}} ~\\
	Il \textit{Progettista} è responsabile delle attività di progettazione\glossario attraverso la gestione degli aspetti tecnici 	del progetto.\\
	Più nello specifico si occupa di:
	\begin{itemize}
	\item Definire l'Architettura\glossario del prodotto\glossario, applicando quanto più possibile norme di 							best practice\glossario e prestando attenzione alla manutenibilità del prodotto;
	\item Suddividere il problema, e di conseguenza il sistema, in parti di complessità trattabile.
	\end{itemize}

\paragraph{\textit{Programmatore}} ~\\
	Il \textit{Programmatore} si occupa delle attività di codifica, le quali portano alla realizzazione effettiva del 						prodotto.\\
	Egli ha dunque il compito di:
	\begin{itemize}
	\item Implementare l'architettura definita dal \textit{Progettista}, prestando attenzione a scrivere codice 						coerente con ciò che è stato stabilito nelle \textit{Norme di Progetto v2.0.0};
	\item Produrre codice documentato e manutenibile;
	\item Realizzare le componenti necessarie per la verifica e la validazione\glossario del codice;
	\item Redarre il \textit{Manuale Utente v1.0.0}.
	\end{itemize}

\paragraph{\textit{Verificatore}} ~\\
	Il \textit{Verificatore}, figura presente per l'intera durata del progetto, è responsabile delle attività di 				verifica.\\
	Nello specifico egli:
	\begin{itemize}
	\item Verifica l'applicazione ed il rispetto delle \textit{Norme di Progetto v2.0.0};
	\item Segnala al \textit{Responsabile di Progetto} l'emergere di eventuali discordanze tra quanto presentato nel 			\textit{Piano di Progetto v2.0.0} e quanto effettivamente realizzato;
	\item Ha il compito di redarre il \textit{Piano di Qualifica v2.0.0}.
	\end{itemize}

\paragraph{Rotazione dei Ruoli} ~\\
	Come da istruzioni ogni membro del gruppo dovrà ricoprire, per un periodo di tempo limitato, ciascun ruolo, nel 			rispetto delle seguenti regole:
	\begin{itemize}
	\item Ciascun membro dovrà svolgere esclusivamente le attività proprie del ruolo a lui assegnato;
	\item Al fine di evitare conflitti di interesse nessun membro potrà ricoprire un ruolo che preveda la 									verifica di quanto da lui svolto, nell'immediato passato;
	\item Vista l'ampia differenza di compiti e mansioni tra i vari ruoli, e al fine di valorizzare l'attività 						collaborativa all'interno del gruppo, ogni componente che abbia ricoperto in precedenza un ruolo ora destinato 			a qualcun altro dovrà fornire supporto al compagno in caso di necessità, fornendogli consigli e, se possibile, 			affiancandolo in situazioni critiche.
	\end{itemize}


\subsection{Procedure}
\subsubsection{Gestione degli Strumenti di Versionamento}
	Come strumento per la gestione del versionamento dei files si è scelto di utilizzare una \textit{Repository}\glossario su \textit{GitHub}\glossario. La gestione di tale \textit{Repository}\glossario è compito dell'\textit{Amministratore di Progetto}.

\paragraph{Uso dei Branch} ~\\
	Al fine di agevolare il più possibile il parallelismo, evitando al contempo quanto più possibile eventuali problematiche in fase di \textit{merge}\glossario, sono stati creati i seguenti branch\glossario:
	\begin{itemize}
	\item \textbf{master}: questo branch contiene solamente i documenti che si trovano in stato \textbf{"Approvato"}, i quali formano dunque la \textit{baseline}\glossario;
	\item \textbf{develop}: questo è il branch di "sviluppo". Esso contiene tutti i documenti che, sebbene siano considerati ultimati per quanto riguarda la loro stesura, sono in attesa di approvazione o in fase di verifica;
	\item \textbf{feature/"nomeDocumento"}: vi sono quattro branch\glossario distinti questo tipo, nello specifico: "feature/analisiRequisiti", "feature/normeProgetto, "feature/pianoProgetto" e "feature/pianoQualifica". Ciascuno di questi è specializzato nell'avanzamento della stesura del documento a cui si riferisce;
	\item \textbf{feature/revisione"nomeDocumento"}: questi branch vengono sfruttati qualora un documento presente nel branch \textit{develop}, in sede di verifica, dovesse necessitare di modifiche non immediate.
	\end{itemize}

\paragraph{Norme delle Commit} ~\\
	Ogni \textit{commit}\glossario, ovvero ciascuna modifica alla repository\glossario \textit{Agents-of-S.W.E.} deve essere caratterizzata da una descrizione sensata, eventualmente accompagnata ad un riferimento esplicito ad una \textit{issue}\glossario aperta, al fine di agevolare una piena, e non onerosa, comprensione da parte di ogni membro del gruppo.

\paragraph{Norme dei Merge tra Branch} ~\\
	Al fine di rispettare la caratterizzazione che si è deciso di dare al branch \textbf{master}, e per agevolare un lavoro sistematico ed organizzato del lavoro si sono stabilite le seguenti norme in sede di \textit{merge}\glossario tra diversi branch\glossario:
	\begin{itemize}
	\item \textbf{Merge develop-feature/"nomeDocumento"}: questo merge avviene solo quando gli incaricati alla stesura del documento a cui si riferisce il branch: feature/"nomeDocumento" ritengono ultimata questa prima attività. Il documento in questione, dunque, si considera completo di ogni sua parte essenziale, eccezzion fatta per eventuali modifiche, anche cospicue, da apportare a seguito di un'attività di verifica e/o approvazione;
	\item \textbf{Merge develop-feature/revisione"nomeDocumento"}: questo merge avviene qualora le modifche necessarie al documento in questione siano risolte con successo;
	\item \textbf{Merge master-develop}: questo merge avviene solo quando ogni documento contenuto nel branch: "develop" è stato verificato ed approvato, e si considera dunque pronto per il rilascio.
	\end{itemize}


\subsubsection{Gestione degli Strumenti di Coordinamento}

\paragraph{Tasks} ~\\
	La suddivisione del lavoro in \textit{tasks}\glossario è compito del \textit{Responsabile di Progetto}. Lo strumento scelto per la creazione e gestione di questi \textit{tasks}\glossario è lo stesso Git\glossario, il quale mette a disposizione l'utile strumento delle \textit{issue}\glossario.\\
	La creazione di un \textit{task}\glossario da parte del \textit{Responsabile di Progetto}, risulterà dunque essere l'istanziazione di una \textit{issue}\glossario caratterizzata dalle seguenti proprietà:
	\begin{itemize}
	\item \textit{Titolo} significativo;
	\item \textit{Descrizione} concisa ma caratteristica ed esplicativa del problema da affrontare;
	\item \textit{Uno o più tags} associati a particolarità del \textit{task} in questione, e/o al/ai documento/i a cui si riferiscono. Tali etichette consentono una rapida catalogazione delle \textit{issue} stesse;
	\item \textit{Data di scadenza} che rappresenta il termine ultimo entro cui tale \textit{issue} deve essere chiusa.
	\end{itemize}
	E' importante far notare che, sebbene l'onere della suddivisione del lavoro, e dunque la creazione e gestione dei \textit{tasks}\glossario e dei conseguenti \textit{tickets}\glossario, ricada sul \textit{Responsabile di progetto}, ciascun membro del gruppo ha la facoltà di creare \textit{tasks}\glossario, a patto che tale compito veda lui come unico assegnatario. Tali \textit{tasks}\glossario dovranno inoltre essere approvati dal \textit{Responsabile} per essere validi.

\paragraph{Tickets} ~\\
	I \textit{tickets}\glossario rappresentano l'operazione di assegnazione dei \textit{tasks}\glossario, e quindi in questo caso delle \textit{issue}\glossario, ad uno o più specifici membri del gruppo. Tale operazione è responsabilità unica del \textit{Responsabile di Progetto} a meno che non si tratti di \textit{tasks}\glossario (approvati dal Responsabile) creati da un membro del gruppo, ed assegnati autonomamente a sè stesso.\\
	Questa operazione di \textit{ticketing} può avvenire in due modalità distinte:
	\begin{itemize}
	\item \textbf{Proattivamente}: nel caso in cui l'assegnatario del task in questione sia già noto, ed indicato come tale, alla creazione della issue da parte del \textit{Responsabile}. E' importante notare come questa sia l'unica modalità di \textit{ticketing} possibile nel caso in cui il creatore del task sia un membro diverso dal \textit{Responsabile di Progetto};
	\item \textbf{Retroattivamente}: nel caso in cui uno o più membri del gruppo vengano designati, in un secondo momento, come assegnatari di una issue già precedentemente esistente. Questa modalità di ticketing consente di gestire situazioni in cui non è utile individuare subito un assegnatario, come nel caso di task di importanza manginale e/o scandeze molto permissive, oppure invece si tratti di un compito le cui complessità sono emerse solo in seconda battuta, e necessiti dunque di un maggior apporto lavorativo per rispettarne le scadenze.
	\end{itemize}


\subsection{Strumenti}











\subsection{Formazione del Gruppo}
	La formazione dei componenti del gruppo \texttt{Agents of S.W.E.} è da considerarsi individuale. Ogni membro del 			team è infatti tenuto a documentarsi autonomamente riguardo le tecnologie coinvolte nello sviluppo del progetto. 			Tuttavia, nell'ottica di un ambiente di lavoro sano e collaborativo, nel caso in cui fosse necessario, è compito 			degli \textit{Amministratori} mettere a disposizione di chi ne avesse bisogno risorse basilari, al fine di 						agevolare la formazione dei restanti componenti del gruppo.
