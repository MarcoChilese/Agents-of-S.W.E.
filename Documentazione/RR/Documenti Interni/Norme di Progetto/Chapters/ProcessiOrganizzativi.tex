\section{Processi Organizzativi}\label{ProcessiOrganizzativi}

\subsection{Processi di Coordinamento}\label{ProcessiOrganizzativi_ProcessiCoordinamento}
\subsubsection{Gestione Comunicazioni}

\paragraph{Comunicazioni Interne}

\paragraph{Comunicazioni Esterne}


\subsubsection{Gestione Riunioni}

\paragraph{Riunioni Interne}

\paragraph{Riunioni Esterne}

\paragraph{Verbale di Riunione}

\subsection{Ruoli di Progetto}\label{ProcessiOrganizzativi_RuoliProgetto}
	Nell'ottica di un lavoro ben organizzato e collaborativo tra i membri del gruppo, ad ogni componente, in ogni 					momento, è attribuito un ruolo per un periodo di tempo limitato.\\
	Questi ruoli, che corrispondono ad una figura aziendale ben precisa, sono:
	\begin{itemize}
	\item \textit{Responsabile di Progetto};
	\item \textit{Amministratore di Progetto};
	\item \textit{Analista};
	\item \textit{Progettista};
	\item \textit{Programmatore};
	\item \textit{Verificatore}.
	\end{itemize}
	\paragraph{\textit{Responsabile di Progetto}} ~\\
	Detto anche \textit{"Project Manager"}, è il rappresentate del progetto\glossario, agli occhi sia del committente che del 			fornitore. Egli risulta dunque essere, in primo luogo, il responsabile ultimo dei risultati del proprio gruppo. 			Figura di grande responsabilità, partecipa al progetto per tutta la sua durata, ha il compito di prendere 						decisioni 	e approvare scelte collettive.\\
	Nello specifico egli ha la responsabilità di:
	\begin{itemize}
	\item Coordinare le attività del gruppo, attraverso la gestione delle risorse umane;
	\item Approvare i documenti redatti, e verificati, dai membri del gruppo;
	\item Elaborare piani e scadenze, monitorando i progressi nell'avanzamento del progetto;
	\item Redigere l'organigramma del gruppo e il \textit{Piano di Progetto v3.0.0}.
	\end{itemize}

\paragraph{\textit{Amministratore di Progetto}} ~\\
	L'\textit{Amministratore} è la figura chiave per quanto concerne la produttività. Egli ha infatti come primaria 							responsabilità il garantire l'efficienza\glossario del gruppo, fornendo strumenti utili e occupandosi 								dell'operatività delle risorse. Ha dunque il compito di gestire l'ambiente lavorativo.\\
	Tra le sue responsbilità specifiche figurano:
	\begin{itemize}
	\item Redigere documenti che normano l'attività lavorativa, e la loro verifica;
	\item Redigere le \textit{Norme di Progetto v3.0.0};
	\item Scegliere ed amministrare gli strumenti di versionamento\glossario;
	\item Ricercare strumenti che possano agevolare il lavoro del gruppo;
	\item Attuare piani e procedure di gestione della qualità\glossario.
	\end{itemize}

\paragraph{\textit{Analista}} ~\\
	L'\textit{Analista} deve essere dotato di un'ottima conoscenza riguardo al dominio del problema. Egli ha infatti il 					compito di analizzare tale dominio e comprenderlo appieno, affinchè possa avvenire una corretta 											progettazione\glossario.\\
	Ha il compito di:
	\begin{itemize}
	\item Comprendere al meglio il problema, per poi poterlo esporre in modo chiaro attraverso specifici 									requisiti\glossario;
	\item Redarre lo \textit{Studio di Fattibilità v1.0.0} e l'\textit{Analisi dei Requisiti v3.0.0}.
	\end{itemize}

\paragraph{\textit{Progettista}} ~\\
	Il \textit{Progettista} è responsabile delle attività di progettazione attraverso la gestione degli aspetti tecnici 	del progetto.\\
	Più nello specifico si occupa di:
	\begin{itemize}
	\item Definire l'Architettura\glossario del prodotto\glossario, applicando quanto più possibile norme di 							best practice\glossario e prestando attenzione alla manutenibilità del prodotto;
	\item Suddividere il problema, e di conseguenza il sistema, in parti di complessità trattabile.
	\end{itemize}

\paragraph{\textit{Programmatore}} ~\\
	Il \textit{Programmatore} si occupa delle attività di codifica, le quali portano alla realizzazione effettiva del 						prodotto.\\
	Egli ha dunque il compito di:
	\begin{itemize}
	\item Implementare l'architettura definita dal \textit{Progettista}, prestando attenzione a scrivere codice 						coerente con ciò che è stato stabilito nelle \textit{Norme di Progetto v3.0.0};
	\item Produrre codice documentato e manutenibile;
	\item Realizzare le componenti necessarie per la verifica e la validazione del codice;
	\item Redarre il \textit{Manuale Utente v1.0.0}.
	\end{itemize}

\paragraph{\textit{Verificatore}} ~\\
	Il \textit{Verificatore}, figura presente per l'intera durata del progetto, è responsabile delle attività di 				verifica.\\
	Nello specifico egli:
	\begin{itemize}
	\item Verifica l'applicazione ed il rispetto delle \textit{Norme di Progetto v3.0.0};
	\item Segnala al \textit{Responsabile di Progetto} l'emergere di eventuali discordanze tra quanto presentato nel 			\textit{Piano di Progetto v3.0.0} e quanto effettivamente realizzato;
	\item Ha il compito di redarre il \textit{Piano di Qualifica v3.0.0}.
	\end{itemize}

\paragraph{Rotazione dei Ruoli} ~\\
	Come da istruzioni ogni membro del gruppo dovrà ricoprire, per un periodo di tempo limitato, ciascun ruolo, nel 			rispetto delle seguenti regole:
	\begin{itemize}
	\item Ciascun membro dovrà svolgere esclusivamente le attività proprie del ruolo a lui assegnato;
	\item Al fine di evitare conflitti di interesse nessun membro potrà ricoprire un ruolo che preveda la 									verifica di quanto da lui svolto, nell'immediato passato;
	\item Vista l'ampia differenza di compiti e mansioni tra i vari ruoli, e al fine di valorizzare l'attività 						collaborativa all'interno del gruppo, ogni componente che abbia ricoperto in precedenza un ruolo ora destinato 			a qualcun altro dovrà fornire supporto al compagno in caso di necessità, fornendogli consigli e, se possibile, 			affiancandolo in situazioni critiche.
	\end{itemize}


\subsection{Procedure}\label{ProcessiOrganizzativi_Procedure}
\subsubsection{Gestione degli Strumenti di Versionamento}\label{ProcessiOrganizzativi_Procedure_GestioneStrumentiVersionamento}
	Come strumento per la gestione del versionamento dei documenti si è scelto di utilizzare un \textit{Repository} su \textit{GitHub}, mentre come gestione del versionamento del codice si è scelto di utilizzare un repository su \textit{GitLab}. La gestione di tali \textit{Repository} è compito dell'\textit{Amministratore di Progetto}.

\paragraph{Uso dei Branch} ~\\
	Al fine di agevolare il più possibile il parallelismo, evitando al contempo quanto più possibile eventuali problematiche in fase di \textit{merge}\glossario, sono stati creati i seguenti branch\glossario:
	\begin{itemize}
	\item \textbf{master}: questo branch contiene solamente i documenti e file che si trovano in stato "Approvato", i quali formano dunque la baseline\glossario;
	\item \textbf{develop}: questo è il branch di "sviluppo". Esso contiene tutti i documenti e file che, sebbene siano considerati ultimati per quanto riguarda la loro stesura, sono in attesa di approvazione o in fase di verifica;
	\item \textbf{feature/"nomeDocumento"}: vi sono quattro branch distinti questo tipo, nello specifico: "feature/analisiRequisiti", "feature/normeProgetto, "feature/pianoProgetto" e "feature/pianoQualifica". Ciascuno di questi è specializzato nell'avanzamento della stesura del documento a cui si riferisce;
	%TODO: inserire struttura repository su gitlab
	\item \textbf{feature/revisione"nomeFile"}: questi branch vengono sfruttati qualora un documento o file presente nel branch develop, in sede di verifica, dovesse necessitare di modifiche non immediate.
	\end{itemize}

\paragraph{Norme delle Commit} ~\\
	Ogni commit\glossario, ovvero ciascuna modifica alle repository, deve essere caratterizzata da una descrizione sensata, eventualmente accompagnata ad un riferimento esplicito ad una issue\glossario aperta, al fine di agevolare una piena, e non onerosa, comprensione da parte di ogni membro del gruppo.

\paragraph{Norme dei Merge tra Branch} ~\\
	Al fine di rispettare la caratterizzazione che si è deciso di dare al branch master, e per agevolare un lavoro sistematico ed organizzato del lavoro si sono stabilite le seguenti norme in sede di merge tra diversi branch:
	\begin{itemize}
	\item \textbf{Merge develop-feature/"nomeDocumento"}: questo merge avviene solo quando gli incaricati alla stesura del documento a cui si riferisce il branch: feature/"nomeDocumento" ritengono ultimata questa prima attività. Il documento in questione, dunque, si considera completo di ogni sua parte essenziale, eccezzion fatta per eventuali modifiche, anche cospicue, da apportare a seguito di un'attività di verifica e/o approvazione;
	\item \textbf{Merge develop-feature/revisione"nomeDocumento"}: questo merge avviene qualora le modifche necessarie al documento in questione siano risolte con successo;
	\item \textbf{Merge master-develop}: questo merge avviene solo quando ogni documento contenuto nel branch: "develop" è stato verificato ed approvato, e si considera dunque pronto per il rilascio.
	\end{itemize}


\subsubsection{Gestione degli Strumenti di Coordinamento}

\paragraph{Task} ~\\
	La suddivisione del lavoro in task\glossario è compito del \textit{Responsabile di Progetto}. Lo strumento scelto per la creazione e gestione di questi task è lo stesso \textit{Git}\glossario, il quale mette a disposizione l'utile strumento delle issue.\\
	La creazione di un task da parte del \textit{Responsabile di Progetto}, risulterà dunque essere l'istanziazione di una issue caratterizzata dalle seguenti proprietà:
	\begin{itemize}
	\item \textbf{Titolo}: significativo;
	\item \textbf{Descrizione}: concisa ma caratteristica ed esplicativa del problema da affrontare;
	\item \textbf{Uno o più tags}: associati a particolarità del task in questione, e/o al/ai documento/i a cui si riferiscono. Tali etichette consentono una rapida catalogazione delle issue stesse;
	\item \textbf{Data di scadenza}: che rappresenta il termine ultimo entro cui tale issue deve essere chiusa.
	\end{itemize}
	E' importante far notare che, sebbene l'onere della suddivisione del lavoro, e dunque la creazione e gestione dei task e dei conseguenti ticket\glossario, ricada sul \textit{Responsabile di progetto}, ciascun membro del gruppo ha la facoltà di creare task, a patto che tale compito veda lui come unico assegnatario. Tali task dovranno inoltre essere approvati dal \textit{Responsabile} per essere validi.

\paragraph{Ticket} ~\\
	I tickets rappresentano l'operazione di assegnazione dei task, e quindi in questo caso delle issue, ad uno o più specifici membri del gruppo. Tale operazione è responsabilità unica del \textit{Responsabile di Progetto} a meno che non si tratti di task (approvati dal Responsabile) creati da un membro del gruppo, ed assegnati autonomamente a sè stesso.\\
	Questa operazione di ticketing può avvenire in due modalità distinte:
	\begin{itemize}
	\item \textbf{Proattivamente}: nel caso in cui l'assegnatario del task in questione sia già noto, ed indicato come tale, alla creazione della issue da parte del \textit{Responsabile}. E' importante notare come questa sia l'unica modalità di ticketing possibile nel caso in cui il creatore del task sia un membro diverso dal \textit{Responsabile di Progetto};
	\item \textbf{Retroattivamente}: nel caso in cui uno o più membri del gruppo vengano designati, in un secondo momento, come assegnatari di una issue già precedentemente esistente. Questa modalità di ticketing consente di gestire situazioni in cui non è utile individuare subito un assegnatario, come nel caso di task di importanza manginale e/o scandeze molto permissive, oppure invece si tratti di un compito le cui complessità sono emerse solo in seconda battuta, e necessiti dunque di un maggior apporto lavorativo per rispettarne le scadenze.
	\end{itemize}


\subsection{Strumenti}\label{ProcessiOrganizzativi_Strumenti}
\subsubsection{Sistema Operativo}
	I sistemi operativi utilizzati dai membri del gruppo sono i seguenti:
	\begin{itemize}
	\item Windows 10 x64;
	\item Mac OS 10.14.1;
	\item Majaro Linux - 4.14.85-1 LTS.
	\end{itemize}

\subsubsection{Versionamento e Issue Tracking}

\paragraph{Git} ~\\
	\textit{Git} è il sistema di versionamento\glossario scelto dal gruppo. E' un sistema open source creato da Linus 	Torvalds 2005. Presenta un'interfaccia a riga di comando, tuttavia esistono svariati \textit{tools}\glossario 	che 	ne forniscono una GUI.

\paragraph{GitHub} ~\\
	\textit{Github} è un \textit{Respository Manager}\glossario che usa Git come sistema di versioning, offre inoltre 	servizi di \textit{issue tracking}\glossario.

\subsubsection{Comunicazione}

\paragraph{Telegram} ~\\
	Telegram è una delle maggiori e più note applicazioni di messaggistica istantanea \textit{cross platform}, 					utilizzabile contemporaneamente su più dispositivi. Oltre alla semplice messaggistica offre servizi quali lo 				scambio di files, la creazione di gruppi e le chiamate vocali.

\paragraph{Slack} ~\\
	Slack è un'applicazione di messaggistica istantanea specializzata nella comunicazione interna tra membri di un 			gruppo di lavoro. L'applicazione è organizzata in \textit{workspace}, a loro volta suddivisi in canali, i quali 			consentono di catalogare le conversazione sulla base dell'argomento trattato. Questa struttura, studiata 						appositamente per l'ambito lavorativo, è stata giudicata come positiva e vantaggiosa dal gruppo, visto che 					consente di mantenere chat ordinate e monotematiche.\\
	Nonostante preveda anche abbonamenti a pagamento le funzionalità base di Slack sono gratuite, inoltre, come 					Telegram, risulta essere un'applicazione \textit{cross platform}\glossario.

\subsubsection{Diagrammi di Gantt}
	Lo strumento scelto dal gruppo per la realizzazione dei diagrammi di Gantt\glossario è "Gantt Project". Le 					motivazioni che hanno portato a questa scelta sono molteplici, tra queste spiccano il fatto che sia uno strumento 	gratuito, \textit{open-source}\glossario, e \textit{cross platform}. L'elevata accessibilità è stata infatti 				giudicata come una caratteristica di primaria importanza, considerando i differenti sistemi operativi utilizzati 		dai componenti del gruppo.

\subsubsection{Diagrammi UML}
	Lo strumento scelto dal gruppo per la realizzazione dei diagrammi UML è "Umbrello", un software \textit{open-				source}\glossario per il disegno di diagrammi UML 2.0, creato da KDE Team. Anche in questo caso è stata tenuta in 	grande considerazione l'accessibilità dello strumento. Il software in questione risulta infatti gratuito e 					disponibile sia su Windows, sia su Mac OS tramite porting, sebbene sia nativo Unix.\\
	Umbrello risulta essere sia \textit{User Friendly}, con un'interfaccia utente chiara e consistente, sia 							abbastanza potente da essere utilizzato anche per le industrie. Questo, unito alle dimensioni relativamente 					ridotte del software, e al fatto che non sia richiesta alcuna licenza per l'installazione, hanno fatto propendere 	il gruppo per questo programma, a discapito di alternative, come Papyrus o Astah, comunque considerate valide.


\subsection{Formazione del Gruppo}\label{ProcessiOrganizzativi_FormazioneGruppo}
	La formazione dei componenti del gruppo \texttt{Agents of S.W.E.} è da considerarsi individuale. Ogni membro del 			team è infatti tenuto a documentarsi autonomamente riguardo le tecnologie coinvolte nello sviluppo del 								progetto\glossario. Tuttavia, nell'ottica di un ambiente di lavoro sano e collaborativo, nel caso in cui fosse 				necessario, è compito degli \textit{Amministratori} mettere a disposizione di chi ne avesse bisogno risorse 						basilari, al fine di	agevolare la formazione dei restanti componenti del gruppo.
