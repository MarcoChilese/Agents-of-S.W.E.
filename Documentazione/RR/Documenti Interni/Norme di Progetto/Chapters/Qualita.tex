\subsection{Qualità} \label{qualita}

\subsubsection{Introduzione}
Il contenuto di questa sezione descrive le metriche e i criteri di qualità di processo e prodotto che vengono utilizzati nel documento \textit{Piano di qualifica}.

\subsubsection{Classificazione Processi}
Per garantire una corretta struttura, il gruppo ha deciso di introdurre una nomenclatura per identificare i processi descritti nel documento. I processi saranno identificati così:
\begin{center}
	\textbf{PRC[num]}
\end{center}
dove:
\begin{itemize}
	\item \textbf{num}: rappresenta il numero identificativo del processo formato da due cifre intere a partire da 1, unico per tutto il documento.
\end{itemize}

\subsubsection{Classificazione Metriche}
Per garantire una corretta struttura, il gruppo ha deciso di introdurre una nomenclatura per identificare le metriche utilizzate. Si potranno quindi identificare cosi:
\begin{center}
	\textbf{MT[cat][mcat][num]}
\end{center}
\begin{itemize}
		\item \textbf{cat}: identifica la categoria di appartenenza della metrica. Può assumere i seguenti valori:
	\begin{itemize}
		\item \textbf{PC}: per indicare le metriche di processo;
		\item \textbf{PD}: per indicare le metriche di prodotto;
		\item \textbf{TS}: per indicare le metriche di test.
	\end{itemize}
	\item \textbf{mcat}: identifica la macrocategoria di appartenenza, se esiste, altrimenti è vuota. Per le metriche di prodotto può assumere i seguenti valori:
	\begin{itemize}
		\item \textbf{D}: per indicare i documenti;
		\item \textbf{S}: per indicare il software.
	\end{itemize}
	Per le metriche di test invece può assumere:
	\begin{itemize}
		\item \textbf{A}: tutti i tipi di test;
		\item \textbf{M}: test di modulo;
		\item \textbf{H}: test ad alto livello.
	\end{itemize}
	\item \textbf{num}: identificativo univoco formato da due cifre intere a partire da 1.
\end{itemize}

