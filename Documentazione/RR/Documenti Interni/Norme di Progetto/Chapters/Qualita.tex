\subsection{Qualità} \label{qualita}

\subsubsection{Introduzione}
Il contenuto di questa sezione descrive le metriche e i criteri di qualità di processo e prodotto che vengono utilizzati nel documento \textit{Piano di qualifica}.

\subsubsection{Classificazione Processi}
Per garantire una corretta struttura, il gruppo ha deciso di introdurre una nomenclatura per identificare i processi descritti nel documento. I processi saranno identificati così:
\begin{center}
	\textbf{PRC[num]}
\end{center}
dove:
\begin{itemize}
	\item \textbf{num}: rappresenta il numero identificativo del processo formato da due cifre intere a partire da 1, unico per tutto il documento.
\end{itemize}

\subsubsection{Classificazione Metriche}
Per garantire una corretta struttura, il gruppo ha deciso di introdurre una nomenclatura per identificare le metriche utilizzate. Si potranno quindi identificare cosi:
\begin{center}
	\textbf{MT[cat][mcat][num]}
\end{center}
\begin{itemize}
		\item \textbf{cat}: identifica la categoria di appartenenza della metrica. Può assumere i seguenti valori:
	\begin{itemize}
		\item \textbf{PC}: per indicare le metriche di processo;
		\item \textbf{PD}: per indicare le metriche di prodotto;
		\item \textbf{TS}: per indicare le metriche di test.
	\end{itemize}
	\item \textbf{mcat}: identifica la macrocategoria di appartenenza, se esiste, altrimenti è vuota. Per le metriche di prodotto può assumere i seguenti valori:
	\begin{itemize}
		\item \textbf{D}: per indicare i documenti;
		\item \textbf{S}: per indicare il software.
	\end{itemize}
	Per le metriche di test invece può assumere:
	\begin{itemize}
		\item \textbf{A}: tutti i tipi di test;
		\item \textbf{M}: test di modulo;
		\item \textbf{H}: test ad alto livello.
	\end{itemize}
	\item \textbf{num}: identificativo univoco formato da due cifre intere a partire da 1.
\end{itemize}

\subsubsection{Controllo qualità di processo}
La qualità di processo è raggiunta tramite l'utilizzo di metodi e modelli che garantiscono il corretto procedimento delle fasi di sviluppo del processo. Il team sfrutta le potenzialità del metodo PDCA, descritto nell'appendice A, ottenendo miglioramenti continui nelle qualità di processo e verifica in modo da avere una maggiore qualità nel prodotto risultante. Inoltre verrà utilizzato lo standard ISO/IEC 15504, comunemente conosciuto con l'acronimo SPICE, che misurerà il livello di maturità dei processi.

\subsubsection{Metriche per la qualità di processo}
Le seguenti metriche sono utilizzate per la valutazione dell'efficacia e dell'efficienza dei processi.

\paragraph{MTPC01 Schedule Variance(SV)}\--\\
\'E un indice che consente di rilevare se l'andamento del progetto è in linea con quanto pianificato nella baseline\glossario. Può essere utile al cliente per verificare quantitativamente se il team di sviluppo del progetto intero sono in linea con i tempi. \-\\
Calcolo:\-\\
\begin{center}
	\item \textbf{SV = BCWP - BCWS}
\end{center}
dove:
\begin{itemize}
	\item BCWP: valore delle attività realizzate alla data odierna(in euro o giorni);
	\item BCWS: costo pianificato per realizzare le attività alla data odierna(in euro o giorni).
\end{itemize}
Se SV > 0 significa che il progetto sta producendo con maggiore velocità prevista.

\paragraph{MTPC02 Budget Variance(BV)}\-\\
\'E un indice calcolato che permette di rilevare la differenza tra i costi previsti e quelli reali alla data odierna. 
Calcolo:
\begin{center}
	\item \textbf{BV = BCWS - ACWP}
\end{center} 
dove:
\begin{itemize}
	\item BCWS: costo pianificato per realizzare le attività alla data odierna(in euro o giorni);
	\item ACWP: costo effettivo sostenuto per completare le attività alla data odierna(in euro o giorni).
\end{itemize}
Se BV > 0 significa che il progetto sta risparmiando sui costi prestabiliti, se BV = 0 significa che il progetto sta mantenendo i costi prefissati, se BV = 0 significa che il progetto sta superando il budget imposto.

\paragraph{MTPC03 Estimated At Completion(EAC)}
Indice che rappresenta la stima dei costi mancanti. Viene calcolato man mano che il progetto procede ed è fondamentale per attività di pianificazione.
Calcolo:
\begin{center}
	\item \textbf{EAC = ACWP + ETC}
\end{center}
dove:
\begin{itemize}
	\item ACWP: costo effettivo sostenuto per completare le attività alla data odierna(in euro o giorni);
	\item ETC: valore stimato per la realizzazione delle attività mancanti(in euro o giorni).
\end{itemize}

\paragraph{MTPC04 Cost Variance(CV)}
Indice che quantifica la produttività o efficienza monitorando se il valore del costo realmente maturato è minore, maggiore o uguale al costo effettivo.
Calcolo:
\begin{center}
	\item \textbf{CV = BCWP - ACWP}
\end{center}
dove:
\begin{itemize}
	\item BCWP: valore delle attività realizzate alla data odierna(in euro o giorni);
	\item ACWP: costo effettivo sostenuto per completare le attività alla data odierna(in euro o giorni).
\end{itemize}
Se CV > 0 significa che il progetto produce con maggior efficienza e minor costo.

\paragraph{Code Coverage}\-\\
Le operazioni di verifica della qualità sul software vengono eseguite seguendo le metriche principali sotto elencate:
\begin{itemize}
	\item \textbf{MTPC05 Function Coverage}: verificare che una funzione sia chiamata;
	\item \textbf{MTPC06 Statement Coverage}: verificare che ogni statement del codice sia eseguito, e non ci sia quindi codice che non verrà mai preso in considerazione;
	\item \textbf{MTPC07 Branch Coverage}: verificare che tutti i possibili percorsi delle strutture di controllo (del tipo if,case,ecc) siano stati eseguiti.
	\item \textbf{MTPC08 Condition Coverage}: verificare che ogni condizione booleana sia considerata sia vera che falsa.
\end{itemize}

\subsubsection{Controllo qualità di prodotto}
La qualità di prodotto è raggiunta mediante l'utilizzo dello standard ISO/IEC 9126:2001. TO DO

\subsubsection{Metriche per la qualità di prodotto}
TO DO


