\section{Introduzione}\label{Intro}

\subsection{Scopo del Documento}
Il documento \textit{Norme di Progetto} si propone di definire le regole che i membri del gruppo \texttt{Agents of S.W.E.} sono tenuti a rispettare durante tutto lo svolgimento del progetto "\textit{G\&B}", al fine di garantire quanto più possibile l'uniformità del materiale prodotto.\\
Le norme presenti in questo documento verranno prodotte incrementalmente, al progressivo maturare delle esigenze e delle attività di progetto. Di conseguenza il documento, allo stato corrente, non è da considerarsi completo.\\
Il documento sarà perciò aggiornato e sottoposto a più revisioni.

\subsection{Ambiguità e Glossario}
I termini che potrebbero risultare ambigui all'interno del documento sono siglati tramite pedice rappresentante la lettera \textmd{G}, tale terminologia trova una sua più specifica definizione nel \textit{Glossario v1.0.0} che viene fornito tra i Documenti Esterni.

\subsection{Riferimenti}
\subsubsection{Referimenti Normativi}
\begin{itemize}
\item \textbf{Capitolato d'Appalto C3:}\\ \url{https://www.math.unipd.it/~tullio/IS-1/2018/Progetto/C3.pdf};
\item \textbf{ISO/IEC 9126:2001:}\\ \url{https://www.math.unipd.it/~tullio/IS-1/2018/Dispense/L13.pdf};
\item \textbf{Metriche:}
	\begin{itemize}
	\item \url{https://www.qasymphony.com/blog/64-test-metrics};
	\item \url{https://www.softwaretestinghelp.com/software-test-metrics-and-mesurements}.
	\end{itemize}
\end{itemize}

\subsubsection{Referimenti Informativi}
\begin{itemize}
\item \textbf{Presentazione Capitolato:}\\ \url{https://www.math.unipd.it/~tullio/IS-1/2018/Progetto/C3p.pdf};
\item \textbf{Materiale didattico del corso di Ingegneria del Software:}
	\begin{itemize}
	\item \textbf{Gestione di Progetto:}\\ \url{https://www.math.unipd.it/~tullio/IS-1/2018/Dispense/L06.pdf};
	\item \textbf{Regole del progetto didattico:}\\ \url{https://www.math.unipd.it/~tullio/IS-1/2018/Dispense/P01.pdf};
	\item \textbf{Regolamento organigramma:}\\ \url{https://www.math.unipd.it/~tullio/IS-1/2018/Progetto/RO.html}.
	\end{itemize}
\end{itemize}

All'interno del documento eventuali ulteriori riferimenti normativi o informativi vengono contrassegnati da apice\footnote{} e riportati a piè pagina.