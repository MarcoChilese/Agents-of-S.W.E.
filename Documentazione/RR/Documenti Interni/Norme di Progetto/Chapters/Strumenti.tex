\subsubsection{Sistema Operativo}
	I sistemi operativi utilizzati dai membri del gruppo sono i seguenti:
	\begin{itemize}
	\item Windows 10 x64;
	\item Mac OS 10.14.1;
	\item Manjaro Linux - 4.14.85-1 LTS.
	\end{itemize}

\subsubsection{Versionamento e Issue Tracking}

\paragraph{Git} ~\\
	\textit{Git} è il sistema di versionamento\glossario scelto dal gruppo. E' un sistema open source creato da Linus 	Torvalds 2005. Presenta un'interfaccia a riga di comando, tuttavia esistono svariati \textit{tools}\glossario 	che 	ne forniscono una GUI.

\paragraph{GitHub} ~\\
	\textit{Github} è un \textit{Respository Manager}\glossario che usa Git come sistema di versioning, offre inoltre 	servizi di \textit{issue tracking}\glossario.

\subsubsection{Comunicazione}

\paragraph{Telegram} ~\\
	Telegram è una delle maggiori e più note applicazioni di messaggistica istantanea \textit{cross platform}, 					utilizzabile contemporaneamente su più dispositivi. Oltre alla semplice messaggistica offre servizi quali lo 				scambio di files, la creazione di gruppi e le chiamate vocali.

\paragraph{Slack} ~\\
	Slack è un'applicazione di messaggistica istantanea specializzata nella comunicazione interna tra membri di un 			gruppo di lavoro. L'applicazione è organizzata in \textit{workspace}, a loro volta suddivisi in canali, i quali 			consentono di catalogare le conversazione sulla base dell'argomento trattato. Questa struttura, studiata 						appositamente per l'ambito lavorativo, è stata giudicata come positiva e vantaggiosa dal gruppo, visto che 					consente di mantenere chat ordinate e monotematiche.\\
	Nonostante preveda anche abbonamenti a pagamento le funzionalità base di Slack sono gratuite, inoltre, come 					Telegram, risulta essere un'applicazione \textit{cross platform}\glossario.

\subsubsection{Diagrammi di Gantt}
	Lo strumento scelto dal gruppo per la realizzazione dei diagrammi di Gantt\glossario è "Gantt Project". Le 					motivazioni che hanno portato a questa scelta sono molteplici, tra queste spiccano il fatto che sia uno strumento 	gratuito, \textit{open-source}\glossario, e \textit{cross platform}. L'elevata accessibilità è stata infatti 				giudicata come una caratteristica di primaria importanza, considerando i differenti sistemi operativi utilizzati 		dai componenti del gruppo.

\subsubsection{Diagrammi UML}
	Lo strumento scelto dal gruppo per la realizzazione dei diagrammi UML è "Umbrello", un software \textit{open-				source}\glossario per il disegno di diagrammi UML 2.0, creato da KDE Team. Anche in questo caso è stata tenuta in 	grande considerazione l'accessibilità dello strumento. Il software in questione risulta infatti gratuito e 					disponibile sia su Windows, sia su Mac OS tramite porting, sebbene sia nativo Unix.\\
	Umbrello risulta essere sia \textit{User Friendly}, con un'interfaccia utente chiara e consistente, sia 							abbastanza potente da essere utilizzato anche per le industrie. Questo, unito alle dimensioni relativamente 					ridotte del software, e al fatto che non sia richiesta alcuna licenza per l'installazione, hanno fatto propendere 	il gruppo per questo programma, a discapito di alternative, come Papyrus o Astah, comunque considerate valide.
