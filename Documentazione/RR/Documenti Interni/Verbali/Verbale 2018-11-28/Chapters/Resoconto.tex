\section{Resoconto}

\subsection{Punto 1}

La struttura dello \textit{Studio di Fattibilità} scelta è stata concordata con un voto unanime da parte del gruppo. Tale struttura è mirata alla facile compressione dei capitolati, esponendo i pro ed i contro di ognuno di essi, le tecnologie proposte e un breve conclusione sulle motivazioni della scelta in base alle competenze pregresse e alle tecnologie richieste.  


\subsection{Punto 2}
L'assegnazione dei ruoli per la prima parte della fase di Revisione dei Requisiti \'e stata effettuata tramite votazioni. I ruoli per ogni membro sono i seguenti:

\begin{center}
\begin{tabular}{|c|c|c|}
\hline
\rowcolor{bluelogo}\textbf{\textcolor{white}{Membro}} & \textbf{\textcolor{white}{Ruolo}} \\
\hline
Luca Violato & Amministratore\\
\hline
\rowcolor{grigio}Bogdan Stanciu &  Responsabile\\
\hline
Marco Chilese & Verificatore \\
\hline
\rowcolor{grigio} Carlotta Segna & Verificatore \\
\hline
Marco Favaro & Analista\\
\hline
\rowcolor{grigio} Diego Mazzalovo & Analista\\
\hline
Matteo Slanzi & Analista \\
\hline
\end{tabular}
\end{center}


\subsection{Punto 3}
Discussione del contenuto riguardante la struttura dei documenti e le norme tipografiche da mantenere per essi.

\subsection{Punto 4}
Discussione sulla struttura del glossario, al fine di cominciare ad inserire termini. Successivo inserimento dei termini.

\subsection{Punto 5}
Discussione dei strumenti tecnologici da utilizzare nella realizzazione, mantenimento, condivisione, verifica dei documenti interni e dei canali di comunicazione tra i membri del gruppo. \\
Tali strumenti vengono qua solamente accennati, ma sono analizzati nel dettaglio  all'interno delle \textit{Norme di Progetto v1.0.0}:
\begin{itemize}
	\item \textbf{Git}: sistema di versionamento utilizzato da più membri del gruppo. Ciò ha indotto alla scelta di questo come sistema da utilizzare; 
	\item \textbf{Slack}: hub di comunicazione tra i vari membri di un team o di una realtà entreprise di grandi dimensioni. Tale hub e molto famoso tra le comunità di sviluppatori e fornisce l'integrazione di vari strumenti, quali  				\textit{GitHub} e \textit{Google Calendar}. Viene scelto come strumento di comunicazione interna e organizzazione di task e scadenze con i membri del team; 
	\item \textbf{LaTeX}: linguaggio di markup per la preparazione di testi, basato su  composizione tipografica utilizzato per la redazione di ogni documento interno-esterno; 
	\item \textbf{GanttProject}: software per la creazione dei diagrammi di Gantt;
	\item \textbf{Umbrello}: software di modellazione UML con supporto a tutti i sistemi operativi più utilizzati \textit{(Linux, MacOS, Windows)}.
\end{itemize}

I sistemi operativi utilizzati dai componenti del gruppo sono:
\begin{itemize}
	\item \textbf{MacOS};
	\item \textbf{Windows 10};
	\item \textbf{Manjaro Linux}.
\end{itemize}

\subsection{Punto 6}
Le \textit{Norme di Progetto} sono in fase di redazione da parte di: 
\begin{itemize}
	\item \textbf{Bogdan Stanciu};
	\item \textbf{Marco Chilese};
	\item \textbf{Luca Violato};
	\item \textbf{Marco Favaro}.
\end{itemize} 

La redazione del \textit{Piano di Progetto} è assegnata a:
\begin{itemize}
	\item \textbf{Carlotta Segna};
	\item \textbf{Matteo Slanzi};
	\item \textbf{Diego Mazzalovo}.
\end{itemize}

