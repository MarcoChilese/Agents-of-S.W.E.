\section{Resoconto}

\subsection{Punto1}

La \textbf{struttura dello studio di fattibilità} scelta è stata concordata con un voto unanime da parte del gruppo. Tale struttura è mirata alla facile compressione dei capitolati, esponendo principalmente i pro e i contro di ognuno di essi, le tecnologie proposte e un breve conclusione sulle motivazioni della scelta in base alle competenze pregresse e alle tecnologie richieste.  


\subsection{Punto2}
L'assegnazione dei ruoli per la fase di \textbf{RR} è stata effettuata, scegliendo il candidato per ogni ruolo, a votazione da parte del gruppo. I ruoli per ogni membro sono i seguenti: 
\begin{itemize}
	\item \textbf{Amministratore}: Luca Violato.
	\item \textbf{Responsabile}: Bogdan Stanciu.
	\item \textbf{Verificatore}: Marco Chilese, Carlotta Segna. 
	\item \textbf{Analista}: Diego Mazzalovo, Marco Favaro, Matteo Slanzi. 
\end{itemize}


\subsection{Punto 3}

\subsection{Punto 4}


\subsection{Punto 5}
Discussione dei strumenti tecnologici da utilizzare nella realizzazione, mantenimento, condivisione, verifica dei documenti interni e dei canali di comunicazione tra i membri del gruppo. \\
\textbf{Git}: Sistema di versionamento più utilizzato tra i membri del gruppo. Ciò ha indotto a una scelta \textit{"ovvia"} del sistema da utilizzare, poiché tutti i membri del gruppo possiedono conoscenze pregresse di questa tecnologia. \\ 
\textbf{Slack}: hub di comunicazione tra i vari membri di un team o di una realtà entreprise di grandi dimensioni. Tale hub e molto famoso tra le comunità di sviluppatori e fornisce l'integrazione di vari strumenti, quali \textit{GitHub} e \textit{Google Calendar}. Viene scelto come strumento di comunicazione interna e organizzazione di task e scadenze con i membri del team. \\
\textbf{LaTeX}: che dire (?) \\
\textbf{GanttProject}: software di gestione processi e tempistiche. (?) 
\textbf{Umbrello}: software di modellazione UML con supporto a tutti i sistemi operativi più utilizzati \textit{(Linux, MacOS, Windows)} \\
\textbf{HTML e CSS Validator} \\
\textbf{Elenco SO}: (?) \\ 
\textbf{MacOS}: (?) \\
\textbf{Windows 10}:(?) \\ 
\textbf{Manjaro Linux}: (?)



