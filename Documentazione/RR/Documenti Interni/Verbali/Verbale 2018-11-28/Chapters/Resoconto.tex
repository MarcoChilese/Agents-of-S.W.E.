\section{Resoconto}

\subsection{Punto1}
Il \textbf{nome} scelto è stato deciso tramite un sondaggio vista la presenza di più possibilità ed indecisione sul quale utilizzare. Abbiamo deciso di identificarci con il nome \textbf{Agents of S.W.E.}.\\
La scelta del \textbf{logo} è avvenuta con lo stesso processo di scelta vista, nuovamente, la presenza di più possibilità e l'indecisione sul quale utilizzare.

\subsection{Punto2}
La \textbf{discussione e l'analisi del capitoli} proposti sono stati il punto principale di questo incontro. \\
Precedentemente avevamo analizzato singolarmente i contenuti delle varie proposte e durante questo incontro abbiamo effettuato un brainstorming andando ad analizzare, collettivamente, ogni singolo capitolato. Durante questa analisi abbiamo valutato i pro ed i contro di ogni offerta, le tecnologie da utilizzare, la fattibilità di ogni progetto ed, infine, le conoscenze pregresse relative ad ognuno di essi. \\
Per ogni capitolato è stato assegnato un livello di difficoltà soggettivo, che andava ad unire i fattori analizzati precedentemente elencati.\\
Alla fine di questo incontro sono state espresse delle preferenze di gruppo che si sono concluse con la preferenza del capitolato \textbf{C3 - G\&B: monitoraggio intelligente di processi DevOps} offerto da Zucchetti, seguito subito dopo da \textbf{Butterfly: monitor per processi CI/CD} offerto da Imola Informatica ed infine \textbf{MegAlexa: arricchitore di skill di Amazon Alexa}, offerto dall'azienza Zero12.\\
