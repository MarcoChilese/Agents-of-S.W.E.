\section{Resoconto}

\subsection{Punto 1}
Il gruppo ha discusso sulle segnalazioni individuate dal committente in sede di Revisione dei Requisiti e ha proceduto con l'assegnazione di task ad ogni membro del gruppo per correggerli.
\\
I task sono stati assegnati dal \textit{Responsabile} ad ogni membro del gruppo in modo tale che ogni componente, si occupi di correggere le segnalazioni sui documenti che ha redatto.

\subsection{Punto 2}
I ruoli all'interno del team, dopo la Revisione dei Requisiti, ruotano nel seguente modo:\\
\begin{center}
\begin{longtable}[c]{|m{.20\textwidth}|m{.20\textwidth}|m{.20\textwidth}|} 
\hline
\rowcolor{bluelogo}\textbf{\textcolor{white}{Membro}} & \textbf{\textcolor{white}{Vecchio Ruolo}} & \textbf{\textcolor{white}{Nuovo Ruolo}}\\
\hline
Matteo Slanzi & Amministratore & Analista\\
\hline
\rowcolor{grigio}Carlotta Segna & Responsabile & Verificatore\\
\hline
Marco Favaro & Verificatore & Analista\\
\hline
\rowcolor{grigio}Diego Mazzalovo & Verificatore & Analista\\
\hline
Luca Violato & Analista & Amministratore\\
\hline
\rowcolor{grigio}Marco Chilese & Analista & Verificatore\\
\hline
Bogdan Stanciu & Analista & Responsabile\\
\hline
\end{longtable}
\end{center}

\subsection{Punto 3}
Nella presente riunione, il gruppo ha discusso sulle tecnologie da utilizzare per lo sviluppo del progetto. 
\\
In particolare si è deciso di utilizzare come IDE\glossario il programma \textit{WebStorm}, un programma sviluppato da JetBrains ottimizzato per il linguaggio \textit{JavaScript}. 
\\
Inoltre il gruppo ha iniziato a discutere sull'utilizzo di strumenti per la continuous integration\glossario. 
Il gruppo ha valutato come possibili strumenti di integrazione Docker\glossario e Travis CI\glossario, nella prossima riunione il gruppo sceglierà lo strumento più adatto da utilizzare.

\subsection{Punto 4}
Il gruppo ha deciso di incontrarsi nuovamente in data 2019-02-04.