\section{Resoconto}

\subsection{Punto 1}
Al committente è stata richiesta la presentazione di alcuni plugin-in sviluppati o utilizzati dall'azienda. 

Al committente è stata richiesta la presentazione di alcuni plug-in  per grafana -TODO CHECK GRAFANA- sviluppati dall'azienda stessa, ma negativo poiché il committente utilizza solo plug-in da terze parti usando una politica di \textit{"riutilizzo del software"}. Dunque, l'azienda, ci ha presentato alcuni plug-in che utilizzano come indicatori di stato e soglie di allertamento, mettendo
in risalto la complessità di estrapolare informazioni mirate ed efficienti per capire il problema all'origine di tali indicatori, 
da cui è nata la necessità di un sistema più avanzato è \textit{"intelligente"} come quello richiesto nel capitolato. \\
Inoltre, oltre a mettere in risalto le varie difficoltà, ci sono stati mostrati i casi d'uso in cui il plug-in andrebbe adottato, evidenziato i principali punti di forza e le difficoltà di implementazione
richieste.\\ 
Per compensare le mancate conoscenze di grafana e delle reti bayesiane, ci sono state consigliate delle dispense a cui fare riferimento per aumentare le nostre conoscenze sull'argomento. 

\subsection{Punto 2}
Si sono concordati gli standard da utilizzare nell'implementazione del plug-in, quali:
\begin{itemize}
	\item La versione di \textbf{JavaScript} da utilizzare; 
	\item Lo standard \textbf{HTML} necessario; 
	\item La versione di \textbf{CSS} da rispettare; 
\end{itemize}
Tutti gli standard ci sono stati indicati nella documentazione di \textbf{Grafana}.

\subsection{Punto 3}
Abbiamo chiesto un campione di dati da analizzare e da utilizzare come \glossario{\textbf{mock}} per 
futuri test interni. Sfortunatamente l'azienda non poteva darci alcun tipo di campione, ma ci ha fornito 
una esaustiva presentazione del flusso di dati che deve essere analizzato dal plug-in.
Ciò è bastato per analizzare il carico da sopportare nel processo di elaborazione dei dati. 

\subsection{Punto 4}
L'azienda non ci fornice alcun tipo di ambiente di sviluppo, testing o deploy. Tutto questo deve essere 
gestito dal \texttt{gruppo}. Per i test ci è stato consigliato di creare dei \textbf{mock} appositi adibiti a tale scopo. 

\subsection{Punto 5}
Si sono inoltre discusse altre tecniche di implementazione, quali l'utilizzo di reti neurali come \textit{RNN}, 
le quali potrebbero offrire prestazioni migliori delle attuali reti bayesiane, ma poiché la complessità di 
strutturazione di tali reti neurali è molto elevata, tenendo conto del fatto che un minimo errore potrebbe 
portare alla perdita di dati sensibili, si e deciso di accantonare l'idea delle RNN, poiché l'utilizzo delle reti 
bayesiane era più che sufficiente per svolgere il lavoro richiesto con degli standard di qualità elevati. 
