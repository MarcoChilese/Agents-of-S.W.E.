\section{Resoconto}

\subsection{Punto 1}
Il nome scelto è stato deciso tramite un sondaggio vista la presenza di più possibilità ed indecisione su quale utilizzare. Il gruppo ha dunque deciso di identificarci con il nome \texttt{Agents of S.W.E.}.\\
La scelta del logo è avvenuta con lo stesso processo di scelta vista, nuovamente, la presenza di più possibilità e l'indecisione su quale utilizzare.

\subsection{Punto 2}
La discussione e l'analisi del capitoli proposti sono stati il punto principale della riunione. \\
Precedentemente si erano analizzati singolarmente i contenuti delle varie proposte e durante questo incontro si è effettuato un brainstorming andando ad analizzare, collettivamente, ogni singolo capitolato. Durante questa analisi si sono valutati i pro ed i contro di ogni offerta, le tecnologie da utilizzare, la fattibilità di ogni progetto ed, infine, le conoscenze pregresse relative ad ognuno di essi. Lo studio effettuato per i capitolati sarà poi formalizzato nel documento \textit{Studio di Fattibilità}. \\
Per ogni capitolato è stato assegnato un livello di difficoltà soggettivo, che andava ad unire i fattori analizzati precedentemente elencati.\\
Alla fine di questo incontro sono state espresse alcune preferenze, che si sono concluse con la scelta del capitolato  C3 - G\&B: monitoraggio intelligente di processi DevOps, offerto da Zucchetti. In seguito si sono piazzati, in ordine di preferenza, i capitolati C1 - Butterfly: monitor per processi CI/CD, offerto da Imola Informatica e C4  - MegAlexa: arricchitore di skill di Amazon Alexa, offerto dall'azienda Zero12. 
