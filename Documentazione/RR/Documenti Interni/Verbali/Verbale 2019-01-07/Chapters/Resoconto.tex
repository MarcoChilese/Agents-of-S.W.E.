\section{Resoconto}

\subsection{Punto 1}
Nella presente riunione il gruppo ha, come prima cosa, analizzato e valutato lo stato dei dei documenti: \textit{Piano di Progetto} e \textit{Piano di Qualifica}.\\
Per entrambi i documenti il gruppo ha valutato che questi necessitassero unicamente di un'ulteriore revisione a livello di sintassi, al fine di evitare refusi di ogni sorta prima dell'approvazione. I contenuti di tali documenti sono dunque stati considerati come definitivi e soddisfacenti da parte del team.\\

\subsection{Punto 2}
E' stato successivamente valutato separatamente lo stato di avanzamento dell'\textit{Analisi dei Requisiti}.\\
Dopo attenta analisi tale documento è stato giudicato come non completo, soprattutto per quanto riguarda la sezione "Requisiti". E' stato inoltre valutato come la parte riguardante i Casi d'Uso necessitasse di revisione e correzioni a livello di forma e contenuto, soprattutto per quanto concerne i diagrammi.\\ 
Nonostante ci;, il team non ha giudicato negativamente lo stato di avanzamento generale, valutando come accettabile il tempo necessario per l'ultimazione del documento in esame.

\subsection{Punto 3}
Facendo riferimento a quanto valutato nei due punti precedenti, il gruppo ha dunque deciso di assegnare i seguenti compiti ai vari membri del gruppo:
\begin{itemize}
	\item \textit{Piano di Progetto}: Ultima revisione e conseguente verifica, a cura di Marco Favaro;
	\item \textit{Piano di Qualifica}: Ultima revisione, a cura di Carlotta Segna e Diego Mazzalovo;
	\item \textit{Analisi dei Requisti}: Modifiche dei Casi d'Uso errati, a cura di Matteo Slanzi;
	\item \textit{Analisi dei Requisiti}: Correzione dei diagrammi errati, a cura di Bogdan Stanciu;
	\item \textit{Analisi dei Requisiti}: Stesura definitiva della sezione "Requisiti", a cura di Luca Violato, Marco Chilese e Bogdan Stanciu.
\end{itemize}

\subsection{Punto 4}
Il team ha deciso di incontrarsi nuovamente in data 2019-01-10.

