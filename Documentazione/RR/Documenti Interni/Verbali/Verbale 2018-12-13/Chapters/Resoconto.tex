\section{Resoconto}

\subsection{Punto 1}
Nella presente riunione ha discusso e analizzato le modifiche proposte dai revisori per ciascun documento in analisi: Piano di Progetto e Norme di Progetto.\\
Le modifiche proposte al documento delle Piano di Progetto riguardano principalmente:
\begin{itemize}
	\item Lingua e forma;
	\item Riformulazione codice sintetico dei vari rischi. Si è deciso di passare da un codice del tipo XYZ (es: 005) ad un codice più esplicativo ed immediato. Ad esempio, in ambito di Rischio tecnologico, il codice di ogni rischio sarà del tipo TXYZ (es. T005);
	\item Uniformazione struttura documento.
\end{itemize}
Le modifiche proposte al documento delle Norme di Progetto riguardano principalmente:
\begin{itemize}
	\item Lingua e forma;
	\item Uniformazione struttura documento.
\end{itemize}

\subsection{Punto 2}
Le modifiche discusse al punto precedente vengono quindi assegnate a:
\begin{itemize}
	\item Piano di Progetto: Carlotta Segna, Diego Mazzalovo;
	\item Norme di Progetto: Luca Violato, Bogdan Stanciu, Marco Favaro.
\end{itemize}

\subsection{Punto 3}
I ruoli all'interno del team ruotano nel seguente modo:\\
\begin{center}
\begin{tabular}{|c|c|c|}
\hline
\textbf{Membro} & \textbf{Vecchio Ruolo} & \textbf{Nuovo Ruolo}\\
\hline
Matteo Slanzi & Analista & Amministratore\\
\hline
Carlotta Segna & Verificatore & Responsabile\\
\hline
Marco Favaro & Analista & Verificatore\\
\hline
Diego Mazzalovo & Analista & Verificatore\\
\hline
Luca Violato & Amministratore & Analista\\
\hline
Marco Chilese & Verificatore & Analista\\
\hline
Bogdan Stanciu & Responsabile & Analista\\
\hline
\end{tabular}
\end{center}

\subsection{Punto 4}
La stesura dei nuovi documenti Piano di Qualifica e Analisi dei Requisiti viene di conseguenza assegnata a:
\begin{itemize}
	\item Piano di Qualifica: Marco Favaro, Diego Mazzalovo, Carlotta Segna;
	\item Analisi dei Requisiti: Luca Violato, Bogdan Stanciu, Marco Chilese, Matteo Slanzi.
\end{itemize}

\subsection{Punto 5}
Il team ha deciso di incontrarsi nuovamente in data:
Le modifiche discusse al punto precedente vengono quindi assegnate a:
\begin{itemize}
	\item 18/12/2018;
	\item 20/12/2018.
\end{itemize}