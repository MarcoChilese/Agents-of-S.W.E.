\subsection{Capitolato scelto: C3}

\subsubsection{Descrizione generale}
Il terzo capitolato propone di sviluppare un \textit{plugin} per la piattaforma, preesistente, \textit{Grafana} per la gestione dinamica di alert in situazioni di potenziale rischio all'interno di un contesto d'uso di macchine virtuali, e segnalazioni tra gli operatori del servizio Cloud e gli operatori della linea di produzione software.

\subsubsection{Finalit� del progetto}
In particolare si chiede di sviluppare un plugin per Grafana, che utilizzer� dati in input forniti, ad intervalli regolari o con continuit�, ad una \textit{rete bayesiana} per stimare la probabilit� di alcuni eventi, segnalandone quindi il rischio in modo dinamico, prevenendo situazioni di stallo.

\subsubsection{Tecnologie interessate}
\begin{itemize}
	\item \textbr{JavaScript}: linguaggio di scripting indicato per lo sviluppo del plugin;
	\item \textbr{JSON}: formato dati utilizzato per l'acquisizione dei dati;
	\item \textbr{Rete Bayesiana}: modello probabilistico utilizzato per stimare la probabilit� degli eventi di interesse;
	\item \textbr{jsbayes\footnote{\hyperref[Link al repository GitHub]{''https://github.com/vangj/jsbayes''}}}: libreria open-source consigliata dal fornitore per la gestione dei calcoli della rete Bayesiana;
	\item \textbr{HTML \& CSS}: linguaggi utilizzati per lo sviluppo del front-end del plugin.
\end{itemize}

\paragraph{Conclusioni}
Grazie alla tematica interessante, la possibilit� di contribuire con un plugin ad una piattaforma preesistente ampiamente utilizzata e la tematica relativa alle reti bayesiane, tema innovativo ed attuale, il team � portato a preferire il capitolato in oggetto. A favorire, inoltre, la scelta hanno contribuito la disponibilit� dell'azienda proponente e la chiarezza dei temi e dei requisiti esposti.\\
Le tecnologie coinvolte devono essere necessariamente approfondite dall'intero team, ciononostante sembrano  ampiamente affrontabili. 