\section{Analisi dei Rischi}
\label{RischiIntroduzione}

L'analisi dei rischi prevede la valutazione preventiva dei possibili problemi che possono verificarsi durante lo svolgimento del progetto. \\
I rischi sono catalogati in base a tipologie stabilite a priori all'interno del gruppo. 
Ogni rischio sarà inserito in una particolare categoria:
\begin{itemize}
\item Rischi relativi ai Componenti del Gruppo \texttt{Agents of S.W.E.} (G);
\item Rischi relativi alle Tecnologie da utilizzare(T);
\item Rischi relativi alla Gestione del Lavoro(L);
\item Rischi relativi ai Requisiti Richiesti(R).  
\end{itemize}

L'utilizzo dei numeri è incrementale e la suddivisione dei rischi in sottocategorie non interferisce con l'incremento numerico, in modo tale da avere una visione complessiva del numero di possibili rischi che possono intercorrere durante lo svolgimento del progetto. \\
Si e' scelto di applicare un'ulteriore suddivisione: in primo luogo è presente una tabella dove vengono descritti i possibili rischi, mentre la seconda spiega come il gruppo pensa di rilevare i rischi e successivamente come comportarsi in caso si verifichino.

\subsection{Identificazione}
\label{RischiIdentificazione}

\begin{longtable}{|C{.11\textwidth}|C{.47\textwidth}|C{.15\textwidth}|C{.15\textwidth}|}
\hline
\rowcolor{bluelogo}\textbf{\textcolor{white}{Rischio}} & \textbf{\textcolor{white}{Descrizione}} & \textbf{\textcolor{white}{Occorrenza}} & \textbf{\textcolor{white}{Pericolosità}}\\
\hline \hline
\endfirsthead
G01 &  Alcuni componenti del gruppo non si conoscono tra di loro. Questo potrebbe causare problematiche relative alla comunicazione intragruppo. & Media & Media \\
\hline
\rowcolor{grigio}G02 &  Alcuni membri del gruppo hanno impegni lavorativi e questo comporterà una presenza minore durante lo svolgimento delle componenti progettistiche.  & Bassa &  Alta\\
\hline
G03 &  Quasi nessuno dei membri del gruppo ha avuto precedenti esperienze lavorative in team. Questo potrebbe implicare problemi nello svolgimento delle attività e conseguenti ritardi.  & Alta & Alta\\
\hline
\rowcolor{grigio}T04 &  Certe tecnologie da usare sono sconosciute a tutti i membri del gruppo, mentre altre solo a certi membri. Ciò potrebbe causare ritardi nell'avanzamento del progetto. & Media & Alta \\
\hline
T05 &  L'utilizzo da parte del gruppo di strumentazione software di terzi per la gestione del progetto, comporta che, in caso di malfunzionamenti, potrebbero verificarsi ritardi o eventuali perdite di dati.  & Bassa & Alta\\
\hline
\rowcolor{grigio}T06 & I PC dei singoli membri del gruppo potrebbero guastarsi ed eventuali malfunzionamenti hardware potrebbero causare la perdita di dati, oltre a possibili ritardi nel ripristinare lo stato ottimale dei mezzi di lavoro.  & Bassa &  Media\\
\hline
T07 & La necessità
I membri del gruppo utilizzano PC con sistemi operativi diversi, il che comporta l'utilizzo di software di sviluppo compatibili con il sistema utilizzato.  & Bassa & Bassa\\
\hline
\rowcolor{grigio}T08 & Per l'apprendimento di certe tecnologie non conosciute al gruppo potrebbe essere richiesto più tempo rispetto a quello programmato.  & Media & Media\\
\hline
L09 & Nessun membro del gruppo ha precedenti esperienze nella pianificazione delle attività. & Alta & Alta \\
\hline
\rowcolor{grigio}L10 & La pianificazione prevede un costo per le attività. Tutti i membri del gruppo sono alla prima esperienza su un progetto simile, questo potrebbe portare a delle valutazioni errate sui costi complessivi del progetto. & Media & Media\\
\hline
R11 & Potrebbero venir rivalutati i requisiti da parte della proponente \textit{Zucchetti S.p.A.}\glossario o anche da parte del gruppo in caso di necessità. Ciò richiederebbe una revisione parziale o completa dell'Analisi dei requisiti, portando così ritardi nelle consegne. & Bassa & Alta \\
\hline
\rowcolor{grigio}R12 & L'analisi del capitolato scelto e la valutazione dei suoi requisiti possono essere fraintesi, questo può creare delle divergenze tra le aspettative della proponente e la visione del gruppo sul progetto. & Bassa & Bassa\\
\hline
\caption{Identificazione dei Rischi
\label{Tabella Identificazione dei Rischi}}
\end{longtable}

\subsection{Piano di Contenimento}
\label{RischiContenimento}

\begin{longtable}{|C{.11\textwidth}|C{.40\textwidth}|C{.40\textwidth}|}
\hline
\rowcolor{bluelogo}\textbf{\textcolor{white}{Rischio}} & \textbf{\textcolor{white}{Rilevamento}} & \textbf{\textcolor{white}{Risoluzione del Rischio}}\\
\hline \hline
\endfirsthead
G01 &  Il responsabile avrà il compito di controllare e risolvere eventuali diatribe che si potrebbero venire a creare. & La rotazione dei compiti avverrà in modo tale da far conoscere tutte le componenti del gruppo tra di loro, per trovare il corretto equilibrio in termini di efficienza ed efficacia. \\
\hline
\rowcolor{grigio}G02 &  Creazione di un calendario comune accedibile da tutti i membri del gruppo così da essere a conoscenza degli impegni altrui. & Le componenti del gruppo con impegni lavorativi suddivideranno i loro incarichi, in caso non siano in grado di portarli a termine nei tempi prestabiliti, con componenti che hanno maggior tempo a disposizione.\\
\hline
G03 &  Mancato rispetto delle milestones\glossario prestabilite.  & Analisi della milestone non rispettata, al fine di migliorare la gestione del tempo per le milestones successive.\\
\hline
\rowcolor{grigio}T04 &  Il responsabile avrà il compito di raccogliere le conoscenze dei singoli membri e monitorare la loro preparazione. & Coloro che conoscono determinate tecnologie sconosciute ad altri dovranno aiutare chi non le conosce. Per quelle sconosciute a tutti, lo studio verrà suddiviso in maniera equa e successivamente ogni membro condividerà ciò che ha appreso con gli altri. \\
\hline
T05 & Il problema non sarà facilmente riscontrabile, in quanto dipende da fattori esterni ed eventuali guasti non sono identificabili prima che si verifichino. & Per far fronte ad eventuali malfunzionamenti di Github\glossario et similia, ogni membro del gruppo provvederà ad effettuare dei backup ad intervalli regolari dei dati, in modo da poterli recuperare in caso di bisogno.\\
\hline
\rowcolor{grigio}T06 & In caso di problemi, l'interessato provvederà ad avvisare gli altri componenti del gruppo. &  Ogni membro del gruppo effettuerà dei backup settimanali dei dati, in modo da poter recuperare la maggior parte del lavoro in caso di guasti.\\
\hline
T07 & I membri forniranno all'amministratore il sistema operativo in uso. & Prima dell'effettivo inizio del progetto, l'amministratore provvederà a ricercare software disponibili in ognuno di questi sistemi.\\
\hline
\rowcolor{grigio}T08 & Sarà compito del responsabile valutare la complessità di tali tecnologie. & Per far fronte a ciò, qualora sia possibile, saranno richiesti incontri con la proponente per la spiegazione di eventuali tecnologie di difficile apprendimento.\\
\hline
L09 & Il responsabile terrà traccia degli effettivi tempi di sviluppo. & Verranno stabilite delle milestones, in modo che ad ognuna di esse possa essere valutato l'adempimento o meno dei compiti prestabiliti. In caso negativo, il lavoro arretrato verrà ridistribuito tra i membri del gruppo. Inoltre, sarà possibile ripianificare le attività fino alla prossima milestone con maggior precisione. \\
\hline
\rowcolor{grigio}L10 & Il responsabile avrà il compito di monitorare costantemente lo stato delle attività. & Il responsabile valuterà di volta in volta le attività in modo tale da far quadrare i costi prestabiliti nel preventivo.\\
\hline
R11 & Già dalle prime fasi, il gruppo cercherà di comunicare il più possibile con la proponente, in modo da far emergere ogni possibile necessità nella modifica dei requisiti. & Il gruppo cercherà di discutere apertamente con la proponente le modifiche sui requisiti, cercando di trovare un punto comune che soddisfi entrambe le parti. \\
\hline
\rowcolor{grigio}R12 & Durante l'attività di Analisi dei requisiti, verranno effettuati degli incontri con la proponente per ridurre al minimo possibili incomprensioni ed errori.  & Il gruppo esporrà negli incontri con la proponente, eventuali dubbi in modo da chiarire ogni requisito richiesto per il corretto sviluppo del progetto. Possibili errori dovranno essere corretti in seguito all'esito di ogni revisione.\\
\hline
\caption{Piani di Contenimento dei Rischi
\label{table:Contenimento Rischi}}
\end{longtable}
