\section{Consuntivo di Fine Periodo e Preventivo a Finire}
\label{CFPPAF}

Lo scopo di questa sezione è quello di stilare un rendiconto delle attività svolte fino ad ora, al fine di poter confrontare il preventivo stilato con l'effettivo numero di ore svolte da ogni membro del gruppo per ogni periodo.\\
Questa sezione verrà incrementata alla fine di ogni periodo.\\
Il conteggio verrà inserito in una tabella, così da rendere meglio visibili ore preventivate ed ore effettivamente svolte.\\
Le prime verranno rappresentate con un semplice numero mentre le seconde verranno rappresentate tra parentesi graffe, seguendo i criteri qui sotto riportati:
\begin{itemize}
	\item \textbf{Positivi}: sono state necessarie più ore di quelle preventivate;
	\item \textbf{Negativi}: sono state necessarie meno ore di quelle preventivate;
	\item \textbf{Invariato}: qualora le ore non dovessero cambiare, verrà semplicemente inserito il numero di ore preventivato.	
\end{itemize}

Inoltre, verranno anche aggiornati i dati preventivi andando a uniformarli con le ore effettivamente utilizzate durante la fase.\\
Questo ricalcolo è utile solamente al fine di poter verificare la qualità delle attività svolte dal gruppo e non andrà in alcun modo a variare i costi preventivati per la committente.

\newpage

\subsection{Consuntivo di Fine Periodo}
\label{CFP}
\subsubsection{Revisione dei Requisiti}

\paragraph{Prospetto orario} \-\\

\begin{longtable}{|C{.30\textwidth}|C{.06\textwidth}|C{.06\textwidth}|C{.06\textwidth} | C{.06\textwidth}| C{.06\textwidth} | C{.06\textwidth} | C{.10\textwidth} |}
\hline
\rowcolor{bluelogo}	\textbf{\textcolor{white}{Nome}} & \textbf{\textcolor{white}{RE}} & \textbf{\textcolor{white}{AM}} & \textbf{\textcolor{white}{AN}} & \textbf{\textcolor{white}{PJ}} & \textbf{\textcolor{white}{PR}} & \textbf{\textcolor{white}{VE}} & \textbf{\textcolor{white}{Totale}}\\
\hline 
Marco Chilese & - & - & 10 \{+5\} & - & - & 15 \{-4\} & 25 \{+1\} \\
\hline
\rowcolor{grigio}Marco Favaro & - & - & 14 \{-3\} & - & - & 11 \{+4\} & 25 \{+1\} \\
\hline
Diego Mazzalovo & - & - & 18 \{-5\} & - & - & 7 \{+6\} & 25 \{+1\} \\
\hline
\rowcolor{grigio}Carlotta Segna & 13 \{+3\} & - & - & - & - & 12 \{-2\} & 25 \{+1\} \\
\hline
Matteo Slanzi & - & 10 \{-3\} & 8 \{+4\} & - & - & 7 & 25 \{+1\} \\
\hline
\rowcolor{grigio}Bogdan Stanciu & 15 \{-3\} & - & 10 \{+4\} & - & - & - & 25 \{+1\}\\
\hline
Luca Violato & - & 15 \{+3\} & 10 \{-2\} & - & - & - & 25 \{+1\} \\
\hline


\caption{Consuntivo di Fine Periodo: Avvio ed Analisi dei Requisiti}
\label{tab:cfp_aar}
\end{longtable}

\paragraph{Prospetto economico} \-\\

\begin{longtable}{| C{.30\textwidth}| C{.15\textwidth}| C{.20\textwidth}|}
\hline
\rowcolor{bluelogo}\textbf{\textcolor{white}{Ruolo}} & \textbf{\textcolor{white}{Ore}} & \textbf{\textcolor{white}{Costo in \euro}} \\
\hline
Responsabile & 28 & \EUR{840.00} \\
\hline
\rowcolor{grigio}Amministratore & 25 & \EUR{500.00} \\
\hline
Analista & 73 \{+3\} & \EUR{1825.00} \{+\EUR{75.00}\}\\
\hline
\rowcolor{grigio}Progettista & - & - \\
\hline
Programmatore & - & - \\
\hline
\rowcolor{grigio}Verificatore & 56 \{+4\} & \EUR{840.00} \{+\EUR{60.00}\}\\
\hline
\textbf{Totale} & 182 \{+7\}& \EUR{4005.00} \{+\EUR{135.00}\}\\
\hline
\caption{Consuntivo di Fine Periodo dei ruoli: Avvio ed Analisi dei Requisiti}
\label{tab:doaar}
\end{longtable}

\paragraph{Conclusioni} \-\\
Il numero di ore preventivato differisce da quelle realmente utilizzate in quanto, approcciandosi ad un nuovo modo di svolgere un progetto, il gruppo non è stato in grado di preventivare correttamente il numero di ore che si sarebbero poi andate, effettivamente, ad utilizzare. Ciò è dovuto al fatto che siano state utilizzate più ore, rispetto a quanto preventivato, per la stesura del \textit{Piano di Progetto}, andando così ad aumentare le ore di \textit{Responsabile}. Un discorso equivalente può essere effettuato per il ruolo di \textit{Analista} , in quanto il documento \textit{Analisi dei Requisiti} ha richiesto un'analisi più lunga ed approfondita di quanto precedentemente preventivato. \\
Il \textit{Verificatore} ha necessitato più ore in quanto i documenti sono stati rivisti più volte, al fine di controllare le modifiche che venivano effettuate. \\
Alla fine di questo periodo il numero si è quindi verificato un incremento di 7 ore totali, andando ad aggiungere \EUR{135.00} al costo totale della prima fase, a carico del gruppo. 


