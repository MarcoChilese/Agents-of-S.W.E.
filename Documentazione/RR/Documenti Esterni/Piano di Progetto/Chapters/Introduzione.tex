\section{Introduzione}

\subsection{Scopo del Documento}
Il documento ha lo scopo di pianificare gli eventi che avranno luogo durante lo svolgimento del progetto \textit{G\&B: monitoraggio intelligente per i processi DevOps\glossario} per il gruppo \texttt{Agents of S.W.E.}.\\
Il documento presenterà, inoltre, un'analisi dei rischi e dei costi collegati allo svolgimento di esso. \\
Nello specifico esso comprende le seguenti sezioni:
\begin{itemize}
\item Breve analisi del modello di sviluppo per il progetto;
\item Analisi dei rischi che possono incorrere durante lo svolgimento del progetto;
\item Pianificazione approfondita dei tempi e delle attività;
\item Valutazione anticipata e ipotetica dell'uso delle risorse.
\end{itemize}
 
\subsection{Scopo del Prodotto} 
Lo scopo del prodotto è la creazione di un plug-in\glossario  per la piattaforma, preesistente,
Grafana\glossario per la gestione dinamica di alert in situazioni di potenziale rischio all’interno
di un contesto d’uso di macchine virtuali, e segnalazioni tra gli operatori del servizio
Cloud\glossario e gli operatori della linea di produzione software. In particolare, il plug-in
utilizzerà dati in input forniti ad intervalli regolari o con continuità, ad una rete
bayesiana\glossario per stimare la probabilità di alcuni eventi, segnalandone quindi il rischio
in modo dinamico, prevenendo situazioni di stallo.
 
\subsection{Ambiguità e Glossario}
Le parole ambigue all'interno del documento saranno contrassegnate tramite pedice rappresentante la lettera \textmd{G}, la quale rimanda al documento di nome Glossario nel quale saranno spiegati tutti i termini ambigui o che necessitano di specifiche.

\subsection{Riferimenti}

\newpage

\subsection{Scadenze}
Il gruppo \texttt{Agents of S.W.E.} si prepone di rispettare le seguenti scadenze temporali per la consegna degli incrementi di progetto e la consegna finale di esso:

\begin{longtable}{m{6cm} m{2cm} p{2cm}}
\hline

\rowcolor{bluelogo}\color{white}\textbf{Consegna} & \color{white}\textbf{Acronimo} & \color{white}\textbf{Data} \\
\hline

\rowcolor{beigechiaro}\color{black} \textbf{Revisione dei Requisiti} & \centering RR & 21/01/2019 \\
\hline
\rowcolor{beigescuro}\color{black} \textbf{Revisione di Progettazione} & \centering RP & 15/03/2019 \\
\hline
\rowcolor{beigechiaro}\color{black} \textbf{Revisione di Qualifica} & \centering RQ & 19/04/2019 \\
\hline
\rowcolor{beigescuro}\color{black} \textbf{Revisione di Accettazione} & \centering RA & 17/05/2019 \\
\hline
	  
\end{longtable}