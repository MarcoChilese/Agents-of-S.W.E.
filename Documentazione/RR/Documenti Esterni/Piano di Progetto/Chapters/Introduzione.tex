\section{Introduzione
\label{Introduzione}}

\subsection{Scopo del Documento}
Il documento ha lo scopo di pianificare gli eventi che avranno luogo durante lo svolgimento del progetto \textit{G\&B: monitoraggio intelligente per i processi DevOps\glossario} per il gruppo \texttt{Agents of S.W.E.}.\\
Il documento presenterà, inoltre, un'analisi dei rischi e dei costi collegati allo svolgimento di esso. \\
Nello specifico esso comprende le seguenti sezioni:
\begin{itemize}
\item Breve analisi del modello di sviluppo per il progetto;
\item Analisi dei rischi che possono incorrere durante lo svolgimento del progetto;
\item Pianificazione approfondita dei tempi e delle attività;
\item Valutazione anticipata e ipotetica dell'uso delle risorse.
\end{itemize}
 
\subsection{Scopo del Prodotto} 
Lo scopo del prodotto è la creazione di un plug-in per la piattaforma open source\glossario di visualizzazione e gestione dati, denominata Grafana, 
con l'obiettivo di creare un sistema di alert dinamico per monitorare la "liveliness"\glossario del sistema a supporto dei processi
devops\glossario e per consigliare interventi nel sistema di produzione del software.
In particolare, il plug-in utilizzerà dati in input forniti ad intervalli regolari o con continuità, ad una rete bayesiana\glossario per stimare la probabilità di alcuni eventi, segnalandone quindi il rischio in modo dinamico, prevenendo situazioni di stallo.   

 
\subsection{Ambiguità e Glossario}
I termini che potrebbero risultare ambigui all'interno del documento sono siglati tramite pedice rappresentante la lettera \textmd{G}, tale terminologia trova una sua più specifica definizione nel \textit{Glossario v1.0.0} che viene fornito tra i Documenti Esterni.


\subsection{Riferimenti}
\subsubsection{Riferimenti Normativi}
\begin{itemize}
	\item \textbf{Regolamento organigramma e specifica tecnico-economica}:\-\\ \url{www.math.unipd.it/~tullio/IS-1/2018/Progetto/RO.html};
	\item \textbf{\textit{Norme di Progetto v1.0.0}}:
		\begin{itemize}
			\item §3.1 "Documentazione";
			\item §4.2 "Ruoli di Progetto".
		\end{itemize}
	\item \textbf{Capitolato d'Appalto: C3 - G\&B}:\-\\ \url{https://www.math.unipd.it/~tullio/IS-1/2018/Progetto/C3.pdf}.
\end{itemize}

\subsubsection{Riferimenti Informativi}
\begin{itemize}
	\item \textbf{Slide del Corso "Ingegneria del Software" - Gestione di progetto:}\-\\ \url{www.math.unipd.it/~tullio/IS-1/2018/Dispense/L06.pdf}
	\begin{itemize}
		\item "Pianificazione di Progetto", slides 10-11; 
		\item "Allocazione delle Risorse", slides 17-18;
		\item "Stima dei Costi di Progetto", slide 19;
		\item "Piano di Progetto", slides 25-26;
		\item "Gestione dei Rischi", slides 29-30.
	\end{itemize}
	\item \textbf{Slide del Corso "Ingegneria del Software" - Regole del progetto didattico:}\-\\ \url{www.math.unipd.it/~tullio/IS-1/2018/Dispense/P01.pdf}
		\begin{itemize}
			\item "Scadenziario del Progetto", slide 9;
			\item "\textit{Technology Baseline}", slide 12;
			\item "\textit{Product Baseline}", slide 13;
			\item "Ciclo Revisioni", slides 17, 18, 19, 20. 
		\end{itemize}
	\item \textbf{Ian Sommerville - Software Engineering};
	\item \textbf{ISO/IEC 12207:1995}\footnote{\url{en.wikipedia.org/wiki/ISO/IEC_12207}}.
\end{itemize}




\newpage

\subsection{Scadenze}
Il gruppo \texttt{Agents of S.W.E.} si prepone di rispettare le seguenti scadenze temporali per la consegna degli incrementi di progetto e la consegna finale di esso:


\begin{longtable}{|C{.40\textwidth}|C{.15\textwidth}|C{.20\textwidth}|}
\hline
\rowcolor{bluelogo}\textbf{\textcolor{white}{Consegna}} & \textbf{\textcolor{white}{Acronimo}} & \textbf{\textcolor{white}{Data}}\\
\hline \hline
\endfirsthead

\textbf{Revisione dei Requisiti} & RR & 2019-01-21 \\
\hline
\rowcolor{grigio}\textbf{Revisione di Progettazione} & RP & 2019-03-15 \\
\hline
\textbf{Revisione di Qualifica} & RQ & 2019-04-19 \\
\hline
\rowcolor{grigio}\textbf{Revisione di Accettazione} & RA & 2019-05-17 \\
\hline
\caption{Scadenze delle Consegne \label{table:Scadenze}}
\end{longtable}