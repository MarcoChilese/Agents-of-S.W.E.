\section{Pianificazione}

Basandosi sulle scadenze esposte nel capitolo §1.5 si è deciso di sviluppare il progetto suddividendolo sulla base dello standard ISO/IEC 12207:1995, che prevede le seguenti fasi:
\begin{itemize}
	\item Attività preliminari di avvio ed analisi dei requisiti;
	\item Progettazione architetturale;
	\item Progettazione di dettaglio e codifica;
	\item Validazione e collaudo.
\end{itemize}

\begin{longtable}{ m{8cm} m{3cm} p{3cm} }
\hline

\rowcolor{bluelogo}\color{white}\textbf{Fase} & \color{white}\textbf{Inizio} & \color{white}\textbf{Fine} \\
\hline
\rowcolor{beigechiaro} \color{black} Avvio ed analisi requisiti & 15/11/2018 & 14/01/2019 \\
\rowcolor{beigescuro} \color{black} Progettazione architetturale & 22/01/2019 & 08/03/2019 \\
\rowcolor{beigechiaro} \color{black} Progettazione di dettaglio e codifica & 16/03/2019 & 12/04/2019 \\
\rowcolor{beigescuro} \color{black} Validazione e collaudo & 20/04/2019 & 10/05/2019 \\

\hline
\end{longtable}

La prima fase sarà a carico del gruppo \texttt{Agents of S.W.E.} mentre le ulteriori tre fasi saranno a carico del committente.

\subsection{Attività preliminari di avvio ed analisi dei requisiti}

Il periodo di analisi va dal 15/11/2018, data di formazione dei gruppi, e termina il 14/01/2019 con la consegna della documentazione relativa alla RR.

\subsubsection{Incrementi}

Il primo periodo prevede 6 incrementi e le principali attività svolte sono: 
\begin{itemize}
	\item \textbf{Analisi dei Requisiti}: all'interno del documento \textit{Analisi dei Requisiti} vengono inseriti tutti i requisiti individuati dagli Analisti, analizzando il capitolato d'appalto. Questa risulta essere un'attività particolarmente importante poiché l'errata analisi comporterebbe un impedimento nell'avanzamento del progetto.
	\item \textbf{Glossario}: il documento \textit{Glossario} racchiuderà tutti i termini ambigui o poco chiari che vengono individuati durante la redazione dei documenti;
	\item \textit{Lettera di Presentazione}: l'attività prevede la stesura della \textit{Lettera di Presentazione} dichiarando il gruppo \texttt{Agents of S.W.E.} come fornitore;
	\item \textbf{Norme di Progetto}: tutte le norme che vengono stabilite saranno inserite all'interno del documento \textit{Norme di progetto} individuate dall'Amministratore. Ha lo scopo di uniformare le modalità di lavoro che dovranno essere attuate da tutti i membri del gruppo. Consiste in un'attività critica in quanto fondamentali per la stesura della documentazione;
	\item \textbf{Piano di Progetto}: è compito del Responsabile analizzare attività e scadenze al fine di ottenere una buona riuscita del progetto ed è compito dell'Amministratore analizzare i rischi nei quali si può incorrere. Le attività e le risorse vengono suddivise per l'intera durata del progetto ed inserite all'interno del documento \textit{Piano di Progetto}, necessario e vincolante per la stesura della \textit{Lettera di presentazione};
	\item \textbf{Piano di Qualifica}: i Progettisti avranno il compito di cercare un elenco di attività e metodi utili al fine di garantire una buona qualità di prodotto. Questi verranno racchiusi all'interno del documento \textit{Piano di Qualifica};
	\item \textbf{Studio di fattibilità}: consiste nell'analisi preliminare dei vari capitolati proposti ed è essenziale alla fine della scelta del capitolato da svolgere. L'analisi verrà inserita all'interno del documento \textit{Studio di fattibilità}; questa risulta essere un'attività bloccante per l'inizio dell'attività di Analisi dei Requisiti.  
\end{itemize}