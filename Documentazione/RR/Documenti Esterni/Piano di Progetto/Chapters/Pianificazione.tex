\section{Pianificazione}

Basandosi sulle scadenze esposte nel capitolo §1.5 si è deciso di sviluppare il progetto suddividendolo sulla base dello standard ISO/IEC 12207:1995, che prevede le seguenti fasi:
\begin{itemize}
	\item Attività preliminari di avvio ed analisi dei requisiti;
	\item Progettazione architetturale;
	\item Progettazione di dettaglio e codifica;
	\item Validazione e collaudo.
\end{itemize}

\begin{longtable}{ m{8cm} m{3cm} p{3cm} }
\hline

\rowcolor{bluelogo}\color{white}\textbf{Fase} & \color{white}\textbf{Inizio} & \color{white}\textbf{Fine} \\
\hline
\rowcolor{beigechiaro} \color{black} Avvio ed analisi requisiti & 15/11/2018 & 14/01/2019 \\
\rowcolor{beigescuro} \color{black} Risanamento criticità & 22/01/2019 & 28/01/2019 \\
\rowcolor{beigechiaro} \color{black} Progettazione architetturale & 29/01/2019 & 08/03/2019 \\
\rowcolor{beigescuro} \color{black} Risanamento criticità & 16/03/2019 & 19/03/2019 \\
\rowcolor{beigechiaro} \color{black} Progettazione di dettaglio e codifica & 20/03/2019 & 12/04/2019 \\
\rowcolor{beigescuro} \color{black} Validazione e collaudo & 20/04/2019 & 10/05/2019 \\

\hline
\end{longtable}

La prima fase sarà a carico del gruppo \texttt{Agents of S.W.E.} mentre le ulteriori tre fasi saranno a carico del committente. \\
Nonostante la scelta di adottare lo standard ISO/IEC, abbiamo deciso di aggiungere un incremento tra le prime due fasi che consiste in un periodo di sanazione delle criticità trovate dopo un'attenta analisi della documentazione, che avviene non solo da parte del gruppo \texttt{Agents of S.W.E.}, ma anche da parte della proponente. 

\subsection{Attività preliminari di avvio ed analisi dei requisiti}

Il periodo di analisi va dal 15/11/2018, data di formazione dei gruppi, e termina il 14/01/2019 con la consegna della documentazione relativa alla RR.

\subsubsection{Incrementi}

Il primo periodo prevede 6 incrementi e le principali operazioni svolte sono: 
\begin{itemize}
	\item \textbf{Analisi dei Requisiti}: all'interno del documento \textit{Analisi dei Requisiti} vengono inseriti tutti i requisiti individuati dagli Analisti, analizzando il capitolato d'appalto. Questa risulta essere un'attività particolarmente importante poiché l'errata analisi comporterebbe un impedimento nell'avanzamento del progetto.
	\item \textbf{Glossario}: il documento \textit{Glossario} racchiuderà tutti i termini ambigui o poco chiari che vengono individuati durante la redazione dei documenti;
	\item \textbf{Lettera di Presentazione}: l'attività prevede la stesura della \textit{Lettera di Presentazione} dichiarando il gruppo \texttt{Agents of S.W.E.} come fornitore;
	\item \textbf{Norme di Progetto}: tutte le norme che vengono stabilite saranno inserite all'interno del documento \textit{Norme di progetto} individuate dall'Amministratore. Ha lo scopo di uniformare le modalità di lavoro che dovranno essere attuate da tutti i membri del gruppo. Consiste in un'attività critica in quanto fondamentali per la stesura della documentazione;
	\item \textbf{Piano di Progetto}: è compito del Responsabile analizzare attività e scadenze al fine di ottenere una buona riuscita del progetto ed è compito dell'Amministratore analizzare i rischi nei quali si può incorrere. Le attività e le risorse vengono suddivise per l'intera durata del progetto ed inserite all'interno del documento \textit{Piano di Progetto}, necessario e vincolante per la stesura della \textit{Lettera di presentazione};
	\item \textbf{Piano di Qualifica}: i Progettisti avranno il compito di cercare un elenco di attività e metodi utili al fine di garantire una buona qualità di prodotto. Questi verranno racchiusi all'interno del documento \textit{Piano di Qualifica};
	\item \textbf{Studio di fattibilità}: consiste nell'analisi preliminare dei vari capitolati proposti ed è essenziale alla fine della scelta del capitolato da svolgere. L'analisi verrà inserita all'interno del documento \textit{Studio di fattibilità}; questa risulta essere un'attività bloccante per l'inizio dell'attività di Analisi dei Requisiti.  
\end{itemize}

\subsection{Risanamento criticità}

Con \textit{Risanamento delle criticità}, fase all'infuori dello standard ISO, intendiamo una periodo da noi programmato in cui andremo ad analizzare le varie criticità che sono emerse alla fine del primo incremento, dopo la sua analisi sia da parte della proponente che da parte del gruppo. \\
La scelta di inserire questo tra la prima e la seconda fase è guidata dal fatto che per procedere è necessario aver risolto tutti i problemi riscontrati nelle produzioni precedenti, in particolar modo la seconda fase, individuata dallo standard, prevede la \textit{Progettazione architetturale} ed è quindi fondamentale sanare i problemi sorti nel primo periodo. 

\subsection{Progettazione architetturale}

Il periodo di \textit{Progettazione architetturale} inizia dalla fine del periodo di \textit{Risanamento criticità} e termina con la consegna del nuovo incremento (quindi dal 29/01/2018 al 08/03/2018), il quale prevede le operazioni riportate nella sottosezione seguente.

\subsubsection{Incrementi}
\begin{itemize}
	\item \textbf{Incremento e verifica}: all'inizio del periodo vengono svolte attività di incremento e verifica su vari documenti (\textit{Norme di progetto, Piano di progetto, Piano di qualifica});
	\item \textbf{Studio tecnologie}: prevede un continuo approfondimento delle tecnologie necessarie allo svolgimento del progetto; 
	\item \textbf{Technology Baseline}\glossario: questa attività prevede l'analisi e la scelta di tecnologie, framework\glossario e librerie\glossario da utilizzare. In questa fase, inoltre, è previsto lo sviluppo del \textit{Proof of concept}\glossario;
	\item \textbf{Glossario}: aggiunta e modifica dei termini in itinere;  
	\item \textbf{Verifica}: cinque giorni prima della fine del periodo sarà compito dei Verificatori analizzare i risultati della seconda fase segnalando, a chi di dovere, gli errori o le imprecisioni riscontrate.
\end{itemize}

\subsection{Risanamento criticità}
Con \textit{Risanamento delle criticità}, fase all'infuori dello standard ISO, intendiamo una periodo da noi programmato in cui andremo ad analizzare le varie criticità che sono emerse alla fine del primo incremento, dopo la sua analisi sia da parte della proponente che da parte del gruppo. \\

\subsection{Progettazione di dettaglio e codifica}
Il periodo di \textit{Progettazione di dettaglio e codifica} va dal giorno dopo la fine del periodo di \textit{Risanamento delle criticità}, cioè il 20/03/2019, e termina con la consegna dei documenti per la RQ, cioè il 12/04/2019.\\

\subsubsection{Incrementi}
Gli incrementi che si andranno ad attuare durante questo periodo sono:
\begin{itemize}
	\item \textbf{Incremento e verifica}: \item \textbf{Incremento e verifica}: all'inizio del periodo vengono svolte attività di incremento e verifica su vari documenti (\textit{Norme di progetto, Piano di progetto, Piano di qualifica e Technology Baseline});
	\item \textbf{Glossario}: prevede l'aggiunta di nuovi termini al \textit{Glossario} ed il suo miglioramento;
	\item \textbf{Product Baseline}\glossario: presenta la baseline\glossario architetturale del prodotto, coerente rispetto a quando riportato nella \textit{Technoly Baseline}. Al suo interno contiene i diagrammi delle classi e di sequenza, la contestualizzazione dei design pattern adottati nell'architettura del prodotto. 
	\item \textbf{Codifica}: prevede la scrittura del codice e relativa verifica\glossario di esso;
	\item \textbf{Manuale utente}: consiste nella redazione del \textit{Manuale utente}, contente le indicazioni d'utilizzo del prodotto;
	\item \textbf{Lettera di Presentazione}: prevede la stesura della \textit{Lettera di presentazione} per la partecipazione alla RQ.
\end{itemize}

\subsection{Validazione e collaudo}
Il periodi di \textit{Validazione e collaudo} inizia il 20/04/2018 e termina il 10/05/2018 con la consegna dei documenti per la RA. 

\subsubsection{Incrementi}
Durante questo periodo saranno svolti i seguenti incrementi:
\begin{itemize}
	\item \textbf{Incremento e Verifica}: \item \textbf{Incremento e verifica}: all'inizio del periodo vengono svolte attività di incremento e verifica su vari documenti (\textit{Norme di progetto, Piano di progetto, Piano di qualifica e Technology Baseline});
	\item \textbf{Glossario}: prevede l'aggiunta di nuovi termini al \textit{Glossario} ed il suo miglioramento;
	\item \textbf{Validazione e collaudo}: prevede lo sviluppo ultimo del prodotto, inserendovi miglioramenti e svolgendo ulteriori test al fine di assicurare e completare il completo soddisfacimento dei requisiti;
	\item \textbf{Manuale utente}: prevede il miglioramento e completamento del \textit{Manuale utente}, contente le indicazioni di utilizzo del prodotto.
\end{itemize}


