\section{Resoconto}

\subsection{Punto 1}

Al committente è stata richiesta la presentazione di alcuni plug-in  per \textit{Grafana} sviluppati dall'azienda stessa, ma si è avuto un riscontro negativo poiché il committente utilizza solo plug-in da terze parti usando una politica di \textit{"riutilizzo del software"}. Dunque, l'azienda, ci ha presentato alcuni plug-in che utilizzano come indicatori di stato e soglie di allertamento, mettendo
in risalto la complessità di estrapolare informazioni mirate ed efficienti per capire il problema all'origine di tali indicatori, 
da cui è nata la necessità di un sistema più avanzato e \textit{"intelligente"} come quello richiesto nel capitolato. \\
Inoltre, oltre a mettere in risalto le varie difficoltà, ci sono stati mostrati i casi d'uso in cui il plug-in andrebbe adottato, evidenziato i principali punti di forza e le difficoltà di implementazione
richieste.\\ 
Per compensare le mancate conoscenze di \textit{Grafana} e delle reti bayesiane, ci sono state consigliate delle dispense a cui fare riferimento. 

\subsection{Punto 2}
Si sono concordati gli standard da utilizzare nell'implementazione del plug-in, quali:
\begin{itemize}
	\item La versione di \textit{JavaScript} da utilizzare; 
	\item Lo standard \textit{HTML} necessario; 
	\item La versione di \textit{CSS} da rispettare. 
\end{itemize}
Tutti gli standard sono indicati nella documentazione di \textit{Grafana}.

\subsection{Punto 3}
Abbiamo chiesto un campione di dati da analizzare e da utilizzare come mock per futuri test interni. Sfortunatamente l'azienda non poteva darci alcun tipo di campione, ma ci ha fornito 
un'esaustiva presentazione del flusso di dati che deve essere analizzato dal plug-in.
Ciò è bastato per analizzare il carico da sopportare nel processo di elaborazione dei dati. 

\subsection{Punto 4}
L'azienda non ci fornice alcun tipo di ambiente di sviluppo, testing o deploy. Tutto questo deve essere 
gestito dal gruppo. Per i test ci è stato consigliato di creare dei mock appositi adibiti a tale scopo. 

\subsection{Punto 5}
Si sono inoltre discusse altre tecniche di implementazione, quali l'utilizzo di reti neurali come \textit{RNN}, 
le quali potrebbero offrire prestazioni migliori delle attuali reti bayesiane, ma poiché la complessità di 
strutturazione di tali reti neurali è molto elevata, tenendo conto del fatto che un minimo errore potrebbe 
portare alla perdita di dati sensibili, si è deciso di accantonare l'idea delle \textit{RNN}, poiché l'utilizzo delle reti 
bayesiane risulta essere più che sufficiente per svolgere il lavoro richiesto con degli standard di qualità elevati. 
