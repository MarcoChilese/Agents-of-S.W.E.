\section{Qualità di Prodotto}
\label{qualitaProdotto}

\subsection{Scopo}

Nello standard ISO/IEC 9126:2001, il gruppo \texttt{Agents of S.W.E.} ha individuato i principali obiettivi da dover perseguire per garantire un'ottima qualità di prodotto.  

\subsection{Qualità dei Documenti}

I documenti dovranno rispettare i pilastri della scrittura che prevedono la leggibilità e la comprensibilità del documento, le quali derivano dalla correttezza grammaticale, ortografica, logica e semantica.

\subsubsection{Comprensione}

\begin{itemize}
	\item \textbf{Leggibilità}: vista la natura molto tecnica dei documenti prodotti, essi verranno considerati leggibili se comprensibili da persone con licenza di istruzione superiore;
	\item \textbf{Correzione ortografica}: i documenti non conterranno errori ortografici;
	\item \textbf{Correttezza logica e semantica}: non essendo disponibili sistemi automatici al fine di controllare 
la correttezza logica e semantica, la comprensione totale del prodotto letto identificherà anche tale correttezza, in quanto un documento viene considerato leggibile solamente se è corretto.
\end{itemize}

\subsection{Qualità del Software}

Seguendo lo standard ISO/IEC 9126:2001, il gruppo \texttt{Agents of S.W.E.} ha deciso di perseguire i seguenti obiettivi di qualità del prodotto software finale:

\subsubsection{Funzionalità}

Con \textit{Funzionalità} si intendono le qualità riguardanti le funzioni offerte dal software.
\begin{itemize}
	\item \textbf{Appropriatezza}: le funzioni offerte devono essere in grado di ricoprire tutte le funzionalità proposte inizialmente all'utente;
	\item \textbf{Accuratezza}: il prodotto finale sarà in grado di svolgere tutti i compiti richiesti;
	\item \textbf{Interoperabilità}: il software deve essere in grado di eseguire su diversi sistemi;
	\item \textbf{Sicurezza}: i dati sensibili utilizzati dal prodotto devono essere disponibili solo agli utenti che li hanno generati o a chi da loro richiesto. 
\end{itemize}

\subsubsection{Affidabilità}

Con \textit{Affidabilità} si intende la garanzia di funzionamento del software sotto determinate condizioni d'uso. 

\begin{itemize}
	\item \textbf{Maturità}: il prodotto deve essere sviluppato in modo da evitare l'insorgere di failure\glossario derivati dalla sua esecuzione;
	\item \textbf{Tolleranza agli Errori}: anche in presenza di errori o usi impropri, il software deve comunque garantire determinate prestazioni;
	\item \textbf{Recuperabilità}: al verificarsi di un malfunzionamento, il software deve essere in grado di ripristinare uno stato funzionante del sistema in un tempo ragionevole e recuperando i dati persi;
\end{itemize}

\subsubsection{Efficienza}

Con \textit{Efficienza} si intendono le prestazioni raggiungibili sotto specifiche condizioni di utilizzo. 

\begin{itemize}
	\item \textbf{Comportamento rispetto al Tempo}: il software deve garantire determinati tempi di risposta, elaborazione e velocità di attraversamento;
	\item \textbf{Utilizzo di Risorse}: uso non eccessivo di risorse;
\end{itemize}

\subsubsection{Usabilità}

Con \textit{Usabilità} si intende il livello di comprensione del prodotto da parte dell'utilizzatore.

\begin{itemize}
	\item \textbf{Comprensibilità}: la facilità di comprensione delle funzionalità offerte dal prodotto, atta a fungere da spiegazione per l'utente che desideri utilizzarlo. 
	\item \textbf{Apprendibilità}: livello di impegno richiesto dall'utente per imparare ad utilizzare il software;
	\item \textbf{Operabilità}: capacità del software di mettere l'utente in condizione di utilizzarlo per i suoi scopi; 
	\item \textbf{Attrattività}: il software deve essere di piacevole utilizzo da parte dell'utente.
\end{itemize}

\subsubsection{Manutenibilità}

Con \textit{Manutenibilità} si intende il livello di semplicità richiesto al fine di eseguire interventi di modifica, correzione o adattamento.

\begin{itemize}
	\item \textbf{Analizzabilità}: facilità di lettura del codice per localizzare errori al suo interno; 
	\item \textbf{Modificabilità}: facilità nella modifica delle componenti del software;
	\item \textbf{Stabilità}: il software deve garantire il corretto funzionamento anche a fronte di modifiche errate;
	\item \textbf{Testabilità}: il codice deve essere sviluppato in maniera tale da garantire facilità in creazione ed esecuzione dei test.
\end{itemize}

\subsubsection{Portabilità}

Con \textit{Portabilità} si intende la capacità del software di funzionare in diversi sistemi, che siano essi software o hardware. 

\begin{itemize}
	\item \textbf{Adattabilità}: capacità del software di funzionare su sistemi diversi senza dover implementare nuove funzionalità, oltre a quelle già fornite;
	\item \textbf{Installabilità}: possibilità di installare il software in specifici ambienti;
	\item \textbf{Sostituibilità}: capacità del software di essere utilizzato al posto di un altro software per lo svolgimento dei medesimi compiti nel medesimo ambiente.
\end{itemize}























































