\section{Introduzione}
\label{introduzione}

\subsection{Scopo del Documento}

Lo scopo del documento \textit{Piano di Qualifica v1.0.0} è quello di stabilire gli obbiettivi metrici da dover rispettare nello sviluppo di processi e prodotti sviluppati dal gruppo \texttt{Agents of S.W.E.} per la verifica\glossario e validazione\glossario di essi. In particolare verranno seguite le norme descritte nel documento \textit{Norme di Progetto v1.0.0}. Sarà compito dei verificatori del gruppo provvedere ad una continua verifica dei processi e dei prodotti in modo da ottenere incrementi parziali, fino ad arrivare alla realizzazione completa del progetto, senza l'inserimento di errori che possano compromettere il risultato finale. 

\subsection{Scopo del Prodotto}
Lo scopo del prodotto è la creazione di un plug-in\glossario  per la piattaforma, preesistente, Grafana\glossario per la gestione dinamica di alert in situazioni di potenziale rischio all’interno di un contesto d’uso di macchine virtuali e segnalazioni tra gli operatori del servizio Cloud\glossario e gli operatori della linea di produzione software. In particolare, il plug-in utilizzerà dati in input forniti ad intervalli regolari o con continuità, ad una rete bayesiana\glossario per stimare la probabilità di alcuni eventi, segnalandone quindi il rischio in modo dinamico, prevenendo situazioni di stallo.   

\subsection{Ambiguità e Glossario}
I termini che potrebbero risultare ambigui all'interno del documento sono siglati tramite pedice rappresentante la lettera \textmd{G}, tale terminologia trova una sua più specifica definizione nel \textit{Glossario v1.0.0} che viene fornito tra i Documenti Esterni.

\subsection{Riferimenti}
\subsubsection{Riferimenti Normativi}

	\begin{itemize}
		\item \textbf{Norma di Progetto v1.0.0};
		\item 
	\end{itemize}

