\documentclass[12pt]{article}

\usepackage[utf8]{inputenc}	%Encoding UTF8 per accenti

\usepackage[T1]{fontenc}

\usepackage[italian]{babel}

\usepackage[onehalfspacing]{setspace}	%Interlinea di 1.5

\usepackage[hyperfootnotes=false]{hyperref}	%Pacchetto per coll. ipertestuali

\usepackage{tabularx}
\usepackage{longtable} %Per Tabelle potenzialmente multipagina
\usepackage{fancyhdr}  % per gli header e footer
\usepackage{graphicx}  %per le immagini
\usepackage{flafter}
\usepackage[font=small,labelfont=bf]{caption}
\usepackage[dvipsnames, table]{xcolor}

\usepackage{framed}

\usepackage[headheight=2cm, headsep=0.5cm, a4paper, margin=3cm]{geometry}
\usepackage[bottom]{footmisc}

\usepackage{tcolorbox} %Per riquadrare il testo in un box
\usepackage{tabularx} 

\usepackage{listings} %per i listati di codice
%define Javascript language
\lstdefinelanguage{JavaScript}{
  keywords={typeof, new, true, false, catch, function, return, null, catch, switch, var, if, in, while, do, else, case, break},
  keywordstyle=\color{blue}\bfseries,
  ndkeywords={class, export, boolean, throw, implements, import, this},
  ndkeywordstyle=\color{darkgray}\bfseries,
  identifierstyle=\color{black},
  sensitive=false,
  comment=[l]{//},
  morecomment=[s]{/*}{*/},
  commentstyle=\color{purple}\ttfamily,
  stringstyle=\color{red}\ttfamily,
  morestring=[b]',
  morestring=[b]"
}

\lstset{
basicstyle=\footnotesize,
numbers=left, 
numberstyle=\small, 
numbersep=8pt, 
frame = single, 
language=JavaScript, 
framexleftmargin=15pt}



\hypersetup{colorlinks=true}		%Configurazione colore link INTERNI documento
\hypersetup{linkcolor= blue}
\renewcommand\UrlFont{\color{blue}\rmfamily\itshape} %Forzato colore blu su tutti i link ESTERNI

\lhead{\includegraphics[scale=0.08]{./images/logo.png}}

\renewcommand{\footrulewidth}{0.4pt}
\newcommand{\glossario}{\textsubscript{G} }

\definecolor{bluelogo}{HTML}{415A66}
\definecolor{grigio}{HTML}{D0D0D0}

\usepackage{makecell}

\pagestyle{fancy}

\makeindex
\setcounter{secnumdepth}{4}
\setcounter{tocdepth}{4}

\usepackage{float} %Per le immagini

% \lhead{\includegraphics[scale=]{}} TODO LOGO ripetuto angolo Top SX
