\section{Casi d'Uso}\label{CasiUso}
\subsection{Introduzione}\label{CasiUso_Introduzione}
Nella seguente sezione verranno identificati i casi d'uso individuati dal team \texttt{Agents of S.W.E.}.\\
Il numero di casi analizzati è limitato poiché il plug-in fornisce funzionalità aggiuntive ad una piattaforma preesistente, per la quale non è fornita documentazione in quanto già disponibile presso il sito web del fornitore della piattaforma: \href{http://docs.grafana.org/}{\textit{Grafana Labs}}.\\

\subsection{Attori}
\subsubsection{Attori primari}


\subsubsection{Attori secondari}

\subsection{Elenco dei casi d'uso}
\subsubsection{UC1 - Aggiunta della rete bayesiana al plug-in G\&B}\label{UC1}
\begin{itemize}
	\item \textbf{Attore primario}: Utente;
	\item \textbf{Precondizioni}:
		\begin{enumerate}
			\item L'utente deve aver effettuato il login nella piattaforma Grafana, deve aver selezionato una Dashboard e aggiunto il pannello "G\&B Panel".
		\end{enumerate}
	\item \textbf{Postcondizioni}: 
	\begin{enumerate}
		\item L'utente ha aggiunto la rete bayesiana al plug-in. Attraverso UC2 (§\ref{UC2}) può selezionare quali nodi sorgente collegare alla rete.
	\end{enumerate}
	\item \textbf{Scenario principale:}
	\begin{enumerate}
		\item L'utente accede alla piattaforma Grafana, si trova nella dashboard preferita ed ha aggiunto il pannello "G\&B Panel";
		\item L'utente seleziona e clicca sul bottone con simbolo di "+";
		\item L'utente si trova davanti una finestra presso cui selezionare il file JSON contenente la rete (UC1.1, §\ref{UC1.1}) e seleziona "Aggiungi".
	\end{enumerate}
	\item \textbf{Estensioni}:
	\begin{itemize}
		\item \hyperref[UC6]{UC6 (\ref*{UC6})} La mappa selezionata non è corretta per estensione o per contenuto;
		\begin{enumerate}
			\item L'aggiunta della rete al plug-in fallisce;
			\item Viene visualizzato un messaggio di errore esplicito che spieghi l'errore;
			\item Viene fornita all'utente un'altra possibilità per selezionare il file corretto.
		\end{enumerate}
	\end{itemize}
\end{itemize}

\subsubsection{UC1.1 - Selezione della rete bayesiana}\label{UC1.1}
\begin{itemize}
	\item \textbf{Attore primario}: Utente;
	\item \textbf{Precondizioni}:
	\begin{enumerate}
		\item L'utente ha cliccato il bottone con etichetta "+".
	\end{enumerate}
	\item \textbf{Postcondizioni}: 
	\begin{enumerate}
		\item L'utente ha selezionato la rete bayesiana desiderata e ha premuto il pulsante con etichetta "Aggiungi".
	\end{enumerate}
	\item \textbf{Scenario principale:}
	\begin{enumerate}
		\item L'utente seleziona dalla finestra il file da importare;
		\item L'utente clicca il pulsante con etichetta "Aggiungi".
	\end{enumerate}
	\item \textbf{Estensioni}:
\end{itemize}

\subsubsection{UC2 - ...}\label{UC2}

\subsubsection{UC3 - ...}\label{UC3}

\subsubsection{UC4 - ...}\label{UC4}

\subsubsection{UC5 - ...}\label{UC5}

\subsubsection{UC6 - ...}\label{UC6}