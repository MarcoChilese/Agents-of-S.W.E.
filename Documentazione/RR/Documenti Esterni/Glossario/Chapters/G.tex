\section*{G}
\addcontentsline{toc}{section}{G}

\subsection{\index{Gamification}Gamification}
Rappresenta uno strumento in grado di veicolare messaggi di vario tipo e di indurre a comportamenti attivi da parte dell’utenza, permettendo di raggiungere specifici obiettivi. Al centro di questo approccio va sempre collocato l'utente ed il suo coinvolgimento attivo.

\subsection{\index{Git}Git}
Software di controllo versione distribuito utilizzabile da interfaccia a riga di comando. Può considerarsi il software di versionamento più diffuso.

\subsection{\index{GitHub}GitHub}
Servizio di hosting per progetti software, è un'implementazione dello strumento di controllo di versionamento Git. Mette a disposizione degli utenti del repository un'interfaccia web per vedere l'albero di directory, i file inseriti e aggiornati e dispone di un sistema di tracciamento delle issue che permette di mantenere una lista delle issue relative al progetto, svolte e da svolgere.

\subsection{\index{GitLab}GitLab}
Come GitHub è un servizio di hosting per progetti software che implementa il controllo di versionamento Git. 

\subsection{\index{Google Drive}Google Drive}
Google Drive è un servizio web, in ambiente cloud computing, di memorizzazione e sincronizzazione online introdotto da Google.

\subsection{\index{Grafana}Grafana}
Piattaforma open source che permette il monitoraggio e l'analisi di dati che vengono visualizzati in dashboard operative. 