\section*{I}
\addcontentsline{toc}{section}{I}

\subsection{\index{IA}IA}
"Intelligenza Artificiale". Disciplina appartenente all'informatica che studia i fondamenti teorici, le metodologie e le tecniche che consentono la progettazione di sistemi hardware e software
capaci di fornire all'elaboratore elettronico prestazioni, che a un osservatore comune, sembrerebbero essere di pertinenza esclusivamente umana. 

\subsection{\index{Indice di Gulpease}Indice di Gulpease}
Indice di leggibilità di un testo tarato sulla lingua italiana. Si calcola valutando la lunghezza delle parole in lettere, il numero delle parole e il numero delle frasi totali. L'indice risultante è compreso tra 0 e 100: più il valore si avvicina a 100 più il testo è leggibile.

\subsection{\index{InfluxDB}InfluxDB}
InfluxDB è un database ottimizzato per le serie temporali, sviluppato da InfluxData.

\subsection{\index{Issue}Issue}
Problema, attività o compito da svolgere che viene sollevata da un utente, di solito il responsabile, e che viene assegnata oppure auto-assegnata ad un membro del team, può avere una data di scadenza, uno o più tag, una priorità e una breve descrizione.

\subsection{\index{Issue Tracking}Issue Tracking}
Metodo che permette di rintracciare da dove derivano le Issue, a chi sono assegnate, e tracciarne uno storico dello sviluppo.

\subsection{\index{Iterazione}Iterazione}
Atto di ripetere un processo con l'obiettivo di avvicinarsi a un risultato desiderato.

