\section*{U}
\addcontentsline{toc}{section}{U}

\subsection{\index{User Case}Use Case} 
E' una tecnica usata nei processi di ingegneria del software per effettuare in maniera esaustiva e non ambigua, la raccolta dei requisiti. Consiste nel valutare ogni requisito focalizzandosi sugli attori che interagiscono col sistema, valutandone le varie interazioni.

\subsection{\index{User-friendly}User-friendly} 
Qualità, solitamente di un prodotto software, che rappresenta l'immediatezza di utilizzo anche per chi non è esperto.
