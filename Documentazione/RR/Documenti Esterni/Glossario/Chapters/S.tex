\section*{S}
\addcontentsline{toc}{section}{S}

\subsection{\index{Script}Script}
Particolare tipo di programma, scritto con appositi linguaggi interpretati (di scripting per l'appunto), senza interfaccia grafica, aventi solitamente complessità relativamente bassa.

\subsection{\index{SCSS}SCSS}
Estensione di CSS non direttamente interpretabile dal browser.

\subsection{\index{Slack}Slack}
Applicazione di messaggistica istantanea, specializzata nella collaborazione aziendale, utilizzata tra i membri del team di lavoro.

\subsection{\index{SonarQube}SonarQube} 
È uno strumento per la continuous inspection per la qualità del codice.

\subsection{\index{SPICE}SPICE}
Standard ISO/IEC 15504, relativo ad un insieme di documenti tecnici standard per i processi di sviluppo software e le funzioni di gestione di management.

\subsection{\index{SSH}SSH}
È un protocollo che permette di stabilire una sessione remota cifrata tramite interfaccia a riga di comando con un altro host di una rete informatica.

\subsection{\index{Stakeholder}Stakeholder}
Con stakeholder si intende ciascuno dei soggetti direttamente o indirettamente coinvolti in un progetto o nell'attività di un'azienda.

\subsection{\index{Swift}Swift}
È un linguaggio di programmazione orientato agli oggetti sviluppato da Apple per sistemi macOS, iOS.

\subsection{\index{Systemjs}Systemjs}
Caricatore modulare configurabile che consente flussi di lavoro retrocompatibili per moduli ECMAScript nei browser.