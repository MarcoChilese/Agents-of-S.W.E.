\section*{E}
\addcontentsline{toc}{section}{E}

\subsection{\index{ECMA}ECMA}
Acronimo di European Computer Manufacturers Association, è un'associazione fondata nel 1961 e dedicata alla standardizzazione nel settore informatico e dei sistemi di comunicazione. Dal 1994 viene chiamata ECMA International. E' responsabile di molti standard come JSON, ECMAScript.

\subsection{\index{ECMAScript}ECMAScript}
E' un linguaggio di programmazione standardizzato e mantenuto da Ecma International nell'ECMA-262 ed ISO/IEC 16262. Le implementazioni più conosciute di questo linguaggio sono JavaScript, JScript e ActionScript che sono entrati largamente in uso, inizialmente, come linguaggi lato client nello sviluppo web.

\subsection{\index{Efficacia}Efficacia}
Capacità di raggiungere l'obiettivo prefissato, soddisfacendo tutti i suoi requisiti, impliciti ed espliciti.

\subsection{\index{Efficienza}Efficienza}
Misura della capacità di raggiungere l'obiettivo prefissato impiegando le risorse minime indispensabili.

\subsection{\index{ESLint}ESLint}
E' un utility open source per l’analisi statica del codice e per l’identificazione di pattern in JavaScript,  questo permette di trovare più facilmente codice che non rispetta determinate linee guida.