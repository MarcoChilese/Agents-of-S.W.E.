\section*{C}
\addcontentsline{toc}{section}{C}

\subsection{\index{CamelCase}CamelCase}
La notazione a cammello è la pratica di scrivere parole composte o frasi unendo tutte le parole tra loro, ma lasciando le loro iniziali maiuscole. La prima lettera può essere sia maiuscola (es. CamelCase), come il nome delle classi in Java, che minuscola (es. camelCase), come le proprietà di un oggetto in Java.

\subsection{\index{Cloud}Cloud}
Vasta rete di server remoti ubicati in tutto il mondo, collegati tra loro e che operano come un unico ecosistema. Questi server possono archiviare e gestire dati, eseguire applicazioni o distribuire contenuti o servizi. Anziché accedere a file e dati da un computer locale, vi si può accedere online, da qualsiasi dispositivo con connessione Internet.

\subsection{\index{Commit}Commit}
Operazione, tipica degli strumenti di controllo di versione, in cui si cerca di rendere delle modifiche, ad un file condiviso, permanenti.

\subsection{\index{Continuos Delivery}Continuos Delivery}

\subsection{\index{Continuos Inspection}Continuos Inspection}

\subsection{\index{Continuos Integration}Continuos Integration}
Nell'ingegneria del software, l'integrazione continua o continuous integration è una pratica dove si raccolgono metodologie e processi per che automatizzano l’integrazione di un progetto, ad ogni check-in o ad intervalli schedulati. La necessità alla base di questo processo è quella di minimizzare i problemi che si possono verificare in progetti gestiti da team di sviluppo numerosi.

\subsection{\index{CSS}CSS}
Acronimo di Cascading Style Sheets(fogli di stile a cascata), è un linguaggio usato per definire la formattazione di documenti HTML, XHTML e XML ad esempio i siti web e relative pagine web. Le regole per comporre il CSS sono contenute in un insieme di direttive emanate a partire dal 1996 dal W3C.



