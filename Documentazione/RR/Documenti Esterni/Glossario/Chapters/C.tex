\section*{C}
\addcontentsline{toc}{section}{C}

\subsection{\index{CamelCase}CamelCase}
La notazione a cammello è la pratica di scrivere parole composte o frasi unendo tutte le parole tra loro, ma lasciando le loro iniziali maiuscole. La prima lettera può essere sia maiuscola (es. CamelCase), che minuscola (es. camelCase).

\subsection{\index{Capitolato}Capitolato}
È un documento tecnico che specifica quale sia il problema che si vuole risolvere descrivendo le caratteristiche generali che sono richieste nel prodotto.

\subsection{\index{Caso D'Uso}Caso D'Uso}
Vedi \textit{Use Case}.

\subsection{\index{Cloud}Cloud}
Rete di server remoti ubicati in tutto il mondo, collegati tra loro e che operano come un unico ecosistema. Questi server possono archiviare e gestire dati, eseguire applicazioni o distribuire contenuti o servizi. Anziché accedere a file e dati da un computer locale, vi si può accedere online, da qualsiasi dispositivo con connessione Internet.

\subsection{\index{Commit}Commit}
Operazione, tipica degli strumenti di controllo di versione, in cui si cerca di rendere delle modifiche, ad un file condiviso, permanenti.

\subsection{\index{Continuos Delivery}Continuous Delivery}
La distribuzione continua è un metodo di sviluppo software in cui le modifiche al codice vengono preparate automaticamente per un rilascio in produzione. La distribuzione continua estende l'integrazione continua distribuendo tutte le modifiche al codice all'ambiente di testing e/o di produzione dopo la fase di creazione di build.

\subsection{\index{Continuos Inspection}Continuous Inspection}
Analisi continua del codice con l'obbiettivo di migliorare la qualità e trovare difetti in esso.

\subsection{\index{Continuos Integration}Continuous Integration}
Nell'ingegneria del software, l'integrazione continua o continuous integration è una pratica dove si raccolgono metodologie e processi per che automatizzano l’integrazione di un progetto, ad ogni check-in o ad intervalli schedulati. La necessità alla base di questo processo è quella di minimizzare i problemi che si possono verificare in progetti gestiti da team di sviluppo numerosi.

\subsection{\index{Cross platform}Cross platform}
Software, linguaggio di programmazione, applicazione software compatibile a pieno con più di un sistema o appunto, piattaforma (es. Windows,Linux/Unix, ecc).

\subsection{\index{CSS}CSS}
Acronimo di Cascading Style Sheets (fogli di stile a cascata), è un linguaggio usato per definire la formattazione di documenti HTML, XHTML e XML ad esempio i siti web e relative pagine web. Le regole per comporre il CSS sono contenute in un insieme di direttive emanate a partire dal 1996 dal W3C.

\subsection{\index{Cubit}Cubit}
Criptovaluta open source e decentralizzata che offre transazioni private e trasparenti.
