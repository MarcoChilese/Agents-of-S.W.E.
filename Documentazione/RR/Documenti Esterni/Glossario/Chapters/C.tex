\section*{C}
\addcontentsline{toc}{section}{C}

\subsection{\index{Cloud}Cloud}
Vasta rete di server remoti ubicati in tutto il mondo, collegati tra loro e che operano come un unico ecosistema. Questi server possono archiviare e gestire dati, eseguire applicazioni o distribuire contenuti o servizi. Anziché accedere a file e dati da un computer locale, vi si può accedere online, da qualsiasi dispositivo con connessione Internet.

\subsection{\index{CSS}CSS}

\subsection{\index{CamelCase}CamelCase}
Notazione a cammello è una pratica per scrivere parole composte o frasi unendo tutte le parole tra di loro, ma lasciando 
le loro iniziali in maiuscolo. 

\subsection{\index{Commit}Commit}
Operazione, tipica degli strumenti di controllo di versione, in cui si cerca di rendere delle modifiche, ad un file condiviso, permanenti.

\subsection{\index{Continuos Delivery}Continuos Delivery}
Approccio ingegneristico dove un team produce software in cicli di vita brevi, assicurando che possa essere rilasciato in modo 
affidabile in qualsiasi momento. 

\subsection{\index{Continuos Inspection}Continuos Inspection}
Analisi continua del codice con l'obbiettivo di migliorare la qualità e trovare difetti in esso. 

\subsection{\index{Continuos Integration}Continuos Integration}
Pratica ingegneristica che si applica in contesti in cui lo sviluppo di software avviene attraverso un sistema di versionamento, 
consentendo l'allineamento frequente verso la mainline.  

