\section*{P}
\addcontentsline{toc}{section}{P}

\subsection{\index{Piano di Progetto}Piano di Progetto}
Documento contenente la pianificazione del lavoro di un gruppo di lavoro per portare a termine un determinato progetto. Esso presenta l'organizzazione delle attività e dei tempi, un preventivo delle risorse necessarie al compimento del progetto, consuntivi di periodo ed analisi di rischi e piani di mitigazione per tali rischi.

\subsection{\index{Piano di Qualifica}Piano di Qualifica}
Documento contente le strategie di verifica e validazione adottate da un gruppo di lavoro per assicurare la qualità di prodotto e di processo in un determinato progetto.

\subsection{\index{Prodotto}Prodotto}
Output di un processo. In ambito software è un'entità progettata per essere rilasciata all'utilizzatore finale.

\subsection{\index{Progettazione}Progettazione}
Fase del ciclo di vita del software che risponde alla domanda \textit{"Coma va fatta la cosa giusta?"}. E' un'attività che, sulla base della specifica dei requisiti prodotta dall'analisi, definisce come tali requisiti debbano essere soddisfatti, ricercando una soluzione soddisfacente per tutti gli stakeholders.

\subsection{\index{Progetto}Progetto}
Insieme di attività e compiti che hanno come proprietà caratteristiche:
\begin{itemize}
	\item Obiettivi Prefissati;
	\item Tempi Fissati, ovvero precise scadenze;
	\item Risorse Limitate che vengono consumate dalle attività di progetto.
\end{itemize}