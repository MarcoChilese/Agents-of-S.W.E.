\section{Resoconto}

\subsection{Punto1}
Durante l'incontro, il gruppo ha verificato le task precedentemente assegnate nella riunione precedente. 

\subsection{Punto2}
Il gruppo ha discusso sulle tecnologie di base utilizzate, in particolar modo sui metodi forniti da ognuna di essa per portare a termine un specifico compito. Si e analizzato, in particolar modo, le direttive di AngularJS\glossario quali: ng-repeat, ng-option, ng-if, ngBindTemplate, ng-model e ng-change, analizzando per ognuna di esse i loro vincoli e le funzionalità. Un problema riscontrato da questa analisi preliminare e l'utilizzo di ng-option nei tag \texttt{select} di HTML\glossario il quale, una volta creati i vari select, non aggiornava tali elementi a ogni modifica apportata al model associato con ng-model. Questa problematica è stata risolta con l'utilizzo di ng-repeat, associando sempre il model all'elemento della direttiva AngularJS. Un punto di discussione per il gruppo è stato l'utilizzo delle classi Grafana come rimpiazzo per le chiamate Ajax effettuate con jQuery alle API di Grafana, soluzione che migliorava la qualità del progetto, e che ha portato il gruppo ad utilizzare quest'ultima. Inoltre il gruppo, viste le difficoltà di portabilità dell'interfaccia per il plugin, ha optato per l'utilizzo di classi CSS\glossario predefinite in Grafana, migliorando la qualità e la portabilità del plugin sviluppato sulla piattaforma. 

\subsection{Punto 3}
Il gruppo ha discusso sulle varie informazione da racchiudere all'interno delle presentazione per la Technology Baseline, dibattendo sulle priorità di ognuna di esse. Successivamente si è inizializzata la presentazione effettiva creando le prima slide. Inoltre si sono verificate le funzionalità minime che soddisfano i requisiti da presentare nel PoC al colloquio. 

\subsection{Punto 4}
Il responsabile del gruppo ha inizializzato e assegnato le nuove task per il compimento delle presentazione e delle parti mancanti del PoC. 
