\section{Capitolato scelto: C3}\label{CapScelto}

\subsection{Descrizione generale}
\begin{itemize}
	\item \textbf{Nome}: G$\&$B: monitoraggio intelligente di processi DevOps;
	\item \textbf{Proponente}: Zucchetti S.p.A.
\end{itemize}

\subsection{Descrizione Capitolato e Obiettivo Finale}
Il terzo capitolato propone di sviluppare un plug-in per la piattaforma open source\glossario di visualizzazione e gestione dati, denominata \textit{Grafana}, con l'obiettivo di creare un sistema di alert dinamico per monitorare la "liveliness"\glossario del sistema a supporto dei processi
devops\glossario e per consigliare interventi nel sistema di produzione del software.
In particolare, il plug-in utilizzerà dati in input forniti ad intervalli regolari o con continuità, ad una rete bayesiana\glossario per stimare la probabilità di alcuni eventi, segnalandone quindi il rischio in modo dinamico, prevenendo situazioni di stallo.   

\subsection{Dominio Tecnologico}
\begin{itemize}
	\item \textbf{\textit{ECMAScript6}\footnote{\url{https://www.ecma-international.org/}}\glossario}: linguaggio di scripting indicato per lo sviluppo del plug-in nella documentazione per gli sviluppatori\footnote{\url{http://docs.grafana.org/plugins/developing/development/}} della piattaforma \textit{Grafana};
	\item \textbf{\textit{JSON}\footnote{\url{https://www.json.org/}}\glossario}: formato dati utilizzato per l'acquisizione dei dati da fornire all rete bayesiana;
	\item \textbf{Rete Bayesiana}: modello probabilistico utilizzato per stimare la probabilità degli eventi di interesse;
	\item \textbf{\textit{Jsbayes}\footnote{\hyperref[Link al repository GitHub]{\url{https://github.com/vangj/jsbayes}}}\glossario}: libreria open source consigliata dal fornitore per la gestione dei calcoli della rete Bayesiana.
\end{itemize}

\subsection{Valutazione del Capitolato}
\subsubsection{Aspetti Positivi}
\begin{itemize}
	\item Chiarezza espositiva del problema da affrontare;
	\item Contesto moderno ed interessante;
	\item Piattaforma preesistente;
	\item Utilizzo di Reti Bayesiane;
	\item Dominio tecnologico ben definito, limitato e ben documentato.
\end{itemize}

\subsubsection{Aspetti Negativi}
\begin{itemize}
	\item Il team ha una conoscenza limitata di \textit{JavaScript}\glossario, linguaggio di scripting che ha dato origine alla sua versione standardizzata: \textit{ECMAScript}. Pertanto è necessario un percorso di apprendimento.
\end{itemize}

\subsubsection{Conclusioni e Motivazioni della scelta}
Grazie alla tematica interessante, la possibilità di contribuire con un plug-in ad una piattaforma preesistente ampiamente utilizzata e l'ambito relativo alle reti bayesiane, tema innovativo ed attuale, il team è portato a preferire il capitolato in oggetto. A dare ulteriore sostegno a tale preferenza, un dominio tecnologico ben definito e non eccessivamente ampio, inoltre {Grafana Labs}, azienda che fornisce \textit{Grafana}, mette a disposizione degli sviluppatori un'ampia documentazione.\\
 Nella scelta ha contribuito la disponibilità dell'azienda proponente, che si è dimostrata sin da subito disposta a fornire chiarimenti ove necessario, la chiarezza dei temi e dei requisiti esposti.\\
Le tecnologie coinvolte devono essere necessariamente approfondite dall'intero team, ciononostante sembrano  ampiamente affrontabili e gestibili.
