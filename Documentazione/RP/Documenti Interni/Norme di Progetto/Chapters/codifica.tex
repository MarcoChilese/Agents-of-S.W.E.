Di seguito vengono definite delle norme che devono essere adottate dai Programmatori per garantire una buona leggibilità  e manutenibilità  del codice. Le prime norme che seguiranno sono le più generali, da adottarsi per ogni linguaggio di programmazione adottato all'interno del progetto, in seguito quelle più specifiche per il linguaggio principale ECMAScript 6\glossario.\\
Ogni norma è caratterizzata da un paragrafo di appartenenza, da un titolo, una breve descrizione e, se il caso lo richiede, un esempio.\\
Il rispetto delle seguenti norme è fondamentale per garantire uno stile di codifica uniforme all'interno del progetto, oltre che per massimizzare la leggibilità  e agevolare la manutenzione, la verifica\glossario e la validazione\glossario.

\paragraph{Convenzioni per i nomi:} \label{Nomi}
\begin{itemize}	
	\item I Programmatori devono adottare come notazione per la definizione di cartelle, file, metodi, funzioni e variabili il CamelCase\glossario.\\
	Di seguito un esempio di corretta nomenclatura:
	\begin{lstlisting}[language=JavaScript]
//Cartelle 
./thisIsAFolder	//OK
./ThisIsAFolder //NO

//File
myFile.extension //OK
MyFile.extension //NO

//Funzioni
myFunction() { ... } //OK
MyFunction() { ... } //NO
	\end{lstlisting}

	\item Tutti i nomi devono essere \textbf{unici} ed \textbf{autoesplicativi}, ciò per evitare ambiguità  e limitare la complessità .
\end{itemize}
\paragraph{Convenzioni per la documentazione:}
\begin{itemize}	
	\item Tutti i nomi ed i commenti al codice vanno scritti in \textbf{inglese};
	\item Nel codice è possibile utilizzare un commento con denominazione \textbf{TODO} in cui si vanno ad indicare compiti da svolgere;
	\item L'intestazione di ogni file deve essere la seguente:
	\begin{lstlisting}[language=JavaScript]
/**
* File: nameFile
* Type: fileType
* Creation date: yyyy-mm-gg
* Author: Name Surname
* Author e-mail: email@example.com
* Version: versionNumber 
*
* Changelog:
* #entry || Author || Date || Description
*/
	\end{lstlisting}
	\item La versione del file nell'intestazione deve rispettare la seguente formulazione: $X.Y.Z$, dove X rappresenta la versione principale, Y la versione parziale della relativa versione principale e Z l'avanzamento rispetto ad Y.\\ I numeri di versione del tipo $X.0.0$, dalla $1.0.0$, vengono considerate versioni stabili e quindi versioni da testare per saggiarne la qualità .
\end{itemize}
\paragraph{ECMAScript 6}\label{EcmaScript6} \-\\
Seguendo le indicazioni presenti nella documentazione\footnote{\texttt{\url{http://docs.grafana.org/plugins/developing/development/}}} dell'azienda fornitrice di \textit{Grafana}, la piattaforma  per cui si intende sviluppare il plug-in, il team ha deciso di adottare come linguaggio di programmazione principale ECMAScript 6\footnote{Linguaggio divenuto standard ISO: ISO/IEC 16262:2011, e relativo aggiornamento ISO/IEC 22275:2018.}.\\
ECMAScript 6 viene stardardizzato da \textbf{ECMA}\glossario\footnote{\texttt{\url{http://www.ecma-international.org/}}} nel giugno 2015 con la sigla \textbf{ECMA-262\footnote{\texttt{\url{https://www.ecma-international.org/ecma-262/6.0/}}}}.\\
Come stile di codifica si adottano le linee guida proposte da \textbf{Airbnb JavaScript Style Guide}\footnote{\texttt{\url{https://github.com/airbnb/javascript}}}. Per la verifica dell'adesione a tali norme, i Programmatori devono utilizzare, come suggerito dalla documentazione proposta da \textit{Grafana}, \textbf{ESLint}\glossario\footnote{\texttt{\url{https://eslint.org/}}}.\\
In particolare i Programmatori devono rispettare 5 linee guida proposte dalla documentazione ufficiale di \textit{Grafana}:
\begin{enumerate}
	\item Se una variabile non viene riutilizzata, deve essere dichiarata come \texttt{\textbf{const}};
	\item Utilizzare preferibilmente, per la definizione di variabili, la keyword  \texttt{\textbf{let}}, anziché  \texttt{\textbf{var}};
	\item Utilizzare il marcatore freccia (\texttt{\textbf{=>}}), in quanto non oscura il \texttt{\textbf{this}}:
	\begin{lstlisting}[language=JavaScript]
testDatasource() {
  return this.getServerStatus()
  .then(status => {
    return this.doSomething(status);
  })
}	
	\end{lstlisting}
	Invece che:
	\begin{lstlisting}[language=JavaScript]
testDatasource() {
  var self = this;
  return this.getServerStatus()
    .then(function(status) {
  return self.doSomething(status);
  })
}
	\end{lstlisting}
	\item Utilizzare l'oggetto \textit{Promise}:
	\begin{lstlisting}[language=JavaScript]
metricFindQuery(query) {
  if (!query) {
    return Promise.resolve([]);
  }
}	
	\end{lstlisting}
	Invece che:
	\begin{lstlisting}[language=JavaScript]
metricFindQuery(query) {
  if (!query) {
    return this.$q.when([]);
  }
}
	\end{lstlisting}
	%conseguenti?
	\item Se si utilizza \textit{Lodash}, per coerenza, lo si preferisca alle array function native di ES6.
\end{enumerate}
Verranno esaminate di seguito le norme in merito allo stile di codifica che i Programmatori dovranno adottare.

\subparagraph{Indentazione}\-\\
\textbf{Norma 1}
L'indentazione è da eseguirsi con tabulazione la cui larghezza sia impostata a due (2) spazi per ogni livello.\\
Di seguito un esempio da ritenersi corretto:
\begin{lstlisting}[language=JavaScript]
function() {
..let x = 2;
..if (x > 0)
....return true;
..else
....return false;
}
\end{lstlisting}
Qualsiasi altro tipo di indentazione è da ritenersi scorretta.\\
\-\\
\textbf{Norma 2}
Dopo la graffa principale va inserito uno (1) spazio. Nel seguente modo:
\begin{lstlisting}[language=JavaScript]
function() { ... }
\end{lstlisting}
\-\\
\textbf{Norma 3}
Dopo la keyword di un dato statement (\texttt{if, while}, etc.) va inserito uno (1) spazio. Per un esempio corretto si veda la norma successiva.\\
\-\\
\textbf{Norma 4}
Prima dell'apertura della graffa negli statement di controllo va inserito uno (1) spazio. Nel seguente modo:
\begin{lstlisting}[language=JavaScript]
function() {
  if (condition) {
    ...  
  }
  while (condition) {
    ...
  }
}
\end{lstlisting}
\-\\
\textbf{Norma 5}
Negli statement di controllo (\texttt{if, while}, etc.) le condizioni concatenate o annidate, mediante operatori logici, che diventano eccessivamente lunghe NON vanno espresse in un'unica linea, bensì spezzate in più righe. Nel seguente modo:
\begin{lstlisting}[language=JavaScript]
function() {
  if (condition && condition) {
    ...  
  }
  
  if (
   veryLongCondition
   && longCondition
   && condition
    ) {
    doSomething();
  }
}
\end{lstlisting}
\-\\
\textbf{Norma 6}
Dopo blocchi, o prima di un nuovo statement va lasciata una riga vuota. Nel seguente modo:
\begin{lstlisting}[language=JavaScript]
function1() {
  if (condition) {
    doSomething():  
  }
  
  return toReturn;  
}

function2(){
  ...
}
\end{lstlisting}
\-\\
\textbf{Norma 7}
I blocchi di codice multi-riga devono essere contenuti all'interno di graffe. Blocchi costituiti da una singola riga non è necessario che siano contenuti tra graffe: nel caso non vengano utilizzate, la definizione deve essere \textit{inline}, cioè sulla stessa riga.\\
Nel seguente modo:
\begin{lstlisting}[language=JavaScript]
if (condition) return true;

if (condtion) {
  return true;
}
\end{lstlisting}

\subparagraph{Commenti al codice}\-\\
Il codice va commentato nel seguente modo:
\begin{itemize}
	\item "\texttt{//}" se il commento occupa una sola riga;
	\item "\texttt{/** ... */} " se il commento occupa più righe.
\end{itemize}
Nel seguente modo:
\begin{lstlisting}[language=JavaScript]
// single line comment
if (condition) return true;

/**
* multi line comment, line 1
* multi line comment, line 2
*/
if (condtion) {
  return true;
}
\end{lstlisting}

\subparagraph{Variabili}\-\\
\textbf{Norma 1}
Fare riferimento alle norme 1 e 2, all'inizio della sezione §\ref{EcmaScript6}.

\-\\
\textbf{Norma 2}
Non utilizzare dichiarazioni multiple di variabili, dichiarare una variabile per riga.\\
Nel seguente modo:
\begin{lstlisting}[language=JavaScript]
// OK
var x = 1;
var y = 0;

// NO
var x = 1, y = 0;
\end{lstlisting}


%va linkato il paragrafo
\subparagraph{Nomi}\-\\
\textbf{Norma 1} Oltre a quanto enunciato nel secondo punto del paragrafo §\ref{Nomi}, tutti i nomi di funzioni o variabili composti da una singola lettera, o che indichino temporaneità della variabile sono \textit{vietati}: ogni nome deve essere significativo.\\
\-\\
\textbf{Norma 2} 
\begin{enumerate}
	\item I nomi delle variabili, funzioni ed istanze devono utilizzare il CamelCase;
	\item I nomi delle classi deve avere lo stile \texttt{capWords}.
\end{enumerate}
Nel seguente modo:
\begin{lstlisting}[language=JavaScript]
// OK
var thisIsAVariable;

function thisIsAFunction() { ... }

class ThisIsAClass() {
  ...
}

// NO
var Variable;

function Function() { ... }

class myClass() {
  ...
}
\end{lstlisting}

\paragraph{CSS}
\label{css} \-\\

Per la parte front-end\glossario del plug-in il gruppo \texttt{Agents of S.W.E.} ha scelto di utilizzare i linguaggi \textit{CSS} ed \textit{HTML5}.\\
Qui di seguito verrà descritta la Style Guide resa disponibile da \textbf{Airbnb CSS-in-JavaScript Style Guide}\footnote{\texttt{\url{http://airbnb.io/javascript/css-in-javascript}}}.

\subparagraph{Indentazione} \-\\
\textbf{Norma 1}
Tra blocchi adiacenti dello stesso livello deve essere lasciata una linea bianca, tra l'una e l'altra. In questo modo il codice risulterà essere più leggibile. \\
Di seguito, un esempio della corretta indentazione: 


\begin{lstlisting} [language=JavaScript]
//OK
{
  bigBang: {
    display: 'inline-block',
    '::before': {
      content: "''",
    },
  },
  universe: {
    border: 'none',
  },
}

//NO
{
  bigBang: {
    display: 'inline-block',

    '::before': {
      content: "''",
    },
  },

  universe: {
    border: 'none',
  },
}
\end{lstlisting}

Per le restanti regole si farà riferimento allo standard imposto dalla World Wide Web Consortium\footnote{\href{http://www.w3.org/}{\texttt{W3C}}}. Per essere considerato corretto il file \textit{.css} dovrà essere validato tramite il sito reso disponibile dall'organizzazione, precedentemente nominata. La validazione potrà essere effettuata tramite il sistema disponibile al seguente \href{http://validator.w3.org/}{link}.

\paragraph{HTML}
\label{html}

Insieme al \textit{CSS}, per lo sviluppo della parte front-end del progetto, il gruppo ha scelto di utilizzare \textit{HTML5}. Per la stesura della Style Guide verrà utilizzata la documentazione resa disponibile da W3C\footnote{\url{https://www.w3schools.com/html/html5_syntax.asp}}.\\

Di seguito saranno riportate le principali norme da seguire: \\
\begin{itemize}

\item \textbf{Norma 1}: 
il tipo di documento deve sempre essere dichiarato all'inizio del documento, in maiuscolo o minuscolo a piacere;
\item \textbf{Norma 2}: 
i nomi degli elementi ed attributi devono essere sempre scritti in minuscolo;
\item \textbf{Norma 3}:
tutti gli elementi HTML devono essere chiusi, anche se vuoti;
\item \textbf{Norma 4}:
il valore degli attributi deve essere racchiuso dentro doppio apice;
\item \textbf{Norma 5}: 
intorno al simbolo di uguaglianza (=) non devono essere presenti spazi;
\item \textbf{Norma 6}:
devono essere usati due (2) spazi e non le tabulazioni. Gli spazi bianchi vanno utilizzati solamente per ampli blocchi di codice, per una maggiore leggibilità;
\item \textbf{Norma 7}:
nonostante i tag \texttt{html}, 	\texttt{body} e \texttt{head} possano essere omessi, vanno inseriti affinché siano compatibili con tutti i browser;
\item \textbf{Norma 8}:
le classi che verranno dichiarate all'interno dell'\textit{HTML}, al fine di andare poi a ridefinire il \textit{CSS}, dovranno essere quelle rese disponibili dalla documentazione di Grafana\footnote{\url{http://docs.grafana.org/plugins/developing/code-styleguide/}}.
\end{itemize}

Equivalentemente al \textit{CSS}, anche il codice \textit{HTML} dovrà essere validato. Questa validazione avverrà tramite il sito reso disponibile dalla W3C, al seguente \href{https://validator.w3.org/}{link}.

%AGGIUNGERE E SISTEMARE DA QUA IN GIU

\paragraph{Npm}

Npm è un gestore di pacchetti per \textit{JavaScript} e gestore dei quelli presenti di default nell'ambiente \textit{Node.js}. \\
Risulta essere strutturato in due parti: un client a riga di comando ed un database, contente pacchetti sia pubblici che privati. Lo scopo di no è quello di incoraggiare il riuso del codice tra team ed il suo utilizzo è mirato alla gestione delle dipendenze. Il suo utilizzo è necessario al fine di installare \textbf{ESLint}.

\paragraph{WebStorm}

Per lo sviluppo del plug-in, si è scelto di utilizzare L'IDE WebStorm \footnote{\url{https://www.jetbrains.com/webstorm/}}, facente parte della famiglia di prodotti di JetBrains. \\
WebStorm è una cross-platform, sviluppata principalmente per lo sviluppo web, JavaScript e TypeScript. \\
La configurazione della piattaforma di sviluppo dovrà essere equivalente per tutti i componenti del gruppo. All'interno della piattaforma dovrà essere installato \textbf{ESLint}, la cui funzione verrà spiegata nella sezione sequente. \\

\paragraph{ESLint}

Il plug-in \textbf{ESLint}, scaricabile dall'apposita sezione all'interno dell'IDE utilizzato, ha la funzione di segnalare errori che vengono effettuati durante la stesura del codice, in modo da evitare bug e rendere il codice più coerente. \\ 
Il plug-in è reperibile tra i pacchetti resi disponibili da NPM\footnote{\url{https://www.npmjs.com/package/eslint}}.\\
Le regole utilizzate all'interno del plug-in saranno quelle esplicitate nella sezione §\ref{EcmaScript6}, riguardanti lo stile e le regole da rispettare per la stesura del codice con \textit{EcmaScript6}.




