\subsubsection{Gestione Comunicazioni}
	In questa sezione vengono descritte le norme che regolano le comunicazioni del gruppo \texttt{Agents of S.W.E.}, 		sia interne, tra i suoi componenti, sia verso entità esterne, come committenti e proponenti.

\paragraph{Comunicazioni Interne} ~\\
	Le comunicazioni interne ai membri del gruppo vengono gestite principalmente attraverso un gruppo 										\textit{Telegram}\glossario, presso il quale vengono discusse le tematiche riguardanti gli aspetti più generali o collettivi riguardanti il progetto. \\
	Per facilitare una comunicazione più specifica, monotematica, ed efficiente tra alcuni membri del gruppo, e per 			gestire meglio la stesura dei documenti, sono stati inoltre predisposti svariati canali tematici all'interno 				dell'app di messaggistica: \textit{Slack}\glossario. Tali canali sono:
	\begin{itemize}
	\item \textbf{\#general}: Per discussioni riguardanti rotazione di ruoli e decisioni degli argomenti principali 				da discure nelle riunioni;
	\item \textbf{\#normeprogetto}: Per discutere riguardo le regole del \textit{Way of Working} del gruppo, le norme 		da seguire e, di conseguenza, la stesura in collaborazione del documento \textit{Norme di Progetto}\glossario;
	\item \textbf{\#pianoprogetto}: Per confrontarsi riguardo il monte ore dei vari ruoli e per facilitare la stesura 		del documento \textit{Piano di Progetto}\glossario;
	\item \textbf{\#analisirequisiti}: Per discutere gli \textit{Use Case}\glossario e i requisiti necessari alla 				stesura dell'\textit{Analisi dei Requisiti};
	\item \textbf{\#pianoqualifica}: Per discutere strategie da attuare per garantire qualità attraverso 									verifica\glossario e validazione\glossario.
	\end{itemize}

\paragraph{Comunicazioni Esterne} ~\\
	Le comunicazioni esterne avvengono attraverso la casella di posta elettronica del gruppo: 														\textit{agentsofswe@gmail.com}, gestita principalmente dal \textit{Responsabile} del gruppo, ma accessibile da 			ogni membro e configurata per eseguire un inoltro automatico delle mail ricevute ad ogni membro del gruppo.


\subsubsection{Gestione Riunioni}
	Durante ogni riunione, interna o esterna, verrà nominato, tra i componenti del gruppo, un segretario che avrà il 		compito di far rispettare l'ordine del giorno, stilato dal \textit{Responsabile di Progetto}, ed occuparsi della 		stesura del	Verbale di Riunione\glossario.

\paragraph{Riunioni Interne} ~\\
	E' compito del \textit{Responsabile di Progetto} organizzare riunioni interne al gruppo \texttt{Agents of 						S.W.E.}. Ciò prevede, più nello specifico, la stesura dell'ordine del giorno, stabilire data, orario e luogo di 			incontro, ed assicurarsi, attraverso la comunicazione mediante i mezzi propri del gruppo, che ogni componente sia 	pienamente a conoscenza della riunione in tutti i suoi dettagli. \\
	D'altro canto ogni membro del gruppo deve presentarsi puntuale agli appuntamenti, e comunicare in anticipo 					eventuali ridardi o assenze adeguatamente giustificate. \\
	Una riunione non è da ritenersi valida se i partecipanti risultino essere in numero inferiore a cinque.

\paragraph{Riunioni Esterne} ~\\
	E' nuovamente compito del \textit{Responsabile di Progetto} organizzare riunioni esterne. Nello specifico egli 			deve preoccuparsi di contattare l'azienda proponente per fissare incontri qualora sia necessario, tenendo conto anche delle preferenze 			di date e orario espresse dagli altri membri del gruppo. La partecipazione a tali riunioni deve essere, a meno di 	casi eccezionali, unanime.\\
	Ogni membro del gruppo può, inoltre, esprimere al \textit{Responsabile} una richiesta, adeguatamente motivata, 			di fissare una riunione esterna. A questo punto sarà compito dello stesso \textit{Responsabile} giudicare come 			valida o meno la richiesta presentatagli ed agire di conseguenza.

\paragraph{Verbale di Riunione} ~\\
	Ad ogni riunione, interna o esterna, è compito del Segretario designato redigere il Verbale di Riunione 							corrispondente, che deve essere poi approvato dal \textit{Responsabile}.\\
	Tale Verbale avrà la seguente \textbf{Struttura:}
	\begin{itemize}
	\item \textbf{Informazioni Generali:} questa prima sezione composta di:
		\begin{itemize}
		\item \textbf{Luogo};
		\item \textbf{Data};
		\item \textbf{Ora};
		\item \textbf{Membri del team partecipanti};
		\item \textbf{Segretario}.
		\end{itemize}
	\item \textbf{Ordine del Giorno:} sotto forma di elenco puntato rappresentante gli argomenti discussi;
	\item \textbf{Resoconto}: tale sezione rappresenta il riassunto, redatto dal Segretario, punto per punto di tutti 		gli argomenti di discussione, sia quelli preventivamente presenti nell'ordine del giorno sia eventuali spunti 				di riflessione maturati autonomamente durante la riunione.
	\end{itemize}
