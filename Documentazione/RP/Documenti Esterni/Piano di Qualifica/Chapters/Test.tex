\section{Test di Unità}
\label{test_u}
\subsection{Pianificazione dei Test}

La pianificazione dei test ne descrive l'implementazione specificando: 
\begin{itemize}
	\item \begin{center}
	T[Tipo][Priorità]-[Codice]

	\end{center}
	Dove: 
	
	\begin{itemize}
		\item \textbf{Tipo}: indica il tipo di test ed è identificato da una lettera a scelta tra: 

		\begin{itemize}
			\item \textbf{V}: validazione; 
			\item \textbf{S}: sistema; 
			\item \textbf{I}: integrazione; 
			\item \textbf{U}: unità.
		\end{itemize}
		
		\item \textbf{Priorità}: indica la priorità, identificato da una lettere a scelta tra: 
		\begin{itemize}
			\item \textbf{0}: obbligatorio; 
			\item \textbf{1}: desiderabile; 
			\item \textbf{2}: opzionale.
		\end{itemize}				
		\item \textbf{Codice}: indica il codice del test, rispettando una struttura gerarchica. 
		% \item \textbf{Indice alternativa}: indica uno scenario alternativo per lo svolgimento del test. ??!
	\end{itemize}
	
	\item L'identificatore del requisito associato al test, descritto nell' \textit{Analisi dei requisiti v2.0.0}; 
	\item Breve descrizione dello scopo del test; 
	\item Esisto del test, che può assumere i seguenti valori: 
	\begin{itemize}
		\item \texttt{"N.I."} (non implementato);
		\item \texttt{"N.S."} (non superato); 
		\item \texttt{"S."} (superato). 
	\end{itemize}
\end{itemize}

\subsubsection{Test di validazione}

\begin{longtable}{|C{.10\textwidth}|C{.12\textwidth}|C{.52\textwidth}|C{.12\textwidth}|}
\hline
\rowcolor{bluelogo}\textbf{\textcolor{white}{Test}} & \textbf{\textcolor{white}{Requisito}} & \textbf{\textcolor{white}{Descrizione}} & \textbf{\textcolor{white}{Esito}}\\
\hline \hline
\endhead

TV0-1 & ROF1 & 
	\textbf{Obiettivo}: verificare che il sistema permetta l'aggiunta di una rete bayesiana, caricando un file in formato \texttt{.json}
	\textbf{Procedura}: 
	\begin{enumerate}
		\item L'utente, dal panello di configurazione del plugin, preme sul bottone di caricamento; 
		\item L'utente seleziona il file da caricare; 
		\item L'utente conferma il file selezionato; 
		\item Il sistema carica e inizializza il file caricato. 
	\end{enumerate} & N.I. \\
\hline
\rowcolor{grigio} TV0-1.1 & ROF1.1 & 
	\textbf{Obiettivo}: verificare che il sistema metta a disposizione un pulsante per avviare la procedura di caricamento della rete bayesiana. 
	\textbf{Procedura}: 
	\begin{enumerate}
		\item L'utente, dal panello di configurazione del plugin, individua il bottone di caricamento; 
		\item L'utente preme sul bottone di caricamento; 
		\item L'utente carica il file desiderato; 
		\item Il sistema prende in carico il file selezionato dall'utente e lo inizializza. 
	\end{enumerate}
	& N.I \\
\hline
\makecell{ TV0-1.2}   & ROF1.2 &
	\textbf{Obiettivo}: verificare che il sistema permetta la selezione di un file in formato \texttt{.json} dal sistema dell'utente 
	\textbf{Procedura}:
	\begin{enumerate}
		\item L'utente, dal panello di configurazione del plugin, preme sul bottone di caricamento; 
		\item Il sistema disabilita la scelta di tutti i file che non rispettano l'estensione richiesta; 
		\item L'utente a seconda della directory in cui si trova, seleziona un file abilitato al caricamento; 
		\item L'utente conferma il file selezionato; 
		\item Il sistema prendere in carico il file selezionato e  lo inizializza. 
	\end{enumerate}
	& N.I \\
\hline

\caption{Test di validazione previsti}
\label{testvalidazioneprevisti}
\end{longtable}



% Questa sezione verrà completata nel momento in cui verranno svolti i test. La descrizione di questo tipo di test è riportata nel documento \textit{Norme di Progetto v2.0.0}, nell'appendice §D che tratta del \textit{Modello a V}.

\newpage
\section{Test di Integrazione}
\label{test_i}
Questa sezione verrà completata nel momento in cui verranno svolti i test. La descrizione di questo tipo di test è riportata nel documento \textit{Norme di Progetto v2.0.0}, nell'appendice §D che tratta del \textit{Modello a V}.

\newpage
\section{Test di Sistema}
\label{test_s}
Questa sezione verrà completata nel momento in cui verranno svolti i test. La descrizione di questo tipo di test è riportata nel documento \textit{Norme di Progetto v2.0.0}, nell'appendice §D che tratta del \textit{Modello a V}.
\newpage