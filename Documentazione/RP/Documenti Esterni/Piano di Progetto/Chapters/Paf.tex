\section{Consuntivo di Fine Periodo e Preventivo a Finire}
\label{CFPPAF}

Lo scopo di questa sezione è quello di stilare un rendiconto delle attività svolte fino ad ora, al fine di poter confrontare il preventivo stilato con l'effettivo numero di ore svolte da ogni membro del gruppo per ogni periodo.\\
Questa sezione verrà incrementata alla fine di ogni periodo.\\
Il conteggio verrà inserito in una tabella, così da rendere meglio visibili ore preventivate ed ore effettivamente svolte.\\
Le prime verranno rappresentate con un semplice numero mentre le seconde verranno rappresentate tra parentesi graffe, seguendo i criteri qui sotto riportati:
\begin{itemize}
	\item \textbf{Positivi}: sono state necessarie più ore di quelle preventivate;
	\item \textbf{Negativi}: sono state necessarie meno ore di quelle preventivate;
	\item \textbf{Invariato}: qualora le ore non dovessero cambiare, verrà semplicemente inserito il numero di ore preventivato.	
\end{itemize}

Inoltre, verranno anche aggiornati i dati preventivi andando a uniformarli con le ore effettivamente utilizzate durante la fase.\\
Questo ricalcolo è utile solamente al fine di poter verificare la qualità delle attività svolte dal gruppo e non andrà in alcun modo a variare i costi preventivati per la committente.

\newpage

\subsection{Consuntivo di Fine Periodo}
\label{CFP}
\subsubsection{Revisione dei Requisiti}

\paragraph{Prospetto Orario} \-\\

\begin{longtable}{|C{.30\textwidth}|C{.06\textwidth}|C{.06\textwidth}|C{.06\textwidth} | C{.06\textwidth}| C{.06\textwidth} | C{.06\textwidth} | C{.10\textwidth} |}
\hline
\rowcolor{bluelogo}	\textbf{\textcolor{white}{Nome}} & \textbf{\textcolor{white}{RE}} & \textbf{\textcolor{white}{AM}} & \textbf{\textcolor{white}{AN}} & \textbf{\textcolor{white}{PJ}} & \textbf{\textcolor{white}{PR}} & \textbf{\textcolor{white}{VE}} & \textbf{\textcolor{white}{Totale}}\\
\hline 
Marco Chilese & - & - & 10 \{+5\} & - & - & 15 \{-4\} & 25 \{+1\} \\
\hline
\rowcolor{grigio}Marco Favaro & - & - & 14 \{-3\} & - & - & 11 \{+4\} & 25 \{+1\} \\
\hline
Diego Mazzalovo & - & - & 18 \{-5\} & - & - & 7 \{+6\} & 25 \{+1\} \\
\hline
\rowcolor{grigio}Carlotta Segna & 13 \{+3\} & - & - & - & - & 12 \{-2\} & 25 \{+1\} \\
\hline
Matteo Slanzi & - & 10 \{-3\} & 8 \{+4\} & - & - & 7 & 25 \{+1\} \\
\hline
\rowcolor{grigio}Bogdan Stanciu & 15 \{-3\} & - & 10 \{+4\} & - & - & - & 25 \{+1\}\\
\hline
Luca Violato & - & 15 \{+3\} & 10 \{-2\} & - & - & - & 25 \{+1\} \\
\hline


\caption{Consuntivo di Fine Periodo: Avvio ed Analisi dei Requisiti}
\label{tab:cfp_aar}
\end{longtable}

\paragraph{Prospetto Economico} \-\\

\begin{longtable}{| C{.30\textwidth}| C{.15\textwidth}| C{.20\textwidth}|}
\hline
\rowcolor{bluelogo}\textbf{\textcolor{white}{Ruolo}} & \textbf{\textcolor{white}{Ore}} & \textbf{\textcolor{white}{Costo in \euro}} \\
\hline
Responsabile & 28 & \EUR{840.00} \\
\hline
\rowcolor{grigio}Amministratore & 25 & \EUR{500.00} \\
\hline
Analista & 73 \{+3\} & \EUR{1825.00} \{+\EUR{75.00}\}\\
\hline
\rowcolor{grigio}Progettista & - & - \\
\hline
Programmatore & - & - \\
\hline
\rowcolor{grigio}Verificatore & 56 \{+4\} & \EUR{840.00} \{+\EUR{60.00}\}\\
\hline
\textbf{Totale} & 182 \{+7\}& \EUR{4005.00} \{+\EUR{135.00}\}\\
\hline
\caption{Consuntivo di Fine Periodo dei Ruoli: Avvio ed Analisi dei Requisiti}
\label{tab:doaar}
\end{longtable}

\paragraph{Conclusioni} \-\\
Il numero di ore preventivato differisce da quelle realmente utilizzate in quanto, approcciandosi ad un nuovo modo di svolgere un progetto, il gruppo non è stato in grado di preventivare correttamente il numero di ore che si sarebbero poi andate, effettivamente, ad utilizzare. Ciò è dovuto al fatto che siano state utilizzate più ore, rispetto a quanto preventivato, per la stesura del \textit{Piano di Progetto v1.0.0}, andando così ad aumentare le ore di \textit{Responsabile}. Un discorso equivalente può essere effettuato per il ruolo di \textit{Analista} , in quanto il documento \textit{Analisi dei Requisiti v1.0.0} ha richiesto un'analisi più lunga ed approfondita di quanto precedentemente preventivato. \\
Il \textit{Verificatore} ha necessitato più ore in quanto i documenti sono stati rivisti più volte, al fine di controllare le modifiche che venivano effettuate. \\
Alla fine di questo periodo il numero si è quindi verificato un incremento di 7 ore totali, andando ad aggiungere \EUR{135.00} al costo totale della prima fase, a carico del gruppo. 

\subsubsection{Risanamento Criticità}
\label{RA1}

\paragraph{Prospetto Orario} \-\\

\begin{longtable}{|C{.30\textwidth}|C{.06\textwidth}|C{.06\textwidth}|C{.06\textwidth} | C{.06\textwidth}| C{.06\textwidth} | C{.06\textwidth} | C{.10\textwidth} |}
\hline
\rowcolor{bluelogo}	\textbf{\textcolor{white}{Nome}} & \textbf{\textcolor{white}{RE}} & \textbf{\textcolor{white}{AM}} & \textbf{\textcolor{white}{AN}} & \textbf{\textcolor{white}{PJ}} & \textbf{\textcolor{white}{PR}} & \textbf{\textcolor{white}{VE}} & \textbf{\textcolor{white}{Totale}}\\
\hline 
Marco Chilese & - & - & - & - & - & 3 \{-2\} & 3 \{-2\}\\
\hline
\rowcolor{grigio}Marco Favaro & - & - & 4 \{-1\} & - & - & - & 4 \{-1\} \\
\hline
Diego Mazzalovo & - & - & 4 \{-1\} & - & - & - & 4 \{-1\} \\
\hline
\rowcolor{grigio}Carlotta Segna & - & - & - & - & - & 3 \{-2\} & 3 \{-2\}\\
\hline
Matteo Slanzi & - & - & 3 \{-2\} & - & - & - & 3 \{-2\}\\
\hline
\rowcolor{grigio}Bogdan Stanciu & 3 \{-2\} & - & - & - & - & - & 3 \{-2\} \\
\hline
Luca Violato & - & 3 \{-2\}& - & - & - & - & 3 \{-2\} \\
\hline

\caption{Consuntivo di Fine Periodo: Risanamento Criticità 1}
\label{Distribuzione oraria del periodo di rc1}
\end{longtable}

\paragraph{Prospetto Economico} \-\\

\begin{longtable}{| C{.30\textwidth}| C{.15\textwidth}| C{.20\textwidth}|}
\hline
\rowcolor{bluelogo}\textbf{\textcolor{white}{Ruolo}} & \textbf{\textcolor{white}{Ore}} & \textbf{\textcolor{white}{Costo in \euro}} \\
\hline 
Responsabile & 3 \{-2\} & \EUR{90.00} \{-\EUR{60.00}\}\\
\hline
\rowcolor{grigio}Amministratore & 3 \{-2\} & \EUR{60.00} \{-\EUR{40.00}\} \\
\hline
Analista & 11 \{-4\} & \EUR{275.00} \{-\EUR{100.00}\} \\
\hline
\rowcolor{grigio}Progettista & - & - \\
\hline
Programmatore & - & - \\
\hline
\rowcolor{grigio}Verificatore & 6 \{-4\} & \EUR{90.00} \{-\EUR{60.00}\}\\
\hline
\textbf{Totale} & 23 \{-12\} & \EUR{525.00} \{-\EUR{260.00}\} \\
\hline
\caption{Consuntivo di Fine Periodo dei ruoli: Risanamento Criticità 1}
\label{Distribuzione oraria Ruoli del Periodo di rc1}
\end{longtable}

\paragraph{Conclusioni} ~\\

Durante il periodo di \textit{Risanamento Criticità}, il monte orario preventivato si è dimostrato superiore alle reali necessità. Durante il periodo sono stati corretti gli errori segnalati alla fine del periodo precedente. Le correzioni si sono svolte sulla documentazione contente errori. Un impegno maggiore è stato richiesto dal documento \textit{Analisi dei Requisiti v1.0.0}, corretto dagli \textit{Analisti}. \\

\newpage

\subsubsection{Progettazione Architetturale}
\label{pa}

\paragraph{Prospetto Orario} ~\\

\begin{longtable}{|C{.30\textwidth}|C{.06\textwidth}|C{.06\textwidth}|C{.06\textwidth} | C{.06\textwidth}| C{.06\textwidth} | C{.06\textwidth} | C{.10\textwidth} |}
\hline
\rowcolor{bluelogo}	\textbf{\textcolor{white}{Nome}} & \textbf{\textcolor{white}{RE}} & \textbf{\textcolor{white}{AM}} & \textbf{\textcolor{white}{AN}} & \textbf{\textcolor{white}{PJ}} & \textbf{\textcolor{white}{PR}} & \textbf{\textcolor{white}{VE}} & \textbf{\textcolor{white}{Totale}}\\
\hline 
Marco Chilese & - & 12 \{+2\} & - & 12 & - & - & 24 \{+2\} \\
\hline
\rowcolor{grigio}Marco Favaro & 5 & - & 18 \{+1\}  & - & - & - & 23 \{+1\} \\
\hline
Diego Mazzalovo & 6 \{-2\}& - & - & 17 \{+3\} & - & - & 23 \{+1\} \\ 
\hline
\rowcolor{grigio}Carlotta Segna & - & - & - & 10 & 13 \{+1\} & - & 23 \{+1\} \\
\hline
Matteo Slanzi & - & - & - & - & 7 \{-4\} & 9 \{-2\} & 16 \{-6\} \\
\hline
\rowcolor{grigio}Bogdan Stanciu & - & 8 \{-2\}& - & - & 15 \{+3\} & - & 23 \{+1\} \\
\hline
Luca Violato & - & - & 11 \{+1\} & - & - & 14 \{+2\} & 25 \{+3\} \\
\hline 

\caption{Consuntivo di Fine Periodo: Progettazione Architetturale}
\label{Distribuzione oraria del periodo di pa}
\end{longtable}


\paragraph{Prospetto Economico} ~\\

\begin{longtable}{| C{.30\textwidth}| C{.15\textwidth}| C{.20\textwidth}|}
\hline
\rowcolor{bluelogo}\textbf{\textcolor{white}{Ruolo}} & \textbf{\textcolor{white}{Ore}} & \textbf{\textcolor{white}{Costo in \euro}} \\
\hline 
Responsabile & 11 \{-2\}& \EUR{330.00} \{-\EUR{60.00}\} \\
\hline
\rowcolor{grigio}Amministratore & 20 & \EUR{400.00} \\
\hline
Analista & 29 \{+2\} & \EUR{725.00} \{+\EUR{50.00}\}\\
\hline
\rowcolor{grigio}Progettista & 39 \{+3\} & \EUR{858.00} \{+\EUR{66.00}\}\\
\hline
Programmatore & 35 & \EUR{525.00} \\
\hline
\rowcolor{grigio}Verificatore & 23 & \EUR{345.00}\\
\hline
\textbf{Totale} & 157 \{+3\} & \EUR{3183.00} \{+\EUR{56.00}\}\\ 
\hline

\caption{Consuntivo di Fine Periodo dei Ruoli: Progettazione Architetturale}
\label{Distribuzione oraria per ruoli del periodo di pa}
\end{longtable}

\paragraph{Conclusioni} ~\\
L'incremento del documento \textit{Analisi dei Requisiti v2.0.0}, ha richiesto un numero di ore maggiore da parte degli \textit{Analisti} per il suo incremento. L'utilizzo di nuovo tecnologie ha portato gli \textit{Amministratori}, che avevano già una precedente conoscenza di queste. La preparazione del \textit{PoC}\glossario ha richiesto maggior tempo rispetto a quanto preventivato e, quindi, le ore di \textit{Programmatori} e \textit{Progettisti} sono aumentate. 

\pagebreak

\subsection{Preventivo a Finire}\label{caption_paf}

Nella seguente sezione inseriremo il preventivo a finire. Qualora non fosse presente il consuntivo di fine periodo utilizzeremo il preventivo. \\

\begin{longtable}{| C{.30\textwidth}| C{.15\textwidth}| C{.20\textwidth}|}
\hline
\rowcolor{bluelogo}\textbf{\textcolor{white}{Periodo}} & \textbf{\textcolor{white}{Preventivo in \euro}} & \textbf{\textcolor{white}{Consuntivo in \euro}} \\
\hline
Avvio ed Analisi dei Requisiti & \EUR{3870.00} & \EUR{4005.00} \\
\hline
\rowcolor{grigio}Risanamento Criticità & \EUR{775.00}  & \EUR{ 515.00} \\
\hline
Progettazione Architetturale & \EUR{3127.00} & \EUR{3183.00} \\
\hline
\rowcolor{grigio} Risanamento Criticità & \EUR{426.00} & \EUR{426.00} \\
\hline
Progettazione di Dettaglio e Codifica & \EUR{6669.00} & \EUR{6464.00} \\
\hline
\rowcolor{grigio} Risanamento Criticità &  \EUR{426.00} & \EUR{526.00} \\
\hline
Validazione e Collaudo & \EUR{2620.00}  & \EUR{2448.00} \\
\hline
\rowcolor{grigio}\textbf{Totale} & \EUR{14043.00}  & \textbf{\EUR{13634.00}}  \\
\hline
\caption{Preventivo a Finire}
\label{paf}
\end{longtable}

\subsubsection{Conclusioni}
Il costo finale del prodotto si attesta, quindi, ad \EUR{13634.00}, con una riduzione di \EUR{409.00} rispetto a quanto inizialmente preventivato. Tale differenza è dovuta principalmente a causa di un apporto lavorativo inferiore rispetto a quello preventivato da parte di un membro del gruppo. 

