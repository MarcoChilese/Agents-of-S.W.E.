\subsection{Preventivo a Finire}
\label{caption_paf}

Nella seguente sezione inseriremo il preventivo a finire. Qualora non fosse presente il consuntivo di fine periodo utilizzeremo il preventivo. \\
Dove non presenti i costi finali del periodo, utilizzeremo i valori preventivati. \\

\begin{longtable}{| C{.30\textwidth}| C{.15\textwidth}| C{.20\textwidth}|}
\hline
\rowcolor{bluelogo}\textbf{\textcolor{white}{Periodo}} & \textbf{\textcolor{white}{Preventivo in \euro}} & \textbf{\textcolor{white}{Consuntivo in \euro}} \\
\hline
Avvio ed Analisi dei Requisiti & \EUR{3870.00} & \EUR{4005.00} \\
\hline
\rowcolor{grigio}Risanamento Criticità & \EUR{775.00}  & \EUR{ 515.00} \\
\hline
Progettazione Architetturale & \EUR{3127.00} & \EUR{3183.00} \\
\hline
\rowcolor{grigio} Risanamento Criticità & \EUR{426.00} & - \\
\hline
Progettazione di Dettaglio e Codificica & \EUR{6669.00} & - \\
\hline
\rowcolor{grigio} Risanamento Criticità &  \EUR{426.00} & - \\
\hline
Validazione e Collaudo & \EUR{2620.00}  & - \\
\hline
\rowcolor{grigio}\textbf{Totale} & \EUR{17913.00}  & \EUR{17844.00}  \\
\hline
\caption{Preventivo a Finire}
\label{paf}
\end{longtable}

