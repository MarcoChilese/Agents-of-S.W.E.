\section{Resoconto}

\subsection{Punto 1}
Al gruppo è stato fornito il materiale richiesto in precedenza tramite e-mail.
Nello specifico il team aveva richiesto di poter visionare il lavoro svolto da uno studente in stage\glossario presso l'azienda proponente. Lo scopo di questa richiesta era quello di poter comprendere ed analizzare meglio le funzionalità messe a disposizione dalla libreria proposta dal committente per la definizione ed i calcoli riguardanti le reti bayesiane.\\
Il materiale in questione è stato consegnato al gruppo attraverso l'uso di una chiavetta USB.

\subsection{Punto 2}
Il gruppo ha descritto nel dettaglio le funzionalità che il prodotto finale metterà a disposizione dell'utente. Sono stati dunque presentati in modo accurato i casi d'uso fondamentali atti a coprire i principali requisiti obbligatori richiesti dal committente. A tal fine il gruppo ha fatto uso di qualche semplice immagine in formato \textit{png}\glossario per rendere meglio l'idea anche a livello di rappresentazione visiva.\\
Nello specifico sono stati descritti con particolare enfasi i casi d'uso UC1, UC2, UC3 e UC4, descritti nel documento \textit{Analisi dei Requisiti}.

\subsection{Punto 3}
Il committente non ha avuto particolari perplessità riguardanti la descrizione del prodotto in divenire esposta dal gruppo, e anzi ha evidenziato particolare interesse verso il caso d'uso UC2.2.\\
Più nello specifico il committente ha apprezzato l'idea di dover definire livelli di soglia associati ai nodi da collegare al flusso dati. Tali livelli di soglia consentirebbero all'utente, nell'ottica del funzionamento della rete bayesiana, di poter definire in dettaglio l'osservazione di una variabile di evidenza.\\
Riflettendo su questa prospettiva, e considerando anche quanto descritto nel caso d'uso UC3, il committente ha avanzato una richiesta aggiuntiva riguardante il ricalcolo delle probabilità della rete bayesiana.
Nello specifico il committente ha richiesto la possibilità che, fermo restando la politica temporale definita dall'utente del plug-in nel caso d'uso UC3, il sistema effettui il ricalcolo delle probabilità della rete bayesiana al verificarsi del superamento di una qualunque soglia associata ad un nodo collegato al flusso di monitoraggio (UC2.2).\\
Sempre discutendo dello stesso caso d'uso (UC2.2) il committente si è dimostrato favorevole all'idea iniziale presentata dal gruppo, la quale prevede di lasciare totale libertà all'utente riguardo la scelta del flusso dati  a cui collegare ogni nodo. In tale ottica, pur presupponendo una certa consapevolezza e competenza da parte dell'utente, il commitente ha posto la condizione che il nominativo stesso del nodo debba essere sufficientemente autoesplicativo a riguardo del flusso dati a cui deve essere collegato. In tal senso il gruppo ha concordato con il committente che tale compito sia da attribuirsi agli esperti che si occuperanno della definizione effettiva della rete bayesiana in formato \textit{json}\glossario.