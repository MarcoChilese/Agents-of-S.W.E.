\section*{B}
\addcontentsline{toc}{section}{B}

\subsection{\index{Baseline}Baseline}
Una specifica o un prodotto, formalmente esaminato e su cui si è giunti ad un accordo, che da quel momento serve come base per ulteriore sviluppo. Può essere modificato solo attraverso procedure di controllo delle modifiche che devono essere approvate in modo formale.

\subsection{\index{Best Practice}Best Practice}
Nota metodologia che, secondo l'esperienza professionale o studi autorevoli, abbia mostrato di garantire i migliori risultati in circostanze note e specifiche.

\subsection{\index{Blockchain}Blockchain}
È un protocollo di comunicazione che implementa una tecnologia peer-to-peer basata su un database distribuito tra gli utenti.

\subsection{\index{Branch}Branch}
Termine che viene usato nel sistema di versionamento Git, è un ramo di lavoro che permette l'implementazione di funzionalità tra loro isolate, cioè sviluppate in modo indipendente l'una dall'altra a partire dalla medesima radice.

\subsection{\index{Build Automation}Build Automation}
Il processo di Build Automation è il processo di automatizzazione della build del software ed i processi correlati, quali: compilazione del codice sorgente in codice binario, impacchettamento del codice binario e l'esecuzione dei test automatici.  

\subsection{\index{Build System}Build System}
Sistema che consente la Build Automation.