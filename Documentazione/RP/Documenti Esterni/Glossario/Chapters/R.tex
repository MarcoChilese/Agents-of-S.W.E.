\section*{R}
\addcontentsline{toc}{section}{R}

\subsection{\index{React}React}
È una libreria JavaScript per la creazione di interfacce grafiche.

\subsection{\index{Redmine}Redmine}
È una piattaforma web open source che permette la gestione di progetti.

\subsection{\index{Redux}Redux}
È una libreria JavaScript per la scrittura di applicazione web basate su React.

\subsection{\index{Repository}Repository}
Struttura che memorizza metadati per un set di file o strutture di directory utilizzato dal sistema di versionamento, centralizzato o distribuito, che è accessibile a più utenti, i quali possono clonare il repository e modificare i file al suo interno per poi reinserirli nel server dove erano presenti.

\subsection{\index{Repository Manager}Repository Manager}
Un proxy per repository remoti che memorizza nella memoria cache gli artefatti risparmiando sia la larghezza di banda che il tempo richiesto per recuperare un artefatto software da un repository remoto.

\subsection{\index{Requisiti}Requisiti}
Capacità che devono essere possedute, oppure condizioni che devono essere soddisfatte, da un sistema per adempiere ad un obbligo.

\subsection{\index{Rete Bayesiana}Rete Bayesiana}
Rappresentazione grafica delle relazioni di dipendenza tra le variabili di un sistema. In statistica la rete bayesiana è utilizzata per individuare più agevolmente le relazioni di dipendenza assoluta e condizionale tra le variabili, al fine di ridurre il numero delle combinazioni delle variabili da analizzare.