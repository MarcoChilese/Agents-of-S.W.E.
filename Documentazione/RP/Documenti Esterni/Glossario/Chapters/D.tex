\section*{D}
\addcontentsline{toc}{section}{D}

\subsection{\index{Dashboard}Dashboard}
Nella Software Engineering con tale termine (che letteralmente potrebbe essere tradotto come "Cruscotto") si intente solitamente una pagina  informatica dedicata alla visualizzazione, anche eventualemte storica, di metriche, dati o informazioni, allo scopo di comprendere l'andamento di un progetto.

\subsection{\index{Database}Database}
È una collezione strutturata di dati che hanno una relazione tra loro.

\subsection{\index{DevOps}DevOps}
Acronimo di Development e Operations, DevOps è un approccio allo sviluppo e all’implementazione di applicazioni in azienda, che enfatizza la collaborazione tra il team di sviluppo vero e proprio e quello delle operations, ossia che gestirà le applicazioni dopo il loro rilascio.

\subsection{\index{Diagrammi di Gantt}Diagrammi di Gantt}
Il diagramma di Gantt serve a pianificare un insieme di attività in un certo periodo di tempo. La sua struttura è quella di un semplice diagramma cartesiano in cui le ascisse rappresentano una scala temporale, e le ordinate le cose da fare per portare a termine il progetto. Il tempo per svolgere le attività viene rappresentato con una barra colorata che va dalla data di inizio alla data di fine dell'attività.

\subsection{\index{Digital Ocean}Digital Ocean}
DigitalOcean, è un fornitore di cloud computing americana, fornisce agli sviluppatori servizi cloud che aiutano a distribuire e scalare applicazioni che vengono eseguite simultaneamente su più computer

\subsection{\index{Docker}Docker}
È una piattaforma software che permette di creare build, testare e distribuire applicazioni in tempi brevi. Docker raccoglie il software in unità chiamate container che offrono tutto il necessario per la loro corretta esecuzione.

\subsection{\index{DOM}DOM}
Acronimo di Document Object Model,  è una forma di rappresentazione dei documenti strutturati come modello orientato agli oggetti. Un esempio di albero DOM è quello generato da un browser web nell'interpretazione di un documento HTML.

\subsection{\index{DynamoDB}DynamoDB}
È un servizio di AWS per database non relazionali che offre prestazioni di pochi millisecondi a qualsiasi livello.
