\section{Introduzione}\label{Intro}

\subsection{Scopo del Documento}
Il presente documento ha come obiettivo l'esposizione completa e dettagliata dei casi d'uso\glossario e corrispondenti requisiti\glossario individuati a seguito dell'analisi del capitolato C3, proposto dalla Zucchetti S.p.A.\\
Le informazioni contenute nel presente documento derivano dalla comprensione del capitolato precedentemente citato e dagli incontri con la proponente.

\subsection{Ambiguità e Glossario}
I termini che potrebbero risultare ambigui all'interno del documento sono siglati tramite pedice rappresentante la lettera \textmd{G}, tale terminologia trova una sua più specifica definizione nel \textit{Glossario v1.0.0} che viene fornito tra i Documenti Esterni.

\subsection{Riferimenti}\label{Riferimenti}
\subsubsection{Referimenti Normativi}
\begin{itemize}
\item \textbf{\textit{Norme di Progetto v1.0.0}};
\item \textbf{Verbale esterno con la proponente del 10 Dicembre 2018}: \-\\ "Verbale 2018-12-10.pdf";
\item \textbf{Capitolato d'Appalto C3:}\\ \url{https://www.math.unipd.it/~tullio/IS-1/2018/Progetto/C3.pdf}.
\end{itemize}

\subsubsection{Referimenti Informativi}
\begin{itemize}
\item \textbf{Presentazione Capitolato:}\\ \url{https://www.math.unipd.it/~tullio/IS-1/2018/Progetto/C3p.pdf};
\item \textbf{Materiale didattico del corso di Ingegneria del Software:}
	\begin{itemize}
	\item \textbf{Analisi dei Requisiti}:\\ \url{https://www.math.unipd.it/~tullio/IS-1/2018/Dispense/L08.pdf};
	\item \textbf{Diagrammi dei Casi d'Uso}:\\ \url{https://www.math.unipd.it/~tullio/IS-1/2018/Dispense/E05b.pdf}.
	\end{itemize}
\end{itemize}
