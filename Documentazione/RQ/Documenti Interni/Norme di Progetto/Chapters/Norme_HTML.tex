\label{html}\-\\

Insieme al \textit{CSS3}, per lo sviluppo della parte front-end del progetto, il gruppo ha scelto di utilizzare \textit{HTML5}. Per la stesura della Style Guide verrà utilizzata la documentazione resa disponibile da W3C.\\
Di seguito saranno riportate le principali norme da seguire: \\
\-\\
\textbf{Norma 1}\\
Il tipo di documento deve sempre essere dichiarato all'inizio del documento in maiuscolo, in questo modo:\\
	\begin{lstlisting}[language=HTML]
<!DOCTYPE html>
	\end{lstlisting}

\textbf{Norma 2}\\
I nomi degli elementi ed attributi devono essere sempre scritti in minuscolo: \\
	\begin{lstlisting}[language=HTML]
<tag attr1="value"/> <!-- OK -->

<TAG ATTR1="value"/> <!-- NO -->
\end{lstlisting}
\-\\
\textbf{Norma 3}\\
Tutti gli elementi \textit{HTML} devono essere chiusi, anche se vuoti: \\
	\begin{lstlisting}[language=HTML]
<tag /> 	<!-- OK -->
<tag></tag> 	<!-- OK -->

<tag> 		<!-- NO -->

\end{lstlisting}
\-\\
\textbf{Norma 4}\\
Il valore degli attributi deve essere racchiuso dentro doppio apice:\\
	\begin{lstlisting}[language=HTML]
<tag attr1="value"/> <!-- OK -->

<tag attr1='value'/> <!-- NO -->
\end{lstlisting}
\-\\
\textbf{Norma 5}\\
Intorno al simbolo di uguaglianza (=) non devono essere presenti spazi:\\
\begin{lstlisting}[language=HTML]
<tag attr1="value"/> <!-- OK -->

<tag attr1 = 'value'/> <!-- NO -->
\end{lstlisting}
\-\\
\textbf{Norma 6}\\
Devono essere usati due (2) spazi e non le tabulazioni. Gli spazi bianchi vanno utilizzati solamente per ampli blocchi di codice, per una maggiore leggibilità; \\
\begin{lstlisting}[language=HTML]
<!-- OK -->
<rootTag>
..<childTag>
....<subChildTag/>
..</childTag>  
</rootTag>

<!-- NO -->
<rootTag>
	<childTag>
		<subChildTag/>
	</childTag>  
</rootTag>
\end{lstlisting}
\-\\
\textbf{Norma 7}\\
Nonostante i tag html, body e head possano essere omessi, vanno inseriti affinché siano compatibili con tutti i browser:\\
\begin{lstlisting}[language=HTML]
<!-- OK -->
<html>
  <head>
    ...
  </head>
  <body>
    ...
  </body>
</html>
\end{lstlisting}
\-\\
\textbf{Norma 8}\\
Le classi che verranno dichiarate all'interno dell'\textit{HTML}, al fine di andare poi a ridefinire il \textit{CSS}, dovranno essere quelle rese disponibili dalla documentazione di \textit{Grafana}.	\\

Equivalentemente al \textit{CSS}, anche il codice \textit{HTML} dovrà essere validato. Questa validazione avverrà tramite il sito reso disponibile dalla W3C, al seguente \href{https://validator.w3.org/}{link}.

%AGGIUNGERE E SISTEMARE DA QUA IN GIU