Seguendo le indicazioni presenti nella documentazione dell'azienda fornitrice di \textit{Grafana}, la piattaforma  per cui si intende sviluppare il plug-in, il team ha deciso di adottare come linguaggio di programmazione principale \textit{ECMAScript 6}\footnote{Linguaggio divenuto standard ISO: ISO/IEC 16262:2011, e relativo aggiornamento ISO/IEC 22275:2018.}.\\
\textit{ECMAScript 6} viene stardardizzato da \textit{ECMA}\glossario nel giugno 2015 con la sigla "ECMA-262".\\
Come stile di codifica si adottano le linee guida proposte da \textit{Airbnb JavaScript Style Guide}. Per la verifica dell'adesione a tali norme, i \textit{Programmatori} devono utilizzare, come suggerito dalla documentazione proposta da \textit{Grafana}, \textit{ESLint}\glossario.\\
In particolare i \textit{Programmatori} devono rispettare 5 linee guida proposte dalla documentazione ufficiale di \textit{Grafana}:
\begin{enumerate}
	\item Se una variabile non viene riutilizzata, deve essere dichiarata come \texttt{const};
	\item Utilizzare preferibilmente, per la definizione di variabili, la keyword  \texttt{let}, anziché  \texttt{var};
	\item Utilizzare il marcatore freccia (\texttt{\textbf{=>}}), in quanto non oscura il \texttt{this}:
	\begin{lstlisting}[language=JavaScript]
testDatasource() {
  return this.getServerStatus()
  .then(status => {
    return this.doSomething(status);
  })
}	
	\end{lstlisting}
	Invece che:
	\begin{lstlisting}[language=JavaScript]
testDatasource() {
  var self = this;
  return this.getServerStatus()
    .then(function(status) {
  return self.doSomething(status);
  })
}
	\end{lstlisting}
	\item Utilizzare l'oggetto \textit{Promise}:
	\begin{lstlisting}[language=JavaScript]
metricFindQuery(query) {
  if (!query) {
    return Promise.resolve([]);
  }
}	
	\end{lstlisting}
	Invece che:
	\begin{lstlisting}[language=JavaScript]
metricFindQuery(query) {
  if (!query) {
    return this.$q.when([]);
  }
}
	\end{lstlisting}
	%conseguenti?
	\item Se si utilizza \textit{Lodash}\glossario, per coerenza, lo si preferisca alle array function native di \textit{ES6}.
\end{enumerate}
Verranno esaminate di seguito le norme in merito allo stile di codifica che i \textit{Programmatori} dovranno adottare.

\subparagraph{Indentazione}\-\\
\textbf{Norma 1}\\
L'indentazione è da eseguirsi con tabulazione la cui larghezza sia impostata a due (2) spazi per ogni livello.\\
Di seguito un esempio da ritenersi corretto:
\begin{lstlisting}[language=JavaScript]
function() {
..let x = 2;
..if (x > 0)
....return true;
..else
....return false;
}
\end{lstlisting}
Qualsiasi altro tipo di indentazione è da ritenersi scorretta.\\
\-\\
\textbf{Norma 2}\\
Dopo la graffa principale va inserito uno (1) spazio. Nel seguente modo:
\begin{lstlisting}[language=JavaScript]
function() { ... }
\end{lstlisting}
\-\\
\textbf{Norma 3}\\
Dopo la keyword di un dato statement (\texttt{if, while}, etc.) va inserito uno (1) spazio. Per un esempio corretto si veda la norma successiva.\\
\-\\
\textbf{Norma 4}\\
Prima dell'apertura della graffa negli statement di controllo va inserito uno (1) spazio. Nel seguente modo:
\begin{lstlisting}[language=JavaScript]
function() {
  if (condition) {
    ...  
  }
  while (condition) {
    ...
  }
}
\end{lstlisting}
\-\\
\textbf{Norma 5}\\
Negli statement di controllo (\texttt{if, while}, etc.) le condizioni concatenate o annidate, mediante operatori logici, che diventano eccessivamente lunghe NON vanno espresse in un'unica linea, bensì spezzate in più righe. Nel seguente modo:
\begin{lstlisting}[language=JavaScript]
function() {
  if (condition && condition) {
    ...  
  }
  
  if (
   veryLongCondition
   && longCondition
   && condition
    ) {
    doSomething();
  }
}
\end{lstlisting}
\-\\
\textbf{Norma 6}\\
Dopo blocchi, o prima di un nuovo statement va lasciata una riga vuota. Nel seguente modo:
\begin{lstlisting}[language=JavaScript]
function1() {
  if (condition) {
    doSomething():  
  }
  
  return toReturn;  
}

function2(){
  ...
}
\end{lstlisting}
\-\\
\textbf{Norma 7}\\
I blocchi di codice multi-riga devono essere contenuti all'interno di graffe. Blocchi costituiti da una singola riga non è necessario che siano contenuti tra graffe: nel caso non vengano utilizzate, la definizione deve essere inline, cioè sulla stessa riga.\\
Nel seguente modo:
\begin{lstlisting}[language=JavaScript]
if (condition) return true;

if (condtion) {
  return true;
}
\end{lstlisting}

\subparagraph{Commenti al Codice}\-\\
\textbf{Norma 1}\\
Il codice va commentato nel seguente modo:
\begin{itemize}
	\item "\texttt{//}" se il commento occupa una sola riga;
	\item "\texttt{/** ... */} " se il commento occupa più righe.
\end{itemize}
Nel seguente modo:
\begin{lstlisting}[language=JavaScript]
// single line comment
if (condition) return true;

/**
* multi line comment, line 1
* multi line comment, line 2
*/
if (condtion) {
  return true;
}
\end{lstlisting}
\textbf{Norma 2}\label{Docs_Codice}\\
Il codice sviluppato va adeguatamente commentato con i commenti di documentazione appositi:
\begin{lstlisting}[language=JavaScript]
/**
 * Function or class description ...
 */
\end{lstlisting}
Il team fa riferimento a quanto indicato dal sito \textit{@use JSDoc}.\\
In particolari vengono ritenuti necessari, qualora utilizzato, il riferimento alle seguenti parti:
\begin{itemize}
	\item \textbf{@extends}: da utilizzare nell'intestazione della classe se la stessa ne estende un'altra;
	\item \textbf{@param}: da utilizzare per descrivere l'uso di un parametro di una funzione;
	\item \textbf{@return}: da utilizzare per descrivere il tipo di valore ritornato da una funzione; 
	\item \textbf{@module}: da utilizzare per indicare un modulo importato;
	\item \textbf{@throws}: da utilizzare per descrivere quali eccezioni possono essere lanciate.
\end{itemize}
Di seguito un esempio di utilizzo:
\begin{lstlisting}[language=JavaScript]
**
* Class representing a dot.
* @extends Point
*/
class Dot extends Point {
  /**
  * Create a dot.
  * @param {number} x - The x value.
  * @param {number} y - The y value.
  * @param {number} width - The width of the dot, in pixels.
  */
  constructor(x, y, width) {
    // ...
  }

  /**
  * Get the dot's width.
  * @return {number} The dot's width, in pixels.
  */
  getWidth() {
    // ...
  }
  
  /**
  * Set the dot's width.
  * @param {number} width - The width of the dot, in pixels.
  * @throws {Error} Will throw an error if the argument is lower than 0.
  */
  setWidth(w) {
    if(w < 0) throw new Error('Negative width');
    this.width = w;
  }
}
\end{lstlisting}

\subparagraph{Variabili}\-\\
\textbf{Norma 1}\\
Fare riferimento alle norme 1 e 2, all'inizio della sezione §\ref{EcmaScript6}.

\-\\
\textbf{Norma 2}\\
Non utilizzare dichiarazioni multiple di variabili, dichiarare una variabile per riga.\\
Nel seguente modo:
\begin{lstlisting}[language=JavaScript]
// OK
var x = 1;
var y = 0;

// NO
var x = 1, y = 0;
\end{lstlisting}


%va linkato il paragrafo
\subparagraph{Nomi}\-\\
\textbf{Norma 1}\\
 Oltre a quanto enunciato nel secondo punto del paragrafo §\ref{Nomi}, tutti i nomi di funzioni o variabili composti da una singola lettera, o che indichino temporaneità della variabile sono vietati: ogni nome deve essere significativo.\\
\-\\
\textbf{Norma 2} 
\begin{enumerate}
	\item I nomi delle variabili, funzioni ed istanze devono utilizzare il CamelCase;
	\item I nomi delle classi deve avere lo stile \texttt{capWords}.
\end{enumerate}
Nel seguente modo:
\begin{lstlisting}[language=JavaScript]
// OK
var thisIsAVariable;

function thisIsAFunction() { ... }

class ThisIsAClass() {
  ...
}

// NO
var Variable;

function Function() { ... }

class myClass() {
  ...
}
\end{lstlisting}
