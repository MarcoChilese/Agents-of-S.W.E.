\section{Introduzione}\label{Intro}

\subsection{Scopo del Documento}
Il documento \textit{Norme di Progetto v2.0.0} si propone di definire le regole che i membri del gruppo \texttt{Agents of S.W.E.} sono tenuti a rispettare durante tutto lo svolgimento del progetto "\textit{G\&B}", al fine di garantire quanto più possibile l'uniformità del materiale prodotto.\\
Le norme presenti in questo documento verranno prodotte incrementalmente, al progressivo maturare delle esigenze e delle attività di progetto. Di conseguenza il documento, allo stato corrente, non è da considerarsi completo.\\
Il documento sarà perciò aggiornato e sottoposto a più revisioni.

\subsection{Ambiguità e Glossario}
I termini che potrebbero risultare ambigui all'interno del documento sono siglati tramite pedice rappresentante la lettera \textmd{G}, tale terminologia trova una sua più specifica definizione nel \textit{Glossario v2.0.0} che viene fornito tra i Documenti Esterni.

\subsection{Riferimenti}\label{Riferimenti}
\subsubsection{Referimenti Normativi}
\begin{itemize}
\item \textbf{Capitolato d'Appalto C3:}\\ \url{https://www.math.unipd.it/~tullio/IS-1/2018/Progetto/C3.pdf};
\item \textbf{ISO/IEC 9126:2001:}\\ \url{https://www.math.unipd.it/~tullio/IS-1/2018/Dispense/L13.pdf};
\item \textbf{Modello a V:}
\begin{itemize}
	\item \url{https://www.math.unipd.it/~tullio/IS-1/2018/Dispense/L16.pdf};
	\item \url{https://www.hintsw.com/it/safety-engineering/pianificazione-e-concezione-del-sw/modello-a-v-di-sviluppo-del-sw.html}.
\end{itemize}
\item \textbf{Metriche:}
	\begin{itemize}
	\item \url{https://www.qasymphony.com/blog/64-test-metrics};
	\item \url{https://www.softwaretestinghelp.com/software-test-metrics-and-measurements/}.
	\end{itemize}
\item \textbf{Codifica:}
	\begin{itemize}
	\item \url{http://www.ecma-international.org/};
	\item \url{https://www.ecma-international.org/ecma-262/6.0/};
	\item \url{https://github.com/airbnb/javascript};
	\item \url{https://eslint.org/};
	\item \url{http://airbnb.io/javascript/css-in-javascript};
	\item \url{http://usejsdoc.org};
	\item \url{https://www.w3schools.com/html/html5_syntax.asp};
	\item \url{http://docs.grafana.org/plugins/developing/code-styleguide/};
	\item \url{http://docs.grafana.org/plugins/developing/development/};
	\item \url{https://www.npmjs.com/package/eslint}.
	\end{itemize}
\item \textbf{Ambiente:}
	\begin{itemize}
		\item \url{http://www.xm1math.net/texmaker/}.
	\end{itemize}
\item \textbf{Leggibilità:}
	\begin{itemize}
		\item \url{https://it.wikipedia.org/wiki/Indice\_Gulpease}.
	\end{itemize}
\item \textbf{Diagrammi UML:}
\begin{itemize}
	\item \url{https://www.draw.io/}.
\end{itemize}

\end{itemize}

\subsubsection{Referimenti Informativi}
\begin{itemize}
\item \textbf{Presentazione Capitolato:}\\ \url{https://www.math.unipd.it/~tullio/IS-1/2018/Progetto/C3p.pdf};
\item \textbf{Materiale Didattico del Corso di Ingegneria del Software:}
	\begin{itemize}
	\item \textbf{Gestione di Progetto:}\\ \url{https://www.math.unipd.it/~tullio/IS-1/2018/Dispense/L06.pdf};
	\item \textbf{Regole del Progetto Didattico:}\\ \url{https://www.math.unipd.it/~tullio/IS-1/2018/Dispense/P01.pdf};
	\item \textbf{Regolamento Organigramma:}\\ \url{https://www.math.unipd.it/~tullio/IS-1/2018/Progetto/RO.html}.
	\end{itemize}
\item \textbf{Sviluppo:}
	\begin{itemize}
	\item \url{https://www.influxdata.com/time-series-platform/telegraf/};
	\item \url{https://www.influxdata.com/time-series-platform/influxdb/};
%	\item \url{https://sourceforge.net/projects/unbbayes/};
	\item \url{https://www.npmjs.com/};
	\item \url{https://jestjs.io/};
	\item \url{https://gitlab.com};
	\item \url{https://www.jetbrains.com/webstorm/};
	\item \url{https://webpack.js.org/}.
	\item \url{https://angularjs.org/};
	\item \url{http://codecov.io};
	\item \url{https://jquery.com/};
	\item \url{http://validator.w3.org/};
	
	\end{itemize}
\item \textbf{Ambiente di Test del Prodotto:}
\begin{itemize}
	\item \url{https://www.digitalocean.com/}.
\end{itemize}
\end{itemize}

All'interno del documento eventuali ulteriori riferimenti normativi o informativi vengono contrassegnati da apice\footnote{} e riportati a piè pagina.