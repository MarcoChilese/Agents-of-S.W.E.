\textit{NodeJS} è una runtime di \textit{JavaScript} Open source multipiattaforma orientato agli eventi per l'esecuzione di codice \textit{JavaScript} Server-side, costruita sul motore \textit{JavaScript V8} di \textit{Google Chrome}. Molti dei suoi moduli base sono scritti in \textit{JavaScript}, e gli sviluppatori possono scrivere nuovi moduli in \textit{JavaScript}\\
In particolare, tale tecnologia viene utilizzata dal team \texttt{Agents of S.W.E.} per l'implementazione del server che permette il funzionamento del \textit{plug-in}. \\
Le norme di codifica da seguire per NodeJS sono le stesse per ECMAScript 6 con ulteriori norme specifiche riportate di seguito:

\textbf{Norma 1} \\
Il codice deve essere partizionato in componenti ognuno nella propria cartella, garantendo così che ogni unità sia piccola e semplice. 
\-\\

\textbf{Norma 2} \\
Usare Async-Await\glossario o promise\glossario per la gestione di errori asincroni, che consente di avere un codice più compatto e familiare.
\-\\

\textbf{Norma 3} \\
Richiedere i moduli all'inizio di ogni file, evitando di richiederli all'interno di funzioni, permettendo così di individuare facilmente le dipendenze all'inizio del file.
\-\\

\textbf{Norma 4} \\
