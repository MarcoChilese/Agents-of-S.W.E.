\subsubsection{Sistema Operativo}
	I sistemi operativi utilizzati dai membri del gruppo sono i seguenti:
	\begin{itemize}
	\item Windows 10 x64;
	\item Mac OS 10.14.x;
	\item Manjaro Linux - 4.14.85-1 LTS.
	\end{itemize}

\subsubsection{Ambiente di Test del Prodotto}\label{Server}
Il team \texttt{Agents Of S.W.E.} ha deciso, per avere un ambiente comune su cui testare il prodotto in via di sviluppo, di dotarsi di un server noleggiato online attraverso la nota piattaforma di hosting DigitalOcean\glossario.\\
Il server in oggetto è raggiungibile all'indirizzo IP \texttt{142.93.102.115} ha le seguenti caratteristiche:
\begin{itemize}
	\item SO: Ubuntu 18.04 LTS;
	\item CPU: 1x Intel(R) Xeon(R) CPU E5-2650 v4 @ 2.20GHz;
	\item RAM: 2GiB ECC DIMM;
	\item SSD: 24GiB.
\end{itemize}
Il team ha provveduto ad installare all'interno del server i seguenti pacchetti:
\begin{itemize}
	\item Grafana;
	\item InfluxDB;
	\item Telegraf;
	\item Apache\glossario.
\end{itemize}

\subsubsection{Versionamento e Issue Tracking}

\paragraph{\textit{Git}} ~\\
	\textit{Git} è il sistema di versionamento\glossario scelto dal gruppo. E' un sistema open source creato da Linus 	Torvalds 2005. Presenta un'interfaccia a riga di comando, tuttavia esistono svariati tools\glossario 	che 	ne forniscono una GUI.

\paragraph{\textit{GitHub}} ~\\
	\textit{Github} è un Respository Manager\glossario che usa \textit{Git} come sistema di versioning, offre inoltre 	servizi di issue tracking\glossario;
	
\paragraph{\textit{GitLab}} \-\\
	\textit{Gitlab} è un Respository Manager che usa \textit{Git} come sistema di versioning, a differenza di GitHub offre servizi di integrazione test e pipeline di sviluppo. 

\subsubsection{Comunicazione}

\paragraph{\textit{Telegram}} ~\\
	\textit{Telegram} è una delle maggiori e più note applicazioni di messaggistica istantanea cross platform, 					utilizzabile contemporaneamente su più dispositivi. Oltre alla semplice messaggistica offre servizi quali lo 				scambio di files, la creazione di gruppi e le chiamate vocali.

\paragraph{\textit{Slack}} ~\\
	\textit{Slack} è un'applicazione di messaggistica istantanea specializzata nella comunicazione interna tra membri di un gruppo di lavoro. L'applicazione è organizzata in workspace, a loro volta suddivisi in canali, i quali 			consentono di catalogare le conversazione sulla base dell'argomento trattato. Questa struttura, studiata 						appositamente per l'ambito lavorativo, è stata giudicata come positiva e vantaggiosa dal gruppo, visto che 					consente di mantenere chat ordinate e monotematiche.\\
	Nonostante preveda anche abbonamenti a pagamento le funzionalità base di \textit{Slack} sono gratuite, inoltre, come \textit{Telegram}, risulta essere un'applicazione cross platform\glossario.

\subsubsection{Diagrammi di Gantt}
	Lo strumento scelto dal gruppo per la realizzazione dei diagrammi di Gantt\glossario è "Gantt Project". Le 					motivazioni che hanno portato a questa scelta sono molteplici, tra queste spiccano il fatto che sia uno strumento 	gratuito, open-source\glossario, e cross platform. L'elevata accessibilità è stata infatti 				giudicata come una caratteristica di primaria importanza, considerando i differenti sistemi operativi utilizzati 		dai componenti del gruppo.

\subsubsection{Diagrammi UML}
	Lo strumento scelto dal gruppo per la realizzazione dei diagrammi UML è \textit{Draw.io}, una applicazione online per il disegno di diagrammi UML 2.0. Anche in questo caso è stata tenuta in grande considerazione l'accessibilità dello strumento. Il software in questione risulta infatti gratuito ed essendo online non genera problemi per l'uso su più piattaforme.\\
	\textit{Draw.io} risulta essere sia User Friendly, con un'interfaccia utente chiara e consistente, sia abbastanza potente da essere utilizzato anche scopi più evoluti. Questo, unito alle dimensioni relativamente ridotte del software, e al fatto che sia gratuito, hanno fatto propendere il gruppo per questa applicazione online, a discapito di alternative, come \textit{Papyrus o Astah}, comunque considerate valide.
