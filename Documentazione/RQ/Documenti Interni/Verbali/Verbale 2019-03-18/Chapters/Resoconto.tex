\section{Resoconto}

\subsection{Punto 1}
Il gruppo ha discusso sulle segnalazioni individuate dal committente in sede di Revisione di Progettazione e ha proceduto con l'individuazione di task da assegnare ad ogni membro del gruppo per correggerli.
\\
I task sono stati assegnati e suddivisi dal \textit{Responsabile} ad ogni membro del gruppo.

\subsection{Punto 2}
I ruoli all'interno del team, dopo la Revisione di Progettazione, ruotano nel seguente modo:\\

\begin{center}
	\begin{longtable}[c]{|m{.20\textwidth}|m{.20\textwidth}|m{.20\textwidth}|} 
		\hline
		\rowcolor{bluelogo}\textbf{\textcolor{white}{Membro}} & \textbf{\textcolor{white}{Vecchio Ruolo}} & \textbf{\textcolor{white}{Nuovo Ruolo}}\\
		\hline
		\hline
		Luca Violato & Verificatore & Analista \\
		\hline
		\rowcolor{grigio}Bogdan Stanciu & Programmatore & Amministratore \\
		\hline
		Marco Chilese & Amministratore & Progettista\\
		\hline
		\rowcolor{grigio}Carlotta Segna & Progettista & Programmatore\\
		\hline
		Marco Favaro & Analista & Responsabile \\
		\hline
		\rowcolor{grigio} Diego Mazzalovo & Responsabile & Progettista\\
		\hline
		Matteo Slanzi & Programmatore & Verificatore\\
		\hline
		\caption{Rotazione dei Ruoli}
	\end{longtable}

\end{center}
	

\subsection{Punto 3}
Il gruppo non utilizzerà più due servizi diversi di repository online, avere una repository per la documentazione su \textit{GitHub}\glossario e una per il codice sorgente su \textit{GitLab} crea dispersione e confusione. Il gruppo utilizzerà solamente \textit{GitLab} in quanto permette la gestione di pipeline di CI/CD\glossario. La documentazione verrà caricata in una nuova repository di \textit{GitLab}.

\subsection{Punto 4}
È stata valutata l'opzione di suddividere la progettazione in due parti principali, una di queste è il lato server scritto in NodeJS. Questa scelta architetturale comporta notevoli vantaggi nel calcolo delle probabilità e nella suddivsione modulare del codice, suddividendo la gestione dei database e delle reti al server, la raccolta, il caricamento dei dati e la visualizzazione delle probabilità al plug-in. Questa scelta ha agevolato l'applicazione di pattern come l'Adapter e il proxy per la parte server, consentendo lo sviluppo futuro della gestione di nuovi database.

\subsection{Punto 5}
Il gruppo si incontrerà nuovamente in data 2019-03-20.