\begin{longtable}{|C{.15\textwidth}|C{.14\textwidth}|m{.52\textwidth}|m{.08\textwidth}|}
\hline
\rowcolor{bluelogo}\textbf{\textcolor{white}{Test}} & \textbf{\textcolor{white}{Requisito}} & \textbf{\textcolor{white}{Descrizione}} & \textbf{\textcolor{white}{Esito}}\\
\hline \hline
\endhead

TS0-1 & ROF1 & Verificare se il sistema permette l'aggiunta di una rete bayesiana tramite il caricamento di un file & S. \\
\hline
\rowcolor{grigio} TS0-02 & ROF1.1 & Verificare che il sistema di caricamento della rete metta a disposizione dell'utente un pulsante per avviare il procedimento di caricamento della rete bayesiana & S.\\
\hline
TS0-1.2 & ROF1.2 & Verificare se il sistema permette la selezione di una rete in formato \texttt{.json} dal sistema dell'utente & S. \\
\rowcolor{grigio} TS0-04 & ROF1.3 & Verifica che il sistema metta a disposizione un bottone di caricamento del file il quale avvia la procedura di caricamento & S. \\
\hline
TS0-1.4 &  ROF1.4 & Verificare che il sistema faccia apparire un messaggio di errore nel caso in cui l'operazione di caricamento del file non sia andata a buon fine & S. \\
\hline
\rowcolor{grigio} TS-1.4.1 & ROF1.4.1 & Verificare che il sistema si accerti che il file caricato dall'utente sia solo con estensione \texttt{.json} & S. \\
\hline
TS0-1.4.2 & ROF1.4.2 & Verificare l'autenticità del file da parte del sistema & S. \\
\hline 
\rowcolor{grigio} TS0-1.5  & ROF1.5 & Verificare che il sistema, una volta caricato il file, inizializzi la rete bayesiana & S. \\
\hline 
TS1-1.6 & ROF1.6 & Verifica che il sistema memorizzi la rete bayesiana precedentemente caricata & S. \\
\hline
\rowcolor{grigio} TS0-1.7 & ROF1.7 & Verifica che il sistema visualizzi un messaggio di avvenuto caricamento della reta bayesiana & S. \\
\hline
TS0-2 & ROF2 & Verifica che il sistema permetta il collegamento di un flusso di dati a ogni nodo desiderato della rete bayesiana caricata dall'utente & S. \\
\hline
\rowcolor{grigio} TS0-2.1 & ROF2.1 & Verifica che il sistema interpreti la rete bayesyana caricata da file & S. \\
\hline
TS0-2.1.1 & ROF2.1.1 & Verificare che il Sistema mostri a interfaccia utente il nominativo per ogni nodo della rete.& S.\\
\hline
\rowcolor{grigio} TS0-2.1.2 & ROF2.1.2 & Verificare che il Sistema mostri, per ogni nodo della rete bayesiana, la corrispondente checkbox per identificare se un nodo è collegato ad un flusso dati o meno. & S. \\
\hline
TS0-2.5 & ROF2.5 & Verificare che il Sistema metta a disposizione le impostazioni necessarie per effettuare correttamente il collegamento desiderato. & S. \\ 
\hline
\rowcolor{grigio}TS0-2.5.3 & ROF2.5.3 & Verificare che il Sistema mostri un elenco di flussi dati coerente con la sorgente dati selezionata dall'utente. & S. \\
\hline
TS0-2.5.3.1 & ROF2.5.3.1 & Verificare che il sistema permetta all'utente di selezionare un database. & S. \\ 
\hline
\rowcolor{grigio}TS0-17 & ROF2.5.3.2 & Verificare che il sistema metta a disposizione dell'utente una lista dei database disponibili. & S. \\ 
\hline
TS0-2.5.3.3 & ROF2.5.3.3 & Verificare che il sistema notifichi all'utente tramite un messaggio la conferma di collegamento al database. & S. \\
\hline 
\rowcolor{grigio}TS0-19 & ROF2.5.3.4 & Verificare che il sistema metta a disposizione dell'utente un elenco delle tabelle del database disponibili. & S. \\
\hline
TS0-2.5.3.5 & ROF2.5.3.5 & Verificare che l'utente possa selezionare una tabella del database precedentemente selezionato. & S. \\
\hline
\rowcolor{grigio}TS0-2.5.3.6 & ROF2.5.3.6 & Verificare che il sistema aggiorni i flussi dati disponibili in base alla selezione della tabella del database. & S. \\ 
\hline
TS0-2.5.4 & ROF2.5.4 & Verificare che l'utente abbia la possibilità di selezionare un flusso dati desiderato coerente con la sorgente dati e  una corrispondente tabella precedentemente selezionate. & S. \\ 
\hline
\rowcolor{grigio}TS0-2.5.5 & ROF2.5.5 & Verificare che il Sistema mostri la lista dei possibili stati del nodo selezionato. & S. \\ 
\hline
TS0-2.5.6 & ROF2.5.6 & Verificare che il Sistema metta a disposizione, per ogni stato del nodo, un pulsante necessario all'aggiunta di un livello di soglia connesso al flusso dati selezionato. & S. \\
\hline
\rowcolor{grigio}TS0-2.5.6.1 & ROF2.5.6.1 & Verificare che il Sistema metta a disposizione un campo dati numerico che permetta la definizione della soglia. & S. \\ 
\hline
TS0-2.5.6.2 & ROF2.5.6.2 & Verificare che il Sistema metta a disposizione un menù a tendina che permetta di definire se il valore numerico definito per la soglia sia un minimo oppure un massimo. & S. \\
\hline
\rowcolor{grigio}TS1-2.5.6.3 & RDF2.5.6.3  & Verificare che il Sistema metta a disposizione un campo dati che permetta di definire se una soglia è critica o meno.  & S. \\
\hline
TS0-2.5.6.4 & ROF2.5.6.4  & Verificare che il Sistema metta a disposizione un pulsante per l'aggiunta di una soglia di un nodo.Il click di tale pulsante deve portare alla comparsa dei campi editabili per la modifica della stessa. & S. \\
\hline
\rowcolor{grigio}TS1-2.5.6.5 & RDF2.5.6.5  & Verificare che il Sistema consenta l'aggiunta di molteplici soglie relative allo stesso stato di un nodo. & S. \\
\hline
TS0-2.5.7 & ROF2.5.7 & Verificare che il Sistema metta a disposizione un campo dati per definire correttamente un livello di soglia al di sotto, o al di sopra del quale la probabilità associata a quel dato stato risulti pari al 100\%, mentre le probabilità associate agli altri stati risultino pari allo 0\%. & S. \\
\hline
\rowcolor{grigio}TS0-2.5.8 & ROF2.5.8 & Verificare che il Sistema metta a disposizione un bottone per la conferma delle soglie definite dall'utente. & S. \\
\hline
TS0-2.5.9 & ROF2.5.9 & Verificare che il Sistema mostri un messaggio d'errore nel caso in cui l'utente abbia confermato le proprie scelte riguardanti il collegamento dei singolo nodo in esame senza aver correttamente definito i livelli di soglia. & S. \\
\hline
\rowcolor{grigio}TS0-2.5.9.1 & ROF2.5.9.1 & Il sistema nega la conferma di avvenuto collegamento di un nodo ad un flusso dati nel caso in cui sia stata confermata la definizione delle soglie senza la scelta di un flusso dati. & S. \\
\hline
TS0-2.5.9.2 & ROF2.5.9.2  & Il sistema nega la conferma di avvenuto collegamento di un nodo ad un flusso dati nel caso in cui sia stata confermata la definizione delle soglie senza la definizione di almeno una soglia. & S. \\
\hline
\rowcolor{grigio}TS0-2.5.9.3 & ROF2.5.9.3  & Il sistema nega la conferma di avvenuto collegamento di un nodo ad un flusso dati nel caso in cui sia stata confermata la definizione delle soglie, avendo definito soglie tra loro in conflitto. & S. \\
\hline
TS0-2.5.9.4 & ROF2.5.9.4  & Il sistema nega la conferma di avvenuto collegamento di un nodo ad un flusso dati nel caso in cui sia stata confermata la definizione delle soglie in maniera errata, mostrando un errore coerente. & S. \\
\hline
\rowcolor{grigio}TS0-2.5.10 & ROF2.5.10 & Verificare che il Sistema aggiorni la lista di checkbox, registrando le modifiche apportate dall'utente. & S. \\
\hline
TS0-2.5.11 & ROF2.5.11  & Verificare che il sistema metta a disposizione dell'utente un pulsante per la rimozione di una soglia qualora l'utente desideri rimuoverla. & S. \\
\hline
\rowcolor{grigio} TS0-2.6 & ROF2.6 & Verificare che l'utente possa scollegare un nodo dal flusso dati. & S. \\
\hline
TS0-2.6.1 & ROF2.6.1 & Verificare che il Sistema metta a disposizione un bottone per eliminare il collegamento di un nodo al flusso dati. & S. \\
\hline
\rowcolor{grigio}TS0-2.6.2 & ROF2.6.2 & Verificare che il Sistema resetti le impostazioni qualora l'utente scolleghi un nodo dal flusso dati. & S. \\
\hline
TS0-2.6.3 & ROF2.6.3 & Verificare che il Sistema aggiorni la checkbox togliendo la spunta relativa al nodo dopo il suo scollegamento dal flusso dati. & S. \\
\hline
\rowcolor{grigio}TS0-2.6.4 & ROF2.6.4 & Verificare che il Sistema faccia sparire il pulsante per lo scollegamento di un nodo dal flusso dati dopo che esso viene scollegato. & S. \\
\hline
TS0-2.7 & ROF2.7 & Verificare che il Sistema faccia apparire un messaggio di conferma di avvenuto collegamento di un nodo al flusso di dati. & S. \\
\hline
\rowcolor{grigio}TS0-3 & ROF3 & Verificare che il Sistema permetta la definizione di una politica temporale per il ricalcolo delle probabilità condizionate associate ai nodi della rete bayesiana. & S. \\
\hline
TS0-3.3 & ROF3.3 & Verificare che il Sistema offra la possibilità di definire una politica temporale. & S. \\
\hline
\rowcolor{grigio}TS0-3.3.1 & ROF3.3.1 & Verificare che il Sistema metta a disposizione un pulsante per accedere al pannello di configurazione di una politica temporale. & S. \\
\hline
TS0-3.3.2 & ROF3.3.2 & Verificare che il Sistema metta a disposizione un pannello di configurazione con i campi dati adeguati per la definizione di una politica temporale. & S. \\
\hline
\rowcolor{grigio}TV0-3.3.2.4 & ROF3.3.2.4 & Verificare che il Sistema metta a disposizione un campo dati per la definizione del numero di secondi della politica temporale. & S. \\
\hline
TS0-3.3.2.5 & ROF3.3.2.5 & Verificare che il Sistema metta a disposizione un campo dati per la definizione del numero di minuti della politica temporale. & S. \\
\hline
\rowcolor{grigio}TS0-3.3.2.6 & ROF3.3.2.6 & Verificare che il Sistema metta a disposizione un campo dati per la definizione del numero di ore della politica temporale. & S. \\
\hline
TS0-3.3.3 & ROF3.3.3 & Verificare che il Sistema dia la possibilità di modificare i campi dati per definire correttamente la politica temporale desiderata. & S. \\
\hline
\rowcolor{grigio}TS0-3.4 & ROF3.4 & Verificare che il Sistema metta a disposizione un bottone per confermare la politica temporale definita dall'utente. & S. \\
\hline
TS0-3.5 & ROF3.5 & Verificare che il Sistema visualizzi un messaggio d'errore nel caso in cui l'utente confermi una politica temporale non correttamente definita. & S. \\
\hline
\rowcolor{grigio}TS0-3.5.1 & ROF3.5.1 & Verificare che il Sistema neghi la creazione della politica temporale qualora l'utente abbia confermato una politica non valida. & S. \\
\hline
TS0-3.5.2 & ROF3.5.2 & Verificare che il Sistema neghi la creazione della politica temporale qualora l'utente non abbia editato almeno uno dei 3 campi. & S. \\
\hline
\rowcolor{grigio}TS0-3.5.3 & ROF3.5.3 & Verificare che il Sistema neghi la creazione della politica temporale qualora l'utente abbia impostato un numero di secondi non valido. & S. \\
\hline
TS0-3.5.4 & ROF3.5.4 & Verificare che il Sistema neghi la creazione della politica temporale qualora l'utente abbia impostato un numero di minuti non valido. & S. \\
\hline
\rowcolor{grigio}TS0-3.5.5 & ROF3.5.5 & Verificare che il Sistema neghi la creazione della politica temporale qualora l'utente abbia impostato un numero di ore non valido. & S. \\
\hline
TS0-3.6 & ROF3.6 & Verificare che il Sistema visualizzi un messaggio di avvenuta selezione della politica temporale qualora l'utente abbia correttamente impostato la politica temporale. & S. \\
\hline
\rowcolor{grigio}TS0-4 & ROF4 & Verificare che il Sistema a interfaccia utente mostri i dati relativi ai nodi della rete bayesiana non collegati a un flusso di dati. & S. \\
\hline
TS0-4.4 & ROF4.4 & Verificare che il Sistema metta a disposizione un pulsante per avviare il monitoraggio dei dati. & S. \\
\hline
\rowcolor{grigio}TS0-4.4.3 & ROF4.4.3 & Verificare che il Sistema mostri un messaggio di errore nel caso in cui l'utente abbia avviato il monitoraggio senza aver preventivamente impostato la politica temporale per il ricalcolo delle probabilità. & S. \\
\hline
TS0-4.4.4 & ROF4.4.4 & Verificare che il Sistema mostri un messaggio di errore nel caso in cui l'utente abbia avviato il monitoraggio senza aver preventivamente  collegato almeno un nodo al flusso dati. & S. \\
\hline
\rowcolor{grigio}TS0-4.4.5 & ROF4.4.5 & Verificare che il Sistema salvi nel Server le impostazioni di collegamento insieme alla rete. & S. \\
\hline
TS0-4.4.6 & ROF4.4.6 & Verificare che il Sistema impedisca all'utente di modificare le impostazioni di una rete sotto monitoraggio. & S. \\
\hline
\rowcolor{grigio}TS2-4.4.7 & RFF4.4.7 & Verificare che il Sistema consenta all'utente di monitorare più reti contemporaneamente. & S. \\
\hline
TS2-4.4.7.1 & RFF4.4.7.1 & Verificare che il Sistema consenta all'utente l'avvio del monitoraggio di una rete qualora ci siano già altre reti sotto monitoraggio. & S. \\
\hline
\rowcolor{grigio}TS0-4.4.8 & ROF4.4.8 & Verificare che il Sistema mostri un messaggio di corretto inizio del monitoraggio della rete. & S. \\
\hline
TS0-4.5 & ROF4.5 & Verificare che il Sistema fornisca all'utente una lista di probabilità dinamiche associate ai nodi della rete. & S. \\
\hline
\rowcolor{grigio}TS2-4.5.1 & RFF4.5.1 & Verificare che il Sistema consenta all'utente di selezionare una rete tra quelle al momento in monitoraggio, per la visualizzazione delle sue probabilità dinamiche. & S. \\
\hline
TS2-4.5.1.1 & RFF4.5.1.1 & Verificare che il Sistema fornisca un menù a  tendina contenente le reti bayesiane sotto monitoraggio. & S. \\
\hline
\rowcolor{grigio}TS0-4.6 & ROF4.6 & Verificare che il Sistema, attraverso il Server, aggiorni periodicamente le probabilità in base a quanto definito nella politica temporale per il ricalcolo delle probabilità. & S. \\
\hline
TS1-4.6.1 & RDF 4.6.1 & Verificare che il Sistema, indipendentemente dalla politica temporale definita dall'utente, ricalcoli le probabilità al verificarsi del superamento di una soglia critica associata ad uno stato di un nodo collegato al flusso di dati di monitoraggio in base al timeout di refresh impostato nella dashboard al momento dell'avvio del monitoraggio. & S. \\
\hline
\rowcolor{grigio}TS0-4.7 & ROF4.7 & Verificare che il Sistema dia all'utente la possibilità di interrompere il monitoraggio di una rete bayesiana. & S. \\
\hline
TS0-4.7.1 & ROF4.7.1 & Verificare che il Sistema metta a disposizione un pulsante per interrompere il monitoraggio di una rete bayesiana. & S. \\
\hline
\rowcolor{grigio}TS0-4.7.2 & ROF4.7.2 & Verificare che il Sistema visualizzi un messaggio di corretta interruzione del monitoraggio di una rete. & S. \\
\hline
TS0-7 & ROF7 & Verificare che il Sistema consenta all'utente di collegare il plug-in al Server. & S. \\
\hline
\rowcolor{grigio}TV0-7.1 & ROF7.1 & Verificare che il Sistema metta a disposizione dell'utente una sezione "Server Settings" all'interno del menù "Edit" del pannello del plug-in. & S. \\
\hline
TS0-7.1.1 & ROF7.1.1 & Verificare che il Sistema metta a disposizione dell'utente una sezione "Server Settings" all'interno del menù "Edit" del pannello del plug-in, nella quale è presente un campo dati per modificare l'IP del Server per la connessione allo stesso. & S. \\
\hline
\rowcolor{grigio}TS0-7.1.2 & ROF7.1.2 & Verificare che il Sistema metta a disposizione dell'utente una sezione "Server Settings" all'interno del menù "Edit" del pannello del plug-in, nella quale è presente un campo dati per modificare la porta del Server per la connessione allo stesso. & S. \\
\hline
TS0-7.1.3 & ROF7.1.3 & Verificare che il Sistema metta a disposizione dell'utente una sezione "Server Settings" all'interno del menù "Edit" del pannello del plug-in, nella quale è presente un pulsante per confermare i parametri di connessione al Server e connettervi il plug-in. & S. \\
\hline
\rowcolor{grigio}TS0-7.2 & ROF7.2 & Il Sistema mostra all'utente un messaggio di errore qualora il collegamento al Server non sia avvenuto correttamente. & S. \\
\hline
TS0-7.2.1 & ROF7.2.1 & Il Sistema deve verificare che l'IP inserito per il collegamento al Server sia corretto. & S. \\
\hline
\rowcolor{grigio}TS0-7.2.2 & ROF7.2.2 & Il Sistema deve verificare che la porta inserita per il collegamento al Server sia corretta. & S. \\
\hline
TS0-7.3 & ROF7.3 & Il Sistema deve notificare all'utente l'avvenuta connessione al Server qualora sia andata a buon fine.  & S. \\
\hline
\rowcolor{grigio}TS2-8 & RDF8 & Il Sistema deve dare all'utente la possibilità di caricare una rete precedentemente salvata sul Server. & S. \\
\hline
TS2-8.1 & RDF8.1 & Il Sistema deve mettere a disposizione dell'utente un menù a tendina coi nomi delle reti salvate sul Server. & S. \\
\hline
\rowcolor{grigio}TS2-8.2 & RDF8.2 & Il Sistema deve mettere a disposizione dell'utente un pulsante per caricare una rete salvata sul Server. & S. \\
\hline
TS2-8.3 & RDF8.3 & Il Sistema, prima di caricare una nuova rete, deve salvare le impostazioni nel Server. & S. \\
\hline
\rowcolor{grigio}TS2-8.3.1 & RDF8.3.1 & Il Sistema, prima di caricare una nuova rete, deve salvare le impostazioni nel Server e, qualora sia già presente la rete, la sovrascrive. & S. \\
\hline
TS2-8.3.2 & RDF8.3.2 & Il Sistema, prima di salvare una rete sul Server, controlla se è sotto monitoraggio e, in caso positivo, al momento del cambio di contesto, non la salva nel Server. & S. \\
\hline
\rowcolor{grigio}TS2-8.4 & RDF8.4 & Il Sistema, in seguito alla scelta dell'utente di caricare una rete salvata nel Server, deve visualizzare le impostazioni della suddetta rete. & S. \\
\hline
TS0-9 & ROF9 & Il Sistema deve mettere a disposizione dell'utente una sezione unicamente adibita alla visualizzazione dei dati di monitoraggio. & S. \\
\hline
\rowcolor{grigio}TS0-9.1 & ROF9.1 & Il Sistema deve mettere a disposizione dell'utente un pulsante che consenta di passare alla sezione adibita alla visualizzazione dei monitoraggi attivi. & S. \\
\hline
TS0-9.1.1 & ROF9.1.1 & Il Sistema deve mettere a disposizione dell'utente un pulsante modifichi la schermata, visualizzando la sezione dei monitoraggi attivi. & S. \\
\hline
\rowcolor{grigio}TS0-9.2 & ROF9.2 & Il Sistema deve mettere a disposizione dell'utente un pulsante modifichi la schermata, passando da quella dei monitoraggi attivi a quella delle impostazioni delle reti. & S. \\
\hline
TS2-10 & RDF10 & Il Sistema deve dare all'utente la possibilità di eliminare una rete salvata nel Server. & S. \\
\hline
\rowcolor{grigio}TS2-10.1 & RDF10.1 & Il Sistema deve mettere a disposizione dell'utente un pulsante per eliminare una rete selezionata nell'apposito menù a tendina e salvata nel Server. & S. \\
\hline
TS2-10.2 & RDF10.2 & Il Sistema deve notificare all'utente l'impossibilità di eliminare una rete sotto monitoraggio qualora l'utente provi ad eliminarne una. & S. \\
\hline
\rowcolor{grigio}TS2-10.2.1 & RDF10.2.1 & Il Sistema deve verificare che l'utente, in fase di eliminazione della rete, non abbia selezionato una rete sotto monitoraggio. & S. \\
\hline
TS2-10.3 & RDF10.3 & Il Server deve rimuovere dalle reti salvate la rete che l'utente desidera eliminare. & S. \\
\hline

\rowcolor{grigio}TS2-10.4 & RDF10.4 & Il Sistema deve mostrare all'utente un messaggio di avvenuta eliminazione della rete. & S. \\
\hline

\caption{Test di sistema}
\label{testdisistema}
\end{longtable}