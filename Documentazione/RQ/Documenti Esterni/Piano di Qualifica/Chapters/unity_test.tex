\begin{longtable}{|C{.15\textwidth}|m{.65\textwidth}|C{.20\textwidth}|}
\hline
\rowcolor{bluelogo}\textbf{\textcolor{white}{Test}}  & \textbf{\textcolor{white}{Descrizione}} & \textbf{\textcolor{white}{Stato}}\\
\hline \hline
\endhead

TU0-0 & Viene verificato che il file di configurazione esista all'interno della directory & Superato\\
\hline 
\rowcolor{grigio} TU0-1 & Viene verificato che i parametri di configurazione obbligatori siano presenti nel file di configurazione & Superato \\ 
\hline
TU0-2 & Viene verificato che le configurazioni rispettino la sintassi & Supereato \\ 
\hline
\rowcolor{grigio} TU0-3 & Viene verificata la conformità della sintassi alle configurazioni non obbligatorie & Superato \\ 
\hline 
TU0-4 & Viene verificato che siano passati i parametri obbligatori all'avvio del server & Superato \\ 
\hline
\rowcolor{grigio} TU0-5 & Viene verificata l'autenticità della porta obbligatoria all'avvio del server & Superato \\ 
\hline
TU0-6 & Viene verificato  che l'incapsulamento dei parametri sia avvenuto con successo & Superato \\
\hline 
\rowcolor{grigio} TU0-7 & Viene verificato il lancio di un'eccezione nel caso in cui la porta non sia disponibile & Superato \\ 
\hline 
TU0-8 & Viene verificato il lancio di un'eccezione nel caso in cui la porta non sia un numero intero & Superato \\
\hline
\rowcolor{grigio}  TU0-9 & Viene verificato il lancio di un'eccezione nel caso in cui manchino parametri obbligatori nel file di configurazione & Superato \\ 
\hline
TU0-10 & Viene verificata l'inizializzazione del proxy server & Superato \\ 
\hline 
\rowcolor{grigio} TU0-11 & Viene verificato che la richiesta di \texttt{root} al server del server ritorni l'ora corrente & Superato \\ 
\hline
TU0-12 & Viene verificato che il tipo di ritorno dalla richiesta \texttt{root} al server sia di tipo json & Superato \\ 
\hline 
TU0-13 & Viene verificato che il tipo di ritorno della richiesta \texttt{alive} al server sia di tipo json & Superato \\ 
\hline
\rowcolor{grigio} TU0-14 & Viene verificata che la richiesta \texttt{alive} ritorni data corrente e numero della porta in ascolto del server & Superato \\ 
\hline 
TU0-15 & Viene verificata che la richiesta \texttt{networks} ritorni un json & Superato \\ 
\rowcolor{grigio} TU0-16 & Viene verificato che venga chiamato il metodo \texttt{getNetworks()} & Superato \\
TU0-17 & Viene verificato che il metodo \texttt{getNetworks()} ritorni un array di json & Superato \\ 
\hline
\rowcolor{grigio} TU0-18 & Viene verificato che per ogni json appartenente all'array ritornato da \texttt{getNetworks()} abbia un campo \textit{name} di tipo \textbf{string} ed un campo \textit{monitoring} di tipo \textbf{boolean} & Superato \\ 
\hline
TU0-19 & Viene verificato il lancio di un'eccezione dal metodo \texttt{getNetworks()} nel caso in cui l'accesso al filesystem sia proibito & Superato \\ 
\hline 
\rowcolor{grigio}TU0-20 & Viene verificata che la richiesta al server \texttt{uploadnetwork} chiami il metodo \texttt{saveNetworkToFile} passando un parametro di tipo json & Superato \\ 
\hline 
TU0-21 & Viene verificato, nel caso in cui la direcotry di salvataggio delle reti non sia presente, venga creata secondo le configurazioni & Superato \\ 
\hline 
\rowcolor{grigio} TU0-22 & Viene verificato il lancio di un'eccezione nel caso in cui la creazione della cartella fallisca & Superato \\ 
\hline 
TU0-23 & Viene verificato che la rete venga sovrascritta nel caso in cui l'utente cerca di caricare la stessa rete & Superato \\
\hline
 \rowcolor{grigio} TU0-24 & Viene verificato il lancio di un'eccezione nel caso in cui la cancellazione di una rete sia fallito & Superato \\ 
\hline
TU0-25 & Viene verificato che la rete caricata disponga del campo \texttt{name} di tipo \textbf{stringa} & Superato \\ 
\hline 
\rowcolor{grigio} TU0-26 & Viene verificato il lancio di un'eccezione nel caso in cui il campo dati \texttt{name} sia assente & Superato \\ 
\hline 
TU0-27	 & Viene verificata la creazione del file con la definizione della rete & Superato \\ 
\hline 
\rowcolor{grigio} TU0-28 & Viene verificata il lancio di un'eccezione nel caso in cui la scritta su filesystem sia fallita & Superato \\ 
\hline 
TU0-29 & Viene verificata l'invocazione del metodo \texttt{initBayesianNetwork(net)} all'interno del metodo \texttt{saveNetworkToFile()} & Superato \\ 
\rowcolor{grigio} TU0-30 & Viene verificata la creazione di un nuovo oggetto di tipo \texttt{Network} con la rete caricata dall'utente & Superato \\ 
\hline
% TU0-31 & Viene verificata 












\caption{Test di unità}
\label{testunita}
\end{longtable}
