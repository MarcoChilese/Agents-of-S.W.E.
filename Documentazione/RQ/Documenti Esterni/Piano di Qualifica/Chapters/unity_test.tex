\begin{longtable}{|C{.15\textwidth}|m{.65\textwidth}|C{.20\textwidth}|}
\hline
\rowcolor{bluelogo}\textbf{\textcolor{white}{Test}}  & \textbf{\textcolor{white}{Descrizione}} & \textbf{\textcolor{white}{Stato}}\\
\hline \hline
\endhead

TU0-0 & Viene verificato che il file di configurazione esista all'interno della directory & N.I.\\
\hline 
\rowcolor{grigio} TU0-1 & Viene verificato che i parametri di configurazione obbligatori siano presenti nel file di configurazione & N.I. \\ 
\hline
TU0-2 & Viene verificato che le configurazioni rispettino la sintassi & N.I. \\ 
\hline
\rowcolor{grigio} TU0-3 & Viene verificata la conformità della sintassi alle configurazioni non obbligatorie & N.I. \\ 
\hline 
TU0-4 & Viene verificato che siano passati i parametri obbligatori all'avvio del server & N.I. \\ 
\hline
\rowcolor{grigio} TU0-5 & Viene verificata l'autenticità della porta obbligatoria all'avvio del server & N.I. \\ 
\hline
TU0-6 & Viene verificato  che l'incapsulamento dei parametri sia avvenuto con successo & N.I. \\
\hline 
\rowcolor{grigio} TU0-7 & Viene verificato il lancio di un'eccezione nel caso in cui la porta non sia disponibile & N.I. \\ 
\hline 
TU0-8 & Viene verificato il lancio di un'eccezione nel caso in cui la porta non sia un numero intero & N.I. \\
\hline
\rowcolor{grigio}  TU0-9 & Viene verificato il lancio di un'eccezione nel caso in cui manchino parametri obbligatori nel file di configurazione & N.I. \\ 
\hline
TU0-10 & Viene verificata l'inizializzazione del proxy server & N.I. \\ 
\hline 
\rowcolor{grigio} TU0-11 & Viene verificato che la richiesta di \texttt{root} al server del server ritorni l'ora corrente & N.I. \\ 
\hline
TU0-12 & Viene verificato che il tipo di ritorno dalla richiesta \texttt{root} al server sia di tipo json & N.I. \\ 
\hline 
\rowcolor{grigio}TU0-13 & Viene verificato che il tipo di ritorno della richiesta \texttt{alive} al server sia di tipo json & N.I. \\ 
\hline
 TU0-14 & Viene verificata che la richiesta \texttt{alive} ritorni data corrente e numero della porta in ascolto del server & N.I. \\ 
\hline 
\rowcolor{grigio}TU0-15 & Viene verificata che la richiesta \texttt{networks} ritorni un json & N.I. \\ 
\hline 
 TU0-16 & Viene verificato che venga chiamato il metodo \texttt{getNetworks()} & N.I. \\
\hline 
\rowcolor{grigio}TU0-17 & Viene verificato che il metodo \texttt{getNetworks()} ritorni un array di json & N.I. \\ 
\hline
TU0-18 & Viene verificato che per ogni json appartenente all'array ritornato da \texttt{getNetworks()} abbia un campo \textit{name} di tipo \textbf{string} ed un campo \textit{monitoring} di tipo \textbf{boolean} & N.I. \\ 
\hline
\rowcolor{grigio}TU0-19 & Viene verificato il lancio di un'eccezione dal metodo \texttt{getNetworks()} nel caso in cui l'accesso al filesystem sia proibito & N.I. \\ 
\hline 
TU0-20 & Viene verificata che la richiesta al server \texttt{uploadnetwork} chiami il metodo \texttt{saveNetworkToFile} passando un parametro di tipo json & N.I. \\ 
\hline 
\rowcolor{grigio}TU0-21 & Viene verificato, nel caso in cui la direcotry di salvataggio delle reti non sia presente, venga creata secondo le configurazioni & N.I. \\ 
\hline 
 TU0-22 & Viene verificato il lancio di un'eccezione nel caso in cui la creazione della cartella fallisca & N.I. \\ 
\hline 
\rowcolor{grigio}TU0-23 & Viene verificato che la rete venga sovrascritta nel caso in cui l'utente cerca di caricare la stessa rete & N.I. \\
\hline
 TU0-24 & Viene verificato il lancio di un'eccezione nel caso in cui la cancellazione di una rete sia fallito & N.I. \\ 
\hline
\rowcolor{grigio}TU0-25 & Viene verificato che la rete caricata disponga del campo \texttt{name} di tipo \textbf{stringa} & N.I. \\ 
\hline 
 TU0-26 & Viene verificato il lancio di un'eccezione nel caso in cui il campo dati \texttt{name} sia assente & N.I. \\ 
\hline 
\rowcolor{grigio}TU0-27	 & Viene verificata la creazione del file con la definizione della rete & N.I. \\ 
\hline 
 TU0-28 & Viene verificata il lancio di un'eccezione nel caso in cui la scritta su filesystem sia fallita & N.I. \\ 
\hline 
\rowcolor{grigio}TU0-29 & Viene verificata l'invocazione del metodo \texttt{initBayesianNetwork(net)} all'interno del metodo \texttt{saveNetworkToFile()} & N.I. \\ 
\hline 
TU0-30 & Viene verificata la creazione di un nuovo oggetto di tipo \texttt{Network} con la rete caricata dall'utente & N.I. \\ 
\hline
\rowcolor{grigio} TU0-31 & Viene verificato il lancio di un'eccezione nel caso in cui il metodo \texttt{saveNetworkToFile(net)} fallisca & N.I. \\
\hline
TU0-31 & Viene verificato che la richiesta di \texttt{uploadnetwork} ritorni una risposta con stato \textit{404} in caso di fallimento & N.I. \\ 
\hline
\rowcolor{grigio} TU0-32 & Viene verificato che la richiesta \texttt{uploadnetwork} ritorni un messaggio di successo nel caso in cui il metodo non ritorni errori & N.I. \\
\hline
TU0-33 & Viene verificata che la richiesta di \texttt{getnetwork/:net} al server, chiami il metodo \texttt{parserNetworkNameURL} & N.I. \\ 
\hline
\rowcolor{grigio} TU0-34 & Viene verificato che il metodo \texttt{parserNetworkNameURL} ritorni il nome della rete parsato & N.I. \\ 
\hline 
TU0-35 & Viene verificato che il metodo \texttt{parserNetworkNameURL} ritorni \textit{false} nel caso in cui: il parametro passato sia stringa vuota, la rete non esiste oppure il parametro non è definito & N.I. \\ 
\hline 
\rowcolor{grigio}TU0-36 & Viene verificato che la richiesta \texttt{getnetwork/:net} ritorni un json con la definizione della rete richiesta & N.I. \\ 
\hline 
TU0-37 & Viene verificato il lancio di un'eccezione nel caso in cui la lettura della rete da filesystem sia fallita & N.I. \\
\hline
 \rowcolor{grigio} TU0-38 & Viene verificato, in caso di eccezione nella richiesta \texttt{getnetwork/:net} che quest'ultima ritorni un messaggio di errore con stato \textit{404} & N.I. \\ 
 \hline
 TU0-39 & Viene verificato che la richiesta \texttt{networkslive} al server, ritorni un json contenente le reti monitorate in un dato istante di tempo & N.I. \\ 
 \hline 
 \rowcolor{grigio}TU0-40 & Viene verificato che la richiesta \texttt{deletenetwork/:net} al server, ritorni un messaggio di successo nel caso in cui la rete sia stata eliminata & N.I. \\
 \hline
 TU0-41 & Viene verificato che la richiesta \texttt{deletenetwork/:net} al server, ritorni un messaggio d'errore nel caso in cui il nome della rete da eliminare sia vuoto,  non definito o non esista nel filesystem del server & N.I. \\ 
 \hline
\rowcolor{grigio}TU0-42 & Viene verificato il lancio di un'eccezione nel caso in cui la rete da eliminare non esista & N.I. \\ 
\hline
 TU0-43 & Viene verificato il lancio di un'eccezione nel caso in cui il file di salvataggio della rete non esista nel filesystem del server & N.I. \\ 
 \hline
 \rowcolor{grigio} TU0-44 & Viene verificato che la richiesta \texttt{addtopool/:net} al server richiami il metodo \texttt{parserNetworkNameURL} & N.I. \\ 
 \hline
 TU0-45 & Viene verificato che la richiesta \texttt{addtopool/:net} al server ritorni un messaggio di errore nel caso in cui il nome della rete non sia valido o che non esista la rete desiderata & N.I. \\ 
 \hline
 \rowcolor{grigio}TU0-46 & Viene verificata che la richiesta \texttt{addtopool/:net} al server richiami il metodo \texttt{addToPool(net)} & N.I. \\ 
 \hline
 TU0-47 & Viene verificato che il metodo \texttt{addToPool(net)} ritorni aggiunga nel pool di monitoraggio la rete desiderata & N.I. \\
\hline
\rowcolor{grigio} TU0-48 & Viene verificato che il metodo \texttt{addToPool(net)} ritorni true nel caso in cui la rete sia stata aggiunta al pool di monitoraggio con successo & N.I. \\ 
\hline 
TU0-49 & Viene verificato che il metodo \texttt{addToPool(net)} ritorni false nel caso in cui la rete da monitorare e già monitorata & N.I. \\ 
\hline
\rowcolor{grigio} TU0-50 & Viene verificato che la richiesta \texttt{addtopool/:net} ritorni un messaggio di successo nel caso in cui la rete sia stata aggiunta al pool di monitoraggio & N.I. \\ 
\hline 
TU0-51 & Viene verificato che la richiesta \texttt{addtopool/:net} al server ritorni un messaggio di errore nel caso in cui la rete desiderata è già aggiunta al pool di monitoraggio & N.I. \\ 
\hline 
\rowcolor{grigio} TU0-52 & Viene verificato che la richiesta \texttt{getnetworkprob/:net} al server richiami il metodo \texttt{parserNetworknameURL(name)} &  N.I. \\ 
\hline 
TU0-53 & Viene verificato che la richiesta \texttt{getnetworkprob/:net} al server ritorni un messaggio di errore con stato \textit{404} nel caso in cui il nome passato a parametro non sia valido & N.I. \\ 
\hline
\rowcolor{grigio} TU0-54 & Viene verificato che la richiesta \texttt{getnetworkprob/:net} al server ritorni un messaggio di errore nel caso in cui la rete desiderata non sia in monitoraggio & N.I. \\ 
\hline 
TU0-55 & Viene verificato che la richiesta \texttt{getnetworkprob/:net} al server ritorni un json contenente le probabilità calcolate per la rete desiderata & N.I. \\ 
\hline 
\rowcolor{grigio} TU0-56 & Viene verificato che la richiesta \texttt{deletenetpool/:net} al server richiami il metodo \texttt{parserNetworkNameURL(name)} & N.I. \\ 
\hline
TU0-57 & Viene verificato che la richiesta \texttt{deletenetpool/:net} al server richiami il metodo \texttt{deleteFromPool(net)} & N.I. \\ 
\hline 
\rowcolor{grigio} TU0-58 & Viene verificato che il metodo \texttt{deleteFromPool(net)} ritorni true nel caso in cui la rete viene tolta dal monitoraggio & N.I. \\ 
\hline
TU0-59 & Viene verificato che il metodo \texttt{deleteFromPool(net)} ritorni false nel caso in cui l'eliminazione della rete dal monitoraggio fallisca & N.I. \\ 
\hline 
\rowcolor{grigio} TU0-60 & Viene verificato che la richiesta \texttt{deletenetpool/:net} ritorni un messaggio di successo nel caso in cui la rete viene tolta dal monitoraggio & N.I. \\ 
\hline 
TU0-61 & Viene verificato che la richiesta \texttt{deletenetpool/:net} al server, ritorni un messaggio di errore con stato \textit{404} nel caso in l'eliminazione della rete dal monitoraggio sia fallita & N.I. \\ 
\hline 
\rowcolor{grigio} TU0-62 & Viene verificato che all'avvio del server venga invocato il metodo \texttt{startServer()} & N.I. \\ 
\hline
TU0-63 & Viene verificato che all'avvio del server vengano inizializzate le reti salvate nel filessytem & N.I. \\ 
\hline
\rowcolor{grigio}TU0-64 & Viene verificato il lancio di un'eccezione nel caso in cui la lettura da filesystem all'inizializzazione delle reti salvata fallisca & N.I. \\ 
\hline
TU0-65 & Viene verificata l'inizializzazione delle connessioni necessarie al database per ogni rete & N.I. \\ 
\hline
\rowcolor{grigio}TU0-66 &  Viene verificata la creazione di una connessione al database al richiamo del metodo \texttt{initDatabaseConnection(connection)} & N.I.\\ 
\hline
TU0-67 & Viene verificato il lancio di un'eccezione nel caso in cui i parametri di connessione al database siano errati & N.I. \\
\hline
\rowcolor{grigio} TU0-68 & Viene verificato il lancio di un'eccezione nel caso in cui la creazione della connessione al database fallisca & N.I. \\ 
\hline
TU0-69 & Viene verificato che il metodo di creazione di connessione al database ritorni true se la connessione è avvenuta con successo & N.I. \\ 
\hline 
\rowcolor{grigio} TU0-70 & Viene verificata l'istanziazione effettiva della connessione al database nell'array di connessioni disponili sul server & N.I. \\ 
\hline 
TU0-71 & Viene verificato che il monitoraggio di una rete si ripeta secondo le politiche temporali definite dall'utente & N.I. \\ 
\hline 
\rowcolor{grigio} TU0-72 & Viene verificato che le probabilità calcolate nella rete bayesiana vengano salvate all'interno del database  & N.I. \\ 
\hline 
TU0-73 & Viene verificato che il metodo \texttt{writeMesure(net)}  ritorni true nel caso in cui la scrittura delle probabilità sul database sia avvenuta con successo & N.I. \\
\hline
\rowcolor{grigio}TU0-74 & Viene verificato che il metodo \texttt{writeMesure(net)} ritorni false nel caso in cui la scrittura delle probabilità sul database sia fallita & N.I. \\ 
\hline
TU0-75 & Viene verificata la costruzione corretta di una rete & N.I.  \\ 
\hline 
\rowcolor{grigio}TU0-76 & Viene verificata la creazione della tabella di probabilità per ogni nodo della rete bayesiana & N.I. \\ 
\hline

\caption{Test di unità}
\label{testunita}
\end{longtable}




























