\section{Installazione}\label{Installazione}
Per poter utilizzare il prodotto all'interno di Grafana, e poiché quest'ultimo sia riconosciuto come plug-in, è necessario per prima cosa eseguire la build del prodotto.
La build del prodotto avviene attraverso due fasi: la prima che va a risolvere le dipendenze, la seconda che esegue la build.\\
Il processo di build avviene attraverso l'intervento del module bundler \textit{WebPack}.
È necessario quindi eseguire i seguenti comandi dalla directory che ospita il plug-in:
\begin{center}
	\texttt{npm install}\\
	\texttt{npm run build}.
\end{center}

\subsection{Requisiti}\label{Requisiti}
I requisiti minimi richiesti per il funzionamento del plug-in non sono dovuti al prodotto in sè, ma sono dovuti alle tecnologie che vengono utilizzate. Pertanto si rimanda ai requisiti minimi delle seguenti tecnologie:
\begin{itemize}
	\item \textit{InfluxDB};
	\item \textit{Telegraf};
	\item \textit{Grafana};
	\item \textit{NodeJS}.
\end{itemize}

\subsection{Esecuzione}\label{run}
Per poter eseguire il prodotto è necessario seguire i seguenti passi:
\begin{enumerate}
	\item Arrestare l'esecuzione di Grafana, qualora fosse in esecuzione;
	\item Copiare la directory del progetto all'interno della cartella predestinata da Grafana per ospitare i plug-in aggiuntivi da installare;
	\item Eseguire i comandi descritti in §\ref{Installazione};
	\item Assicurarsi che il server esterno a cui sono delegati i calcoli sia attivo e raggiungibile;
	\item Avviare Grafana e procedere all'aggiunta del plug-in.
\end{enumerate}
Per informazioni relative al funzionamento del plug-in si rimanda al Manuale Utente.