\section{Test}\label{Test}

\subsection{Scopo del Capitolo}\label{Test_scopo}
Questo capitolo ha lo scopo di indicare agli sviluppatori come controllare in modo automatico il proprio codice e la sintassi.

\subsection{Test sul Codice \textit{JavaScript}}\label{test_JS}
Per eseguire i test sul codice è necessario eseguire i seguenti comandi:
\begin{center}
	\texttt{npm run test}\\
	o\\
	\texttt{jest}.
\end{center}
Per verificare che il codice sia coerente con le linee guida adottate è necessario eseguire:
\begin{center}
	\texttt{npm run eslint}.
\end{center}
Per correggere in modo automatico alcuni dei potenziali problemi, eseguire:
\begin{center}
	\texttt{npm run eslint--fix}.
\end{center}
Questi script eseguono un controllo del codice all'interno di "\texttt{./src}", la directory dov'è contenuto il codice.\\
Se si desidera eseguire \textit{ESLint} su un unico file, si rimanda il lettore alla documentazione ufficiale.

\subsection{Code Coverage}\label{test_codecoverage}
Per eseguire il controllo di copertura del codice \textit{JavaScript}, è necessario eseguire il seguente comando:
\begin{center}
	\texttt{npm run codecov}.
\end{center}
Nel caso tale comando fallisca la principale motivazione è data dall'assenza di report su cui generare il code coverage. Per rimediare a ciò basterà eseguire il primo comando nella sezione §\ref{test_JS} il quale provvederà a generare i report necessari.
