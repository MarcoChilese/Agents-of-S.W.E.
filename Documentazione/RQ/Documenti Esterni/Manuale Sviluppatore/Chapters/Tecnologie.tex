\section{Principali Tecnologie Utilizzate}\label{tecnologie}
\subsection{Scopo del Capitolo}
In questo capitolo verranno esposte e contestualizzate le tecnologie impiegate all'interno del progetto, in modo tale da offrire agli sviluppatori un quadro più completo del progetto.

\subsection{Tecnologie Lato Client}\label{clientTec}
Lato client vengono adottate le seguenti tecnologie:
\begin{itemize}
	\item \textit{EcmaScript 6}: principale linguaggio con cui il plug-in è strutturato, in particolar modo viene utilizzato per lo sviluppo del pannello;
	\item \textit{AngularJS}: framework adottato da \textit{Grafana} per l'interazione con l'utente;
	\item \textit{HTML \& CSS}: rispettivametne nelle versioni 5 e 3, utilizzati per modellare l'interfaccia del pannello.
\end{itemize}

\subsection{Tecnologie Lato Server}\label{serverTec}
Lato server vengono adottate le seguenti tecnologie:
\begin{itemize}
	\item \textit{NodeJS}: tecnologia utilizzata per implementare il server.
\end{itemize}

\subsection{Tecnologie per il Testing}\label{testTec}
La tecnologia utilizzata per testare il codice è \textit{Jest}, framework di test per codice \textit{JavaScript}.