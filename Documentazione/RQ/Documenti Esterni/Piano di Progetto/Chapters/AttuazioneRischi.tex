\section{Attualizzazione dei Rischi}
\label{AR}

In questa sezione descriveremo i rischi attualizzatisi durante ogni periodo e le tecniche utilizzate al fine di ripristinare uno stato corretto di avanzamento del progetto.


\subsection{Avvio e Analisi dei Requisiti}\label{ARAvvioAnalisi}

\begin{longtable}{|C{.15\textwidth}|C{.79\textwidth}|}
\hline
\rowcolor{bluelogo}\textbf{\textcolor{white}{Rischio}} & \textbf{\textcolor{white}{Descrizione}} \\
\hline \hline
\endfirsthead
G01 & Non è stato chiarito dal principio il modo in cui avanzare nella stesura dei documenti, questo ha causato l'utilizzo di più ore rispetto a quelle preventivate.\\ 
\hline
\rowcolor{grigio}T04 &  L'utilizzo di \textit{GitHub} per i neofiti non è stato immediato e questo ha portato alcuni problemi nell'utilizzo di questa tecnologia. \\
\hline
T08 & L'utilizzo di \LaTeX per la stesura della documentazione ha causato alcuni ritardi in quanto necessitava prima di uno studio delle sue funzionalità base. \\
\hline

\caption{Rischi Verificatisi, periodo Avvio e Analisi dei Requisiti}
\label{tab:rischiVerificatisiAvvioAnalisi}
\end{longtable}

\begin{longtable}{|C{.15\textwidth}|C{.79\textwidth}|}
\hline
\rowcolor{bluelogo}\textbf{\textcolor{white}{Rischio}} & \textbf{\textcolor{white}{Risoluzione}} \\
\hline \hline
\endfirsthead
G01 & Miglioramento della comunicazione e condivisione delle scelte intraprese con tutti i membri del gruppo.\\ 
\hline
\rowcolor{grigio}T04 & I neofiti di \textit{GitHub} sono stati aiutati dai membri del gruppo che già lo avevano utilizzato. \\
\hline
T08 & Utilizzo di tempo personale al fine di apprendere \LaTeX. \\
\hline

\caption{Risoluzione Rischi Verificatisi, periodo Avvio e Analisi dei Requisiti}
\label{tab:risoluzioneRischiAvvioAnalisi}
\end{longtable}
\pagebreak

\subsection{Progettazione Architetturale}\label{ARProgArchi}

\begin{longtable}{|C{.15\textwidth}|C{.79\textwidth}|}
\hline
\rowcolor{bluelogo}\textbf{\textcolor{white}{Rischio}} & \textbf{\textcolor{white}{Descrizione}} \\
\hline \hline
\endfirsthead
T04 & La documentazione non sempre precisa della piattaforma \textit{Grafana} ha portato difficoltà inaspettate per l'operazione di impacchettamento del plug-in. Sono state testate svariate tecnologie prima di giungere ad una soluzione accettabile attraverso l'uso di \textit{Webpack}\glossario \\ 
\hline
\rowcolor{grigio}L09 &  La suddivisione dei compiti in fase di codifica, e l'attuazione di una metodologia di lavoro di stampo collaborativo e coordinato si è rivelata essere di difficile realizzazione. In particolare, di fronte a problemi inaspettati, il gruppo ha faticato a seguire la pianificazione delle attività mantenendo una metodologia di lavoro disciplinata\\
\hline
\caption{Rischi Verificatisi, periodo Progettazione Architetturale}
\label{tab:rischiVerificatisiPArchitetturale}
\end{longtable}

\begin{longtable}{|C{.15\textwidth}|C{.79\textwidth}|}
\hline
\rowcolor{bluelogo}\textbf{\textcolor{white}{Rischio}} & \textbf{\textcolor{white}{Risoluzione}} \\
\hline \hline
\endfirsthead
T04 & Ci siamo documentati attraverso esempi già disponibili in rete ed attraverso una più attenta lettura della documentazione. \\ 
\hline
\rowcolor{grigio}L09 & È stata stabilita la necessità di avere una maggiore comunicazione all'interno del gruppo per le consegne a venire, al fine di evitare il ripetersi del rischio in esame. Il \textit{Responsabile} dovrà porre una maggiore attenzione sul lavoro in corso. \\
\hline
\caption{Risoluzione Rischi Verificatisi, periodo Progettazione Architetturale}
\label{tab:risoluzioneRischiAvvioAnalisi}
\end{longtable}

\subsection{Progettazione di Dettaglio e Codifica}\label{ARRQ}
\begin{longtable}{|C{.15\textwidth}|C{.79\textwidth}|}
	\hline
	\rowcolor{bluelogo}\textbf{\textcolor{white}{Rischio}} & \textbf{\textcolor{white}{Risoluzione}} \\
	\hline \hline
	\endfirsthead
	NP01 & A causa della non totale comprensione dei requisiti impliciti collegati al monitoraggio dei dati, è nata la necessità di introdurre un'elaborazione dati lato server che ha comportato l'introduzione di una tecnologia non preventivata: NodeJS.\\ 
	\hline
	\caption{Rischi Verificatisi, periodo Progettazione di Dettaglio e Codifica}
	\label{tab:analisiRischiRQ}
\end{longtable}

\begin{longtable}{|C{.15\textwidth}|C{.79\textwidth}|}
	\hline
	\rowcolor{bluelogo}\textbf{\textcolor{white}{Rischio}} & \textbf{\textcolor{white}{Risoluzione}} \\
	\hline \hline
	\endfirsthead
	NP01 & Il team, per far fronte al problema ha cercato di comprendere ed imparare ad utilizzare la nuova tecnologia in modo soddisfacente per poter introdurre in modo ottimale la nuova componente del progetto.\\ 
	\hline
	\caption{Risoluzione Rischi Verificatesi, periodo Progettazione di Dettaglio e Codifica}
	\label{tab:risoluzioneRischiRQ}
\end{longtable}

