\section*{U}
\addcontentsline{toc}{section}{U}

\subsection{\index{UML 2.0}UML 2.0}
UML, \textbf{U}nified \textbf{M}odeling \textbf{L}anguage, è un linguaggio di modellazione e specifica, basato sul paradigma orientato agli oggetti. Inizialmente nato come un linguaggio "libero" e non standardizzato, nel 1996 vennero unificati i diversi approcci creando uno standard. \-\\
La versione 2.0 fu ufficializzata nel 2005, includendo notevoli miglioramenti alla versione precedente (1.5). Attualmente la versione più recente è la 2.5, tuttavia la 2.0 è la più diffusa e supportata.

\subsection{\index{UnBBayes}UnBBayes}
UnBBayes è un framework per reti probabilistiche scritto in Java. \\
È reperibile al link: \url{https://sourceforge.net/projects/unbbayes}.

\subsection{\index{Use Case}Use Case}
Tecnica usata nei processi di ingegneria del software per effettuare in maniera esaustiva e non ambigua, la raccolta dei requisiti. Consiste nel valutare ogni requisito focalizzandosi sugli attori che interagiscono col sistema, valutandone le varie interazioni.

\subsection{\index{User-Friendly}User-Friendly} 
Qualità, solitamente di un prodotto software, che rappresenta l'immediatezza di utilizzo anche per chi non è esperto.
