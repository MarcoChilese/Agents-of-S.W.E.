\section*{P}
\addcontentsline{toc}{section}{P}

\subsection{\index{Part-of-Speech Tagger}Part-of-Speech Tagger}
È il processo di identificare una parola in un testo come corrispondente a una particolare parte del discorso, in base sia alla sua definizione che al suo contesto, cioè la sua relazione con parole adiacenti e correlate in una frase o paragrafo.

\subsection{\index{Pattern Publisher/Subscriber}Pattern Publisher/Subscriber}
L'espressione publish/subscribe si riferisce a un design pattern o stile architetturale utilizzato per la comunicazione asincrona fra diversi processi, oggetti o altri agenti.

\subsection{\index{PDCA}PDCA}
È un metodo di gestione iterativo in quattro fasi, utilizzato per il controllo e il miglioramento continuo dei processi e dei prodotti.

\subsection{\index{Piano di Progetto}Piano di Progetto}
Documento contenente la pianificazione del lavoro di un gruppo di lavoro per portare a termine un determinato progetto. Esso presenta l'organizzazione delle attività e dei tempi, un preventivo delle risorse necessarie al compimento del progetto, consuntivi di periodo ed analisi di rischi e piani di mitigazione per tali rischi.

\subsection{\index{Piano di Qualifica}Piano di Qualifica}
Documento contente le strategie di verifica e validazione adottate da un gruppo di lavoro per assicurare la qualità di prodotto e di processo in un determinato progetto.

\subsection{\index{Plug-in}Plug-in}
Componente aggiuntivo che interagisce con un altro programma per ampliarne le funzioni.

\subsection{\index{Prodotto}Prodotto}
Output di un processo. In ambito software è un'entità progettata per essere rilasciata all'utilizzatore finale.

\subsection{\index{Product Baseline}Product Baseline}
Presenta la baseline architetturale del prodotto, coerente rispetto a quando riportato nella Technology Baseline. Al suo interno contiene i diagrammi delle classi e di sequenza, la contestualizzazione dei design pattern adottati nell'architettura del prodotto.

\subsection{\index{Progettazione}Progettazione}
Fase del ciclo di vita del software che risponde alla domanda \textit{"Coma va fatta la cosa giusta?"}. È un'attività che, sulla base della specifica dei requisiti prodotta dall'analisi, definisce come tali requisiti debbano essere soddisfatti, ricercando una soluzione soddisfacente per tutti gli stakeholders.

\subsection{\index{Progetto}Progetto}
Insieme di attività e compiti che hanno come proprietà caratteristiche:
\begin{itemize}
	\item Obiettivi Prefissati;
	\item Tempi Fissati, ovvero precise scadenze;
	\item Risorse Limitate che vengono consumate dalle attività di progetto.
\end{itemize}

\subsection{\index{Promise}Promise}
Una \textit{promise} rappresenta un'operazione che non è stata ancora completata, ma che lo sarà in futuro. Gli oggetti \textit{promise} vengono utilizzati per computazioni in differita e asincrone.

\subsection{\index{PoC}Poc}
Vedi \textbf{Proof of Concept}.

\subsection{\index{Proof of Concept}Proof of Concept}
In ambito informatico, consiste nella dimostrazione pratica dei funzionamenti di base di un applicativo o intero sistema.

\subsection{\index{Prototipo}Prototipo}
Modello realizzato nell'ultima fase della progettazione e sperimentazione, e destinato a divenire il punto di partenza della produzione.

\subsection{\index{Pull}Pull}
Comando del sistema di versionamento Git, scarica dal repository remoto tutti i commit effettuati dagli altri membri del team e li applica ai propri file.

\subsection{\index{Push}Push}
Comando del sistema di versionamento Git, permette di inviare al repository remoto le modifiche per cui esiste un commit in locale.

\subsection{\index{Python}Python}
Linguaggio di programmazione ad alto livello.
