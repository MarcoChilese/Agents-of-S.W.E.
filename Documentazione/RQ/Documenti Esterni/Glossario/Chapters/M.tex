\section*{M}
\addcontentsline{toc}{section}{M}

\subsection{\index{Maven}Maven}
È uno strumento di build automation che permette la gestione di progetti software scritti in Java.

\subsection{\index{Merge}Merge} 
Operazione fondamentale dei sistemi di controllo di versione. Concilia in modo appropiato svariate modifiche apportate ad una raccolta di files controllata tramite versioning.

\subsection{\index{MetaMask}MetaMask}
Permette agli utenti l'interfacciamento con la rete ethereum tramite browser.

\subsection{\index{Milestone}Milestone} 
Indica un importante traguardo intermedio nello svolgimento del progetto. Può derivare da un obbligo contrattuale o da un'opportunità decisa dal gruppo.

\subsection{\index{Module Bundler}Module Bundler}
In generale è una libreria in grado di gestire il caricamento e la gestione dei moduli interdipendenti. Durante il processo di build viene creato lo schema delle dipendenze dei moduli, a differenza del load bundler che crea lo schema delle dipendenze durante la fase di esecuzione. Alcuni esempi di module bundler sono Webpack e Browserify.

\subsection{\index{MVC}MVC (Model-View-Controller)}
È un pattern architetturale molto diffuso nella programmazione orientata agli oggetti in grado di separare la logica di presentazione dalla logica di business. È composto da tre componenti: il "model" che fornisce i metodi per accedere ai dati all'applicazione, la "View" visualizza i dati contenuti nel model e si occupa dell'iterazione con gli utenti, e infine il "Controller" che riceve i comandi dall'utente attraverso la view e va a modificare lo stato degli altri due componenti (model e view).
