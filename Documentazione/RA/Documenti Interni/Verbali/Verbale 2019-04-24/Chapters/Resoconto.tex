\section{Resoconto}

\subsection{Punto 1}
Durante l'incontro si è discusso sulle segnalazioni individuate dal committente in sede di Revisione di Qualifica e si è proceduto con l'individuazione di task da assegnare ad ogni membro del gruppo per correggerli.
\\
Principalmente le criticità sulle \textit{Norme di Progetto v3.0.0} riguardano l'adesione allo standard ISO/IEC 12207, Matteo Slanzi si occuperà delle correzioni necessarie, mentre Diego Mazzalavo procederà con la correzione del ciclo di Deming. \\
Nel documento  \textit{Piano di Qualifica v3.0.0} non è presente la metrica di Code Coverage, Carlotta Segna aggiungerà la metrica mancante. Inoltre Luca Violato e Carlotta Segna aggiungeranno ulteriori test di unità. \\
Nel documento \textit{Piano di Progetto v3.0.0}, Marco Favaro si occuperà di aggiungere le riflessioni sul miglioramento della pianificazione del periodo successivo, mentre Marco Chilese aggiungerà maggior approfondimento negli incrementi della fase di codifica. \\
Per quanto riguarda il documento \textit{Analisi dei Requisiti v3.0.0} non sono emerse criticità rilevanti, ma non può ancora considerarsi definitivo. Luca Violato procederà con la correzione delle criticità segnalate.

\subsection{Punto 2}
Nei documenti \textit{Manuale Sviluppatore v1.0.0} e \textit{Manuale Utente v1.0.0} non è stato inserito il glossario, Marco Favaro aggiungerà il glossario ai manuali. \\
Marco Chilese si occuperà di sanare le criticità del \textit{Manuale Sviluppatore v1.0.0} aggiungendo i requisiti minimi delle varie tecnologie utilizzate, inoltre aggiungerà un paragrafo sull'estensione delle funzionalità del prodotto. \\
Nel \textit{Manuale Utente v1.0.0} è stato valutato positivamente lo stile documentale ricco di immagini, ma che non aderisce allo stile guida/tutorial. Luca Violato apporterà le modifiche necessarie per rendere lo stile più simile a una guida ed inoltre aggiungerà una sezione per la segnalazione dei malfunzionamenti ed un'ulteriore sezione per le possibili domande dell'utente nell'utilizzo del prodotto. \\
Diego Mazzalavo aggiungerà la sezione dove viene definita la struttura del file \textit{.json} per la definizione della rete bayesiana, descrivendo all'utente il modo corretto di definire il file \textit{.json} da importare nel plug-in.  

\subsection{Punto 3}
I ruoli all'interno del team, dopo la Revisione di Qualifica, ruotano nel seguente modo:\\

\begin{center}
	\begin{longtable}[c]{|m{.20\textwidth}|m{.20\textwidth}|m{.20\textwidth}|} 
		\hline
		\rowcolor{bluelogo}\textbf{\textcolor{white}{Membro}} & \textbf{\textcolor{white}{Vecchio Ruolo}} & \textbf{\textcolor{white}{Nuovo Ruolo}}\\
		\hline
		\hline
		Luca Violato & Responsabile & Responsabile \\
		\hline
		\rowcolor{grigio}Bogdan Stanciu & Progettista & Progettista \\
		\hline
		Marco Chilese & Programmatore & Analista\\
		\hline
		\rowcolor{grigio}Carlotta Segna & Amministratore & Amministratore\\
		\hline
		Marco Favaro & Progettista & Verificatore \\
		\hline
		\rowcolor{grigio} Diego Mazzalovo & Verificatore & Programmatore\\
		\hline
		Matteo Slanzi & Analista & Programmatore\\
		\hline
		\caption{Rotazione dei Ruoli}
	\end{longtable}

\end{center}

\subsection{Punto 4}
Si è discusso sui primi test da implementare e sulla suddivisione tra i vari membri del gruppo. 
I test di unità e validazione verranno suddivisi tra Marco Favaro e Bogdan Stanciu, Diego Mazzalovo si occuperà di scrivere alcuni test di integrazione.
	
\subsection{Punto 5}
A seguito delle criticità emerse in sede di Revisione di Qualifica, in modo unanime si è deciso di contattare il Prof. Tullio Vardanega per un colloquio per chiarire le ultime criticità da risanare.

\subsection{Punto 6}
La prossima riunione è stata fissata in data 2019-04-28.