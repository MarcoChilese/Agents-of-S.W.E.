\section{Resoconto}

\subsection{Punto 1}
Come primo punto di discussione sono state analizzate le criticità segnalate dal commitente riguardanti la sezione §2.1 del documento \textit{Norme di Progetto}, ovvero, nello specifico, il processo di fornitura.\\
In tal senso il committente ha portato alla luce quella che è stata una suddivisione errata, da parte del gruppo, delle attività tra i processi di Fornitura e Sviluppo. Nello specifico infatti svariate attività attinenti al processo di Sviluppo sono state inserite sotto quello di Fornitura, evidenziando come il gruppo non si fosse attenuto allo standard ISO/IEC 12207.\\
Il committente dunque, dopo aver ricordato che il processo di Fornitura è caratterizzato da tutte quelle attività che caratterizzano il rapporto con committente e proponente, ha consigliato caldamente di visionare con attenzione lo standard di riferimento ISO/IEC12207 ed adeguare ad esso i contenuti delle \textit{Norme di Progetto}.


\subsection{Punto 2}
Come secondo punto di discussione il gruppo ha chiesto una delucidazione in merito alla gestione della sezione §4.4 del documento \textit{Norme di Progetto v3.0.0}. Tale sezione, denominata "Strumenti", è stata infatti indicata come non appropriata dal committente in sede di RQ. Nello specifico è stato segnalato che porre strumenti allo stesso livello di processi fosse un errore non trascurabile.\\
Il gruppo ha dunque voluto chiedere consiglio su come gestire tale sezione, la cui presenza è ovviamente necessaria all'interno del documento. Nello specifico il gruppo ha domandato se vi fosse uno specifico processo, o attività, all'interno del quale inserire tale sezione, o fosse invece più opportuno suddividerne i contenuti tra le varie attività normate.\\
Il committente ha per prima cosa ricordato come un documento debba essere scritto per il lettore, non per lo scrittore. Di conseguenza, tra le tante qualità che un buon documento dovrebbe possedere, è necessario garantire \textbf{coerenza} e \textbf{coesione}. In tale ottica è stato fatto notare come avere una sezione a sè stante di descrizione degli strumenti adottati andasse contro la \textbf{coesione}, e dunque non fosse appropriata.\\
Il committente ha dunque invitato il gruppo a realizzare, per ogni attività descritta nel documento \textit{Norme di Progetto}, una sottosezione di descrizione degli strumenti a supporto di tali attività. Il committente ha inoltre ricordato come, nell'attuazione di tale correttivo, la presenza di tale sottosezione sia necessaria per ogni attività al fine di garantire la caratteristica di sistematicità del lavoro realizzato.


\subsection{Punto 3}
La parte finale dell'incontro si è spostata come analisi e discussione verso il \textit{Manuale Sviluppatore}.\\
Alcune criticità rilevate dal committente su questo documento riguardano i diagrammi, di classe e package, i quali non vengono adeguatamente corredati da una descrizione testuale. In tale ottica il gruppo ha domandato a che livello di dettaglio sia necessario descrivere testualmente tali diagrammi.\\
Il committente ha nuovamente ribadito che i documenti devono essere scritti nell'ottica di un futuro lettore, in tal senso è dunque necessario che un utente esterno abbia a disposizione tutte le informazioni necessarie alla comprensione. Il committente ha dunque evidenziato la necessità di corredare ogni diagramma con una descrizione testuale, in testa o in coda, sufficientemente dettagliata da far capire ad un esterno il funzionamento del sistema.\\
Infine il committente ha fatto notare come ogni documento dovrebbe contenere solo le informazioni necessarie ad adempiere al proprio scopo. In tal senso dunque il \textit{Manuale Sviluppatore} dovrebbe concentrarsi sul funzionamento del prodotto nell'ottica di descrivere punti di estensione e linee guida per la manutenzione. Il committente ha dunque invitato il gruppo a riflettere sull'effettiva necessità di alcuni diagrammi inclusi nel \textit{Manuale Sviluppatore}.

