Insieme al \textit{CSS} e all'\textit{HTML} il team ha deciso di sviluppare la parte front-end appoggiandosi al framework \textit{JavaScript Angularjs}\glossario.\\
\textit{AngularJS} è un framework per applicazioni web, sviluppato con lo scopo di affrontare le difficoltà incontrate con lo sviluppo di applicazioni su singola pagina di tipo client-side.\-\\
La caratteristica principale è l'introduzione di funzionalità chiamate direttive, che permettono di creare componenti \textit{HTML} personalizzati e riusabili con lo scopo di associare a ogni direttiva un comportamento, così nascondendo la complessità della struttura DOM\glossario. Le direttive sono riconoscibili poiché integrate nei tag con la dicitura "ng-" e susseguite dal nome (scelto dallo sviluppatore). 
Di seguito sono riportate alcune norme da seguire:\\
\textbf{Norma 1}\\
Verrà seguito il pattern Model View Controller(MVC)\glossario che propone la scomposizione in tre componenti:
\begin{itemize}
	\item \textbf{Model}: responsabile per il mantenimento e l'elaborazione dei dati;
	\item \textbf{View}: responsabile per la visualizzazione dei dati all'utente;
	\item \textbf{Controller}: si occupa di controllare l'interazione tra Model e View.
\end{itemize} 
\textbf{Norma 2}\\
Il controller non deve essere sovraccaricato di funzionalità che non gli appartengono. In particolare il controller non deve:
\begin{itemize}
	\item manipolare il DOM, questo renderà i controller difficili da testare. Questo compito è riservato alle direttive;
	\item formattare l'input, a questo scopo vengono usati i form \textit{Angular};
	\item formattare l'output, sarà compito dei filtri;
	\item condividere dati con altri controller;
	\item implementare funzionalità generali, le funzionalità che non riguardano direttamente l'interazione tra dati e la visualizzazione, deve essere fatta nei servizi.
\end{itemize}
\textbf{Norma 3}\\
Il nome dei moduli seguono la notazione lowerCamelCase;
\begin{lstlisting} [language=HTML]
<div ng-app="myModule">...</div>
\end{lstlisting}
\-\\
\textbf{Norma 4}\\
Quando è presente un modulo $b$ che è submodule di $a$, si possono concatenare usando la notazione $a.b$;\\
\-\\
\textbf{Norma 5}\\
Il nome dei controller viene assegnato in base alla loro funzionalità e viene usata la notazione UpperCamelCase;
\begin{lstlisting} [language=HTML]
<div ng-controller="MyController">
  {{ greeting }}
</div>
\end{lstlisting}
\-\\
\textbf{Norma 6}\\
I nomi delle restanti direttive seguono la notazione camelCase;
\begin{lstlisting} [language=HTML]
<select class="myClass" ng-model="myModel">
\end{lstlisting}
\-\\
\textbf{Norma 7}\\
Le direttive \textit{AngularJS} vengono scritte alla fine degli attributi di un tag: 
\begin{lstlisting} [language=HTML]
<!-- OK -->
<p>Nome: <input type="text" ng-model="utente.nome"></p>
	
<!-- NO -->
<p>Nome: <input ng-model="utente.nome" type="text"></p>
\end{lstlisting}
\-\\
\textbf{Norma 8}\\
Il prefisso \$ è riservato ad \textit{AngularJS}, non deve essere usato per nomi di variabili, proprietà o metodi;\\
\-\\
\textbf{Norma 9}\\
I controller non devono essere definiti globali;\\
\-\\
\textbf{Norma 10}\\
Usare prefissi personalizzati per le direttive per evitare collisioni con librerie di terze parti;\\
\-\\
\textbf{Norma 11}\\
In \textit{AngularJS} viene adottato il two-way binding, ogni modifica al modello dati si riflette sulla view e ogni modifica alla view viene riportata sul modello dati.
\-\\
Questa sincronizzazione avviene senza la necessità di scrivere codice particolare, è sufficiente associare il modello allo scope all'interno del controller ed utilizzare la direttiva ng-model nella view.