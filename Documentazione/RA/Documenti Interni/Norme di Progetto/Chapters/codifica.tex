Di seguito vengono definite delle norme che devono essere adottate dai \textit{Programmatori} per garantire una buona leggibilità  e manutenibilità  del codice. Le prime norme che seguiranno sono le più generali, da adottarsi per ogni linguaggio di programmazione adottato all'interno del progetto, in seguito quelle più specifiche per il linguaggio principale \textit{ECMAScript 6}\glossario.\\
Ogni norma è caratterizzata da un paragrafo di appartenenza, da un titolo, una breve descrizione e, se il caso lo richiede, un esempio.\\
Il rispetto delle seguenti norme è fondamentale per garantire uno stile di codifica uniforme all'interno del progetto, oltre che per massimizzare la leggibilità  e agevolare la manutenzione, la verifica\glossario e la validazione\glossario.

\paragraph{Convenzioni per i Nomi} \label{Nomi}
\begin{itemize}
	\item I \textit{Programmatori} devono adottare come notazione per la definizione di cartelle, file, metodi, funzioni e variabili il CamelCase\glossario.\\
	Di seguito un esempio di corretta nomenclatura:
	\begin{lstlisting}[language=JavaScript]
//Cartelle
./thisIsAFolder	//OK
./ThisIsAFolder //NO

//File
myFile.extension //OK
MyFile.extension //NO

//Funzioni
myFunction() { ... } //OK
MyFunction() { ... } //NO
	\end{lstlisting}

	\item Tutti i nomi devono essere unici ed autoesplicativi, ciò per evitare ambiguità  e limitare la complessità .
\end{itemize}
\paragraph{Convenzioni per la Documentazione}
\begin{itemize}
	\item Tutti i nomi ed i commenti al codice vanno scritti in inglese;
	\item Nel codice è possibile utilizzare un commento con denominazione TODO in cui si vanno ad indicare compiti da svolgere;
	\item L'intestazione di ogni file deve essere la seguente:
	\begin{lstlisting}[language=JavaScript]
/**
* File: nameFile
* Type: fileType
* Creation date: yyyy-mm-gg
* Author: Name Surname
* Author e-mail: email@example.com
* Version: versionNumber
*
* Changelog:
* #entry || Author || Date || Description
*/
	\end{lstlisting}
	\item La versione del file nell'intestazione deve rispettare la seguente formulazione: $X.Y.Z$, dove X rappresenta la versione principale, Y la versione parziale della relativa versione principale e Z l'avanzamento rispetto ad Y.\\ I numeri di versione del tipo $X.0.0$, dalla $1.0.0$, vengono considerate versioni stabili e quindi versioni da testare per saggiarne la qualità.
\end{itemize}
\paragraph{\textit{ECMAScript 6}}\label{EcmaScript6} \-\\
Seguendo le indicazioni presenti nella documentazione dell'azienda fornitrice di \textit{Grafana}, la piattaforma  per cui si intende sviluppare il plug-in, il team ha deciso di adottare come linguaggio di programmazione principale \textit{ECMAScript 6}\footnote{Linguaggio divenuto standard ISO: ISO/IEC 16262:2011, e relativo aggiornamento ISO/IEC 22275:2018.}.\\
\textit{ECMAScript 6} viene stardardizzato da \textit{ECMA}\glossario nel giugno 2015 con la sigla "ECMA-262".\\
Come stile di codifica si adottano le linee guida proposte da \textit{Airbnb JavaScript Style Guide}. Per la verifica dell'adesione a tali norme, i \textit{Programmatori} devono utilizzare, come suggerito dalla documentazione proposta da \textit{Grafana}, \textit{ESLint}\glossario.\\
In particolare i \textit{Programmatori} devono rispettare 5 linee guida proposte dalla documentazione ufficiale di \textit{Grafana}:
\begin{enumerate}
	\item Se una variabile non viene riutilizzata, deve essere dichiarata come \texttt{const};
	\item Utilizzare preferibilmente, per la definizione di variabili, la keyword  \texttt{let}, anziché  \texttt{var};
	\item Utilizzare il marcatore freccia (\texttt{\textbf{=>}}), in quanto non oscura il \texttt{this}:
	\begin{lstlisting}[language=JavaScript]
testDatasource() {
  return this.getServerStatus()
  .then(status => {
    return this.doSomething(status);
  })
}	
	\end{lstlisting}
	Invece che:
	\begin{lstlisting}[language=JavaScript]
testDatasource() {
  var self = this;
  return this.getServerStatus()
    .then(function(status) {
  return self.doSomething(status);
  })
}
	\end{lstlisting}
	\item Utilizzare l'oggetto \textit{Promise}:
	\begin{lstlisting}[language=JavaScript]
metricFindQuery(query) {
  if (!query) {
    return Promise.resolve([]);
  }
}	
	\end{lstlisting}
	Invece che:
	\begin{lstlisting}[language=JavaScript]
metricFindQuery(query) {
  if (!query) {
    return this.$q.when([]);
  }
}
	\end{lstlisting}
	%conseguenti?
	\item Se si utilizza \textit{Lodash}\glossario, per coerenza, lo si preferisca alle array function native di \textit{ES6}.
\end{enumerate}
Verranno esaminate di seguito le norme in merito allo stile di codifica che i \textit{Programmatori} dovranno adottare.

\subparagraph{Indentazione}\-\\
\textbf{Norma 1}\\
L'indentazione è da eseguirsi con tabulazione la cui larghezza sia impostata a due (2) spazi per ogni livello.\\
Di seguito un esempio da ritenersi corretto:
\begin{lstlisting}[language=JavaScript]
function() {
..let x = 2;
..if (x > 0)
....return true;
..else
....return false;
}
\end{lstlisting}
Qualsiasi altro tipo di indentazione è da ritenersi scorretta.\\
\-\\
\textbf{Norma 2}\\
Dopo la graffa principale va inserito uno (1) spazio. Nel seguente modo:
\begin{lstlisting}[language=JavaScript]
function() { ... }
\end{lstlisting}
\-\\
\textbf{Norma 3}\\
Dopo la keyword di un dato statement (\texttt{if, while}, etc.) va inserito uno (1) spazio. Per un esempio corretto si veda la norma successiva.\\
\-\\
\textbf{Norma 4}\\
Prima dell'apertura della graffa negli statement di controllo va inserito uno (1) spazio. Nel seguente modo:
\begin{lstlisting}[language=JavaScript]
function() {
  if (condition) {
    ...  
  }
  while (condition) {
    ...
  }
}
\end{lstlisting}
\-\\
\textbf{Norma 5}\\
Negli statement di controllo (\texttt{if, while}, etc.) le condizioni concatenate o annidate, mediante operatori logici, che diventano eccessivamente lunghe NON vanno espresse in un'unica linea, bensì spezzate in più righe. Nel seguente modo:
\begin{lstlisting}[language=JavaScript]
function() {
  if (condition && condition) {
    ...  
  }
  
  if (
   veryLongCondition
   && longCondition
   && condition
    ) {
    doSomething();
  }
}
\end{lstlisting}
\-\\
\textbf{Norma 6}\\
Dopo blocchi, o prima di un nuovo statement va lasciata una riga vuota. Nel seguente modo:
\begin{lstlisting}[language=JavaScript]
function1() {
  if (condition) {
    doSomething():  
  }
  
  return toReturn;  
}

function2(){
  ...
}
\end{lstlisting}
\-\\
\textbf{Norma 7}\\
I blocchi di codice multi-riga devono essere contenuti all'interno di graffe. Blocchi costituiti da una singola riga non è necessario che siano contenuti tra graffe: nel caso non vengano utilizzate, la definizione deve essere inline, cioè sulla stessa riga.\\
Nel seguente modo:
\begin{lstlisting}[language=JavaScript]
if (condition) return true;

if (condtion) {
  return true;
}
\end{lstlisting}

\subparagraph{Commenti al Codice}\-\\
\textbf{Norma 1}\\
Il codice va commentato nel seguente modo:
\begin{itemize}
	\item "\texttt{//}" se il commento occupa una sola riga;
	\item "\texttt{/** ... */} " se il commento occupa più righe.
\end{itemize}
Nel seguente modo:
\begin{lstlisting}[language=JavaScript]
// single line comment
if (condition) return true;

/**
* multi line comment, line 1
* multi line comment, line 2
*/
if (condtion) {
  return true;
}
\end{lstlisting}
\textbf{Norma 2}\label{Docs_Codice}\\
Il codice sviluppato va adeguatamente commentato con i commenti di documentazione appositi:
\begin{lstlisting}[language=JavaScript]
/**
 * Function or class description ...
 */
\end{lstlisting}
Il team fa riferimento a quanto indicato dal sito \textit{@use JSDoc}.\\
In particolari vengono ritenuti necessari, qualora utilizzato, il riferimento alle seguenti parti:
\begin{itemize}
	\item \textbf{@extends}: da utilizzare nell'intestazione della classe se la stessa ne estende un'altra;
	\item \textbf{@param}: da utilizzare per descrivere l'uso di un parametro di una funzione;
	\item \textbf{@return}: da utilizzare per descrivere il tipo di valore ritornato da una funzione; 
	\item \textbf{@module}: da utilizzare per indicare un modulo importato;
	\item \textbf{@throws}: da utilizzare per descrivere quali eccezioni possono essere lanciate.
\end{itemize}
Di seguito un esempio di utilizzo:
\begin{lstlisting}[language=JavaScript]
**
* Class representing a dot.
* @extends Point
*/
class Dot extends Point {
  /**
  * Create a dot.
  * @param {number} x - The x value.
  * @param {number} y - The y value.
  * @param {number} width - The width of the dot, in pixels.
  */
  constructor(x, y, width) {
    // ...
  }

  /**
  * Get the dot's width.
  * @return {number} The dot's width, in pixels.
  */
  getWidth() {
    // ...
  }
  
  /**
  * Set the dot's width.
  * @param {number} width - The width of the dot, in pixels.
  * @throws {Error} Will throw an error if the argument is lower than 0.
  */
  setWidth(w) {
    if(w < 0) throw new Error('Negative width');
    this.width = w;
  }
}
\end{lstlisting}

\subparagraph{Variabili}\-\\
\textbf{Norma 1}\\
Fare riferimento alle norme 1 e 2, all'inizio della sezione §\ref{EcmaScript6}.

\-\\
\textbf{Norma 2}\\
Non utilizzare dichiarazioni multiple di variabili, dichiarare una variabile per riga.\\
Nel seguente modo:
\begin{lstlisting}[language=JavaScript]
// OK
var x = 1;
var y = 0;

// NO
var x = 1, y = 0;
\end{lstlisting}


%va linkato il paragrafo
\subparagraph{Nomi}\-\\
\textbf{Norma 1}\\
 Oltre a quanto enunciato nel secondo punto del paragrafo §\ref{Nomi}, tutti i nomi di funzioni o variabili composti da una singola lettera, o che indichino temporaneità della variabile sono vietati: ogni nome deve essere significativo.\\
\-\\
\textbf{Norma 2} 
\begin{enumerate}
	\item I nomi delle variabili, funzioni ed istanze devono utilizzare il CamelCase;
	\item I nomi delle classi deve avere lo stile \texttt{capWords}.
\end{enumerate}
Nel seguente modo:
\begin{lstlisting}[language=JavaScript]
// OK
var thisIsAVariable;

function thisIsAFunction() { ... }

class ThisIsAClass() {
  ...
}

// NO
var Variable;

function Function() { ... }

class myClass() {
  ...
}
\end{lstlisting}


\paragraph{\textit{CSS}}
\label{css} \-\\
Per la parte front-end\glossario del plug-in il gruppo \texttt{Agents of S.W.E.} ha scelto di utilizzare i linguaggi \textit{CSS3}\glossario ed \textit{HTML5}\glossario.\\
Qui di seguito verrà descritta la Style Guide resa disponibile da \textit{Airbnb CSS-in-JavaScript Style Guide}.

\subparagraph{Indentazione} \-\\
\textbf{Norma 1}\\
Tra blocchi adiacenti dello stesso livello deve essere lasciata una linea bianca, tra l'una e l'altra. In questo modo il codice risulterà essere più leggibile. \\
Di seguito, un esempio della corretta indentazione: 


\begin{lstlisting} [language=JavaScript]
//OK
{
  bigBang: {
    display: 'inline-block',
    '::before': {
      content: "''",
    },
  },
  universe: {
    border: 'none',
  },
}

//NO
{
  bigBang: {
    display: 'inline-block',

    '::before': {
      content: "''",
    },
  },

  universe: {
    border: 'none',
  },
}
\end{lstlisting}

Per le restanti regole si farà riferimento allo standard imposto dalla World Wide Web Consortium. Per essere considerato corretto il file \textit{.css} dovrà essere validato tramite il sito reso disponibile dall'organizzazione, precedentemente nominata. La validazione potrà essere effettuata tramite il sistema messo a disposizione dalla W3C.

\paragraph{\textit{HTML}}
\label{html}\-\\

Insieme al \textit{CSS3}, per lo sviluppo della parte front-end del progetto, il gruppo ha scelto di utilizzare \textit{HTML5}. Per la stesura della Style Guide verrà utilizzata la documentazione resa disponibile da W3C.\\
Di seguito saranno riportate le principali norme da seguire: \\
\-\\
\textbf{Norma 1}\\
Il tipo di documento deve sempre essere dichiarato all'inizio del documento in maiuscolo, in questo modo:\\
	\begin{lstlisting}[language=HTML]
<!DOCTYPE html>
	\end{lstlisting}

\textbf{Norma 2}\\
I nomi degli elementi ed attributi devono essere sempre scritti in minuscolo: \\
	\begin{lstlisting}[language=HTML]
<tag attr1="value"/> <!-- OK -->

<TAG ATTR1="value"/> <!-- NO -->
\end{lstlisting}
\-\\
\textbf{Norma 3}\\
Tutti gli elementi \textit{HTML} devono essere chiusi, anche se vuoti: \\
	\begin{lstlisting}[language=HTML]
<tag /> 	<!-- OK -->
<tag></tag> 	<!-- OK -->

<tag> 		<!-- NO -->

\end{lstlisting}
\-\\
\textbf{Norma 4}\\
Il valore degli attributi deve essere racchiuso dentro doppio apice:\\
	\begin{lstlisting}[language=HTML]
<tag attr1="value"/> <!-- OK -->

<tag attr1='value'/> <!-- NO -->
\end{lstlisting}
\-\\
\textbf{Norma 5}\\
Intorno al simbolo di uguaglianza (=) non devono essere presenti spazi:\\
\begin{lstlisting}[language=HTML]
<tag attr1="value"/> <!-- OK -->

<tag attr1 = 'value'/> <!-- NO -->
\end{lstlisting}
\-\\
\textbf{Norma 6}\\
Devono essere usati due (2) spazi e non le tabulazioni. Gli spazi bianchi vanno utilizzati solamente per ampli blocchi di codice, per una maggiore leggibilità; \\
\begin{lstlisting}[language=HTML]
<!-- OK -->
<rootTag>
..<childTag>
....<subChildTag/>
..</childTag>  
</rootTag>

<!-- NO -->
<rootTag>
	<childTag>
		<subChildTag/>
	</childTag>  
</rootTag>
\end{lstlisting}
\-\\
\textbf{Norma 7}\\
Nonostante i tag html, body e head possano essere omessi, vanno inseriti affinché siano compatibili con tutti i browser:\\
\begin{lstlisting}[language=HTML]
<!-- OK -->
<html>
  <head>
    ...
  </head>
  <body>
    ...
  </body>
</html>
\end{lstlisting}
\-\\
\textbf{Norma 8}\\
Le classi che verranno dichiarate all'interno dell'\textit{HTML}, al fine di andare poi a ridefinire il \textit{CSS}, dovranno essere quelle rese disponibili dalla documentazione di \textit{Grafana}.	\\

Equivalentemente al \textit{CSS}, anche il codice \textit{HTML} dovrà essere validato. Questa validazione avverrà tramite il sito reso disponibile dalla W3C, al seguente \href{https://validator.w3.org/}{link}.

%AGGIUNGERE E SISTEMARE DA QUA IN GIU

\paragraph{\textit{AngularJS}} \label{angularjs} \-\\
Insieme al \textit{CSS} e all'\textit{HTML} il team ha deciso di sviluppare la parte front-end appoggiandosi al framework \textit{JavaScript Angularjs}\glossario.\\
\textit{AngularJS} è un framework per applicazioni web, sviluppato con lo scopo di affrontare le difficoltà incontrate con lo sviluppo di applicazioni su singola pagina di tipo client-side.\-\\
La caratteristica principale è l'introduzione di funzionalità chiamate direttive, che permettono di creare componenti \textit{HTML} personalizzati e riusabili con lo scopo di associare a ogni direttiva un comportamento, così nascondendo la complessità della struttura DOM\glossario. Le direttive sono riconoscibili poiché integrate nei tag con la dicitura "ng-" e susseguite dal nome (scelto dallo sviluppatore). 
Di seguito sono riportate alcune norme da seguire:\\
\textbf{Norma 1}\\
Verrà seguito il pattern Model View Controller(MVC)\glossario che propone la scomposizione in tre componenti:
\begin{itemize}
	\item \textbf{Model}: responsabile per il mantenimento e l'elaborazione dei dati;
	\item \textbf{View}: responsabile per la visualizzazione dei dati all'utente;
	\item \textbf{Controller}: si occupa di controllare l'interazione tra Model e View.
\end{itemize} 
\textbf{Norma 2}\\
Il controller non deve essere sovraccaricato di funzionalità che non gli appartengono. In particolare il controller non deve:
\begin{itemize}
	\item manipolare il DOM, questo renderà i controller difficili da testare. Questo compito è riservato alle direttive;
	\item formattare l'input, a questo scopo vengono usati i form \textit{Angular};
	\item formattare l'output, sarà compito dei filtri;
	\item condividere dati con altri controller;
	\item implementare funzionalità generali, le funzionalità che non riguardano direttamente l'interazione tra dati e la visualizzazione, deve essere fatta nei servizi.
\end{itemize}
\textbf{Norma 3}\\
Il nome dei moduli seguono la notazione lowerCamelCase;
\begin{lstlisting} [language=HTML]
	<div ng-app="myModule">...</div>
\end{lstlisting}
\-\\
\textbf{Norma 4}\\
Quando è presente un modulo $b$ che è submodule di $a$, si possono concatenare usando la notazione $a.b$;\\
\-\\
\textbf{Norma 5}\\
Il nome dei controller viene assegnato in base alla loro funzionalità e viene usata la notazione UpperCamelCase;
\begin{lstlisting} [language=HTML]
	<div ng-controller="MyController">
  		{{ greeting }}
	</div>
\end{lstlisting}
\-\\
\textbf{Norma 6}\\
I nomi delle restanti direttive seguono la notazione camelCase;
\begin{lstlisting} [language=HTML]
	<select class="myClass" ng-model="myModel">
\end{lstlisting}
\-\\
\textbf{Norma 7}\\
Le direttive \textit{AngularJS} vengono scritte alla fine degli attributi di un tag: 
\begin{lstlisting} [language=HTML]
	//OK
	<p>Nome: <input type="text" ng-model="utente.nome"></p>
	
	//NO
	<p>Nome: <input ng-model="utente.nome" type="text"></p>
\end{lstlisting}
\-\\
\textbf{Norma 8}\\
Il prefisso \$ è riservato ad \textit{AngularJS}, non deve essere usato per nomi di variabili, proprietà o metodi;\\
\-\\
\textbf{Norma 9}\\
I controller non devono essere definiti globali;\\
\-\\
\textbf{Norma 10}\\
Usare prefissi personalizzati per le direttive per evitare collisioni con librerie di terze parti;\\
\-\\
\textbf{Norma 11}\\
In \textit{AngularJS} viene adottato il two-way binding, ogni modifica al modello dati si riflette sulla view e ogni modifica alla view viene riportata sul modello dati.
\-\\
Questa sincronizzazione avviene senza la necessità di scrivere codice particolare, è sufficiente associare il modello allo scope all'interno del controller ed utilizzare la direttiva ng-model nella view.

\paragraph{\textit{NodeJS}} \label{nodeJS} \-\\
\textit{NodeJS} è una runtime di \textit{JavaScript} Open source multipiattaforma orientato agli eventi per l'esecuzione di codice \textit{JavaScript} Server-side, costruita sul motore \textit{JavaScript V8} di \textit{Google Chrome}. Molti dei suoi moduli base sono scritti in \textit{JavaScript}, e gli sviluppatori possono scrivere nuovi moduli in \textit{JavaScript}.\\
In particolare, tale tecnologia viene utilizzata dal team \texttt{Agents of S.W.E.} per l'implementazione del server che permette il funzionamento del \textit{plug-in}. \\
Le norme di codifica da seguire per \textit{NodeJS} sono le stesse di \textit{ECMAScript 6} con ulteriori norme specifiche riportate di seguito:

\textbf{Norma 1} \\
Iniziare un nuovo progetto usando il comando \texttt{NPM init} come riportato di seguito:
\begin{lstlisting}[language=JavaScript]
	$ mkdir my-project
	$ cd my-project
	$ npm init
\end{lstlisting}

\-\\

\textbf{Norma 2} \\
Il codice deve essere partizionato in componenti ognuno nella propria cartella, garantendo così che ogni unità sia piccola e semplice. 
\-\\

\textbf{Norma 3} \\
Per i moduli più complessi e composti da più file, è possibile includere l’intero set di file con un unica chiamata alla funzione \textit{require}\glossario. In questo caso è però necessario definire un file descrittore all’interno del modulo che permette al motore V8\glossario di conoscere quali sono i file da includere.
\-\\

\textbf{Norma 4} \\
Richiedere i moduli all'inizio di ogni file, evitando di richiederli all'interno di funzioni, permettendo così di individuare facilmente le dipendenze all'inizio del file.
\-\\

\textbf{Norma 5} \\
Usare \textit{async-await}\glossario o \textit{promise}\glossario per la gestione di errori asincroni, che consente di avere un codice più compatto e familiare.
\-\\

\textbf{Norma 6} \\
Evitare l'uso di funzioni anonime, nominare ogni funzione permetterà una più facile comprensione in fase di lettura.
\-\\

  




