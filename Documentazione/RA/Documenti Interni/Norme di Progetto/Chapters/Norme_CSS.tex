\label{css} \-\\
Per la parte front-end\glossario del plug-in il gruppo \texttt{Agents of S.W.E.} ha scelto di utilizzare i linguaggi \textit{CSS3}\glossario ed \textit{HTML5}\glossario.\\
Qui di seguito verrà descritta la Style Guide resa disponibile da \textit{Airbnb CSS-in-JavaScript Style Guide}.

\subparagraph{Indentazione} \-\\
\textbf{Norma 1}\\
Tra blocchi adiacenti dello stesso livello deve essere lasciata una linea bianca, tra l'una e l'altra. In questo modo il codice risulterà essere più leggibile. \\
Di seguito, un esempio della corretta indentazione: 


\begin{lstlisting} [language=JavaScript]
//OK
{
  bigBang: {
    display: 'inline-block',
    '::before': {
      content: "''",
    },
  },
  universe: {
    border: 'none',
  },
}

//NO
{
  bigBang: {
    display: 'inline-block',

    '::before': {
      content: "''",
    },
  },

  universe: {
    border: 'none',
  },
}
\end{lstlisting}

Per le restanti regole si farà riferimento allo standard imposto dalla World Wide Web Consortium. Per essere considerato corretto il file \textit{.css} dovrà essere validato tramite il sito reso disponibile dall'organizzazione, precedentemente nominata. La validazione potrà essere effettuata tramite il sistema messo a disposizione dalla W3C.