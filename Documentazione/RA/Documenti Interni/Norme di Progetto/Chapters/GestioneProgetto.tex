La gestione di progetto avviene tramite un sistema di task integrato nei servizii di hosting \textit{GitHub e GitLab}. 
Esso permette l'integrazione delle task con il repository stesso, dando la possibilità ai vari commit di chiudere con comandi appositi determinate task, aumentando cosi l'automazione di tutto il processo. 

\subsubsection{Configurazione Strumenti di Organizzazione}
	La configurazione di tutto il processo di organizzazione avviene nel portale di \textit{GitHub}, dove si crea una project board per ogni categoria di processo. 	

\paragraph{Inizializzazione} \-\\
 L'inizializzazione della project board avviene tramite un'istanza vuota oppure selezionando un template di ciclo di vita fornito da \textit{GitHub} largamente utilizzate in molti progetti, quindi testati ed affidabili. Tra i template forniti abbiamo: 

\begin{itemize}
		\item \textbf{Basic Kanban}: presenta le fasi di ToDo, In Progress, e Done; 
		\item \textbf{Automated Kanban}: presenta trigger\glossario predefiniti che permettono lo spostamento di task
		 automatici nei vari cicli di vita, utilizzando il meccanismo di chiusura dei commit;
		\item \textbf{Automated Kanban with Reviews}: tutto cioè che viene incluso nel template \textit{Automated Kanban} con l'aggiunta di trigger per la revisione di nuove componenti; 
		\item \textbf{Bug Triage}: template centrato sulla gestione degli errori, fornendo un ciclo di vita per essi che varia tra ToDo, Alta Priorità, Bassa Priorità e Chiusi. 
\end{itemize}
	  	
\paragraph{Aggiunta di Milestone} \-\\
	Le milestone\glossario sono gruppi di task mirate ad un obbiettivo comune tra esse.
	Possono essere aggiunte in qualsiasi momento, sia prima che dopo la creazione di una task.

\subsubsection{Ciclo di Vita delle Task}

\paragraph{Apertura} \-\\
	Da una specifica project board si possono creare le task o le issue\glossario, le quali possono essere assegnate ad uno o più individui che collaborano al repository, inoltre ogni task può far parte di una milestone, che raggruppa un insieme di task o issue per il raggiungimento di un obbiettivo comune. \\
	Ad ognuna di esse può essere assegnato un colore che ne identifica il tipo come per esempio: bug, ToDo, miglioramenti, ecc.,\\
	Si può creare una nuova task senza l'obbligo di assegnarla ad una project board, mantenendo comunque tutte le funzionalità descritte precedentemente. 

\paragraph{Completamento} \-\\\label{ProcessiOrganizzativi_GestioneProgetto_CicloTask_Completamento}
	Il completamento di una task avviene in diversi modi, a seconda delle impostazioni della project board. 
	Se la project board è automatizzata, il completamento di una task può avvenire tramite commit utilizzando il codice di chiusura. \\
	Questo metodo collega direttamente l'implementazione richiesta alla task. \\
	Se la project board non è automatizzata, il completamento dalla task deve essere manuale spostandola nello stato di "Concluso". 
	
\paragraph{Richiesta di Revisione} \-\\
	Accumulate un certo numero di task o di milestone, si avvia la procedura di revisione da parte del \textit{Verificatore}. \\ Questa può essere notificata e pianificata in modo automatico a seconda del livello di automatizzazione della project board, oppure può essere totalmente gestita dal \textit{Verificatore}. 
	
	
	\paragraph{Chiusura} \-\\
	Una volta che le task o le milestone sono state approvate dal \textit{Verificatore}, esse concludono il loro ciclo di vita nello stato di chiusura, le quali verranno spostate manualmente dal \textit{Verificatore} o automaticamente dalla project board se il merge è avvenuto con successo. 
	
\paragraph{Riapertura} \-\\ \label{ProcessiOrganizzativi_Riapertura}
	Le task in stato di "Chiusura" possono essere riaperte e spostate nello stato di "Apertura" se esse non soddisfanno tutti i parametri di qualità richiesti.