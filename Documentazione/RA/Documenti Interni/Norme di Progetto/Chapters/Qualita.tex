\subsection{Qualità} \label{qualita}

\subsubsection{Introduzione}
Il contenuto di questa sezione descrive le metriche e i criteri di qualità di processo e prodotto che vengono utilizzati nel documento \textit{Piano di Qualifica v3.0.0}.

\subsubsection{Classificazione Processi}
Per garantire una corretta struttura, il gruppo ha deciso di introdurre una nomenclatura per identificare i processi descritti nel documento. I processi saranno identificati così:
\begin{center}
	\textbf{PR[num]}
\end{center}
dove:
\begin{itemize}
	\item \textbf{num}: rappresenta il numero identificativo del processo formato da due cifre intere a partire da 1, unico per tutto il documento.
\end{itemize}

\subsubsection{Classificazione Metriche}
Per garantire una corretta struttura, il gruppo ha deciso di introdurre una nomenclatura per identificare le metriche utilizzate. Si potranno quindi identificare cosi:
\begin{center}
	\textbf{MT[mcat][cat][num]}
\end{center}
\begin{itemize}
		\item \textbf{mcat}: identifica le metriche in base a quale macrocategoria andranno a misurare. Può assumere i seguenti valori:
	\begin{itemize}
		\item \textbf{PC}: per indicare le metriche di processo;
		\item \textbf{PD}: per indicare le metriche di prodotto;
		\item \textbf{TS}: per indicare le metriche di test.
	\end{itemize}
	\item \textbf{cat}: identifica la categoria di appartenenza, se esiste, altrimenti è vuota. Per le metriche di prodotto può assumere i seguenti valori:
	\begin{itemize}
		\item \textbf{D}: per indicare i documenti;
		\item \textbf{S}: per indicare il software.
	\end{itemize}

	\item \textbf{num}: identificativo univoco formato da due cifre intere a partire da 1.
\end{itemize}


\subsubsection{Controllo Qualità di Processo e Metriche} \label{ControlloQualita_Processo}
La qualità di processo è raggiunta tramite l'utilizzo di metodi e modelli che garantiscono il corretto procedimento delle fasi di sviluppo del processo. Il team sfrutta le potenzialità del metodo PDCA, descritto nell'appendice \ref{PDCASection}, ottenendo miglioramenti continui nelle qualità di processo e verifica in modo da avere una maggiore qualità nel prodotto risultante. Inoltre verrà utilizzato lo standard ISO/IEC 15504, comunemente conosciuto con l'acronimo SPICE, che misurerà il livello di maturità dei processi.\-\\
Le metriche utilizzate per la valutazione dell'efficacia e dell'efficienza degli stessi sono presentate nell'appendice \ref{AMe}.
