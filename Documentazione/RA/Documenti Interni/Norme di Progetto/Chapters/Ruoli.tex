\paragraph{\textit{Responsabile di Progetto}} ~\\
	Detto anche \textit{"Project Manager"}, è il rappresentate del progetto\glossario, agli occhi sia del committente che del 			fornitore. Egli risulta dunque essere, in primo luogo, il responsabile ultimo dei risultati del proprio gruppo. 			Figura di grande responsabilità, partecipa al progetto per tutta la sua durata, ha il compito di prendere 						decisioni 	e approvare scelte collettive.\\
	Nello specifico egli ha la responsabilità di:
	\begin{itemize}
	\item Coordinare le attività del gruppo, attraverso la gestione delle risorse umane;
	\item Approvare i documenti redatti, e verificati, dai membri del gruppo;
	\item Elaborare piani e scadenze, monitorando i progressi nell'avanzamento del progetto;
	\item Redigere l'organigramma del gruppo e il \textit{Piano di Progetto v4.0.0}.
	\end{itemize}

\paragraph{\textit{Amministratore di Progetto}} ~\\
	L'\textit{Amministratore} è la figura chiave per quanto concerne la produttività. Egli ha infatti come primaria 							responsabilità il garantire l'efficienza\glossario del gruppo, fornendo strumenti utili e occupandosi 								dell'operatività delle risorse. Ha dunque il compito di gestire l'ambiente lavorativo.\\
	Tra le sue responsbilità specifiche figurano:
	\begin{itemize}
	\item Redigere documenti che normano l'attività lavorativa, e la loro verifica;
	\item Redigere le \textit{Norme di Progetto v4.0.0};
	\item Scegliere ed amministrare gli strumenti di versionamento\glossario;
	\item Ricercare strumenti che possano agevolare il lavoro del gruppo;
	\item Attuare piani e procedure di gestione della qualità\glossario.
	\end{itemize}

\paragraph{\textit{Analista}} ~\\
	L'\textit{Analista} deve essere dotato di un'ottima conoscenza riguardo al dominio del problema. Egli ha infatti il 					compito di analizzare tale dominio e comprenderlo appieno, affinchè possa avvenire una corretta 											progettazione\glossario.\\
	Ha il compito di:
	\begin{itemize}
	\item Comprendere al meglio il problema, per poi poterlo esporre in modo chiaro attraverso specifici 									requisiti\glossario;
	\item Redarre lo \textit{Studio di Fattibilità v1.0.0} e l'\textit{Analisi dei Requisiti v4.0.0}.
	\end{itemize}

\paragraph{\textit{Progettista}} ~\\
	Il \textit{Progettista} è responsabile delle attività di progettazione attraverso la gestione degli aspetti tecnici 	del progetto.\\
	Più nello specifico si occupa di:
	\begin{itemize}
	\item Definire l'Architettura\glossario del prodotto\glossario, applicando quanto più possibile norme di 							best practice\glossario e prestando attenzione alla manutenibilità del prodotto;
	\item Suddividere il problema, e di conseguenza il sistema, in parti di complessità trattabile.
	\end{itemize}

\paragraph{\textit{Programmatore}} ~\\
	Il \textit{Programmatore} si occupa delle attività di codifica, le quali portano alla realizzazione effettiva del 						prodotto.\\
	Egli ha dunque il compito di:
	\begin{itemize}
	\item Implementare l'architettura definita dal \textit{Progettista}, prestando attenzione a scrivere codice 						coerente con ciò che è stato stabilito nelle \textit{Norme di Progetto v4.0.0};
	\item Produrre codice documentato e manutenibile;
	\item Realizzare le componenti necessarie per la verifica e la validazione del codice;
	\item Redarre il \textit{Manuale Utente v2.0.0}.
	\end{itemize}

\paragraph{\textit{Verificatore}} ~\\
	Il \textit{Verificatore}, figura presente per l'intera durata del progetto, è responsabile delle attività di 				verifica.\\
	Nello specifico egli:
	\begin{itemize}
	\item Verifica l'applicazione ed il rispetto delle \textit{Norme di Progetto v4.0.0};
	\item Segnala al \textit{Responsabile di Progetto} l'emergere di eventuali discordanze tra quanto presentato nel 			\textit{Piano di Progetto v4.0.0} e quanto effettivamente realizzato;
	\item Ha il compito di redarre il \textit{Piano di Qualifica v4.0.0}.
	\end{itemize}

\paragraph{Rotazione dei Ruoli} ~\\
	Come da istruzioni ogni membro del gruppo dovrà ricoprire, per un periodo di tempo limitato, ciascun ruolo, nel 			rispetto delle seguenti regole:
	\begin{itemize}
	\item Ciascun membro dovrà svolgere esclusivamente le attività proprie del ruolo a lui assegnato;
	\item Al fine di evitare conflitti di interesse nessun membro potrà ricoprire un ruolo che preveda la 									verifica di quanto da lui svolto, nell'immediato passato;
	\item Vista l'ampia differenza di compiti e mansioni tra i vari ruoli, e al fine di valorizzare l'attività 						collaborativa all'interno del gruppo, ogni componente che abbia ricoperto in precedenza un ruolo ora destinato 			a qualcun altro dovrà fornire supporto al compagno in caso di necessità, fornendogli consigli e, se possibile, 			affiancandolo in situazioni critiche.
	\end{itemize}
