\section{Resoconto}


\subsection{Punto 1}
Il gruppo ha presentato alla proponente il prodotto realizzato, al fine di illustrarne le caratteristiche mettendo in luce le scelte che sono state fatte e le motivazioni dietro di esse.\\
La dimostrazione è stata presentata facendo uso del server remoto affittato dal gruppo al fine di simulare un ambiente di utilizzo del prodotto il più possibile verosimile. Su tale server è stata installata la piattaforma \textit{Grafana} e il plug-in realizzato, ovvero il pannello \textit{G\&B}.\\
La dimostrazione del prodotto è dunque stata effettuata presentando tutte le funzionalità dello stesso, in riferimento ai vari requisiti obbligatori ed opzionali, pattuiti. Nello specifico sono state presentate in prima istanza le funzionalità base del prodotto, attenenti per la quasi totalità a requisiti obbligatori, ovvero: 
\begin{itemize}
 \item La connessione ad un server locale per il monitoraggio costante dei dati;
 \item Il caricamento di una rete bayesiana da un file di definizione \textit{JSON};
 \item La definizione di una politica temporale per il ricalcolo delle probabilità;
 \item La definizione di tutte le necessarie impostazioni per il collegamento dei nodi della rete ad un flusso dati;
 \item L'avvio di monitoraggio della rete collegata;
 \item La visualizzazione dei dati di monitoraggio aggiornati in tempo reale.
\end{itemize}
~\\
Dopo aver mostrato le funzionalità basilari il gruppo ha proseguito con la dimostrazione illustrando funzionalità relative a requisiti non obbligatori, sviluppate per aumentare la qualità ed il valore del prodotto. Queste sono:
\begin{itemize}
 \item La possibilità di ricaricare una rete memorizzata in precedenza all'interno del server locale;
 \item La possibilità, discussa già in precedenza in data 2019-02-08, di poter definire soglie "Critiche" in sede di collegamento dei nodi;
 \item La possibilità di interrompere monitoraggi attivi ed eliminare reti memorizzate nel server;
 \item La possibilità di mantenere multipli monitoraggi di reti diverse attivi contemporaneamente, selezionando di volta in volta la rete di cui si desidera visualizzare i dati;
 \item La visualizzazione dei dati di monitoraggio sotto forma grafica relativa alla rete monitorata, anziché tabellare;
 \item La possibilità di definire grafici ed alert, anche con il pannello \textit{G\&B} chiuso o assente, su dati di monitoraggio attivi utilizzando le funzionalità proprie della piattaforma \textit{Grafana}.
\end{itemize}

\subsection{Punto 2}
La proponente, al termine della presentazione del prodotto, non ha espresso criticità di alcun genere. Inoltre, valutando come tutti i requisiti principali fossero stati soddisfatti appieno, non è stata avanzata alcuna richiesta specifica in merito al plug-in realizzato.

\subsection{Punto 3}
Il gruppo aveva come massimo interesse quello di valutare il grado di soddisfazione della proponente in merito al prodotto realizzato, per accertarsi di aver eseguito un lavoro in linea con le aspettative.\\
In tal senso la proponente si è dimostrata ampiamente soddisfatta del lavoro realizzato, elogiando in particolar modo alcune funzionalità, come: la gestione degli alert anche a pannello chiuso, la possibilità di definire soglie critiche e la visualizzazione dei dati di monitoraggio sotto forma grafica di rete bayesiana.\\
La proponente è rimasta colpita soprattutto da quest'ultima caratteristica, sottolineando come fosse una funzionalità extra, non considerata inizialmente tra i requisiti, ma la cui implementazione aumenta notevolmente il valore del prodotto realizzato.

