\section*{J}
\addcontentsline{toc}{section}{J}

\subsection{\index{Java}Java}
Linguaggio di programmazione ad alto livello orientato agli oggetti. 

\subsection{\index{JavaScript}JavaScript}
È un linguaggio di scripting orientato agli oggetti, utilizzato nella programmazione web lato client per la creazione di effetti dinamici interattivi, tramite funzioni di script, invocate da eventi innescati dall'utente sulla pagina web.

\subsection{\index{Jest}Jest} 
È un framework di testing di JavaScript. È compatibile con Babel, TypeScript, Node, React, Angular, Vue. Offre la possibilità di eseguire test in parallelo in maniera affidabile, ed eseguire il code coverage di interi progetti.

\subsection{\index{JFrog Artifactory}JFrog Artifactory} 
È un servzio web per la gestione universale di repository.

\subsection{\index{jQuery}jQuery}
Libreria javascript opensource che ha come obiettivo la semplificazione della gestione degli eventi di elementi del DOM in pagine HTML. 

\subsection{\index{Jsbayes}JsBayes}
È una libreria open source per la gestione dei calcoli della rete Bayesiana sviluppata in \textit{JavaScript}.

\subsection{\index{JsbayesViz}JsBayesViz}
È una libreria open source per la visualizzazione delle probabilità delle reti calcolate con \textit{JsBayes}.

\subsection{\index{JSON}JSON}
È un formato utilizzato per il salvataggio e lo scambio di dati in applicazioni client/server.

\subsection{\index{JUnit}JUnit}
È un framework per il linguaggio Java che permette test di unità.