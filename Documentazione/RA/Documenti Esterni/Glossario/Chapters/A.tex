\section*{A}
\addcontentsline{toc}{section}{A}

\subsection{\index{Amazon Alexa}Amazon Alexa}
Sviluppato da Amazon, Amazon Alexa è un assistente personale intelligente che, interpretando il linguaggio naturale, è in grado di rispondere ai comandi vocali e comunicare con l'utente. 

\subsection{\index{Analisi dei Requisiti}Analisi dei Requisiti}
È un'attività preliminare allo sviluppo di un sistema software, prevede la stesura di una dettagliata specifica dei requisiti che descrivono le funzionalità del software.

\subsection{\index{Android}Android}
È un sistema operativo per dispositivi mobili, come smartphone e tablet, sviluppato da Google e basato su kernel Linux.

\subsection{\index{Apache}Apache}
Piattaforma server web modulare più diffusa che fornisce le funzioni di trasporto delle informazioni, di internetwork e di collegamento, offrendo funzioni di controllo per la sicurezza.

\subsection{\index{Apache Kafka}Apache Kafka}
Piattaforma di streaming distribuito.

\subsection{\index{API Gateway}API Gateway}
È un servizio di AWS per per la creazione, monitoraggio e manutenzione delle API.

\subsection{\index{Apprendimento Supervisionato}Apprendimento Supervisionato}
È una tecnica di apprendimento automatico che mira a istruire un sistema informatico in modo da consentirgli di risolvere dei compiti in maniera autonoma sulla base di una serie di esempi ideali, costituiti da coppie di input e di output desiderati, che gli vengono inizialmente forniti.

\subsection{\index{Architettura}Architettura}
Decomposizione organizzata di un sistema in componenti, nonché l'organizzazione di tali componenti, ovvero la definizione di ruoli, responsabilità e interfacce necessarie all'interazione tra loro stessi.

\subsection{\index{Async}Async}
È una parola chiave introdotta in \textit{ECMAScript} che consente di dichiarare una funziona asincrona.

\subsection{\index{Await}Await}
È una parola chiave introdotta in \textit{ECMAScript} che sospende l'esecuzione di una funzione in attesa che la \textit{promise} associata ad un'attività asincrona, venga risolta o rigettata.

\subsection{\index{AWS}AWS}
Amazon Web Services è una raccolta di servizi di cloud computing sviluppata da Amazon.