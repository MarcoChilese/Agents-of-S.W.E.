\section*{E}
\addcontentsline{toc}{section}{E}

\subsection{\index{ECMA}ECMA}
Acronimo di European Computer Manufacturers Association, è un'associazione che si occupa della standardizzazione nel settore informatico e dei sistemi di comunicazione. È responsabile di molti standard come JSON, ECMAScript.

\subsection{\index{ECMAScript}ECMAScript}
È un linguaggio di programmazione standardizzato e mantenuto da Ecma International nell'ECMA-262 ed ISO/IEC 16262. Le implementazioni più conosciute di questo linguaggio sono JavaScript, JScript e ActionScript che sono utilizzati per lo sviluppo web lato client.

\subsection{\index{ECMAScript 6}ECMAScript 6}
Sesta edizione di ECMAScript definita nel 2015, implementa significativi cambiamenti sintattici per scrivere applicazioni più complesse. Tuttavia il supporto per ES6, da parte dei browser, è ancora incompleto.

\subsection{\index{Efficacia}Efficacia}
Capacità di raggiungere l'obiettivo prefissato, soddisfacendo tutti i suoi requisiti, impliciti ed espliciti.

\subsection{\index{Efficienza}Efficienza}
Misura della capacità di raggiungere l'obiettivo prefissato impiegando le risorse minime indispensabili.

\subsection{\index{ESLint}ESLint}
È un utility open source per l’analisi statica del codice e per l’identificazione di pattern in JavaScript,  questo permette di trovare più facilmente codice che non rispetta determinate linee guida.

\subsection{\index{Ethereum}Ethereum}
È una piattaforma open source di calcolo distribuito per la creazione di contratti intelligenti.

\subsection{\index{Etherescan}Etherescan}
Piattaforma per Ethereum per la ricerca di smart contracts.