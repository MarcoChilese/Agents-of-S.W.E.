\section{Impostare l'Ambiente di Lavoro}\label{AmbienteLavoro}
\subsection{Scopo del Capitolo}\label{AmbienteLavoro_scopo}
In questa sezione viene riportata una guida per la corretta configurazione dell'ambiente di sviluppo, in modo che sia la stessa utilizzata dal gruppo \texttt{Agents of S.W.E.}.\\
Per contribuire al progetto non è strettamente necessario seguire questa sezione, tuttavia è consigliato al fine di ottenere un ambiente di lavoro pronto e correttamente impostato per lo sviluppo.

\section{Requisiti}\label{AmbienteLavoro_requisiti}
Per i requisiti minimi di funzionamento dell'ambiente di lavoro consigliato, si rimanda al sito del produttore di ogni singola tecnologia.

\subsection{\textit{WebStorm}}\label{webstorm}
\textit{WebStorm} è l'ambiente di sviluppo integrato (IDE) di \textit{JetBrains} utilizzato dal team per lo sviluppo del progetto. Esso può essere ottenuto mediante download dal sito ufficiale nella formula di prova gratuita se non si dispone di licenza, che può essere ottenuta attraverso l'indirizzo email universitario.\\
Tale software è disponibile per i sistemi operativi Microsoft Windows, Linux e Apple MacOS.\\
Per ulteriori informazioni si rimanda al sito ufficiale.

\subsection{\textit{NPM}}\label{npm}
\textit{NPM} è il gestore di pachetti utilizzato per gestire il progetto dal team \texttt{Agents of S.W.E.}. Per la relativa installazione si rimanda al sito del produttore.\\
Per l'effettivo utilizzo è necessario avere all'interno della directory del progetto un file denominato "package.json". Esso permette di esprimere le dipendenze e le caratteristiche del prodotto, oltre che alla definizione dei vari comandi da eseguire con radice "\texttt{npm run ...}".\\
Di seguito è riportato, nei punti salienti, il file "package.json" utilizzato durante lo sviluppo del progetto:\\
\begin{lstlisting}[language=JavaScript]
{
  "name": "GrafanaAndBayes",
  "version": "2.0.0",
  "description": "Plug-in per Grafana",
  "main": "src/module.js",
  "scripts": {
    "test": "jest",
    "build": "webpack --config build/webpack.prod.conf.js",
    "dev": "webpack --mode development",
    "eslint": "eslint ./src",
     "codecov": "codecov"
  },
  "jest": {
    "verbose": true,
    "collectCoverage": true,
    "coverageDirectory": "./coverage/"
  },
  "author": "Agents Of S.W.E.",
  "license": "ISC",
  "devDependencies": {
    ...
  },
  "dependencies": {
    ...
  }
}
\end{lstlisting}

\subsection{\textit{ESLint}}\label{eslint}
\textit{ESLint} verrà installato automaticamente attraverso il comando \texttt{npm install}.\\
Per abilitarlo all'interno di \textit{WebStorm} è necessario lanciare l'IDE e recarsi in: File > Settings > \textit{ESLint} e scegliere "Enable". All'interno del campo "Node Interpreter" è necessario inserire il percorso alla directory in cui si trovano i file eseguibili di Node.\\
Se non rilevato in automatico, specificare la posizione del file di configurazione ".eslintrc" all'interno della directory del progetto.

\subsection{\textit{Jest}}\label{jest}
\textit{Jest} è il framework di test utilizzato per testare il codice \textit{JavaScript}. È inoltre il responsabile della generazione dei report di coverage utilizzati in seguito per il calcolo del code coverage.\\
L'installazione di tale componente consta nell'aggiunta del relativo pacchetto \textit{npm}. In particolare si dovranno eseguire i seguenti comandi:
\begin{center}
\texttt{npm install --save-dev jest}.
\end{center}
A ciò seguirà poi l'aggiunta del relativo comando di script all'interno di "package.json" in modo tale da consentire l'esecuzione dei test all'invocazione del comando \texttt{npm run test}. La modifica da apportare al file sopracitato, e sopra riportato, è visibile alla riga 7:
\begin{center}
	\texttt{"test": "jest",}.
\end{center}

\subsection{\textit{Webpack}}\label{webpack}
\textit{Webpack} è il model bundler utilizzato per eseguire la build del progetto.\\
L'installazione di tale componente avviene tramite \textit{NPM}. È necessario eseguire il seguente comando:
\begin{center}
	\texttt{npm install --save-dev webpack}.
\end{center}
In seguito è necessario aggiungere una riga al file "package.json", come riportato in riga 8:
\begin{center}
	\texttt{"build": "webpack --config build/webpack.prod.conf.js",}.
\end{center}


