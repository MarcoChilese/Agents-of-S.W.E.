\section{Introduzione}\label{Intro}
\subsection{Scopo del Documento}
Lo scopo del documento è di fornire tutte le informazioni necessarie agli sviluppatori che intenderanno estendere o migliorare il plug-in \textit{G\&B}.\\
Verranno fornite informazioni relative anche ad un possibile ambiente di sviluppo il più completo possibile da cui sarà possibile partire. In particolare sarà illustrato l'ambiente di sviluppo utilizzato dal team \texttt{Agents of S.W.E.} per lo sviluppo del prodotto in oggetto.\\
La seguente guida può essere utilizzata indifferentemente sia da utenti Microsoft Windows, Linux e Apple MacOs.


\subsection{Scopo del Prodotto}
Lo scopo del prodotto è la creazione di un plug-in per la piattaforma open source\glossario di visualizzazione e gestione dati, denominata \textit{Grafana}\glossario , con l’obiettivo di creare un sistema di alert dinamico per monitorare la "liveliness"\glossario del sistema a supporto dei processi DevOps\glossario e per consigliare interventi nel sistema di produzione del software.
In particolare, il plug-in\glossario utilizzerà dati in input forniti ad intervalli regolari o con continuità, ad una rete bayesiana\glossario per stimare la probabilità di alcuni eventi, segnalandone quindi il rischio in modo dinamico, prevenendo situazioni di stallo.


\subsection{Riferimenti}\label{Riferimenti}
\subsubsection{Referimenti per l'Installazione}
\begin{itemize}
	\item \url{https://nodejs.org/it/};
	\item \url{https://www.npmjs.com/};
	\item \url{https://grafana.com/docs/installation/}.
\end{itemize}

\subsubsection{Referimenti Legali}
\begin{itemize}
	\item \url{https://grafana.com/docs/contribute/cla/}.
\end{itemize}

\subsubsection{Referimenti Informativi}
\begin{itemize}
	\item \url{https://grafana.com/};
	\item \url{https://www.influxdata.com/time-series-platform/telegraf/};
	\item \url{https://www.influxdata.com/time-series-platform/influxdb/};
	\item \url{https://jestjs.io/};
	\item \url{https://www.jetbrains.com/webstorm/};
	\item \url{https://webpack.js.org/}.
	\item \url{https://angularjs.org/};
	\item \url{http://codecov.io};
	\item \url{https://jquery.com/};
	\item \url{http://pm2.keymetrics.io/};
\end{itemize}