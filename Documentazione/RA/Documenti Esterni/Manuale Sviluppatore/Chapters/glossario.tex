\section{Glossario}\label{glox}

\textbf{AngularJS}\-\\
AngularJS è un framework per applicazioni web, sviluppato con lo scopo di affrontare le difficoltà incontrate con lo sviluppo di applicazioni su singola pagina di tipo client-side.
\-\\

\textbf{API}\-\\
Application Programming Interface (API) si indica un insieme di procedure atte all'espletamento di un dato compito.
\-\\

\textbf{CSS}\-\\
Acronimo di Cascading Style Sheets (fogli di stile a cascata), è un linguaggio usato per definire la formattazione di documenti HTML, XHTML e XML ad esempio i siti web e relative pagine web. Le regole per comporre il CSS sono contenute in un insieme di direttive emanate a partire dal 1996 dal W3C. 
\-\\

\textbf{DevOps}\-\\
Acronimo di Development e Operations, DevOps è un approccio allo sviluppo e all'implementazione di applicazioni in azienda, che enfatizza la collaborazione tra il team di sviluppo vero e proprio e quello delle operations, ossia che gestirà le applicazioni dopo il loro rilascio.
\-\\

\textbf{EcmaScript 6}\-\\
Sesta edizione di ECMAScript definita nel 2015, implementa significativi cambiamenti sintattici per scrivere applicazioni più complesse. Tuttavia il supporto per ES6, da parte dei browser, è ancora incompleto.
\-\\

\textbf{Express}\-\\
Framework per applicazioni web NodeJS flessibile e leggero che fornisce una serie di funzioni 
avanzate per le applicazioni web e per dispositivi mobili. 
\-\\


\textbf{Grafana}\-\\
Piattaforma open source che permette il monitoraggio e l'analisi di dati che vengono visualizzati in dashboard operative.
\-\\

\textbf{HTML}\-\\
Acronimo di HyperText Markup Language, è un linguaggio di markup nato per la formattazione e impaginazione di documenti ipertestuali disponibili nel web
\-\\

\textbf{Jest}\-\\
È un framework di testing di JavaScript. È compatibile con Babel, TypeScript, Node, React, Angular, Vue. Offre la possibilità di eseguire test in parallelo in maniera affidabile, ed eseguire il code coverage di interi progetti.
\-\\

\textbf{Liveliness}\-\\
Dall'Inglese: "vivacità". Tutta via si vuole intendere lo stato di vita di un sistema.
\-\\

\textbf{MVC}\-\\
È un pattern architetturale molto diffuso nella programmazione orientata agli oggetti in grado di separare la logica di presentazione dalla logica di business. È composto da
tre componenti: il "model" che fornisce i metodi per accedere ai dati all'applicazione, la "View" visualizza i dati contenuti nel model e si occupa dell'iterazione con gli
utenti, e infine il "Controller" che riceve i comandi dall'utente attraverso la view e va a modificare lo stato degli altri due componenti (model e view).
\-\\

\textbf{NodeJS}\-\\
Piattaforma open source per scrivere applicazione in JavaScript Server-side. 
\-\\

\textbf{Open source}\-\\
Termine utilizzato per indicare programmi software non protetti da copyright. Essenziale per favorire il libero studio e permettere ai programmatori indipendenti di apportarvi modifiche.
\-\\

\textbf{Plug-in}\-\\
Componente aggiuntivo che interagisce con un altro programma per ampliarne le funzioni.
\-\\

\textbf{Promise}
Rappresenta un proxy per un valore non necessariamente noto quando la promise è stata creata. 
Consente di associare degli handlers con il successo o il fallimento di un'azione asincrona (e il "valore" in caso di successo). Questo in pratica consente di utilizzare dei metodi asincroni di fatto 
come se fossero sincroni. 
\-\\

\textbf{REST}\-\\
Representational State Transfer (REST) è uno stile architetturale (di architettura software) per i sistemi distribuiti e rappresenta un sistema di trasmissione di dati su HTTP senza ulteriori livelli.
\-\\

\textbf{Rete bayesiana}\-\\
Rappresentazione grafica delle relazioni di dipendenza tra le variabili di un sistema. In statistica la rete bayesiana è utilizzata per individuare più agevolmente le relazioni di dipendenza assoluta e condizionale tra le variabili, al fine di ridurre il numero delle combinazioni delle variabili da analizzare.
\-\\

\textbf{Route}\-\\
Per route si intende determinare come un'applicazione risponde a una richiesta client a un 
endpoint particolare, il quale è un URI (o percorso) e un metodo di richiesta HTTP specifico (GET, POST, e cosi via).
\-\\

\textbf{wrapper}\-\\
Metodo che avvolge un altro metodo per adattarlo alle esigenze del di una determinata classe.
\-\\