\subsection{Preventivo a Finire}\label{caption_paf}

Nella seguente sezione inseriremo il preventivo a finire. Qualora non fosse presente il consuntivo di fine periodo utilizzeremo il preventivo. \\

\begin{longtable}{| C{.30\textwidth}| C{.15\textwidth}| C{.20\textwidth}|}
\hline
\rowcolor{bluelogo}\textbf{\textcolor{white}{Periodo}} & \textbf{\textcolor{white}{Preventivo in \euro}} & \textbf{\textcolor{white}{Consuntivo in \euro}} \\
\hline
Avvio ed Analisi dei Requisiti & \EUR{3870.00} & \EUR{4005.00} \\
\hline
\rowcolor{grigio}Risanamento Criticità & \EUR{775.00}  & \EUR{ 515.00} \\
\hline
Progettazione Architetturale & \EUR{3127.00} & \EUR{3183.00} \\
\hline
\rowcolor{grigio} Risanamento Criticità & \EUR{426.00} & \EUR{426.00} \\
\hline
Progettazione di Dettaglio e Codifica & \EUR{6669.00} & \EUR{6464.00} \\
\hline
\rowcolor{grigio} Risanamento Criticità &  \EUR{426.00} & \EUR{526.00} \\
\hline
Validazione e Collaudo & \EUR{2620.00}  & \EUR{2448.00} \\
\hline
\rowcolor{grigio}\textbf{Totale} & \EUR{14043.00}  & \textbf{\EUR{13634.00}}  \\
\hline
\caption{Preventivo a Finire}
\label{paf}
\end{longtable}

\subsubsection{Conclusioni}
Il costo finale del prodotto si attesta, quindi, ad \EUR{13634.00}, con una riduzione di \EUR{409.00} rispetto a quanto inizialmente preventivato. Tale differenza è dovuta principalmente a causa di un apporto lavorativo inferiore rispetto a quello preventivato da parte di un membro del gruppo. 