\subsubsection{Risanamento Criticità}
\label{RA2}

\paragraph{Prospetto Orario} \-\\

\begin{longtable}{|C{.30\textwidth}|C{.06\textwidth}|C{.06\textwidth}|C{.06\textwidth} | C{.06\textwidth}| C{.06\textwidth} | C{.06\textwidth} | C{.10\textwidth} |}
\hline
\rowcolor{bluelogo}	\textbf{\textcolor{white}{Nome}} & \textbf{\textcolor{white}{RE}} & \textbf{\textcolor{white}{AM}} & \textbf{\textcolor{white}{AN}} & \textbf{\textcolor{white}{PJ}} & \textbf{\textcolor{white}{PR}} & \textbf{\textcolor{white}{VE}} & \textbf{\textcolor{white}{Totale}}\\
\hline \hline

Marco Chilese & - & - & - & 3 & - & - & 3\\
\hline
\rowcolor{grigio}Marco Favaro & 3 & - & - & - & - & - & 3 \\
\hline
Diego Mazzalovo & - & - & - & 3 & - & - & 3 \\
\hline
\rowcolor{grigio}Carlotta Segna & - & - & - & - & 3 & - & 3\\
\hline
Matteo Slanzi & - & - & - & - & - & 3 & 3\\
\hline
\rowcolor{grigio}Bogdan Stanciu & - & 3 & - & - & - & - & 3 \\
\hline
Luca Violato & - & - & 3 & - & - & - & 3 \\
\hline

\caption{Consuntivo di Periodo: Risanamento Criticità 2}
\label{Distribuzione oraria del periodo di rc2}
\end{longtable}

\paragraph{Prospetto Economico} \-\\

\begin{longtable}{| C{.30\textwidth}| C{.15\textwidth}| C{.20\textwidth}|}
	\hline
	\rowcolor{bluelogo}\textbf{\textcolor{white}{Ruolo}} & \textbf{\textcolor{white}{Ore}} & \textbf{\textcolor{white}{Costo 	in \euro}} \\
	\hline 
	Responsabile & 3 & \EUR{90.00} \\
	\hline
	\rowcolor{grigio}Amministratore & 3 & \EUR{60.00} \\
	\hline
	Analista & 3 & \EUR{75.00} \\
	\hline
	\rowcolor{grigio}Progettista & 3 & \EUR{66.00}\\
	\hline
	Programmatore & 6 & \EUR{90.00} \\
	\hline 
	\rowcolor{grigio}Verificatore & 3 & \EUR{45.00} \\
	\hline
	\textbf{Totale} & 21 & \EUR{426.00} \\
	\hline 

\caption{Consuntivo di Periodo dei ruoli: Risanamento Criticità 2}
\label{Distribuzione ruoli RC2}
\end{longtable}

\paragraph{Variazioni nella Pianificazione} ~\\
Durante lo svolgimento delle attività non sono state commessi ritardi, quindi la pianificazione è rimasta invariata.

\paragraph{Conclusioni} ~\\

Le ore preventivate per questa fase sono state precedentemente correttamente preventivate. Le modifiche apportate sono state quelle segnalate in Revisione di Pianificazione dalla committente. 

\pagebreak

\subsubsection{Progettazione di Dettaglio e Codifica}
\label{PPDC}
\paragraph{Prospetto Orario } ~\\
\begin{longtable}{|C{.30\textwidth}|C{.06\textwidth}|C{.06\textwidth}|C{.06\textwidth} | C{.06\textwidth}| C{.06\textwidth} | C{.06\textwidth} | C{.10\textwidth} |}
	\hline
	\rowcolor{bluelogo}	\textbf{\textcolor{white}{Nome}} & \textbf{\textcolor{white}{RE}} & \textbf{\textcolor{white}{AM}} & \textbf{\textcolor{white}{AN}} & \textbf{\textcolor{white}{PJ}} & \textbf{\textcolor{white}{PR}} & \textbf{\textcolor{white}{VE}} & \textbf{\textcolor{white}{Totale}}\\
	\hline 
	Marco Chilese & - & - & - & 20 & 13 \{-3\} & 17 \{+3\}& 50 \\
	\hline
	\rowcolor{grigio}Marco Favaro &  - & - & - & 14 & 19 \{+4\} & 17 \{-4\} & 50 \\
	\hline
	Diego Mazzalovo & - & 8 \{-3\} & - & 18 & 13 \{-8\} & 11 \{+11\} & 50 \\
	\hline
	\rowcolor{grigio}Carlotta Segna & - & 14 \{-4\} & 15 \{+4\} & - & 21 & - & 50 \\
	\hline
	Matteo Slanzi & 8 & - & - & 11 \{-10\} & 21 & 10 \{+10\} & 50 \\
	\hline
	\rowcolor{grigio}Bogdan Stanciu & - & - & 8 \{-12\} & 12 &  20 \{+20\} & 10 \{-8\} & 50 \\
	\hline
	Luca Violato & 5 & - & - & 22 \{+3\} & 23 \{-3\} & - & 50 \\   
	\hline


\caption{Consuntivo di Periodo dei Ruoli: Progettazione di Dettaglio e Codifica}
\label{CP PDC}
\end{longtable}

\paragraph{Prospetto Economico} ~\\

\begin{longtable}{| C{.30\textwidth}| C{.15\textwidth}| C{.20\textwidth}|}
	\hline
	\rowcolor{bluelogo}\textbf{\textcolor{white}{Ruolo}} & \textbf{\textcolor{white}{Ore}} & \textbf{\textcolor{white}{Costo 	in \euro}} \\
	\hline 
	Responsabile & 13 & \EUR{390.00} \\
	\hline
	\rowcolor{grigio}Amministratore & 22 \{-3\} & \EUR{440.00}  \{-\EUR{60.00}\}\\
	\hline
	Analista & 23 \{-12\}& \EUR{575.00} \{-\EUR{300}\} \\
	\hline
	\rowcolor{grigio}Progettista & 97 \{-10\} & \EUR{2134.00}  \{-\EUR{220.00}\}\\
	\hline
	Programmatore & 130 \{+9\} & \EUR{1950.00} \{+\EUR{135.00\}} \\
	\hline 
	\rowcolor{grigio}Verificatore & 65 \{+16\} & \EUR{975.00} \{+\EUR{240.00\}}\\
	\hline
	\textbf{Totale} & 350 & \EUR{6464.00} \{-\EUR{205.00 }\}\\
	\hline 

\caption{Consuntivo di Periodo dei ruoli: Progettazione di Dettaglio e Codifica}
\label{Distribuzione ruoli pdc}
\end{longtable}

\paragraph{Variazioni nella Pianificazione} ~\\
Durante lo svolgimento delle attività non sono state commessi ritardi, quindi la pianificazione è rimasta invariata.

\paragraph{Conclusioni} ~\\
Nonostante le ore assegnate a ciascuna persona siano, nel totale, rimaste invariate, il numero di ore suddiviso tra persona ha subito delle modifiche. Questo si è verificato in quanto persone con più esperienza sono state assegnate allo svolgimento di compiti con i quali avevano precedentemente familiarizzato. \\
Un aumento delle ore di \textit{Verificatore} è dovuto al continuo controllo di documenti e codice che venivano scritti e sviluppati al fine di annullare gli errori che venivano commessi durante lo sviluppo di questi. L'aumento delle ore del \textit{Programmatore} è dovuto ad un lavoro maggiore rispetto a quanto preventivato. \\
Nella prossima fase vi sarà un incremento minimo dei documenti \textit{Piano di progetto}, \textit{Glossario}, \textit{Norme di Progetto} e \textit{Analisi dei Requisiti}, mentre per quanto riguarda il documento \textit{Piano di Qualifica} ci sarà un incremento notevole che sarà comportato dall'aggiunta di nuovi test di unità. Ci sarà quindi un grosso carico di lavoro per i \textit{Verificatori} che dovranno verificare tutti i documenti e implementare i test di unità e di integrazione precedentemente pianificati. L'esperienza da noi acquisita ci metterà nelle condizioni di non sforare le ore preventivate attuando metodologie di collaborazione tra chi ha scritto il codice e chi dovrà implementare i test. Per quanto riguarda i manuali, eventuali criticità saranno risanate nel minor tempo possibile, e successivamente verrà eseguito un eventuale incremento comportato dall'implementazione di requisiti non obbligatori, con l'obiettivo di rendere il prodotto più appetibile e utilizzabile.\\

\pagebreak