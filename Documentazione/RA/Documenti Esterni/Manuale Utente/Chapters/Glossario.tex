\section{Glossario}\label{glox}

\textbf{Alert}\-\\
È un segnale o un messaggio di avviso, avvertimento. \-\\

\textbf{Bug}\-\\
Errore o guasto che porta al malfunzionamento del software.\-\\

\textbf{Dashboard} \-\\
Nella Software Engineering con tale termine (che letteralmente potrebbe essere tradotto come "Cruscotto") si intente solitamente una pagina informatica dedicata
alla visualizzazione, anche eventualemte storica, di metriche, dati o informazioni, allo scopo di comprendere l'andamento di un progetto.\-\\

\textbf{DevOps}\-\\
Acronimo di Development e Operations, DevOps è un approccio allo sviluppo e all’implementazione di applicazioni in azienda, che enfatizza la collaborazione tra il team di sviluppo vero e proprio e quello delle operations, ossia che gestirà le applicazioni dopo il loro rilascio. \-\\


\textbf{Grafana}\-\\
Piattaforma open source che permette il monitoraggio e l'analisi di dati che vengono visualizzati in dashboard operative.\-\\

\textbf{InfluxDB}\-\\
InfluxDB è un database ottimizzato per le serie temporali, sviluppato da InfluxData.\-\\

\textbf{JavaScript}\-\\
È un linguaggio di scripting orientato agli oggetti, utilizzato nella programmazione web lato client per la creazione di effetti dinamici interattivi, tramite funzioni di script, invocate da eventi innescati dall'utente sulla pagina web. \-\\

\textbf{Jsbayes}\-\\
È una libreria open source per la gestione dei calcoli della rete Bayesiana sviluppata in JavaScript.\-\\

\textbf{JSON}\-\\
È un formato utilizzato per il salvataggio e lo scambio di dati in applicazioni client/server.\-\\

\textbf{Liveliness}\-\\
Dall'Inglese: "vivacità". Tuttavia si vuole intendere lo stato di vita di un sistema. \-\\


\textbf{NodeJS}\-\\
Piattaforma open source per scrivere applicazioni in JavaScript\glossario Server-side.\-\\

\textbf{PM2}\-\\
PM2 è un gestore di processo per Node.js runtime JavaScript. \-\\ 

\textbf{Rete Bayesiana} \-\\
Rappresentazione grafica delle relazioni di dipendenza tra le variabili di un sistema. In statistica la rete bayesiana è utilizzata per individuare più agevolmente le relazioni di dipendenza assoluta e condizionale tra le variabili, al fine di ridurre il numero delle combinazioni delle variabili da analizzare.\-\\


