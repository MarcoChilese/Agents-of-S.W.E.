\subsection{Selezione del Database}\label{SelectDB}

Una volta caricata una rete bayesiana (§\ref{ReteB}), al fine di collegare la stessa al flusso di monitoraggio, l'utente deve selezionare il Database contenente i dati da monitorare.\\
Tale operazione si articola in due passaggi fondamentali:
\begin{enumerate}
	\item \textbf{Passaggio 1:} L'utente seleziona, attraverso un menù a tendina, il database da usare come sorgente dati (Figura \ref{Sorgenti});
	\begin{figure}[H]
	\begin{center}
		\includegraphics[scale=0.68]{./images/Sorgenti.png}
		 \caption{Elenco Database disponibili per il collegamento}	
		 \label{Sorgenti}
	\end{center}
	\end{figure}
	\item \textbf{Passagio 2:} L'utente conferma la propria scelta attraverso il pulsante \textbf{Conferma}, presente in Figura \ref{Sorgenti}.
\end{enumerate}
~\\
A seguito della corretta selezione del Database da usare come sorgente dati, l'utente verrà avvisato del buon esito dell'operazione da un messaggio di notifica (Figura \ref{NotificaSorgente}). 

\begin{figure}[H]
	\begin{center}
		\includegraphics[scale=0.6]{./images/NotificaSorgente.png}
		 \caption{Notifica avvenuto collegamento Database}	
		 \label{NotificaSorgente}
	\end{center}
\end{figure}