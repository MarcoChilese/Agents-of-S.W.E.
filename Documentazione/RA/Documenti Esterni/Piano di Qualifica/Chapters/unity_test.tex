\begin{longtable}{|C{.15\textwidth}|m{.66\textwidth}|C{.10\textwidth}|}
\hline
\rowcolor{bluelogo}\textbf{\textcolor{white}{Test}}  & \textbf{\textcolor{white}{Descrizione}} & \textbf{\textcolor{white}{Stato}}\\
\hline \hline
\endhead

TU0-0 & Viene verificato che il file di configurazione esista all'interno della directory & N.I.\\
\hline 
\rowcolor{grigio} TU0-1 & Viene verificato che i parametri di configurazione obbligatori siano presenti nel file di configurazione & N.I. \\ 
\hline
TU0-2 & Viene verificato che le configurazioni rispettino la sintassi & N.I. \\ 
\hline
\rowcolor{grigio} TU0-3 & Viene verificata la conformità della sintassi alle configurazioni non obbligatorie & N.I. \\ 
\hline 
TU0-4 & Viene verificato che siano passati i parametri obbligatori all'avvio del server & N.I. \\ 
\hline
\rowcolor{grigio} TU0-5 & Viene verificata l'autenticità della porta obbligatoria all'avvio del server & N.I. \\ 
\hline
TU0-6 & Viene verificato  che l'incapsulamento dei parametri sia avvenuto con successo & N.I. \\
\hline 
\rowcolor{grigio} TU0-7 & Viene verificato il lancio di un'eccezione nel caso in cui la porta non sia disponibile & N.I. \\ 
\hline 
TU0-8 & Viene verificato il lancio di un'eccezione nel caso in cui la porta non sia un numero intero & N.I. \\
\hline
\rowcolor{grigio}  TU0-9 & Viene verificato il lancio di un'eccezione nel caso in cui manchino parametri obbligatori nel file di configurazione & N.I. \\ 
\hline
TU0-10 & Viene verificata l'inizializzazione del proxy server & N.I. \\ 
\hline 
\rowcolor{grigio} TU0-11 & Viene verificato che la richiesta di \texttt{root} al server del server ritorni l'ora corrente & N.I. \\ 
\hline
TU0-12 & Viene verificato che il tipo di ritorno dalla richiesta \texttt{root} al server sia di tipo json & N.I. \\ 
\hline 
\rowcolor{grigio}TU0-13 & Viene verificato che il tipo di ritorno della richiesta \texttt{alive} al server sia di tipo json & N.I. \\ 
\hline
 TU0-14 & Viene verificata che la richiesta \texttt{alive} ritorni data corrente e numero della porta in ascolto del server & N.I. \\ 
\hline 
\rowcolor{grigio}TU0-15 & Viene verificata che la richiesta \texttt{networks} ritorni un json & N.I. \\ 
\hline 
 TU0-16 & Viene verificato che venga chiamato il metodo \texttt{getNetworks()} & N.I. \\
\hline 
\rowcolor{grigio}TU0-17 & Viene verificato che il metodo \texttt{getNetworks()} ritorni un array di json & N.I. \\ 
\hline
TU0-18 & Viene verificato che per ogni json appartenente all'array ritornato da \texttt{getNetworks()} abbia un campo \textit{name} di tipo \textbf{string} ed un campo \textit{monitoring} di tipo \textbf{boolean} & N.I. \\ 
\hline
\rowcolor{grigio}TU0-19 & Viene verificato il lancio di un'eccezione dal metodo \texttt{getNetworks()} nel caso in cui l'accesso al filesystem sia proibito & N.I. \\ 
\hline 
TU0-20 & Viene verificata che la richiesta al server \texttt{uploadnetwork} chiami il metodo \texttt{saveNetworkToFile} passando un parametro di tipo json & N.I. \\ 
\hline 
\rowcolor{grigio}TU0-21 & Viene verificato, nel caso in cui la direcotry di salvataggio delle reti non sia presente, venga creata secondo le configurazioni & N.I. \\ 
\hline 
 TU0-22 & Viene verificato il lancio di un'eccezione nel caso in cui la creazione della cartella fallisca & N.I. \\ 
\hline 
\rowcolor{grigio}TU0-23 & Viene verificato che la rete venga sovrascritta nel caso in cui l'utente cerca di caricare la stessa rete & N.I. \\
\hline
 TU0-24 & Viene verificato il lancio di un'eccezione nel caso in cui la cancellazione di una rete sia fallito & N.I. \\ 
\hline
\rowcolor{grigio}TU0-25 & Viene verificato che la rete caricata disponga del campo \texttt{name} di tipo \textbf{stringa} & N.I. \\ 
\hline 
 TU0-26 & Viene verificato il lancio di un'eccezione nel caso in cui il campo dati \texttt{name} sia assente & N.I. \\ 
\hline 
\rowcolor{grigio}TU0-27	 & Viene verificata la creazione del file con la definizione della rete & N.I. \\ 
\hline 
 TU0-28 & Viene verificata il lancio di un'eccezione nel caso in cui la scritta su filesystem sia fallita & N.I. \\ 
\hline 
\rowcolor{grigio}TU0-29 & Viene verificata l'invocazione del metodo \texttt{initBayesianNetwork(net)} all'interno del metodo \texttt{saveNetworkToFile()} & N.I. \\ 
\hline 
TU0-30 & Viene verificata la creazione di un nuovo oggetto di tipo \texttt{Network} con la rete caricata dall'utente & N.I. \\ 
\hline
\rowcolor{grigio} TU0-31 & Viene verificato il lancio di un'eccezione nel caso in cui il metodo \texttt{saveNetworkToFile(net)} fallisca & N.I. \\
\hline
TU0-31 & Viene verificato che la richiesta di \texttt{uploadnetwork} ritorni una risposta con stato \textit{404} in caso di fallimento & N.I. \\ 
\hline
\rowcolor{grigio} TU0-32 & Viene verificato che la richiesta \texttt{uploadnetwork} ritorni un messaggio di successo nel caso in cui il metodo non ritorni errori & N.I. \\
\hline
TU0-33 & Viene verificata che la richiesta di \texttt{getnetwork/:net} al server, chiami il metodo \texttt{parserNetworkNameURL} & N.I. \\ 
\hline
\rowcolor{grigio} TU0-34 & Viene verificato che il metodo \texttt{parserNetworkNameURL} ritorni il nome della rete parsato & N.I. \\ 
\hline 
TU0-35 & Viene verificato che il metodo \texttt{parserNetworkNameURL} ritorni \textit{false} nel caso in cui: il parametro passato sia stringa vuota, la rete non esiste oppure il parametro non è definito & N.I. \\ 
\hline 
\rowcolor{grigio}TU0-36 & Viene verificato che la richiesta \texttt{getnetwork/:net} ritorni un json con la definizione della rete richiesta & N.I. \\ 
\hline 
TU0-37 & Viene verificato il lancio di un'eccezione nel caso in cui la lettura della rete da filesystem sia fallita & N.I. \\
\hline
 \rowcolor{grigio} TU0-38 & Viene verificato, in caso di eccezione nella richiesta \texttt{getnetwork/:net} che quest'ultima ritorni un messaggio di errore con stato \textit{404} & N.I. \\ 
 \hline
 TU0-39 & Viene verificato che la richiesta \texttt{networkslive} al server, ritorni un json contenente le reti monitorate in un dato istante di tempo & N.I. \\ 
 \hline 
 \rowcolor{grigio}TU0-40 & Viene verificato che la richiesta \texttt{deletenetwork/:net} al server, ritorni un messaggio di successo nel caso in cui la rete sia stata eliminata & N.I. \\
 \hline
 TU0-41 & Viene verificato che la richiesta \texttt{deletenetwork/:net} al server, ritorni un messaggio d'errore nel caso in cui il nome della rete da eliminare sia vuoto,  non definito o non esista nel filesystem del server & N.I. \\ 
 \hline
\rowcolor{grigio}TU0-42 & Viene verificato il lancio di un'eccezione nel caso in cui la rete da eliminare non esista & N.I. \\ 
\hline
 TU0-43 & Viene verificato il lancio di un'eccezione nel caso in cui il file di salvataggio della rete non esista nel filesystem del server & N.I. \\ 
 \hline
 \rowcolor{grigio} TU0-44 & Viene verificato che la richiesta \texttt{addtopool/:net} al server richiami il metodo \texttt{parserNetworkNameURL} & N.I. \\ 
 \hline
 TU0-45 & Viene verificato che la richiesta \texttt{addtopool/:net} al server ritorni un messaggio di errore nel caso in cui il nome della rete non sia valido o che non esista la rete desiderata & N.I. \\ 
 \hline
 \rowcolor{grigio}TU0-46 & Viene verificata che la richiesta \texttt{addtopool/:net} al server richiami il metodo \texttt{addToPool(net)} & N.I. \\ 
 \hline
 TU0-47 & Viene verificato che il metodo \texttt{addToPool(net)} ritorni aggiunga nel pool di monitoraggio la rete desiderata & N.I. \\
\hline
\rowcolor{grigio} TU0-48 & Viene verificato che il metodo \texttt{addToPool(net)} ritorni true nel caso in cui la rete sia stata aggiunta al pool di monitoraggio con successo & N.I. \\ 
\hline 
TU0-49 & Viene verificato che il metodo \texttt{addToPool(net)} ritorni false nel caso in cui la rete da monitorare e già monitorata & N.I. \\ 
\hline
\rowcolor{grigio} TU0-50 & Viene verificato che la richiesta \texttt{addtopool/:net} ritorni un messaggio di successo nel caso in cui la rete sia stata aggiunta al pool di monitoraggio & N.I. \\ 
\hline 
TU0-51 & Viene verificato che la richiesta \texttt{addtopool/:net} al server ritorni un messaggio di errore nel caso in cui la rete desiderata è già aggiunta al pool di monitoraggio & N.I. \\ 
\hline 
\rowcolor{grigio} TU0-52 & Viene verificato che la richiesta \texttt{getnetworkprob/:net} al server richiami il metodo \texttt{parserNetworknameURL(name)} &  N.I. \\ 
\hline 
TU0-53 & Viene verificato che la richiesta \texttt{getnetworkprob/:net} al server ritorni un messaggio di errore con stato \textit{404} nel caso in cui il nome passato a parametro non sia valido & N.I. \\ 
\hline
\rowcolor{grigio} TU0-54 & Viene verificato che la richiesta \texttt{getnetworkprob/:net} al server ritorni un messaggio di errore nel caso in cui la rete desiderata non sia in monitoraggio & N.I. \\ 
\hline 
TU0-55 & Viene verificato che la richiesta \texttt{getnetworkprob/:net} al server ritorni un json contenente le probabilità calcolate per la rete desiderata & N.I. \\ 
\hline 
\rowcolor{grigio} TU0-56 & Viene verificato che la richiesta \texttt{deletenetpool/:net} al server richiami il metodo \texttt{parserNetworkNameURL(name)} & N.I. \\ 
\hline
TU0-57 & Viene verificato che la richiesta \texttt{deletenetpool/:net} al server richiami il metodo \texttt{deleteFromPool(net)} & N.I. \\ 
\hline 
\rowcolor{grigio} TU0-58 & Viene verificato che il metodo \texttt{deleteFromPool(net)} ritorni true nel caso in cui la rete viene tolta dal monitoraggio & N.I. \\ 
\hline
TU0-59 & Viene verificato che il metodo \texttt{deleteFromPool(net)} ritorni false nel caso in cui l'eliminazione della rete dal monitoraggio fallisca & N.I. \\ 
\hline 
\rowcolor{grigio} TU0-60 & Viene verificato che la richiesta \texttt{deletenetpool/:net} ritorni un messaggio di successo nel caso in cui la rete viene tolta dal monitoraggio & N.I. \\ 
\hline 
TU0-61 & Viene verificato che la richiesta \texttt{deletenetpool/:net} al server, ritorni un messaggio di errore con stato \textit{404} nel caso in l'eliminazione della rete dal monitoraggio sia fallita & N.I. \\ 
\hline 
\rowcolor{grigio} TU0-62 & Viene verificato che all'avvio del server venga invocato il metodo \texttt{startServer()} & N.I. \\ 
\hline
TU0-63 & Viene verificato che all'avvio del server vengano inizializzate le reti salvate nel filessytem & N.I. \\ 
\hline
\rowcolor{grigio}TU0-64 & Viene verificato il lancio di un'eccezione nel caso in cui la lettura da filesystem all'inizializzazione delle reti salvata fallisca & N.I. \\ 
\hline
TU0-65 & Viene verificata l'inizializzazione delle connessioni necessarie al database per ogni rete & N.I. \\ 
\hline
\rowcolor{grigio}TU0-66 &  Viene verificata la creazione di una connessione al database al richiamo del metodo \texttt{initDatabaseConnection(connection)} & N.I.\\ 
\hline
TU0-67 & Viene verificato il lancio di un'eccezione nel caso in cui i parametri di connessione al database siano errati & N.I. \\
\hline
\rowcolor{grigio} TU0-68 & Viene verificato il lancio di un'eccezione nel caso in cui la creazione della connessione al database fallisca & N.I. \\ 
\hline
TU0-69 & Viene verificato che il metodo di creazione di connessione al database ritorni true se la connessione è avvenuta con successo & N.I. \\ 
\hline 
\rowcolor{grigio} TU0-70 & Viene verificata l'istanziazione effettiva della connessione al database nell'array di connessioni disponili sul server & N.I. \\ 
\hline 
TU0-71 & Viene verificato che il monitoraggio di una rete si ripeta secondo le politiche temporali definite dall'utente & N.I. \\ 
\hline 
\rowcolor{grigio} TU0-72 & Viene verificato che le probabilità calcolate nella rete bayesiana vengano salvate all'interno del database  & N.I. \\ 
\hline 
TU0-73 & Viene verificato che il metodo \texttt{writeMesure(net)}  ritorni true nel caso in cui la scrittura delle probabilità sul database sia avvenuta con successo & N.I. \\
\hline
\rowcolor{grigio}TU0-74 & Viene verificato che il metodo \texttt{writeMesure(net)} ritorni false nel caso in cui la scrittura delle probabilità sul database sia fallita & N.I. \\ 
\hline
TU0-75 & Viene verificata la costruzione corretta di una rete & N.I.  \\ 
\hline 
\rowcolor{grigio}TU0-76 & Viene verificata la creazione della tabella di probabilità per ogni nodo della rete bayesiana & N.I. \\ 
\hline
TU0-77 & Viene verificato che il \textit{Parser} riconosca una rete bayesiana ben formata come tale, quindi priva di errori & S. \\
\hline
\rowcolor{grigio}TU0-78 & Viene verificato che il metodo \texttt{checkMinimumFields} del \textit{Parser} torni \textbf{True} se il file di definizione \textit{JSON} della rete bayesiana ha il corretto numero di campi & S.\\
\hline
TU0-79 & Viene verificato che il metodo \texttt{checkMinimumFields} del \textit{Parser} torni \textbf{False} se il file di definizione \textit{JSON} della rete bayesiana non ha il corretto numero di campi & S.\\
\hline
\rowcolor{grigio}TU0-80 & Viene verificato che il \textit{Parser} riconosca e segnali se un campo dati del file di definizione ha una nomenclatura non conforme a quanto previsto & S.\\
\hline
TU0-81 & Viene verificato che il metodo \texttt{checkNamedNodes} del \textit{Parser} segnali con un messaggio di errore se un certo campo del file di definizione ha un numero di righe non conforme & S.\\
\hline
\rowcolor{grigio}TU0-82 & Viene verificato che il metodo \texttt{checkNamedNodes} del \textit{Parser} segnali con un messaggio di errore se un certo campo del file di definizione ha una definizione mancante & S.\\
\hline
TU0-83 & Viene verificato che il metodo \texttt{countNumberOfValue} del \textit{Parser} ritorni l'effettivo numero di probabilità definite per un dato nodo & S.\\
\hline
\rowcolor{grigio}TU0-84 & Viene verificato che il metodo \texttt{checkStates} del \textit{Parser} ritorni \textbf{True} nel caso in cui il file di definizione \textit{JSON} della rete bayesiana abbia correttamente definito gli stati dei nodi & S.\\
\hline
TU0-85 & Viene verificato che il metodo \texttt{checkStates} del \textit{Parser} segnali con un messaggio di errore se un nodo possiede meno di due stati & S.\\
\hline
\rowcolor{grigio}TU0-86 & Viene verificato che il metodo \texttt{checkStates} del \textit{Parser} segnali con un messaggio di errore se all'interno del file di definizione \textit{JSON} si tenti di ridefinire più volte il medesimo stato & S.\\
\hline
TU0-87 & Viene verificato che il metodo \texttt{checkParents} del \textit{Parser} torni \textbf{True} nel caso in cui il file di definizione \textit{JSON} della rete bayesiana abbia correttamente definito i padri dei nodi & S.\\
\hline
\rowcolor{grigio}TU0-88 & Viene verificato che il metodo \texttt{checkStates} del \textit{Parser} segnali con un messaggio di errore se all'interno del file di definizione \textit{JSON} si tenti di ridefinire più volte il medesimo padre di uno stesso nodo & S.\\
\hline
TU0-89 & Viene verificato che il metodo \texttt{checkStates} del \textit{Parser} segnali con un messaggio di errore se all'interno del file di definizione \textit{JSON} venga indicato come padre un nodo non esistente & S.\\
\hline
\rowcolor{grigio}TU0-90 & Viene verificato che il metodo \texttt{checkStates} del \textit{Parser} segnali con un messaggio di errore se all'interno del file di definizione \textit{JSON} venga indicato come padre di un nodo se stesso & S.\\
\hline
TU0-91 & Viene verificato che il metodo \texttt{checkProbabilities} del \textit{Parser} torni \textbf{True} nel caso in cui il file di definizione \textit{JSON} della rete bayesiana abbia correttamente definito le probabilità dei nodi & S.\\
\hline
\rowcolor{grigio}TU0-92 & Viene verificato che il metodo \texttt{checkProbabilities} del \textit{Parser} segnali con un messaggio di errore se all'interno del file di definizione \textit{JSON} venga definita una probabilità negativa & S.\\
\hline
TU0-93 & Viene verificato che il metodo \texttt{checkProbabilities} del \textit{Parser} segnali con un messaggio di errore se all'interno del file di definizione \textit{JSON} viene definita una probabilità maggiore al 100\% & S.\\
\hline
\rowcolor{grigio}TU0-94 & Viene verificato che il metodo \texttt{setConfirmationToTrue} di \textit{TemporalPolicyCtrl} segnali con un messaggio di errore se viene impostata una politica temporale avete valore associato al campo \textbf{secondi} > 60 & S.\\
\hline
TU0-95 & Viene verificato che il metodo \texttt{setConfirmationToTrue} di \textit{TemporalPolicyCtrl} segnali con un messaggio di errore se viene impostata una politica temporale avete valore negativo associato al campo \textbf{secondi} & S.\\
\hline
\rowcolor{grigio}TU0-96 & Viene verificato che il metodo \texttt{setConfirmationToTrue} di \textit{TemporalPolicyCtrl} segnali con un messaggio di errore se viene impostata una politica temporale avete valore associato al campo \textbf{minuti} > 60 & S.\\
\hline
TU0-97 & Viene verificato che il metodo \texttt{setConfirmationToTrue} di \textit{TemporalPolicyCtrl} segnali con un messaggio di errore se viene impostata una politica temporale avete valore negativo associato al campo \textbf{minuti} & S.\\
\hline
\rowcolor{grigio}TU0-98 & Viene verificato che il metodo \texttt{setConfirmationToTrue} di \textit{TemporalPolicyCtrl} segnali con un messaggio di errore se viene impostata una politica temporale avete valore negativo associato al campo \textbf{ore} & S.\\
\hline
TU0-99 & Viene verificato che il metodo \texttt{setConfirmationToTrue} di \textit{TemporalPolicyCtrl} torni \textbf{true} nel caso in cui venga impostata una politica temporale valida & S.\\
\hline
\rowcolor{grigio}TU0-100 & Viene verificato che il metodo \texttt{checkIfTherIsAtLeastOneTreshold} di \textit{TresholdCtrl} torni \textbf{False} nel caso in cui non venga impostata una tabella per la sorgente dati all'interno del contesto del collegamento di un nodo & S.\\
\hline
TU0-101 & Viene verificato che il metodo \texttt{checkIfTherIsAtLeastOneTreshold} di \textit{TresholdCtrl} torni \textbf{False} nel caso in cui non venga impostata un flusso all'interno del contesto del collegamento di un nodo & S.\\
\hline
\rowcolor{grigio}TU0-102 & Viene verificato che il metodo \texttt{checkIfTherIsAtLeastOneTreshold} di \textit{TresholdCtrl} torni \textbf{False} nel caso in cui non venga impostata alcuna soglia all'interno del contesto del collegamento di un nodo & S.\\
\hline
TU0-103 & Viene verificato che il metodo \texttt{checkNotRepeatedTresholds} di \textit{TresholdCtrl} torni \textbf{False} nel caso in cui vengano impostate due soglie identiche all'interno del contesto del collegamento di un nodo & S.\\
\hline
\rowcolor{grigio}TU0-104 & Viene verificato che il metodo \texttt{checkConflicts} di \textit{TresholdCtrl} torni \textbf{False} nel caso in cui vengano impostate soglie tra loro non coerenti all'interno del contesto del collegamento di un nodo & S.\\
\hline
TU0-105 & Viene verificato che il metodo \texttt{confirmTresholdsChanges} di \textit{TresholdCtrl} torni \textbf{True} nel caso in cui vengano configurate delle impostazioni corrette per il collegamento del nodo & S.\\
\hline
\rowcolor{grigio}TU0-106 & Viene verificato che il metodo \texttt{deleteTreshold} di \textit{TresholdCtrl} torni \textbf{True} nel caso in cui venga eliminata una soglia precedentemente definita & S.\\
\hline
TU0-107 & Viene verificato che il metodo \texttt{deleteTreshold} di \textit{TresholdCtrl} torni \textbf{False} nel caso in cui venga selezionata per l'eliminazione una soglia non esistente & S.\\
\hline
\rowcolor{grigio}TU0-108 & Viene verificato che il metodo \texttt{checkConflictSameSign} di \textit{TresholdCtrl} torni \textbf{False} nel caso in cui vengano definite due soglie di verso opposto che condividono uno stesso estremo all'interno del contesto del collegamento di un nodo & S.\\
\hline
TU0-109 & Viene verificato che il metodo \texttt{setNotLinked} di \textit{TresholdCtrl} elimini il collegamento del nodo al flusso dati & S.\\
\hline
\rowcolor{grigio}TU0-110 & Viene verificato che il metodo \texttt{checkMonitoring} di \textit{ModalCreator} torni \textbf{True} e faccia comparire una modal che segnali il monitoraggio in corso nel caso in cui la rete sia in monitoraggio & S. \\
\hline
TU0-111 & Viene verificato che la funzione \texttt{checkMonitoring} di \textit{ModalCreator} torni \textbf{False} nel caso in cui la rete non sia in monitoraggio & S. \\
\hline
\rowcolor{grigio}TU0-112 & Viene verificato che la funzione \texttt{checkDB} di \textit{ModalCreator} torni \textbf{False} e faccia comparire una modal che segnali il mancato collegamento nel caso in cui la rete non sia collegata ad un database & S. \\
\hline
TU0-113 & Viene verificato che la funzione \texttt{checkDB} di \textit{ModalCreator} torni \textbf{True} nel caso in cui la rete non sia collegata ad un Database & S. \\
\hline
\rowcolor{grigio}TU0-114 & Viene verificato che il metodo \texttt{showMessageModal} di \textit{ModalCreator} faccia correttamente comparire una modal nella quale compaia il messaggio da segnalare & S. \\
\hline
TU0-115 & Viene verificato che il metodo \texttt{showTresholdModal} di \textit{ModalCreator} ritorni come valore \textbf{True} e faccia comparire una modal per la definizione delle soglie se le condizioni per il collegamento del nodo sono verificate & S. \\
\hline
\rowcolor{grigio}TU0-116 & Viene verificato che il metodo \texttt{showTresholdModal} di \textit{ModalCreator} ritorni come valore \textbf{False}, e faccia comparire una modal che segnali l'errore, nel caso in cui si tenti di modificare le impostazioni di collegamento durante un monitoraggio attivo o senza aver connesione al database  & S. \\
\hline
TU0-117 & Viene verificato che il metodo \texttt{selectTemporalPolicy} di \textit{ModalCreator} ritorni come valore \textbf{True} e faccia comparire una modal per la definizione della politica temporale se le condizioni per la sua definizione sono tutte rispettate & S. \\
\hline
\rowcolor{grigio}TU0-118 & Viene verificato che il metodo \texttt{selectTemporalPolicy} di \textit{ModalCreator} ritorni come valore \textbf{False}, e faccia comparire una modal che segnali l'errore, nel caso in cui si tenti di definire una politica temporale durante un monitoraggio attivo o senza aver connesione al database & S. \\
\hline
TU0-119 & Viene verificato che il metodo \texttt{getTables} di \textit{GetApiGrafana} torni come risultato un array contenente tutte le tabelle del database, il quale è un campo dati della classe stessa & S.\\
\hline
\rowcolor{grigio}TU0-120 & Viene verificato che il metodo \texttt{getFields} di \textit{GetApiGrafana} torni come risultato un \textit{JSON} contenente tutti i campi di tutte le tabelle del database & S.\\
\hline
TU0-121 & Viene verificato che il metodo \texttt{divideFields} di \textit{GetApiGrafana} data una certa struttura dati possibilmente non ben formattata torni come risultato un \textit{JSON} contenente tabelle e corrispondenti campi correttamente strutturati & S.\\
\hline
\rowcolor{grigio}TU0-122 & Viene verificato che il metodo \texttt{getData} di \textit{GetApiGrafana} faccia una richiesta alle API \textit{Grafana} e torni un file \textit{JSON} contente le informazioni dei database associati & S.\\
\hline
TU0-123 & Viene verificato che il metodo \texttt{alive} di \textit{ConnectServer} torni un \textit{JSON} contenente informazioni su data, ora e porta del server in ascolto & S.\\
\hline
\rowcolor{grigio}TU0-124 & Viene verificato che il metodo \texttt{networks} di \textit{ConnectServer} torni un \textit{JSON} contenente le reti bayesiane salvate nel server & S.\\
\hline
TU0-125 & Viene verificato che il metodo \texttt{uploadnetwork(net)} di \textit{ConnectServer} carichi sul server la rete "net" passata come parametro e torni \textbf{True} nel caso di buona riuscita dell'operazione & S.\\
\hline
\rowcolor{grigio}TU0-126 & Viene verificato che il metodo \texttt{deletenetwork(net)} di \textit{ConnectServer} elimini dal server la rete "net" passata come parametro e torni \textbf{True} nel caso di buona riuscita dell'operazione & S.\\
\hline
TU0-127 & Viene verificato che il metodo \texttt{getnetworkprob(net)} di \textit{ConnectServer} torni un \textit{JSON} contenente le probabilità aggiornate relative agli stati dei nodi della rete "net" passata come parametro & S.\\
\hline
\rowcolor{grigio}TU0-128 & Viene verificato che il metodo \texttt{getnetwork(net)} di \textit{ConnectServer} torni un \textit{JSON} contenente tutte le informazioni relative alla rete "net" memorizzata nel server & S.\\
\hline
TU0-129 & Viene verificato che il metodo \texttt{calculateSeconds(policy)} di \textit{GBCtrl} torni un \textbf{int} che rappresenta il numero di secondi di cui è composta la politica temporale "policy" passata come parametro & S.\\
\hline
\rowcolor{grigio}TU0-130 & Viene verificato che il metodo \texttt{panelPath} di \textit{GBCtrl}, nel caso in cui il pannello abbia un path definito, torni il path del pannello & S.\\
\hline
TU0-131 & Viene verificato che il metodo \texttt{panelPath} di \textit{GBCtrl}, nel caso in cui il pannello non abbia un path definito, assegni un path di default al pannello e lo ritorni & S.\\
\hline
\rowcolor{grigio}TU0-132 & Viene verificato che il metodo \texttt{onInitEditMode} di \textit{GBCtrl}, aggiunga la tab "server settings" all'editor tab del pannello e torni \textbf{True} & S.\\
\hline
TU0-133 & Viene verificato che il metodo \texttt{splitMonitoringNetworks} di \textit{GBCtrl} ritorni un array di reti bayesiane attualmente sotto monitoraggio & S.\\
\hline
\rowcolor{grigio}TU0-134 & Viene verificato che il metodo \texttt{requestNetworks} di \textit{GBCtrl} ricavi dal server le reti bayesiane caricate e le imposti nel pannello, tornando \textbf{True} nel caso l'operazione sia andata  a buon fine & S.\\
\hline
TU0-135 & Viene verificato che il metodo \texttt{requestNetworks} di \textit{GBCtrl} visualizzi un messaggio di errore e torni \textbf{False} nel caso in cui la connessione al server fallisca & S.\\
\hline
\rowcolor{grigio}TU0-136 & Viene verificato che il metodo \texttt{tryConnectServer} di \textit{GBCtrl} controlli lo stato della connessione al server e visualizzi un messaggio contestuale tornando \textbf{True} nel caso in cui la connessione al server sia attiva & S.\\
\hline
TU0-137 & Viene verificato che il metodo \texttt{tryConnectServer} di \textit{GBCtrl} controlli lo stato della connessione al server e visualizzi un messaggio di errore tornando \textbf{False} nel caso in cui la connessione al server non sia attiva & S.\\
\hline
\rowcolor{grigio}TU0-139 & Viene verificato che il metodo \texttt{checkIfNetworkIsDeletable(net)} di \textit{GBCtrl} controlli che la rete "net" passata come parametro sia eliminabile dal server, tornando \textbf{True} in caso affermativo & S.\\
\hline
TU0-140 & Viene verificato che il metodo \texttt{checkIfNetworkIsDeletable(net)} di \textit{GBCtrl} controlli che la rete "net" passata come parametro sia eliminabile dal server, tornando \textbf{False}, e visualizzando anche un contestuale messaggio di errore, nel caso in cui la rete sia in stato di monitoraggio & S.\\
\hline
\rowcolor{grigio}TU0-141 & Viene verificato che il metodo \texttt{checkIfNetworkIsDeletable(net)} di \textit{GBCtrl} controlli che la rete "net" passata come parametro sia eliminabile dal server, tornando \textbf{False}, e visualizzando anche un contestuale messaggio di errore, nel caso in cui la rete non sia specificata & S.\\
\hline
TU0-142 & Viene verificato che il metodo \texttt{requestNetworkDelete(net)} di \textit{GBCtrl} elimini dal server la rete "net" passata come parametro, tornando \textbf{True} nel caso l'operazione sia andata a buon fine & S.\\
\hline
\rowcolor{grigio}TU0-143 & Viene verificato che il metodo \texttt{requestNetworkDelete(net)} di \textit{GBCtrl} torni \textbf{False}, e mostri un corrispondente messaggio di errore, nel caso l'operazione non sia andata a buon fine a causa del fatto che la rete "net", passata come parametro, non sia eliminabile & S.\\
\hline
TU0-144 & Viene verificato che il metodo \texttt{requestNetworkDelete(net)} di \textit{GBCtrl} torni \textbf{False}, e mostri un corrispondente messaggio di errore, nel caso l'operazione non sia andata a buon fine a causa di un problema con la connesione al server & S.\\
\hline
\rowcolor{grigio}TU0-145 & Viene verificato che il metodo \texttt{updateProbs} di \textit{GBCtrl} aggiorni le probabilità che rappresentano i dati di monitoraggio della rete visualizzata sul pannello tornando \textbf{True} se l'operazione è andata a buon fine & S.\\
\hline
TU0-146 & Viene verificato che il metodo \texttt{changeNetworkToVisualizeMonitoring} di \textit{GBCtrl} \textcolor{red}{Trascrivere Test non comprensibile} & S.\\
\hline
\rowcolor{grigio}TU0-147 & Viene verificato che il metodo \texttt{changeNetworkToVisualizeMonitoring} di \textit{GBCtrl} \textcolor{red}{Trascrivere Test non comprensibile} & S.\\
\hline
TU0-148 & Viene verificato che il metodo \texttt{loadNetworkToServer(net)} di \textit{GBCtrl} carichi la rete "net" passata come parametro sul server, tornando \textbf{True}, e mostrando un corrispondente messaggio di successo, nel caso di buon esito dell'operazione & S.\\
\hline
\rowcolor{grigio}TU0-149 & Viene verificato che il metodo \texttt{loadNetworkToServer(net)} di \textit{GBCtrl} torni \textbf{False} nel caso in cui \textcolor{red}{Trascrivere motivazione} & S.\\
\hline
TU0-150 & Viene verificato che il metodo \texttt{saveActualChanges} di \textit{GBCtrl} \textcolor{red}{Trascrivere Test non comprensibile} & S.\\
\hline
\rowcolor{grigio}TU0-151 & Viene verificato che il metodo \texttt{saveActualChanges} di \textit{GBCtrl} \textcolor{red}{Trascrivere Test non comprensibile} & S.\\
\hline
TU0-152 & Viene verificato che il metodo \texttt{saveActualChanges} di \textit{GBCtrl} \textcolor{red}{Trascrivere Test non comprensibile} & S.\\
\hline
\rowcolor{grigio}TU0-153 & Viene verificato che il metodo \texttt{saveActualChanges} di \textit{GBCtrl} \textcolor{red}{Trascrivere Test non comprensibile} & S.\\
\hline
TU0-154 & Viene verificato che il metodo \texttt{saveActualChanges} di \textit{GBCtrl} \textcolor{red}{Trascrivere Test non comprensibile} & S.\\
\hline
\rowcolor{grigio}TU0-155 & Viene verificato che il metodo \texttt{resetData} di \textit{GBCtrl} resetti tutte le impostazioni del pannello al valore di default, tornando \textbf{True} nel caso di buon esito dell'operazione & S.\\
\hline
TU0-156 & Viene verificato che il metodo \texttt{checkIfConnectableToDB} di \textit{GBCtrl} torni \textbf{True} nel caso sia possibile selezionare un databse come sorgente dati per la rete visualizzata nel pannello & S.\\
\hline
\rowcolor{grigio}TU0-157 & Viene verificato che il metodo \texttt{checkIfConnectableToDB} di \textit{GBCtrl} lanci un'eccezione nel caso in cui la rete visualizzata nel pannello sia sotto monitoraggio & S.\\
\hline
TU0-158 & Viene verificato che il metodo \texttt{checkIfConnectableToDB} di \textit{GBCtrl} lanci un'eccezione nel caso in cui non sia specificato alcun database & S.\\
\hline
\rowcolor{grigio}TU0-159 & Viene verificato che il metodo \texttt{checkIfConnectableToDB} di \textit{GBCtrl} lanci un'eccezione nel caso in cui sia indicato un database non influx & S.\\
\hline
TU0-160 & Viene verificato che il metodo \texttt{getConnectionToDB} di \textit{GBCtrl} \textcolor{red}{Trascrivere Test non comprensibile} & S.\\
\hline
\rowcolor{grigio}TU0-161 & Viene verificato che il metodo \texttt{getConnectionToDB} di \textit{GBCtrl} \textcolor{red}{Trascrivere Test non comprensibile} & S.\\
\hline
TU0-162 & Viene verificato che il metodo \texttt{getFlushes} di \textit{GBCtrl} recuperi dal database tutti i possibili flussi dati salvandoli in un'apposita variabile, tornando \textbf{True} nel caso l'operazione sia andata a buon fine & S.\\
\hline
\rowcolor{grigio}TU0-163 & Viene verificato che il metodo \texttt{getFlushes} di \textit{GBCtrl} recuperi dal database tutti i possibili flussi dati salvandoli in un'apposita variabile, tornando \textbf{False} nel caso in cui non sia possibile la connessione al database & S.\\
\hline
TU0-164 & Viene verificato che il metodo \texttt{connectToDB} di \textit{GBCtrl} effettui il collegamento della rete al database, eseguendo gli opportuni controlli e tornando \textbf{True} nel caso l'operazione sia andata a buon fine & S.\\
\hline
\rowcolor{grigio}TU0-165 & Viene verificato che il metodo \texttt{connectToDB} di \textit{GBCtrl} torni \textbf{False}, mostrando un corrispondente messaggio di errore, nel caso in cui non riesca a ricavare i flussi dati disponibili dal database selezionato & S.\\
\hline
TU0-166 & Viene verificato che il metodo \texttt{connectToDB} di \textit{GBCtrl} lanci un'eccezione nel caso in cui sia impossibile stabilire una connessione con il database & S.\\
\hline
\rowcolor{grigio}TU0-167 & Viene verificato che il metodo \texttt{showTresholdModal(node)} di \textit{GBCtrl} faccia comparire una modal per la definizione delle impostazioni di collegamento del nodo "node" passato come parametro, tornando \textbf{True} nel caso di buon esito dell'operazione & S.\\
\hline
TU0-168 & Viene verificato che il metodo \texttt{selectTemporalPolicy} di \textit{GBCtrl} faccia comparire una modal per la configurazione della politica temporale di ricalcolo delle probabilità, tornando \textbf{True} nel caso di buon esito dell'operazione & S.\\
\hline
\rowcolor{grigio}TU0-169 & Viene verificato che il metodo \texttt{visualizeMonitoring} di \textit{GBCtrl} cambi la visualizzazione del pannello, passando alla visualizzazione dei monitoraggi attivi & S.\\
\hline
TU0-170 & Viene verificato che il metodo \texttt{visualizeSettings} di \textit{GBCtrl} cambi la visualizzazione del pannello, passando alla visualizzazione delle impostazioni di monitoraggio & S.\\
\hline
\rowcolor{grigio}TU0-171 & Viene verificato che il metodo \texttt{loadNetworkFromSaved(net)} di \textit{GBCtrl} carichi nel pannello la rete "net",passata come parametro, insieme alle sue impostazioni di collegamento, tornando \textbf{True} nel caso di buon esito dell'operazione & S.\\
\hline
TU0-172 & Viene verificato che il metodo \texttt{requestNetworkToServer(string)} di \textit{GBCtrl} carichi nel pannello la rete memorizzata nel server, avente come nome la stringa passata come parametro, insieme alle sue impostazioni di collegamento, tornando \textbf{True} nel caso di buon esito dell'operazione & S.\\
\hline
\rowcolor{grigio}TU0-173 & Viene verificato che il metodo \texttt{requestNetworkToServer(string)} di \textit{GBCtrl} torni \textbf{False} nel caso non sia possibile reperire la rete, avente come nome la stringa passata come parametro, dal server & S.\\
\hline
TU0-174 & Viene verificato che il metodo \texttt{loadNetwork(data)} di \textit{GBCtrl} carichi sul pannello la rete bayesiana "data", passata come parametro (\textit{JSON}) al metodo, tornando \textbf{True} nel caso di buon esito dell'operazione & S.\\
\hline
\rowcolor{grigio}TU0-175 & Viene verificato che il metodo \texttt{loadNetwork(data)} di \textit{GBCtrl} torni \textbf{False} nel caso in cui \textcolor{red}{Trascrivere Condizione non comprensibile} & S.\\
\hline
TU0-176 & Viene verificato che il metodo \texttt{resetTresholds} di \textit{GBCtrl} elimini tutte le soglie, ripristinando le impostazioni di default del pannello & S.\\
\hline
\rowcolor{grigio}TU0-177 & Viene verificato che il metodo \texttt{checkIfAtLeastOneTresholdDefined} di \textit{GBCtrl} torni \textbf{True} nel caso sia definita almeno una soglia per il collegamento di un qualche nodo & S.\\
\hline
TU0-178 & Viene verificato che il metodo \texttt{checkIfAtLeastOneTresholdDefined} di \textit{GBCtrl} torni \textbf{False} nel caso non sia definita alcuna soglia per il collegamento di un qualche nodo & S.\\
\hline
\rowcolor{grigio}TU0-179 & Viene verificato che il metodo \texttt{deleteLinkToFlush(node)} di \textit{GBCtrl} scolleghi dal flusso di monitoraggio il nodo "node", passato come parametro al metodo, tornando \textbf{True} in caso di buon esito dell'operazione & S.\\
\hline
TU0-180 & Viene verificato che il metodo \texttt{deleteLinkToFlush(node)} di \textit{GBCtrl} torni \textbf{False} nel caso in cui il nodo "node", passato come parametro al metodo, appartenga ad una rete bayesiana sotto monitoraggio & S.\\
\hline
\rowcolor{grigio}TU0-181 & Viene verificato che il metodo \texttt{freeAllFlushes} di \textit{GBCtrl} riassegni ai flussi dati disponibili per il collegamento tutti i flussi dati occupati da nodi collegati & S.\\
\hline
TU0-182 & Viene verificato che il metodo \texttt{checkIfCanStartComputation} di \textit{GBCtrl} torni \textbf{True} nel caso in cui possa essere avviato il monitoraggio per la rete bayesiana visualizzata nel pannello & S.\\
\hline
\rowcolor{grigio}TU0-183 & Viene verificato che il metodo \texttt{checkIfCanStartComputation} di \textit{GBCtrl} lanci un'eccezione nel caso in cui non sia selezionato alcun database da usare come sorgente dati e dunque, di conseguenza, non possa essere avviato il monitoraggio della rete visualizzata nel pannello & S.\\
\hline
TU0-184 & Viene verificato che il metodo \texttt{checkIfCanStartComputation} di \textit{GBCtrl} lanci un'eccezione nel caso in cui non sia impostata alcuna politica temporale e dunque, di conseguenza, non possa essere avviato il monitoraggio della rete visualizzata nel pannello & S.\\
\hline
\rowcolor{grigio}TU0-185 & Viene verificato che il metodo \texttt{checkIfCanStartComputation} di \textit{GBCtrl} lanci un'eccezione nel caso in cui non sia collegato alcun nodo al flusso dati e dunque, di conseguenza, non possa essere avviato il monitoraggio della rete visualizzata nel pannello & S.\\
\hline
TU0-186 & Viene verificato che il metodo \texttt{startComputation} di \textit{GBCtrl} faccia partire il monitoraggio per la rete impostata nel pannello, tornando \textbf{True} nel caso di buon esito dell'operazione & S.\\
\hline
\rowcolor{grigio}TU0-187 & Viene verificato che il metodo \texttt{startComputation} di \textit{GBCtrl} torni \textbf{False} nel caso non sia possibile avviare il monitoraggio della rete impostata nel pannello & S.\\
\hline
TU0-188 & Viene verificato che il metodo \texttt{startComputation} di \textit{GBCtrl} laci un'eccezione nel caso in cui sia impossibile stabilire una connessione con il server & S.\\
\hline
\rowcolor{grigio}TU0-189 & Viene verificato che il metodo \texttt{closeComputation} di \textit{GBCtrl} interrompa il monitoraggio per la rete impostata nel pannello, tornando \textbf{True} nel caso di buon esito dell'operazione & S.\\
\hline
TU0-190 & Viene verificato che il metodo \texttt{closeComputation} di \textit{GBCtrl} torni \textbf{False} nel caso non sia possibile interrompere il monitoraggio per la rete impostata nel pannello a causa di un'anomalia nella connessione con il server & S.\\
\hline
\caption{Test di unità}
\label{testunita}
\end{longtable}



