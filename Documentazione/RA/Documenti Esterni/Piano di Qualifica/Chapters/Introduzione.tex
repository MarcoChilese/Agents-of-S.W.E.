\section{Introduzione}
\label{introduzione}

\subsection{Scopo del Documento}

Lo scopo del documento \textit{Piano di Qualifica v3.0.0} è di stabilire gli obbiettivi metrici da dover rispettare nello sviluppo di processi e prodotti sviluppati dal gruppo \texttt{Agents of S.W.E.} per la verifica\glossario e validazione\glossario degli stessi. In particolare verranno seguite le norme descritte nel documento \textit{Norme di Progetto v3.0.0}. Sarà compito dei \textit{Verificatori} del gruppo provvedere ad una continua verifica dei processi e dei prodotti in modo da ottenere incrementi parziali, fino ad arrivare alla realizzazione completa del progetto, senza l'inserimento di errori che possano compromettere il risultato finale. 

\subsection{Scopo del Prodotto}
Lo scopo del prodotto è la creazione di un plug-in per la piattaforma open source\glossario di visualizzazione e gestione dati, denominata \textit{Grafana}, 
con l'obiettivo di creare un sistema di alert dinamico per monitorare la "liveliness"\glossario del sistema a supporto dei processi
DevOps\glossario e per consigliare interventi nel sistema di produzione del software.
In particolare, il plug-in utilizzerà dati in input forniti ad intervalli regolari o con continuità, ad una rete bayesiana\glossario per stimare la probabilità di alcuni eventi, segnalandone quindi il rischio in modo dinamico, prevenendo situazioni di stallo.   

\subsection{Incrementalità}
Essendo un documento incrementale la versione 3.0.0 consegnata non è da intendersi come finale: al suo interno sono presenti solamente i contenuti riguardanti argomenti propri del periodo di \textit{Avvio ed Analisi dei Requisiti}, \textit{Revisione di Progettazione} e  \textit{Progettazione di Dettagli e Codifica}. \\
Nella fase di  \textit{Validazione e Collaudo} andremo a finire il prodotto, implementare i test ancora mancanti ed eseguire le misurazioni necessarie a rispettare quanto stabilito dal presente documento.

\subsection{Ambiguità e Glossario}
I termini che potrebbero risultare ambigui all'interno del documento sono siglati tramite pedice rappresentante la lettera \textmd{G}, tale terminologia trova una sua più specifica definizione nel \textit{Glossario v3.0.0} che viene fornito tra i Documenti Esterni.

\subsection{Riferimenti}
\label{riferimenti}
\subsubsection{Riferimenti Normativi}

	\begin{itemize}
		\item \textbf{\textit{Norme di Progetto v3.0.0}};
		\item \textbf{Standard ISO/IEC 9126} : \\ \url{https://it.wikipedia.org/wiki/ISO/IEC_9126};
		\item \textbf{Standard ISO/IEC 15504} : \\ \url{https://en.wikipedia.org/wiki/ISO/IEC_15504};
	\end{itemize}


\subsubsection{Riferimenti Informativi}

	\begin{itemize}
		\item \textbf{PDCA} :\\ \url{https://it.wikipedia.org/wiki/Ciclo_di_Deming};
		\item \textbf{Metriche per il Software} :
			\begin{itemize}
			\item \url{https://metriche-per-il-software-pa.readthedocs.io/it/latest/documento-in-consultazione/metriche-e-strumenti.html#misurazioni-di-portabilita};
			\item \url{https://www.sealights.io/test-metrics/11-test-automation-metrics-and-their-pros-cons/};
			\end{itemize}
		\item \textbf{Metriche per la Scrittura} :\\\url{http://wpage.unina.it/ptramont/Download/Tesi/LAURENZAGABRIELLA.pdf};
		\item \textbf{Tempi di Risposta} :\\ \url{https://www.cdnetworks.com/it/news/6-parametri-critici-da-considerare-relativi-alle-prestazioni-delle-applicazioni-web/479};
		\item \textbf{Indice di Gulpease}: \\ \url{https://it.wikipedia.org/wiki/Indice_Gulpease};
		
		
		\item \textbf{Materiale Didattico del Corso di Ingegneria del Software:}
			\begin{itemize}
				\item \textbf{Qualità di Prodotto}: \\ \url{https://www.math.unipd.it/~tullio/IS-1/2018/Dispense/L13.pdf};
				
				\item \textbf{Qualità di Processo}: \\ \url{https://www.math.unipd.it/~tullio/IS-1/2018/Dispense/L15.pdf};
				
				\item \textbf{Verifica e Validazione: Introduzione}: \\ \url{https://www.math.unipd.it/~tullio/IS-1/2018/Dispense/L16.pdf};
				
				\item \textbf{Verifica e Validazione: Analisi Statica}: \\ \url{https://www.math.unipd.it/~tullio/IS-1/2018/Dispense/L17.pdf};
				\item \textbf{Verifica e Validazione: Analisi Dinamica}: \\ \url{https://www.math.unipd.it/~tullio/IS-1/2018/Dispense/L18.pdf};
			\end{itemize}
			
		\item \textbf{Materiale Didattico del Corso di Tecnologie Open Source\footnote{Tenuto dal Prof. Bertazzo nel corrente A.A. 2018-2019. Il materiale didattico citato è disponibile nella piattaforma di e-learning Moodle.}:}
			\begin{itemize}
				\item \textbf{Test del Software}: Lezione 7;
				\item \textbf{Test di unità, Test di Integrazione e Test di Sistema}: Lezione 8.
			\end{itemize}
	\end{itemize}