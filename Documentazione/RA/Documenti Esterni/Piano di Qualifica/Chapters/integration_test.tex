\begin{longtable}{|C{.15\textwidth}|m{.52\textwidth}|m{.08\textwidth}|}
\hline
\rowcolor{bluelogo}\textbf{\textcolor{white}{Test}} & \textbf{\textcolor{white}{Descrizione}} & \textbf{\textcolor{white}{Esito}}\\
\hline \hline
\endhead

TI0-01 & Verificare che il caricamento di una rete bayesiana e le impostazioni di monitoraggio a essa correlate venga caricata nel server in maniera corretta. & N.I. \\
\hline
\rowcolor{grigio} TI0-02 & Verificare che una rete bayesiana e le relative impostazioni salvate nel server vengano correttamente caricate nel pannello per eventuali modifiche. & N.I. \\
\hline
TI0-03 & Verificare che la ricezione delle probabilità dal server avvenga in maniera corretta. & N.I. \\
\hline
\rowcolor{grigio} TI0-04 & Verificare che vengano visualizzati i database messi a disposizione da Grafana & N.I. \\
\hline
TI0-05 & Verificare che dato un database, vengano visualizzate le relative tabelle e flussi dati & N.I. \\
\hline
\rowcolor{grigio}TI0-06 & Verificare che il server comunichi correttamente con il pannello & N.I. \\
\hline
TI0-07 & Verificare che, una volta selezionata una rete nel pannello, il server la elimini correttamente & N.I. \\
\hline 
\rowcolor{grigio}TI0-08 & Verificare che, una volta fatto partire il monitoraggio dal pannello, il server esegua correttamente il ricalcolo delle probabilità & N.I. \\  
\hline
TI0-09 & Viene verificato che il metodo \texttt{loadNetwork(data)} di \textit{GBCtrl} carichi sul pannello la rete bayesiana "data", passata come parametro (\textit{JSON}) al metodo, tornando \textbf{True} nel caso di buon esito dell'operazione, e che invochi correttamente il parser per verificare la rete & S.\\
\hline
\rowcolor{grigio}TI0-10 & Viene verificato che il metodo \texttt{loadNetwork(data)} di \textit{GBCtrl} torni \textbf{False} nel caso in cui all'interno del metodo ci sia un altro dei metodi richiamati che lancia un'eccezione, o per l'impossibilità di collegarsi al server o per problemi nella struttura della rete  & S.\\
\hline
	
\caption{Test di Integrazione}
\label{testintegrazione}
\end{longtable}